%%%%%%%%%%%%%%%%%%%%%%%%%%%%%%%%%%%%%%%%%%%%%%%%%%%%%%%%%%%%%%%%%%%%%%%%%
%                                                                       %
% ustthesis_test.tex: A template file for usage with ustthesis.cls      %
%                                                                       %
%%%%%%%%%%%%%%%%%%%%%%%%%%%%%%%%%%%%%%%%%%%%%%%%%%%%%%%%%%%%%%%%%%%%%%%%%

\documentclass{ustthesis}

% \usepackage{latexsym}
    % Use the "latexsym" package when encountering the following error:
    %   ! LaTeX Error: Command \??? not provided in base LaTeX2e.
% \usepackage{epsf}
    % Use the "epsf" package for including EPS files.

\usepackage{algorithm}
\usepackage{algorithmic}
\usepackage{amsmath}
\usepackage{amsthm}
\usepackage{amssymb}
\usepackage{booktabs}
\usepackage{caption}
\usepackage{graphicx}
\usepackage{multicol}
\usepackage{subfigure}
%\usepackage{hyperref}
\usepackage{multirow}
\usepackage{rotating}
\usepackage{theapa}
%\usepackage{natbib}
\usepackage{epsfig}

\usepackage[a4paper,top=2.5cm,bottom=2.5cm,left=2.5cm,right=2.5cm]{geometry}
%\usepackage[justification=centering]{caption}
\usepackage{url}

\numberwithin{algorithm}{chapter}
%\renewcommand{\theHalgorithm}{\arabic{chapter}.\arabic{algorithm}}

\def \cite{\shortcite}
%\def \citep{\shortcite}

%\def \cite{\cite}
%\def \citep{\cite}


\def\A{{\bf A}}
\def\B{{\bf B}}
\def\C{{\bf C}}
\def\D{{\bf D}}
\def\E{{\bf E}}
\def\F{{\bf F}}
\def\G{{\bf G}}
\def\H{{\bf H}}
\def\I{{\bf I}}
\def\J{{\bf J}}
\def\K{{\bf K}}
\def\L{{\bf L}}
\def\M{{\bf M}}
\def\N{{\bf N}}
\def\O{{\bf O}}
\def\P{{\bf P}}
\def\Q{{\bf Q}}
\def\R{{\bf R}}
\def\S{{\bf S}}
\def\T{{\bf T}}
\def\U{{\bf U}}
\def\V{{\bf V}}
\def\W{{\bf W}}
\def\X{{\bf X}}
\def\Y{{\bf Y}}
\def\Z{{\bf Z}}

\def\a{{\bf a}}
\def\b{{\bf b}}
\def\c{{\bf c}}
\def\d{{\bf d}}
\def\e{{\bf e}}
\def\f{{\bf f}}
\def\g{{\bf g}}
\def\h{{\bf h}}
\def\i{{\bf i}}
\def\j{{\bf j}}
\def\k{{\bf k}}
\def\l{{\bf l}}
\def\m{{\bf m}}
\def\n{{\bf n}}
\def\o{{\bf o}}
\def\p{{\bf p}}
\def\q{{\bf q}}
\def\r{{\bf r}}
\def\s{{\bf s}}
\def\t{{\bf t}}
\def\u{{\bf u}}
\def\v{{\bf v}}
\def\w{{\bf w}}
\def\x{{\bf x}}
\def\y{{\bf y}}
\def\z{{\bf z}}

\def\0{{\bf 0}}
\def\1{{\bf 1}}

\def\AM{{\mathcal A}}
\def\FM{{\mathcal F}}
\def\TM{{\mathcal T}}
\def\UM{{\mathcal U}}
\def\XM{{\mathcal X}}
\def\YM{{\mathcal Y}}
\def\NM{{\mathcal N}}
\def\OM{{\mathcal O}}
\def\IM{{\mathcal I}}
\def\GM{{\mathcal G}}
\def\PM{{\mathcal P}}
\def\LM{{\mathcal L}}
\def\MM{{\mathcal M}}
\def\DM{{\mathcal D}}
\def\SM{{\mathcal S}}
\def\RB{{\mathbb R}}


%\def\tx{\tilde{\bf x}}
%\def\ty{\tilde{\bf y}}
%\def\tz{\tilde{\bf z}}
%\def\hd{\hat{d}}
%\def\HD{\hat{\bf D}}
%\def\hx{\hat{\bf x}}
%\def\hR{\hat{R}}

\def\Ell{\mbox{\boldmath$\ell$\unboldmath}}
\def\ph{\mbox{\boldmath$\phi$\unboldmath}}
\def\Pii{\mbox{\boldmath$\Pi$\unboldmath}}
\def\pii{\mbox{\boldmath$\pi$\unboldmath}}
\def\Ph{\mbox{\boldmath$\Phi$\unboldmath}}
\def\Ps{\mbox{\boldmath$\Psi$\unboldmath}}
\def\tha{\mbox{\boldmath$\theta$\unboldmath}}
\def\muu{\mbox{\boldmath$\mu$\unboldmath}}
\def\bett{\mbox{\boldmath$\beta$\unboldmath}}
\def\alpp{\mbox{\boldmath$\alpha$\unboldmath}}
\def\Si{\mbox{\boldmath$\Sigma$\unboldmath}}
\def\Gam{\mbox{\boldmath$\Gamma$\unboldmath}}
\def\Lam{\mbox{\boldmath$\Lambda$\unboldmath}}
\def\De{\mbox{\boldmath$\Delta$\unboldmath}}
\def\vps{\mbox{\boldmath$\varepsilon$\unboldmath}}
\def\Lap{\mbox{\boldmath$\LM$\unboldmath}}
\newcommand{\ti}[1]{\tilde{#1}}

\def\tr{\mathrm{tr}}
\def\etal{{\em et al.\/}\,}
\newcommand{\indep}{{\;\bot\!\!\!\!\!\!\bot\;}}
\def\argmax{\mathop{\rm argmax}}
\def\argmin{\mathop{\rm argmin}}
\def\subto{\mathop{\rm s.t.}}
\def\aka{{\em a.k.a.\/}\,}
\def\wrt{{\em w.r.t.\/}\,}
\def\sgn{\mathop{\rm sgn}}

\newtheorem{mydef}{Definition}[section]
\newtheorem{mylem}{Lemma}[section]
\newtheorem{myproof}{Proof}[section]
%\newtheorem{theorem}{\textbf{Theorem}}
%\newtheorem{lemma}{\textbf{Lemma}}
%\newtheorem{definition}{Definition}[section]
%\newtheorem{proposition}{\textbf{Proposition}}
%\newtheorem{cor}{\textbf{Corollary}} 

%%%%%%%%%%%%%%%%%%%%%%%%%%%%%%%%%%%%%%%%%%%%%%%%%%%%%%%%%%%%%%%%%%%%%%%%%
%                                                                       %
% Preambles. DO NOT ERASE THEM. Change to suite your particular purpose.%
%                                                                       %
%%%%%%%%%%%%%%%%%%%%%%%%%%%%%%%%%%%%%%%%%%%%%%%%%%%%%%%%%%%%%%%%%%%%%%%%%

%\title{Relational Factor Modeling\\ \vspace{0.5cm}\hspace{0cm}\small{a new framework for statistical relational learning}}  % Title of the thesis.
\title{SOME RESEARCH ISSUES IN \\HASH FUNCTION LEARNING}
\author{Yi Zhen}     % Author of the thesis.
%\degree{\MPhil}             % Degree for which the thesis is.
%% or


\degree{\PhD}              % Degree for which the thesis is.
\subject{Computer Science and Engineering}      % Subject of the Degree.
\department{Computer Science and Engineering}       % Department to which the thesis
%                    % is submitted.
\advisor{Prof.~Dit-Yan~Yeung}     % Supervisor.
\depthead{Prof.~Mounir~Hamdi}    % department head.
\defencedate{2012}{07}{10}      % \defencedate{year}{month}{day}.

% NOTE:
%   According to the sample shown in the guidelines, page number is
%   placed below the bottom margin.  However, if the author prefers
%   the page number to be printed above the bottom margin, please
%   activate the following command.

% \PNumberAboveBottomMargin

\begin{document}

%%%%%%%%%%%%%%%%%%%%%%%%%%%%%%%%%%%%%%%%%%%%%%%%%%%%%%%%%%%%%%%%%%%%%%%%%
%                                                                       %
% Now the actual Thesis. The order of output MUST be followed:          %
%                                                                       %
%    1) TITLEPAGE                                                       %
%                                                                       %
% The \maketitle command generates the Title page as well as the        %
% Signature page.                                                       %
%                                                                       %
%%%%%%%%%%%%%%%%%%%%%%%%%%%%%%%%%%%%%%%%%%%%%%%%%%%%%%%%%%%%%%%%%%%%%%%%%

\maketitle

%%%%%%%%%%%%%%%%%%%%%%%%%%%%%%%%%%%%%%%%%%%%%%%%%%%%%%%%%%%%%%%%%%%%%%%%%
%                                                                       %
%     2) DEDICATION (Optional)                                          %
%                                                                       %
% The \dedication and \enddedication commands are optional. If          %
% specified it generates a page for dedication.                         %
%
%%%%%%%%%%%%%%%%%%%%%%%%%%%%%%%%%%%%%%%%%%%%%%%%%%%%%%%%%%%%%%%%%%%%%%%%%

%\dedication
%\enddedication

%%%%%%%%%%%%%%%%%%%%%%%%%%%%%%%%%%%%%%%%%%%%%%%%%%%%%%%%%%%%%%%%%%%%%%%%%
%                                                                       %
%     3) ACKNOWLEDGMENTS                                                %
%                                                                       %
% \acknowledgments and \endacknowledgments defines the                  %
% Acknowledgments of the author of the Thesis.                          %
%                                                                       %
%%%%%%%%%%%%%%%%%%%%%%%%%%%%%%%%%%%%%%%%%%%%%%%%%%%%%%%%%%%%%%%%%%%%%%%%%

\acknowledgments

This thesis would never have been possible without the guidance of my supervisor, support of my family and help from my friends.

First and foremost, I am heartily thankful to  my supervisor, Prof. Dit-Yan Yeung, for his excellent guidance, caring help, and meticulous comments on my research writing. As a real model, Prof. Yeung have shown me how to be a rigorous researcher, a passionate teacher, and a responsible person. % He has paid much attention to helping me on English writing. 

I would like to show my appreciation for the thesis committee members, Prof. Irwin King (from Department of Computer Science and Engineering, The Chinese University of Hong Kong), Prof. Weichuan Yu (from Department of Electronic and Computer Engineering, The Hong Kong University of Science and Technology), Prof. Nevin L. Zhang, Prof. James T. Kwok, and Prof. Dit-Yan Yeung, for their insightful comments and encouraging words. I am also grateful to Prof. Raymond Wong and Prof. Qiang Yang for their invaluable suggestions on my research.

I would like to thank the members of Prof. Yeung's research group, including Wu-Jun Li, Yang Ruan, Yu Zhang, Jingni Chen, Craig Yu, Tony Ho, Emprise Chan, Zhi-Yang Wang, Nai-Yan Wang, and Jie Zhou. Many thanks to the members of the Artificial Intelligence Lab, including Bin Cao, Nathan Liu, Evan Xiang, Vincent Zheng, Sinno Pan, Chung Lee, Tengfei Liu, Weike Pan, Ning Ding, and Haodi Zhang. Thanks to the visiting members of our lab, including Yan-Ming Zhang, Guoqiang Zhong, Deming Zhai, and Li Pu. In addition, I am grateful to the mentors during my visiting to Microsoft Research Asia, including Jian-Tao Sun and Gang Wang. The friendship, collaboration, and discussion with these friends have made my research and life not only meaningful but also colorful.


Last but not least, I would like to thank my family for supporting and encouraging me throughout my study in Hong Kong. Especially, I am grateful to my wife, Ning Zhu, who has always been right behind me whenever and wherever. I would also like to thank my daughter, Joy Zhen, for her sweet smile which brings happiness and delight to the whole family. My love for all of you is beyond words. 

\endacknowledgments

%First and foremost, I would like to thank my supervisor, Prof. Dit-Yan Yeung, for his support throughout my PhD study at the HKUST. Without his guidance, encouragement, and kindness, this thesis would never have been possible.
%
%I would like to thank my thesis committee members, Prof. Eric Xing (from Machine Learn- ing Department, Carnegie Mellon University), Prof. Mike Ka Pui So (from Department of In- formation Systems, Business Statistics and Operations Management, HKUST), Prof. Nevin L. Zhang, Prof. Qiang Yang, and Prof. Dit-Yan Yeung, for their insightful comments and suggestions.
%I would like to thank the members of Prof. Yeung’s group for their friendship, encourage- ment, support and collaboration, such as Gang Wang, Guang Dai, Chris Kui Jia, Zhihua Zhang, Wu-Jun Li, Yang Ruan, Jingni Chen, Yi Zhen, Craig Yu, Tony Ho, Emprise Chan, Yan-Ming Zhang, Guoqiang Zhong, and Deming Zhai. My many friends at the HKUST have given me an unforgettable memory of both research and everyday life. They are: Bin Cao, Qian Xu, Nathan Nan Liu, Weizhu Chen, Wei Bi, Sinno Jialin Pan, Vincent Wenchen Zheng, Evan Wei Xiang, Weike Pan, Derek Hao Hu, Si Shen and Tengfei Liu.
%I would also like to thank Prof. Eric Xing for his insightful discussions and guidance on re- search during my visit to Carnegie Mellon University. My many friends helped me in many ways to adapt to life in the USA. They include Jun Zhu, Li Nao, Yanlin Li, Yin Zhang, Pingzhong Tang, Yuandong Tian, Lei Li, Junming Yin, Haiyi Zhu, Cao Shen, and Yu Sheng.
%Finally, I am deeply indebted to my family especially my mother whose love and support have been the source of my courage.

%%%%%%%%%%%%%%%%%%%%%%%%%%%%%%%%%%%%%%%%%%%%%%%%%%%%%%%%%%%%%%%%%%%%%%%%%
%                                                                       %
%     4) TABLE OF CONTENTS                                              %
%                                                                       %
%%%%%%%%%%%%%%%%%%%%%%%%%%%%%%%%%%%%%%%%%%%%%%%%%%%%%%%%%%%%%%%%%%%%%%%%%

\tableofcontents

%%%%%%%%%%%%%%%%%%%%%%%%%%%%%%%%%%%%%%%%%%%%%%%%%%%%%%%%%%%%%%%%%%%%%%%%%
%                                                                       %
%     5) LIST OF FIGURES (If Any)                                       %
%                                                                       %
%%%%%%%%%%%%%%%%%%%%%%%%%%%%%%%%%%%%%%%%%%%%%%%%%%%%%%%%%%%%%%%%%%%%%%%%%

\listoffigures

%%%%%%%%%%%%%%%%%%%%%%%%%%%%%%%%%%%%%%%%%%%%%%%%%%%%%%%%%%%%%%%%%%%%%%%%%
%                                                                       %
%     6) LIST OF TABLES (If Any)
%                                                                       %
%%%%%%%%%%%%%%%%%%%%%%%%%%%%%%%%%%%%%%%%%%%%%%%%%%%%%%%%%%%%%%%%%%%%%%%%%

\listoftables

%%%%%%%%%%%%%%%%%%%%%%%%%%%%%%%%%%%%%%%%%%%%%%%%%%%%%%%%%%%%%%%%%%%%%%%%%
%                                                                       %
%     7) ABSTRACT                                                       %
%                                                                       %
% \abstract and \endabstract are used to define a short Abstract for    %
% the Thesis.                                                           %
%                                                                       %
%%%%%%%%%%%%%%%%%%%%%%%%%%%%%%%%%%%%%%%%%%%%%%%%%%%%%%%%%%%%%%%%%%%%%%%%%

\abstract




Over the past decade, hashing-based methods for large-scale similarity search have sparked considerable research interest in the database, data mining and information retrieval communities. These methods achieve very fast search speed by indexing data with binary codes. Although lots of hash functions for various similarity metrics have been proposed, they often generate very long codes due to their data independence nature. In recent years, machine learning techniques have been applied to \textit{learn} hash functions from data, forming a new research topic called \textit{hash function learning}. 

In this thesis, we study two important issues in hash function learning. On one hand, existing supervised or semi-supervised hash function learning methods, which learn hash functions from labeled data, can be regarded to be passive because they assume that the labeled data are provided in advance. Given that the data labeling process can be very costly in practice and the contribution of labeled data to hash function learning can be quite different, it may be more cost effective for the hash function learning methods to select labeled data from which to learn. To this end, we propose a novel framework, termed \textit{active hashing}, to actively select the most informative data to label for hash function learning. Under the framework, we develop one simple method which queries data labels that the current hash functions are most uncertain about. Experiments conducted on two real data sets show obvious improvement of our active hashing algorithm over previous passive hashing methods. On the other hand, most existing hash function learning methods only work on unimodal data, which are obviously not the case in many applications, e.g., multimedia retrieval and cross-lingual document analysis. To apply hash function learning to multimodal data, we develop three methods under the framework of \textit{multimodal hashing} which hashes data points of multiple modalities into one common Hamming space. For aligned data, the first method is based on spectral analysis of the correlation of the multimodal data. For graph data, the second method falls into the category of latent feature models and the hash codes can be obtained through Bayesian inference. For general data, we propose a boosted co-regularization model which can be efficiently solved by stochastic gradient-based algorithms. The effectiveness of our models is validated through extensive comparative study on crossmodal multimedia retrieval.



\endabstract


%%%%%%%%%%%%%%%%%%%%%%%%%%%%%%%%%%%%%%%%%%%%%%%%%%%%%%%%%%%%%%%%%%%%%%%%%
%                                                                       %
%     8) The Actual Contents                                            %
%                                                                       %
% The command \chapters MUST BE USED to ensure that the entire content  %
% of the Thesis is double-spaced (in version 1.0).                      %
%                                                                       %
% However, in version 2.0, \chapters will be automatically added in     %
% the beginning of the first chapter.                                   %
%                                                                       %
%%%%%%%%%%%%%%%%%%%%%%%%%%%%%%%%%%%%%%%%%%%%%%%%%%%%%%%%%%%%%%%%%%%%%%%%%

%%\chapters         % Not necessary with ustthesis.cls (v2.0).

%%%%%%%%%%%%%%%%%%%%%%%%%%%%%%%%%%%%%%%%%%%%%%%%%%%%%%%%%%%%%%%%%%%%%%%%%
%                                                                       %
% Each chapter is defined via the \chapter command. The usual sectional %
% commands of LaTeX are also available.                                 %
%                                                                       %
%%%%%%%%%%%%%%%%%%%%%%%%%%%%%%%%%%%%%%%%%%%%%%%%%%%%%%%%%%%%%%%%%%%%%%%%%


\input{TexFile/1_introduction}

\chapter{Background}
\label{chap:background}

The history of hashing may be as long as the history of computer science. Hashing methods were originally used to implement dynamic sets that support only the dictionary operations such as insert, search, and delete~\cite{cormen2001book}. Conventionally, people design hash functions which map objects to bins and then construct hash tables. With the hash tables, fast dictionary operations can be performed. For example, the expected time complexity of searching an element in a hash table is only $ O(1) $, although it can be $ \Theta(N) $ in the worst case. A good hash table should have a low collision rate and use as little storage as possible.

In recent years, hashing has been introduced in several new problems such as similarity search~\cite{gionis1999vldb,salakhutdinov2009ijar} and data compression~\cite{shi2009aistats,weinberger2009icml,li2011nips}. In this thesis, our focus is hashing for similarity search.
Unlike conventional hashing methods, hashing-based similarity search methods use collision to incorporate similarity. These methods are common in indexing objects using binary codes, making search extremely fast (without much loss of accuracy) even on very large data sets.

Depending on whether or not machine learning techniques are applied to design the hash functions, existing hashing-based methods can be grouped into two categories: locality sensitive hashing (\aka non-learning based hashing) and hash function learning (\aka learning based hashing). While locality sensitive hashing methods have enjoyed great success over past decades, most of them are data independent and often generate very long hash codes which are practically inefficient for large-scale applications. On the contrary, hash function learning methods learn from data hash functions which reflect data characteristics more accurately and generate very compact codes. As a result, \mbox{HFL} methods are more desirable in real applications. 

To provide a solid background for this thesis, in this chapter, we review the literature of hashing-based methods for similarity search, and two machine learning areas, namely, active learning and metric learning, which are closely related to the methods we introduce later. Specifically, in Section~\ref{background:lsh}, we give a brief introduction of locality sensitive hashing, which is then followed by a survey of hash function learning in Section~\ref{background:hfl}. Section~\ref{background:metric} and Section~\ref{background:active} present a summarization of metric learning and active learning, separately. Finally, we summarize the chapter in Section~\ref{background:sum}.

%-------------------------------------------------------------------------------
\section{Locality Sensitive Hashing}
\label{background:lsh}

The family of \textit{locality sensitive hashing} (\mbox{LSH}) algorithms have been a hot research topic since the 1990s. In this section, we briefly introduce some well-known methods. For a detailed review, the readers are referred to~\cite{andoni2006focs}. 

The concept of \mbox{LSH} is very simple, that is, it aims to hash objects\footnote{In this paper, we use objects, points and documents interchangeably.} using a number of hash functions to ensure that, for each function, the collision probability for objects that are close to each other is much higher than that for far apart objects~\cite{indyk1998stoc}. In other words, given a family of hash functions $\mathcal{H}$ and two objects $ \p $ and $ \q $, \mbox{LSH}~\cite{charikar2002stoc} requires that 
$$ P_{\mathcal{H}}[h(\p)=h(\q)] = \mathit{sim}(\p, \q),$$
where $ h $ is a hash function randomly selected from $\mathcal{H}$, $ P(\cdot) $ indicates the probability and $ \mathit{sim}(\cdot,\cdot) $ returns a similarity value in the range of $ [0,1] $ based on some metric. Usually, $ h $ has only two hash values $ \{0,1\} $ and \mbox{LSH} uses several hash functions to generate binary codes. Therefore, \mbox{LSH} actually maps data objects from the original feature space to a Hamming space where the Hamming distance incorporates the similarity. It is very efficient to perform the search in the Hamming space via bit operations or data structures such as dictionaries.

%More specifically,  is called ($R,cR, P_{1}, P_{2}$)-sensitive (locality sensitive) if for any two points $\p,\q \in \mathbb{R}^{D}$,
%\begin{itemize}
%\item
%if $\|\p-\q\|\le R$, then $P_{\mathcal{H}}[h(\p)=h(\q)]\ge P_{1}$,
%\item
%if $\|\p-\q\|\ge cR$, then $P_{\mathcal{H}}[h(\p)=h(\q)]\le P_{2}$.
%\end{itemize}
%
%In order to be useful, an \mbox{LSH} algorithm has to satisfy $P_{1}>P_{2}$. Actually, all \mbox{LSH} algorithms try to amplify the gap between $P_1$ and $P_2$ by constructing hash functions.

\mbox{LSH} functions are highly dependent on the similarity measures used in the task. Up to now, many \mbox{LSH} functions have been proposed. In the following subsections, we introduce them separately according to the adopted similarity measures.

%are highly dependent on the similarity measure being adopted. Several \mbox{LSH} families have been proposed for different similarity measures, such as the Hamming distance, $\ell_1$ distance, $\ell_s$ distance, Jaccard coefficient and the angle distance. Some examples for each kind of distance are:
%\begin{description}
%  \item[Hamming distance] One simple \mbox{LSH} family is $h_i(\p) = p_i, i\in\{1,\dots,D\},\p\in\{0,1\}^{D}$. The locality-sensitive property is $\rho = 1/c.$
%  
%  \item[$\ell_1$ distance] The \mbox{LSH} hash functions can be constructed as follows, for a fixed real number $w\gg R$, pick random real numbers $s_1,\dots,s_D\in[0,w)$ and define 
%  $$h_{s_1,\dots,s_D}(\p) = (\lfloor(p_1-s_1)/w\rfloor,\dots,\lfloor(p_D-s_D)/w\rfloor),\p\in\mathbb{R}^D.$$ The locality-sensitive property is $\rho = 1/c+O(R/w).$
%  
%  \item[$\ell_s$ distance] The \mbox{LSH} functions can be constructed as follows, choose a random number $w$ and a random projection vector $\r\in \mathbb{R}^D$ (each of whose coordinates is picked from a Gaussian distribution), then $h_{\r,b}(\p) = \lfloor(\r\cdot \p+b)/w\rfloor$ where $\p\in\mathbb{R}^D$ and $b\in[0,w)$ is random. The
%locality-sensitive property is $\rho < 1/c$ for some (carefully chosen) finite values of $w$.
%
%  \item[Jaccard coefficient] Jaccard coefficient is defined as a similarity metric between two sets $\mathcal{A}$ and $\mathcal{B}$: $s(\mathcal{A},\mathcal{B}) = \frac{|\mathcal{A}\cap \mathcal{B}|}{|\mathcal{A}\cup \mathcal{B}|}$. One \mbox{LSH} family is $h_{\pi}(\mathcal{A}) = \min\{\pi(a)\mid a\in \mathcal{A}\}$, where $\pi$ is a random permutation on the ground universe.
%  
%  \item[Angle distance] Angle distance between two points is defined as
%  $$\Theta(\p,\q) = \arccos\left(\frac{\p\cdot \q}{\|\p\|\cdot\|\q\|}\right),\p,\q\in\mathbb{R}^{D},$$ where $\|\cdot\|$ denotes the vector $\ell_2$ norm. One \mbox{LSH} family for this metric can be constructed as follows, pick a random unit vector $\u$ and define $h_{\u}(\p) = \sgn(\u\cdot \p)$.
%\end{description}



%Actually, we should introduce the methods for different similarities.
%\subsection{Hamming Distance}
%
%The seminal work of using hashing for similarity search, we call it Bit Sampling.
%The seminal work is~\cite{indyk1998stoc}, it is said to be for Hamming distance. and followed by some random projection based methods.
%
%What codes does it generate? binary or non-binary?

\subsection{ $\ell_s $ Distance}

The most commonly used distance might be the Euclidean distance (\aka $\ell_2 $ distance). In~\cite{datar2004scg}, \mbox{LSH} functions for the Euclidean distance can be constructed as follows: choose a random number $w$ and a random projection vector $\r\in \mathbb{R}^D$ (each of whose coordinates is picked from a Gaussian distribution), then $h_{\r,b}(\p) = \lfloor(\r\cdot \p+b)/w\rfloor$ where $b\in[0,w)$ is a random threshold and $\p\in\mathbb{R}^D$ is an object to be hashed. There is a publicly available software package which has implemented this algorithm.\footnote{\url{http://www.mit.edu/~andoni/LSH/}} Recently, Dasgupta \etal~\cite{dasgupta2011kdd} further improved the speed of the hash function construction of \mbox{LSH} for Euclidean distance using the Hadamard matrix.

Another common distance is $ \ell_1 $ distance, for which the \mbox{LSH} hash functions are defined as~\cite{andoni2006soda}: for a fixed real number $w\gg R$,\footnote{$ R $ is a user specified number controlling the distance between points which are considered as neighbors.} pick random real numbers $s_1,\dots,s_D\in[0,w)$ and define $h_{s_1,\dots,s_D}(\p) = (\lfloor(p_1-s_1)/w\rfloor,\dots,\lfloor(p_D-s_D)/w\rfloor),\p\in\mathbb{R}^D.$  

Constructions of \mbox{LSH} functions for $ \ell_s $ distance for any $ s\in(0,2] $ are also possible~\cite{datar2004scg}. 




\subsection{Cosine Similarity}

In the data mining and information retrieval communities, especially for documents, the most commonly used metric is cosine similarity. Its value between two vectors $ \p  $ and $ \q  $ is defined as:
$$\Theta(\p,\q) = \arccos\left(\frac{\p\cdot \q}{\|\p\|\cdot\|\q\|}\right),\p,\q\in\mathbb{R}^{D},$$ where $\|\cdot\|$ denotes the vector $\ell_2$ norm. For this similarity measure, Charikar \etal~\cite{charikar2002stoc} defines the following LSH family: pick a random unit vector $\u$ and define $h_{\u}(\p) = \sgn(\u\cdot \p)$. The hash function can also be viewed as partitioning the space into two half-spaces by a random hyperplane. The collision probability in this case is $ P[h(\p )=h(\q )] = 1-\Theta(\p ,\q )/\pi $.

%Here you should make clear the search procedure, which is also applied in \mbox{KLSH}~\cite{kulis2009iccv}.

Manku \etal~\cite{manku2007www} applied the above mentioned hash functions with some modifications to document de-duplication and named the algorithm \mbox{SimHash}. Since then, \mbox{SimHash} has become one of the most well-known hashing-based methods. \mbox{SimHash} has also been implemented under the framework of \mbox{MapReduce} for cross-lingual pairwise similarity~\cite{ture2011sigir}. Eshghi \etal~\cite{eshghi2008kdd} developed a new \mbox{LSH} family for cosine similarity based on concomitant rank order statistics. 

%This group of work was first for cosine similarity and \mbox{EMD}. They are called SimHash and famous because google used this algorithm.
%
%%The procedure for cosine similarity is easy to follow, but for \mbox{EMD} the procedure is hard to follow. But for uniform case, it can be considered a general form of minwise hashing~\cite{broder1998stoc}.
%
%For cosine similarity and \mbox{EMD}, we call it SimHash~\cite{charikar2002stoc}, it generates binary codes. But google used this method for document deduplication~\cite{manku2007www}, but I don't know whether it is binary.

\subsection{Jaccard Similarity}
Jaccard coefficient is widely used to measure similarity between sets. The definition of Jaccard Similarity is: $s(\mathcal{A},\mathcal{B}) = \frac{|\mathcal{A}\cap \mathcal{B}|}{|\mathcal{A}\cup \mathcal{B}|}$, which can be approximated by a simple approach based on random permutations~\cite{broder1997ccs,broder1997www}. Based on this property, Broder \etal proposed the first \mbox{LSH} family for Jaccard similarity which is called minwise Hashing or \mbox{MinHash}~\cite{broder1998stoc}. The hash functions of \mbox{MinHash} can be described simply as follows: $h_{\pi}(\mathcal{A}) = \min\{\pi(a)\mid a\in \mathcal{A}\}$, where $\pi$ is a random permutation on the ground universe. %Actually they come up with a new approximate method for Jaccard similarity. 

%Where does hash play a role?\footnote{Please check~\cite{broder1998stoc} for details and make this part clear.} 

Recently, Li \etal have extended \mbox{MinHash} to generate more compact codes~\cite{li2010www,li2010nips}. Chum \etal~\cite{chum2009cvpr} have proposed a geometric \mbox{MinHash} algorithm and applied it to the image retrieval task. %\footnote{Please check what does this paper say?}

%And give these two papers more detailed information.
%This group of work is called Minwise Hashing.
%For set similarity, it generates non-binary codes. Actually this is Jaccard similarity. The seminal work is, followed by

\subsection{Kernel Similarity}
%\subsubsection{Kernelized LSH}

In many applications, the similarity metric of interest is defined by means of kernels. As a result, developing hashing-based methods for kernel similarity is also very important. For the Pyrimid match kernel, a widely used kernel in vision problems, Pyrimid match hashing~\cite{grauman2007cvpr} was developed. The algorithm can be simply summarized as first generating an embedding of the Pyrimid match kernel, and then using the \mbox{SimHash} to obtain the binary codes for the embedding. Jain \etal~\cite{jain2008cvpr} proposed a similar hashing algorithm for a learned \mbox{Mahalanobis} distance, which is equivalent to a special kernel. More generally, for shift invariant kernels, hash functions can be constructed as follows~\cite{raginsky2009nips}. First get random features using Fourier transforms~\cite{rahimi2007nips} and then threshold them randomly to give the binary codes.

Kulis \etal~\cite{kulis2009iccv} implemented the idea of random projections in the kernel space and formulated \mbox{KLSH}. In general, \mbox{KLSH} uses a random set of points to form random projections and can accommodate any kernels. Recently, \mbox{KLSH} was extended to accept multiple kernels in~\cite{zhang2011mm}.



%\subsection{Special Cases}
%%\subsection{Learned Metric}
%\subsubsection{Non-metric Distance}
%\subsubsection{Asymmetric Distance}
\subsection{Summary}

Besides the aforementioned algorithms, \mbox{LSH} functions have also been extended for some special settings, such as asymmetric Hamming distance~\cite{dong2008sigir,gordo2011cvpr} and non-metric similarity~\cite{athitsos2008icde,mu2010aaai}.

Although theoretically effective and efficient, \mbox{LSH} methods have some apparent limitations. Most of all, they are data independent and thus may not reflect the data characteristics accurately. As a result, these methods always generate very long codes which are very inefficient for large-scale data sets and hence of limited practical use. It is this limitation that motivates hash function learning.


%LSH families actually do not use bit operations to conduct fast search, they use data structures such as HST~\cite{indyk1998stoc} or some procedure introduced in~\cite{charikar2002stoc}.


 % % % % % % % % % % % % % % % % % % % % % % % % % % % % % %
\section{Hash Function Learning}
\label{background:hfl}

The past five years have witnessed increasingly rapid progress in the topic of hash function learning (\mbox{HFL}), or learning-based hashing. The goal of \mbox{HFL} is to learn, rather than design, the hash functions from data so that the hash functions can generate very compact (short) hash codes. Despite being a young area, a number of \mbox{HFL} methods have been proposed up to now. Depending on how the label information or side-information is used in the learning procedures, we roughly classify these methods into three categories: unsupervised, semi-supervised and supervised methods. 
%In the following subsections, we present some representative methods in each category.

%\subsection{Embedding-based Approach}
%\subsubsection{Continuous Embedding Approach}
%\subsubsection{Discrete Embedding Approach}
%\subsection{Model-based Approach}
%\subsubsection{SVM-based}
%\subsubsection{Boosting-based}
\subsection{Unsupervised Hash Function Learning}
Unsupervised \mbox{HFL} methods learn hash functions from unlabeled data only and do not use labels or side-information. Although the model formulations are different from each other, the common idea is to make similar points close to each other and dissimilar points far apart in the Hamming space in which the hash codes live.\footnote{Note that ``similar'' and ``dissimilar'' are determined by features of unlabeled data.}

\subsubsection{Spectral hashing}
The most well-known unsupervised \mbox{HFL} algorithm should be \textit{spectral hashing} (\mbox{SH}). The objective of \mbox{SH} is similar to one popular dimensionality reduction method called \textit{Laplacian Eigenmap}, meaning that points that are similar in the original feature space should have small distance in the embedded space. However, different from \textit{Laplacian} Eigenmap which embeds data into the Euclidean space, \mbox{SH} maps data into the Hamming space. For a code to be efficient, the authors require that each bit has a 50\% chance of being one or zero, and that different bits are independent of each other. 

Putting all things together, \mbox{SH} is formulated as the following optimization problem:
\begin{align}
\min_{\y_i}&~\sum_{ij}W(i,j)\|\y_i-\y_j\|^2\nonumber\\
\subto&~\y_i\in\{+1,-1\}^{M},~\sum\nolimits_i\nolimits^{N}\y_i=0,~\frac{1}{n}\sum\nolimits_i\nolimits^{N}\y_i\y_i^T=\I,\nonumber
\end{align}
where $\y_i$ is the hash code of the $i$th point, $W(i,j)$ is the similarity between the $i$th and $j$th points, $M$ is the length of each code and $ N $ is the number of points. Note that the authors relax the independence assumption and require the bits to be uncorrelated by constraint $ \frac{1}{n}\sum_i\y_i\y_i^T=\I $, because it is hard to enforce independence between bits. Then above problem can be rewritten using matrix notations as follows:
\begin{align}
\min_{\Y}&~\tr(\Y^T\mathcal{L}\Y)\nonumber\\
\subto&~Y_{ij}\in\{+1,-1\}^{M},\Y^T\1=\0,\Y^T\Y=\I,
\label{problem:sh:original}
\end{align}
where $\Y$ is the $N\times M$ code matrix, $\mathcal{L}$ is the graph \textit{Laplacian} defined on the similarity matrix $\W$. 

Actually, the above problem for each single bit is equivalent to a balanced graph partitioning problem and NP-hard, so Problem~(\ref{problem:sh:original}) is NP-hard. However, if we remove the first two constraints, the problem can easily be solved by finding $K$ eigenvectors corresponding to the smallest eigenvalues of $\mathcal{L}$ (ignoring the eigenvector $\1$ corresponding to eigenvalue 0). 

The solution above has two limitations. First, it involves eigen-decomposition which could be very slow when $N$ is large. Moreover, it cannot deal with out-of-sample data points. To overcome these two limitations, the authors use eigenfunctions instead of the eigenvectors. To make the computation tractable, they simply assume that the data points are uniformly distributed in a rectangle and the similarity measure is fixed as $ \exp(\|\x_i-\x_j\|^2/\sigma^2) $. The resultant model is very fast to train and experimental results show that it is superior to \mbox{LSH}. 

Recently, He \etal~\cite{he2010kdd} proposed a kernel extension of \mbox{SH}, which can handle nonvectorial data, incorporate nonlinearity and places no restrictions on the uniform distribution and the fixed similarity measure. Zhang \etal~\cite{zhang2010sigir,zhang2010fgsir} used \mbox{SVM} to enhance \mbox{SH} with the aim of removing the distribution and similarity assumptions.


\subsubsection{Binary reconstructive embeddings}
Kulis \etal~\cite{kulis2009nips} developed another unsupervised \mbox{HFL} algorithm, namely, \textit{binary reconstructive embeddings} (\mbox{BRE}), based on explicitly minimizing the reconstruction error between the original distance and the Hamming distance of the corresponding binary hash codes. \mbox{BRE} can be easily kernelized and does not require restrictive assumptions about the underlining data distribution.

Specifically, let $M$ be the number of hash functions (\aka code length), $N$ be the number of data points, and $Q$ be the number of landmark points. Given a data set $\mathcal{X}$, \mbox{BRE} defines the hash function \wrt (with respect to) the $m$th bit for $\x\in\mathcal{X}$ as:
\begin{align}
h_{m}(\x) = \sgn\left(\sum\nolimits_{q=1}\nolimits^{Q}W(m,q)\kappa(\x_q,\x)\right),\nonumber
\end{align}
where $\W$ is an $M\times Q$ projection matrices, $\{\x_q\}_{q=1}^{Q}\subset\mathcal{X}$ are landmark points, and $\kappa(\cdot,\cdot)$ is a user specified kernel function. Note that defining hash functions this way is very common in kernel methods such as \textit{support vector machines} (\mbox{SVM}) and gives us the flexibility to work on a wide variety of data types. Therefore, given one point $\x\in\mathcal{X}$, we denote its corresponding binary representation as $\tilde{\x}$ such that its $m$th bit can be evaluated by $\tilde{x}(m) = (1+h_{m}(\x))/2$.

Rather than simply choosing the $ \W $ matrix based on random hyperplanes, they construct this matrix to achieve good reconstructions. In particular, they minimize the squared error between the original distance and the reconstructed distance \wrt $ \W $ as follows,
\begin{align}
\mathcal{O}\left(\W\right)=\sum\nolimits_{(\x_i,\x_j)\in\mathcal{N}}\left(d(\x_i,\x_j)-\tilde{d}(\x_i,\x_j)\right)^{2},\nonumber
\end{align}
where $\mathcal{N}$ is a set of point pairs, $ d(\cdot,\cdot) $ is the original distance, $ \tilde{d}(\cdot,\cdot) $ is the Hamming distance. Although the problem is non-convex and hard to optimize, the authors use a heuristic coordinate-descent algorithm to find a locally optimal $ \W $. Experiments show that \mbox{BRE} achieves a state-of-the-art performance, but finding a good local optimum is hard and the performance highly depends on the applications at hand.

%%%%%%%%%%%%%%%%%%%
\subsection{Semi-Supervised Hash Function Learning}
In semi-supervised \mbox{HFL} methods, both unlabeled and labeled data are used. Different from original unlabeled features, the labels or side-information which often carries semantic information might be very useful for hash function learning. The basic idea of semi-supervised \mbox{HFL} is to consider both kinds of information to learn the hash functions.

\subsubsection{Semantic hashing}
To learn the hash codes, semantic hashing uses multiple layers of \textit{restricted Boltzmann machines} (\mbox{RBM}), which are closely related to one popular dimensionality reduction framework based on neural networks~\cite{hinton2006science}. An \mbox{RBM} is an ensemble of binary vectors with a network of stochastic binary units arranged in two layers, one visible and one hidden. Given a layer of visible units $\v = [v_1, v_2, \dots, v_M]$, a layer of hidden units $\h=[h_1, h_2,\dots, h_N]$, and a symmetric weighting matrix $\W$ connecting units in different layers, the energy function of the joint configuration of all visible and hidden units is defined as:
\begin{align}
E(\v, \h) = -\sum\nolimits_{i=1}\nolimits^{M}b_{i}v_{i}-\sum\nolimits_{j=1}\nolimits^{N}b_{j}h_{j}-\sum\nolimits_{i=1}\nolimits^{M}\sum\nolimits_{j=1}\nolimits^{N}v_i h_i W(i,j),
\end{align}
where $v_i$ and $h_j$ are the binary states of visible and hidden units $i$ and $j$, $ W(i,j)$ are the weights and $b_i$ and $b_j$ are bias terms. Using this energy function, a probability can be assigned to a binary vector of the visible
units:
\begin{align}
P(\v) =\sum\nolimits_{\h}\left(\left.e^{-E(\v,\h)}\middle/\left(\sum\nolimits_{\u,\g}e^{-E(\u,\g)}\right)\right.\right).\nonumber
\end{align}

An \mbox{RBM} lacks connections between units within a layer, hence the conditional distributions $P(\h|\v)$ and $P(\v|\h)$ have convenient forms, being products of Bernoulli distributions:
\begin{align}
P(h_j = 1|\v) = \sigma\left(b_j +\sum\nolimits_{i}w_{ij}v_i\right), \ \ & \ \ P(v_i = 1|\h) = \sigma\left(b_i +\sum\nolimits_{j}w_{ij}h_j\right),\nonumber
\end{align}
where $\sigma(x) = 1/(1 + e^{-x})$ is the logistic sigmoid function.

Recently, Salakhutdinov \etal~\cite{salakhutdinov2009aistats} demonstrated methods for stacking \mbox{RBM} into multiple layers, creating ``deep networks" which can capture high order correlations between the visible units at the bottom layer of the network. By choosing an architecture that progressively reduces the number of units in each layer, a high dimensional binary input vector can be mapped to a far smaller binary vector at the output. Thus, at the output each bit maps through multiple layers of nonlinearities to model the complicated subspace of the input data. If the feature values are not binary but real numbers, the first layer of visible units are modified to have a Gaussian distribution. This type of trained networks are capable of capturing higher order correlations between different layers of the network. Since the network structure gradually reduces the number of units in each layer, the high-dimensional input can be projected to a much more compact binary vector space.

A practical implementation of \mbox{RBM} has two major stages, an unsupervised pre-training stage and a supervised fine-tuning stage. The greedy pre-training stage is progressively executed layer by layer from input to output. After achieving convergence of the parameters of a layer via contrastive divergence, the derived activation probabilities are fixed and treated as input to drive the training of the next layer. During the fine-tuning stage, the labeled data is used to help refine the trained network through back-propagation. Specifically, a cost function is first defined to estimate the number of correctly classified points in the training set. Then, the network weights are refined to maximize this objective function through gradient descent. 

Semantic hashing outperforms \mbox{LSH} methods in applications such as document retrieval, but it involves estimating a large number of weights. As such, it not only involves an extremely costly training procedure, but also demands sufficient labeled training data for fine-tuning.

%%%%%%%%%%%%%%%%%%%
\subsubsection{Semi-supervised hashing}
Rather than use unlabeled and labeled data in two separate stages like semantic hashing, Wang \etal~\cite{wang2010cvpr} have developed a simple semi-supervised hashing (\mbox{SSH}) method which uses both kinds of information in a unified framework.

Given a set of $N$ data points, $\X \in \mathbb{R}^{N\times D}$, and the $i$th row of $\X$ corresponds to a $D$-dimensional point that will be denoted by $\x_i$ in the sequel, the authors want to learn $M$ hash functions leading to a $M$-bit binary codes $\Y \in \{0,1\}^{N\times M}$ of $\X$. In the paper, the $m$th hash function is defined as,
$$h_{m}(\x) = \sgn\left(\w^T_{m}\x+b_{m}\right),$$
where $\w$ is a $D$-dimensional projection vector and $b_{m}$ is a scaler.\footnote{Note that one bit of binary code can be easily obtained by setting $y_m(\x) = \left(1+h_{m}(\x)\right)/2$.} Without loss of generality, we assume that $\X$ has zero-mean ($\sum_{i=1}^{N}\x_i=\0$),\footnote{If this is not the case, preprocessing can be applied on the data.} then the hash function can be simplified as,
$$h_{m}(\x) = \sgn(\w^T_{m}\x).$$

Based on the labels or side-information, the authors construct a set of similar point pairs $\mathcal{S}$ and a set of dissimilar pairs $\mathcal{D}$. We use $\mathcal{SD}$ to denote the set of points involved in $\mathcal{S}$ and $\mathcal{D}$,  and $\X_{\mathcal{SD}}\in \mathbb{R}^{l\times D}$ to denote its matrix form, where $l = |\mathcal{SD}|<N$.

%Further, suppose there are $l$ points, $l < n$, each of which is associated with at least one of the two groups $\mathcal{S}$ or $\mathcal{D}$.

In \mbox{SSH}, the hashing functions $\mathcal{H} = \{h_m\}_{m=1}^M$ can be learned by optimizing two objectives. The first objective is to maximize the empirical accuracy on $\mathcal{S}$ and $\mathcal{D}$, which is defined as the sum of difference of the total number of correctly classified pairs and that of wrongly classified pairs by each bit as follows:
\begin{align}
J(\mathcal{H}) = \sum\nolimits_{m=1}\nolimits^M\left(\sum\nolimits_{(\x_i,\x_j)\in\mathcal{S}}h_m(\x_i)h_m(\x_j) - \sum\nolimits_{(\x_i,\x_j)\in\mathcal{D}}h_m(\x_i)h_m(\x_j)\right).\nonumber
\end{align}

Optimizing the above objective function is difficult, since the hash functions $\{h_m\}_{m=1}^M$ are discrete and not differentiable. As such, the $\sgn(\cdot)$ operator is dropped and the following approximate objective is used instead,
\begin{align}
J(\W) &= \sum\nolimits_{m=1}\nolimits^M\left(\sum\nolimits_{(\x_i,\x_j)\in\mathcal{S}}\w_m^T\x_i\x_j^{T}\w_m - \sum\nolimits_{(\x_i,\x_j)\in\mathcal{D}}\w_m^T\x_i\x_j^{T}\w_m\right)\nonumber\\
&= \frac{1}{2}\tr\left(\W^{T}\X_{\mathcal{SD}}^{T}\S\X_{\mathcal{SD}}\W\right),\nonumber
\end{align}
where $\W = [\w_1,\dots,\w_M] \in \mathbb{R}^{D \times M}$ and $\S\in \mathbb{R}^{l\times l}$ is a relational matrix incorporating pairwise relations,
\begin{align}
S(i,j) = \left\{ \begin{array}{ll}
+1 & \mbox{if } (\x_i,\x_j)\in\mathcal{S}\\
-1 & \mbox{if } (\x_i,\x_j)\in\mathcal{D}\\
0 & \textrm{otherwise}
\end{array} \right..\nonumber
\end{align}

The other objective is to maximize the information conveyed by each bit of hash codes. From an information-theoretic point of view, a binary bit, which gives a balanced partition (\textit{maximal entropy partition}) of $\X$ ($\sum_{i=1}^{n}h(\x_i) = 0$), provides the maximum information. Although finding mean-thresholded hash functions that meet the balancing requirement is \mbox{NP-hard}~\cite{weiss2008nips}, it is proved that this objective is equivalent to maximizing the variance of a bit~\cite{wang2010cvpr}, which is lower-bounded by the scaled variance of the projected data. Thus the scaled variance of the projected data is used as the second objective,
\begin{align}
R(\W) = \rho\sum\nolimits_{k=1}\nolimits^{K}\w_m^T\X^T\X\w_m,\nonumber
\end{align}
where $\rho$ is a scaling parameter.

Combining the two objectives into one, the optimization problem of \mbox{SSH} is formulated as follows:
\begin{align}
\label{eqn:ssh}
\max_{\W} & ~\frac{1}{2}\tr\left(\W^{T}\left(\X_{\mathcal{SD}}^{T}\S\X_{\mathcal{SD}}+\rho\X^{T}\X\right)\W\right)\\
\subto & ~\W^T\W = \I,\nonumber
\end{align}
where the constraint $\W^T\W = \I$ is commonly used to incorporate the uncorrelatedness of the projection directions~\cite{weiss2008nips}. Obviously, Problem~(\ref{eqn:ssh}) can be solved by spectral decomposition of the matrix $\X_{\mathcal{SD}}^{T}\S\X_{\mathcal{SD}}+\rho\X^{T}\X$ and $\W$ is actually the eigenvectors corresponding to the largest eigenvalues.

Wang \etal~\cite{wang2010icml} argue that it is better to consider the dependence between contiguous bits rather than treat them independently. To this end, they propose a sequential learning algorithm which obtains one bit of codes at a time and updates the relational matrix $ \S $ before learning the next bit. Experimental studies show that the sequential algorithm achieves much better performance than semantic hashing.

\subsubsection{Label-regularized max-margin partition}
Mu \etal have taken the concept of margin into consideration to propose a semi-supervised \mbox{HFL} method termed \textit{label-regularized max-margin partition} (\mbox{LAMP})~\cite{mu2010cvpr}. Considering each hash function a binary classifier, the authors argue that larger margin between hash-induced partitions usually indicates better generalization ability to out-of-sample data.

For each hash function, the optimization problem of \mbox{LAMP} is formulated as follows:
\begin{align}
\min_{\omega,b, \xi,\zeta,y}&~\frac{1}{2}\|\omega\|^2+\frac{\lambda_1}{N}\sum_{i}\xi_i+\frac{\lambda_2}{N}\sum_{(i,j)\in\Theta}\zeta_{ij}\nonumber\\
\subto&~y_i(\omega^T\x_i+b)+\xi_i\ge1,\xi_i\ge0, \forall ~i,\nonumber\\
&~y_yy_j+\zeta_{ij}\ge0,\zeta_{ij}a\ge0,\forall ~(ij)\in\Theta\nonumber,
\end{align}
where $ \omega $ and $ b $ are the parameters of the hash function, $ y $ is the label vector induced by the hash function, $ \Theta $ denotes the set of constraints which is generated from label or side information. Since $ y_i $ is always set to be $ \sgn(\omega^T\x_i+b) $ in hashing-based methods and $ \omega $ can be represented by a linear combination of random landmark vectors using the kernel-trick, the modified formulation of the problem is used:
\begin{align}
\label{SSLTH:LAMP}
\min_{\nu,b, \xi,\zeta}&~\frac{1}{2}\nu^T\G\nu+\frac{\lambda_1}{N}\sum_{i}\xi_i+\frac{\lambda_2}{N}\sum_{(i,j)\in\Theta}\zeta_{ij}\\
\subto&~|\nu^T\k_i+b)+\xi_i\ge1,\xi_i\ge0, \forall ~i,\nonumber\\
&~(\nu^T\kappa_i+b)(\nu^T\kappa_j+b)+\zeta_{ij}\ge0,\zeta_{ij}a\ge0,\forall ~(ij)\in\Theta\nonumber,
\end{align}
where $ \G $ is a matrix computed from the random landmark vectors and $ \kappa_i $ denotes the kernel similarity between the $ i $th point and all the landmark vectors.

To optimize Problem~(\ref{SSLTH:LAMP}), however, is difficult since it is non-convex and nonlinear. The authors decompose the problem using a \textit{constrained-concave-convex-procedure} (\mbox{CCCP})~\cite{yuille2001nips}, and then solve the relaxed convex sub-problems in each iteration through an efficient cutting-plane based \mbox{QP} solver.

On several real data sets, \mbox{LAMP} outperforms kernelized \mbox{LSH}~\cite{kulis2009iccv} by a large margin. However, \mbox{LAMP} suffers from the local optimality issue and its performance is highly dependent on the quantity and quality of labeled pairs.

\subsection{Supervised Hash Function Learning}
In supervised \mbox{HFL} methods, only labeled data are used. There are several forms of label information, such as data labels and pairwise constraints, for example, similar pairs or dissimilar pairs.

\subsubsection{Boosted similarity sensitive coding}
\textit{Boosted similarity sensitive coding} (\mbox{BoostSSC})~\cite{shakhnarovich2003iccv,shakhnarovich2005thesis} might be, to the best of our knowledge, the first \mbox{HFL} algorithm for similarity search. Taking an embedding viewpoint, the authors propose to learn the embedding of unlabeled data into the Hamming space in which weighted Hamming distance preserves similarity information in the original input space.

In their approach, they first construct two kinds of point pairs based on labels or side information. In a positive point pair $( x_i , x_j )$, $ x_j $ is one of $ N $ nearest neighbors of $ x_i $ or vice versa. In a negative pair, two points are not neighbors. After embedding or mapping, each point is represented by an $M$-bit binary vector $\y_i = [h_{1}(\x_i), h_{2}(\x_i), ..., h_{M}(\x_i)]$, and the weighted Hamming distance between two points is given by $$d(\y_i, \y_j) = \sum^{M}_{m=1}\alpha_{m}|h_{m}(\x_i)-h_{m}(\x_j)|,$$ where the weights $\alpha_{m}$'s and the functions $h_{m}$'s that map the input vector $\x_i$ into binary features are learned using Boosting~\cite{schapire1999ijcai,schapire1999mlj}.

At each iteration of the learning stage, \mbox{BoostSSC} selects the parameters to minimize the following squared loss:
\begin{align}
\sum\nolimits_{k=1}\nolimits^{K}w(k)(z(k)-d(k))^2,\nonumber
\end{align}
where $ K $ is the number of training pairs, $ z(k) $ is the neighborhood label ($ z(k) = 1 $ if the points are neighbors and $ z(k) = −1 $ otherwise) of the $ k $th pair, $ w(k) $ is the weight of the $ k $th pair and $ d(k) $ is the weighted Hamming distance of the $ k $th pair computed based on current model parameters. After each iteration, \mbox{BoostSSC} increases the weights of wrongly classified pairs such that hash functions learned in the next iteration can be improved.

To summarize, \mbox{BoostSSC} is simple to code, relatively fast to train and achieves competitive performance in~\cite{torralba2008cvpr}. However, \mbox{BoostSSC} might be slow to converge and become trapped in the local optimums.

\subsubsection{SPEC hashing}

Inspired by the idea of distribution matching in metric learning, Lin \etal~\cite{lin2010cvpr} have developed another supervised HFL method called \textit{similarity preserving entropy-based coding} (\mbox{SPEC}), for which two linear time learning algorithms exist. Given a sparse semantic similarity matrix $ \S $, the goal of \mbox{SPEC} hashing is to construct a new matrix $ \W $, which is evaluated on the Hamming distance of the learned hash codes, to minimize the KL divergence between $ \S $ and $ \W $. Each hash function is a decision stump and the hash functions are learned in an incremental manner. Since computing the objective function value is quadratic to the number of objects, the authors proposed two approximated linear time algorithms to learn the hash functions.

Experimental results show that this method is better than spectral hashing. However, it is apparently unclear whether or not \mbox{SPEC} is better than its supervised or semi-supervised counterparts and whether or not its underlying assumption works well on general data sets. As such, more comparative studies are needed to validate its effectiveness.

\subsubsection{Minimal loss hashing} 

Norouzi \etal~\cite{norouzi2011icml} proposed a method called \textit{minimal loss hashing} (\mbox{MLH}) to learn hash functions from pairwise constraints. Intuitively speaking, \mbox{MLH} wants similar training points to be mapped onto binary codes that differ by no more than $ \rho $ bits, and dissimilar points to be mapped onto codes that differ by no less than $ \rho $ bits. The objective is represented as follows,
\begin{align}
\mathcal{O}(\w)  &=\sum\nolimits_{(\x_{i},\x_{j})\in\mathcal{S}}\ell_\rho(\|\y_i-\y_j\|_{H},S(i,j)),\nonumber
\end{align}
where $ \mathcal{S} $ is the set of point pairs, $ \y_i $ and $ \y_j $ are codes of $ \x_i $ and $ \x_j $ respectively, $ S(i,j) $ is the label of the point pair $ (\x_i, \x_j) $, and 
\begin{align}
\ell_{\rho}(m,s) = \left\{ \begin{array}{ll}
\max(m-\rho+1,0) & \mbox{if } s = 1\\
\max(\lambda\max(\rho-m+1,0)) & \mbox{if } s=0,
\end{array} \right.\nonumber
\end{align}
where $ \rho $ is a user provided hyperparameter that differentiates neighbors from non-neighbors in the Hamming space, and $ \lambda $ is a loss-hyperparameter controlling the ratio of the slopes of the penalties.

Though the objective is simple to understand, its formulation is discontinuous, non-convex and hence hard to optimize. To solve the optimization problem, the authors borrow the idea of structural \mbox{SVM}~\cite{tsochantaridis2004icml} and formulate a piecewise linear upper bound on the objective function. An alternating algorithm is developed to minimize the bound. Extensive experiments show that \mbox{MLH} achieves a state-of-the-art performance. However, \mbox{MLH} may still suffer from the local optimality problem.

\subsection{Summary}


Besides existing settings, some more complex \mbox{HFL} settings have also been explored recently, such as data with multiple similarities~\cite{zhang2011sigir}, data of multiple modalities~\cite{bronstein2010cvpr} and optimizing time and accuracy in a unified framework~\cite{he2011cvpr}.

The major limitation of \mbox{HFL} methods is that most of them involve a training procedure with complexity $ O(N^2) $ or $ O(D^2) $, meaning that the training data should not be very large. Although some subsampling approaches can be utilized to reduce the high computational cost, developing some cheaper learning methods is a future research issue very worthwhile studying.

%\chapter{Related Areas}
%\label{chap:relatedareas}
%In spite of being a new research topic, hash function learning is closely related to many other research topics in the areas of machine learning, data mining and computer vision. In this chapter, we briefly review two closest research areas, i.e., metric learning and active learning.

 % % % % % % % % % % % % % % % % % % % % % % % % % % % % % %
\section{Metric Learning}
\label{background:metric}
Metric learning is an important problem in the machine learning and pattern recognition communities. The objective is to learn an optimal metric, either linear or nonlinear, in the original feature space or the reproducing kernel Hilbert
space, from the training data. According to whether or not the label information or side-information is used to learn the metric, existing methods can be classified into unsupervised metric learning or supervised metric learning. In the following, we briefly review some typical works in each category. For a detailed review, we refer the interested readers to~\cite{yang2006tech}.
% the categories of 

\subsection{Unsupervised Metric Learning}
One typical case of unsupervised metric learning is linear dimensionality reduction. The most classic methods are \textit{principal component analysis} (\mbox{PCA}) and \textit{multidimensional scaling} (\mbox{MDS}). While \mbox{PCA} finds the subspace that best preserves the variance of the data, \mbox{MDS} finds the projection that best preserves the pairwise distance. The two methods are equivalent when \mbox{MDS} uses Euclidean distance. Despite being simple, efficient, and guaranteed to optimize their criteria, these linear methods could be very limited because they cannot find the nonlinear structure in the data.

To reveal the nonlinear structures of data, lots of nonlinear dimensionality reduction algorithms have been proposed. \mbox{ISOMAP}~\cite{tenenbaum2000science} assumes that isometric properties should be preserved in both the observation space and the intrinsic embedding space. According to this assumption, \mbox{ISOMAP} finds the subspace that best preserves the geodesic inter-point distance. Unlike \mbox{ISOMAP} that tries to preserve the geodesic distance for any pair of data points, \textit{locally linear embedding} (\mbox{LLE})~\cite{Roweis2000science} and \textit{Laplacian Eigenmap}~\cite{belkin2003nc} focus on the preservation of the local neighbor structure. As an extension of~\cite{belkin2003nc}, \textit{locality preserving projection} (\mbox{LPP})~\cite{he2003nips} finds linear projective mappings that optimally preserve the neighborhood structure of the data. Mutual information which measures the difference between probability distributions has also been introduced to dimensionality reduction methods. Related work includes \textit{stochastic neighbor embedding} (\mbox{SNE})~\cite{hinton2002nips} and \textit{manifold charting}~\cite{brand2002nips}.
%It is an optimal linear approximation to the eigenfunctions of the Laplace-Beltrami Operator on the manifold.

\subsection{Supervised Metric Learning}
Supervised metric learning algorithms are designed to learn either from the class labels or the side information which is often cast in the form of pairwise constraints (i.e., must-link constraints and cannot-link constraints). In~\cite{xing2002nips}, Xing \etal propose to learn a distance metric from the pairwise constraints. The optimal kernel is found to minimize the distance between data points in must-link constraints and simultaneously maximize the distance between data points in cannot-link constraints. \textit{Relevance component analysis}~\cite{shental2002eccv} is another popular approach, in which data points in the same classes are grouped in \textit{chunklets}, and the distance metric is computed based on the covariance matrix estimated from each \textit{chunklet}. Goldberger \etal~\cite{goldberger2004nips} develop an algorithm, abbreviated as \textit{neighborhood component analysis}, which combines metric learning with $k$-nearest neighbor (KNN) classification. Globerson \etal~\cite{globerson2005nips} present an algorithm to collapse data samples in the same class into a single point and make samples belonging to different classes far apart. Recently, an information-theoretic based approach~\cite{davis2007icml} developed by Davis \etal has been reported to achieve a state-of-the-art performance.

Empirical studies show that supervised metric learning algorithms usually outperform the unsupervised ones. However, most of the supervised metric learning algorithms need to solve non-trivial optimization problems, and thus are computationally expensive.

%particularly when the data are in large-scale and with high dimensionality.


\subsection{Relationship with \mbox{HFL}}
Both metric learning and hash function learning try to learn a proper metric from data, but they are different in several aspects. First of all, the learned metric in hash function learning is Hamming distance while that of metric learning is usually the Euclidean distance. Secondly, hash function learning maps data into binary codes whereas metric learning often maps data into real vectors. Last but not least, hash function learning and metric learning have quite different applications. Hash function learning aims to speed up approximate similarity search, but metric learning is usually applied to classification and recognition applications. Therefore, hash function learning puts more focus on local similarity than conventional metric learning.

%have quite different goals. Metric learning aims to learn a similarity metric while is the most propose for tasks at hand, while hash function learning targets at learn hash functions that can generate compact binary codes for similarity search. They are related since both of them defines a metric finally.

 % % % % % % % % % % % % % % % % % % % % % % % % % % % % % %
\section{Active Learning}
\label{background:active}
As a research topic originally developed by the machine learning community, active learning has been widely applied in many areas such as computer vision, data mining and information retrieval. The task of active learning is to select the most informative data for experts to label with the goal of reducing the labeling cost, which might be expensive in many tasks. Over the past few decades, a lot of algorithms have been proposed and gained great successes. 

The major challenge of active learning is how to find the most informative data for specific applications effectively and efficiently. In the following, we briefly review some general criteria of data informativeness and corresponding well-known algorithms. The readers are encouraged to read~\cite{Settles2009survey,tong2001thesis} for detailed reviews. Please also note that we use instance, example and data interchangeably in the sequel.

\subsection{Uncertainty-based Active Learning}

Perhaps the simplest and the most commonly used approach is \textit{uncertainty-based active learning} (\mbox{UAL})~\cite{lewis1994icml,Lewis1994sigir}. In this approach, the learner selects the instances whose labels it is the most uncertain about for the experts to label. This method is very straightforward for classifiers with probabilistic outputs. Take binary classification problems for example, when the basic learner could predict the labels in a probabilistic way, such as $P(y=+1\mid \x)=0.8$, \mbox{UAL} will select the instance whose posterior probability of being positive (or negative) is nearest to $0.5$. However, this simple method will not be suitable in settings where there are more than two classes. A more general \mbox{UAL} method selects the data points maximizing the \textit{entropy} defined as follows:
$$\x^*=\argmax\nolimits_{\x} H(\y\mid\x)=\argmax\nolimits_{\x} \left( -\sum\nolimits_i P(y_i\mid \x,\theta)\log P(y_i\mid \x,\theta)\right),$$
where $H(\y\mid\x)=-\sum_i P(y_i\mid \x,\theta)\log P(y_i\mid \x,\theta)$ is called entropy which measures the uncertainty of label $\y$ given instance $\x$, $y_i$ ranges over all possible labels and $\theta$ is the model parameter. The criterion of entropy can be easily generalized to probabilistic models for more complex structured instances, such as sequences~\cite{Settles2008emnlp} and trees~\cite{Hwa2004CL}.

An alternative to entropy-based \mbox{UAL} is selecting the instance whose \textit{most probable} label is the \textit{least confident}, which can be formulated as follows:
$$\x^* = \argmin\nolimits_{\x} P(y^*\mid \x,\theta),$$
where $y^* = \argmax_{y}P(y\mid \x, \theta)$ is the most probable class label of instance $\x$. This method has been shown to work especially well for information extraction tasks~\cite{Culotta2005aaai,Settles2008emnlp}.

For classifiers without probabilistic outputs, \mbox{UAL} can also be applied if the outputs could be mapped to probabilities~\cite{Lindenbaum2004mlj,Fujii1998CL}. Take margin-based classifiers such as \mbox{SVM} for example, the certainty can be defined as the distance to the decision boundary~\cite{Tong2002jmlr}.

Another general \mbox{UAL} method is \textit{query-by-committee} (\mbox{QBC})~\cite{Seung1992colt}. This approach maintains a group of classifiers, called a \textit{committee}, which are trained on the current labeled data. Each committee member represents one classifier, and is allowed to vote on any unlabeled instances. The most uncertain data are those whose labels the committee members have the largest disagreement on. Intuitively, the \mbox{QBC} strategy is to minimize the version space represented by the committee of classifiers.

The \mbox{UAL} approaches are not immune to selecting outliers, which have high uncertainty but are not helpful to the learner when labeled and incorporated into the training set. Examples are provided in~\cite{McCallum1998icml}.

\subsection{Representativeness-based Active Learning}
Although effective and easy to implement in many applications, \mbox{UAL} methods are always prone to querying outliers, which are useless, sometimes even harmful, for classifier training. To overcome this limitation, \textit{representativeness-based active learning} (\mbox{RAL}) has been proposed. The intuition of \mbox{RAL} methods is that the most informative data should be the most representative of the unlabeled data.

Xu \etal~\cite{Xu2003ecir} might be the first to implement the above intuition for \mbox{SVM} classifiers, using some simple heuristics. Instead of selecting the instances closest to the current \mbox{SVM} hyperplane, they first cluster the points in the margin of the current model and then query the labels of the cluster centroid. ~\cite{Nguyen2004icml} first clusters unlabeled instances and tries to avoid querying outliers by propagating label information from cluster centroid to instances in the same cluster. 

Density-based active learning algorithms, which tend to select the instances from dense regions, could also be considered a special case of \mbox{RAL}, because the denser the region is, the more representative the instances (located in the region) are. These \mbox{RAL} approaches are always used in combination with \mbox{UAL} methods~\cite{Xu2007ecir,Settles2008emnlp}. In \cite{Xu2007ecir}, data informativeness is measured by relevance, density and diversity in the relevance feedback tasks. Similarly,~\cite{Settles2008emnlp} develops an information density framework for the sequence labeling task to measure the uncertainty and representativeness of instances.

In recent years, experimental design originated in statistics has been introduced as a new family of \mbox{RAL} methods. The seminal work is \textit{transductive experimental design}~\cite{Yu2006icml}, which extends traditional experimental design methods to the transductive setting for active learning. Some subsequent work includes convex relaxation of the original problem~\cite{Yu2008sigir}, and incorporation of \textit{Laplacian} regularization~\cite{he2007sigir} or label information~\cite{zhen2010sigir}.


\subsection{Minimal Loss Active Learning}

In many real-world applications, the learned classifiers are eventually evaluated on a test set, so a better classifier should make less error on the test set. However, none of previous approaches directly optimize this objective, and this may explain why they do not work well in some circumstances. Intuitively, \textit{minimal loss active learning} (\mbox{MLAL}) aims to select the instances, when labeled and incorporated into the training set, leading to the largest error reduction on the test set. To evaluate the test error, we have to know the true labels of test instances. However, the true labels are unknown during the model training phase, as a result, the estimated (or expected) test error is used.

Cohn \etal proposed a statistically optimal solution, which selects the training examples that result in the lowest error on future test examples~\cite{cohn1996jair}. In their analysis, this goal could be achieved by minimizing the variance of training data. The authors developed two simple algorithms with closed form solutions for regression problems. For classification problems, Roy and McCallum used a sampling approach to estimate the expected error reduction~\cite{Roy2001icml}. Later, this framework was combined with a supervised learning approach to give a dramatic improvement over conventional \mbox{UAL} methods~\cite{Zhu2003icmlws}.

\mbox{MLAL} has the advantages of being near-optimal and independent of the types of classifiers. However, it may be the most prohibitively expensive strategy, because it requires not only estimating the expected future error over the unlabeled data at each learning iteration, but also retraining the classifier for each possible label of the instance. To reduce the computational cost, some researchers have resorted to subsampling the unlabeled data~\cite{Roy2001icml} or approximate training techniques~\cite{guo2007ijcai}.

\subsection{Relationship with \mbox{HFL}}
Active learning and active hashing have in common that both of them aim to find the most informative data for experts to label. As such, the criteria of informativeness might be similar in active learning and active hashing. However, active learning is usually applied to classification, regression and ranking problems, which are very different from the approximate similarity search to which active hashing applies. This may lead to big differences in the definitions of informativeness, the formulations of optimization problems as well as the algorithms.

\section{Summary}
\label{background:sum}
In this chapter, we have reviewed hashing-based methods for similarity search, and two machine learning areas related to hash function learning. The central idea of hashing-based methods is to index data using binary codes which have the advantages of highly reduced storage cost and very fast computation speed. Different from locality sensitive hashing which is based on random projections or permutations, hash function learning aims to learn hash functions from data automatically, and thus, as we see later in more detail, has a close relationship with metric learning and active learning.

In the next chapter, we introduce the framework of active hashing which combines the concepts behind semi-supervised hash function learning and active learning to make \mbox{HFL} more cost effective.


%%%%%%%%%%%%%%%%%%%%%%%%%%%%%%%%%%%%%%%
%\subsection{Hashing for Compression}
%
%Besides speeding up nearest neighbor search, hashing-based methods has also been applied to data compression, which is of essential importance in many large-scale problems.
%
%%%%%%%%%%%%%%%%%%%%%%%%%%%%%%
%\subsubsection{Hash Kernel}
%
%%Shi \etal~\cite{shi2010cvpr} use hash functions to speed up face recognition.
%
%Shi \etal~\cite{shi2009aistats,weinberger2009icml} firstly introduce hashing into vector compression. The major idea is to hash variables (\aka features) into a small number of bins such that linear kernels computed on the old and new feature spaces are guaranteed to be very close to each other .
%
%Given a feature mapping function $\phi(\x)$ and a hash function $h:\mathcal{J}\rightarrow \{1,\dots,M\}$, where $\mathcal{J}$ is the index set of feature dimension, a hash kernel can be defined as,
%\begin{align}
%\bar{k}(\x,\x') = \langle\bar{\phi}(\x),\bar{\phi}(\x')\rangle,
%\end{align}
%where $\bar{\phi}_{j}(\x) = \sum_{i\in\mathcal{J};h(i)=j}\phi_{i}(\x)$.
%
%The expectation of hash kernel is
%\begin{align}
%\mathbb{E}_{h}[\bar{k}^{h}(\x,\x')] = (1-\frac{1}{N})k(\x,\x')+\frac{1}{N}\sum_{i,i'}\phi_{i}(\x)\phi_{i}(\x').\nonumber
%\end{align}
%And the variance of every entry of hash kernel is upper bounded by $O(\frac{1}{N})$.
%
%
%\begin{theorem}
%Assume that the probability of deviation between the hash kernel and its expected value is bounded by an exponential inequality via
%\begin{align}
%p\left(|\bar{k}^{h}(\x,\x') - \mathbb{E}_{h}[\bar{k}^{h}(\x,\x')]|>\epsilon\right)\le c \exp(-c'\epsilon^2 n),\nonumber
%\end{align}
%for some constants $c,c'$ depending on the size of the hash and the kernel used. In this case the error $\epsilon$ arising from ensuring the above inequality for $m$ observations and $M$ classes (for a joint
%feature map $\phi(x, y)$ is bounded by (with $c'' = -\log c - 2\log 2$)
%\begin{align}
%\epsilon\le\sqrt{(2\log(m+1)+2\log(M+1)-\log\delta-c'')/c'}.\nonumber
%\end{align}
%\end{theorem}
%
%Since above hash kernels are biased, an unbiased hash kernel has been proposed~\cite{weinberger2009icml} and defined as follows,
%\begin{align}
%\bar{\phi}_{j}(\x) = \sum_{i\in\mathcal{J};h(i)=j}\phi_{i}(\x)\xi(i),\nonumber
%\end{align}
%where $\xi$ is a hash function with image range $\{\pm1\}$.
%
%%%%%%%%%%%%%%%%%%%%%%%%%%%%%%
%\subsubsection{HashCoFi}
%In previous section, we have introduced an approach to compress vectors, while in this section, we introduce a similar method that compress matrices.
%
%In matrix factorization, which has been proved to be a powerful tool for collaborative filtering, the observations are viewed as a sparse matrix $\Y$ where $Y_{ij}$ indicates the rating user $i$ gave to item $j$. Matrix factorization approaches then
%approximate this matrix $\Y$ with a dense matrix $\F$ and this
%approximation is modeled as a matrix product between
%a matrix $\U\in\mathbb{R}^{N\times D}$ of user factors and a matrix $\V\in\mathbb{R}^{M\times D}$ of item factors so that $\F = \U\V^T$.
%
%One of the key challenges is that storing $\U$ and $\V$ quickly becomes infeasible for increasing $N,M,D$. Thus we would like to find approximated compressed representations of $\U$ and $\V$ in the form of vectors $\u$ and $\v$ respectively.
%
%Let $h,h'$ denote two independent hash
%functions with range $\{1,\dots, N\}$ where $N$ denotes
%the size of hash table. Moreover, denote by $\sigma,\sigma'$ two independent Rademacher functions with range $\{\pm1\}$ with expected value 0 for any argument.
%
%The compressed representation $\u$ and $\v$ can be constructed as follows,
%\begin{align}
%u_i&= \sum_{(j,k):h(j,k)=i}U_{jk}\sigma(j,k),\nonumber\\
%v_i&= \sum_{(j,k):h'(j,k)=i}V_{jk}\sigma'(j,k)\nonumber.
%\end{align}
%In another word, the entries in $\U$ and $\V$ are added randomly into $\u$ and $\v$ respectively. The basic assumption of this scheme is that only a small number of matrix entries are significant. We reconstruct $\U$ and $\V$ via
%\begin{align}
%\tilde{U}_{ij} = u_{h(i,j)}\sigma(i,j),\tilde{V}_{ij} = v_{h'(i,j)}\sigma'(i,j).\nonumber
%\end{align}
%As a result,
%\begin{align}
%\tilde{F}_{ik} = \sum_{j=1}^{D}u_{h(i,j)}v_{h'(k,j)}\sigma(i,j)\sigma'(k,j).\nonumber
%\end{align}
%
%It can be proved that the reconstructed $\U,\V,\F$ are in expectation accurate and the variance of $\F$ is upper bounded by $O(\frac{1}{N})$. The expectation and variance are \wrt $\sigma,\sigma'$.
%
%Since it could be inefficient to store $\u$ and $\v$ separately, a joint compression scheme is proposed,
%\begin{align}
%w_i = \sum_{(a,b):h(a,b)=i}U_{ab}\sigma(a,b)+\sum_{(a,b):h'(a,b)=i}V_{ab}\sigma'(a,b)\nonumber,
%\end{align}
%and the reconstruction is,
%\begin{align}
%\tilde{U}_{ij} &= w_{h(i,j)}\sigma(i,j)\nonumber\\
%\tilde{V}_{ij} &= w_{h'(i,j)}\sigma'(i,j).\nonumber
%\end{align}
%
%The expectation of $\tilde{F}_{ij}$ has a small correction term and its variance still upper bounded by $O(\frac{1}{N})$.
%
%Given the hash functions, we can learn $\w$ using stochastic gradient descent method.
%
%The advantages of HashCoFi are:
%\begin{itemize}
%  \item Scales up to very large collaborative filtering problems, since the storage only depends on memory constraints.
%  \item The compression of factors is very effective when memory is very limited
%\end{itemize}
%
%The disadvantages of HashCoFi:
%\begin{itemize}
%  \item The hash functions, i.e., h, h¡¯, are data-independent, hence N could still be very large
%  \item Computational cost becomes larger due to the use of hashing
%  \item Performs worse than MF models when their D is large, with same memory constraints
%      \item Needs a number of repeats of sigma, sigma¡¯, to achieve good estimation
%\end{itemize}
%



\chapter{Active Hashing}
\label{chap:ah}

 % % % % % % % % % % % % % % % % % % % % % % % % % % % % % %
\section{Introduction}

Existing supervised and semi-supervised \mbox{HFL} methods can be considered \textit{passive hashing} because they assume that the labeled data are provided beforehand. However, given the labeled data may be expensive to acquire,\footnote{Although crowdsourcing has recently been developed to acquire labels in a very cheap manner, for some domains involving privacy or confidentiality, it is inappropriate to involve the ``crowd" in the labeling work.} it is more cost effective if we are able to identify and label the most informative data. In this case, hash functions can be learned efficiently with only a small number of labeled data and the labeling cost, which might be very high in practice, can be greatly reduced. Besides, as we will see in next subsection, the effectiveness of the labeled data may be quite different. In some situations, adding more labels (if not carefully chosen) into the training set may even impair the quality of the learned hash functions. As a result, it is a very worthwhile endeavor to explore methods of selecting and labeling data in an active manner for hash function learning.

To eliminate the aforementioned limitations of passive hashing, in this chapter, we propose a novel framework, called \textit{active hashing} (\mbox{AH}), with the goal of selecting the most informative data to label for hash function learning.\footnote{Active hashing and active learning~\cite{Angluin1988mlj,cohn1994mlj} have quite different goals, although they share some common ideas such as selecting the most informative data for expert to label.} Generally speaking, each active hashing iteration consists of three phases as depicted in Figure~\ref{fig:ah:ah}. Specifically, given the labeled data set $\mathcal{L}$, the unlabeled data set $\mathcal{U}$ and the candidate data set $\mathcal{C}$,
\begin{itemize}
  \item Active selection phase: select the most informative data points from $\mathcal{C}$ to form set $\mathcal{A}$ .
  \item Labeling phase: ask an oracle to label the points in $\mathcal{A}$ and update  $\mathcal{L}$,  $\mathcal{U}$ and  $\mathcal{C}$.
  \item Training phase: Train hash functions based on both $\mathcal{L}$\footnote{The usage of $\mathcal{L}$ may be different in different methods. For example, \mbox{SSH} can construct labeled pairs $\mathcal{S}$ and $\mathcal{D}$  easily from $\mathcal{L}$.} and $\mathcal{U}$.
\end{itemize}

\begin{figure}[htb]
\centering
%\vspace{-0.5cm}
\epsfig{figure=fig/activehashing.pdf, width=0.5\textwidth}
%\vspace{-2cm}
\caption{Three phases of one active hashing iteration. Each iteration starts from the active selection phase and ends by the training phase.}
\label{fig:ah:ah}
\end{figure}



In practice, there can be several iterations before some criteria are met. For example, we have run out of labeling resources or the quality of the learned hash functions is satisfactory.

%Under this framework, we have developed a simple method based on the uncertainty criterion.

The remainder of this chapter is organized as follows. Section~\ref{AH:motivation} illustrates the limitations of passive hashing to motivate our active hashing framework. In Section~\ref{AH:UAH}, we present a simple active hashing method based on the uncertainty criterion. Empirical studies conducted on several real-world data sets are presented in Section~\ref{AH:exps}, followed by some conclusions in Section~\ref{AH:conclusion}.

%In Section~\ref{AH:ITAH}, we present an information theoretic method which select labeled data by maximizing the information gain.
%\section{Related Work}
%\label{section:relatework}
%
%The most well-known hashing-based method is probably locality sensitive hashing (\mbox{LSH})~\cite{andoni2006focs}. \mbox{LSH} simply applies random linear projection followed by thresholding to index data points, with the goal of assigning similar binary codes to data points that are close in the feature space.  An appealing property of LSH is the existence of theoretical guarantee that, as the code length increases, the Hamming distance between two codes will asymptotically approach the metric distance. However, \mbox{LSH} may lead to quite inefficient (long) codes in practice due to its data-independent nature.
%
%Machine learning techniques have recently been introduced to learn data-dependent hash functions which are superior to data-independent ones.  In semantic hashing~\cite{salakhutdinov2009ijar}, a stacked Restricted \mbox{Boltzmann} Machine (\mbox{RBM}) is used to generate compact binary codes for document retrieval. In boosted similarity sensitive coding~\cite{shakhnarovich2005thesis}, the boosting approach is incorporated to learn the codes. These two methods have been demonstrated to be much more effective than \mbox{LSH} for similarity search in a large image database containing millions of pictures~\cite{torralba2008cvpr}. Spectral hashing (\mbox{SH})~\cite{weiss2008nips} treats the hashing problem as a spectral embedding problem and makes use of spectral decomposition to obtain binary codes. Although good performance has been shown when applying \mbox{SH} on some data sets, it makes a restrictive and unrealistic assumption that data are uniformly distributed in a hyper-rectangle.  This restrictive assumption does not hold in general.  To relax this assumption, some new methods have been proposed, such as self-taught hashing~\cite{zhang2010sigir}, binary reconstructive embedding~\cite{kulis2009nips} and distribution matching~\cite{lin2010cvpr}. Besides, several researchers have developed more general hashing methods that can support kernels~\cite{kulis2009iccv,he2010kdd,mu2010cvpr}, non-metric data~\cite{mu2010aaai} and multimodal data~\cite{bronstein2010cvpr}.
%
%As expected, our work is also related to \emph{active learning}~\cite{Angluin1988mlj,cohn1994mlj,Lewis1994sigir,MacKay1992ncj}.  As a topic in machine learning, active learning seeks to reduce the labeling cost by identifying and presenting the most informative examples from the unlabeled data for the human experts to label. Hence the key issue is how to measure data informativeness. Many criteria have been proposed for this. Taking classification problems for example, some methods select the most uncertain data given the current classifier~\cite{Lewis1994sigir}, some select the data with the smallest margin~\cite{Tong2002jmlr}, some select the data on which multiple classifiers disagree most with each other~\cite{Freund2003jmlr,McCallum1998icml,Seung1992colt}, some select the data that optimize some information gain~\cite{guo2007ijcai,MacKay1992ncj,Roy2001icml,Zhu2003icmlws}, and some select data points which are the most representative given the data distribution~\cite{Nguyen2004icml,Yu2006icml}. 
%
%Another hot research issue in active learning is batch mode active learning~\cite{hoi2006www,hoi2006icml,Guo2007nips} which aims at selecting a set of points rather than one point at a time. Batch mode active learning can save much computational cost and thus is very suitable for large-scale and parallel computing applications. More recently, experimental design techniques~\cite{atkinson1992book} in statistics have been introduced to active learning and have achieved state-of-the-art performance in active learning  applications~\cite{Yu2006icml,he2007sigir,Yu2008sigir,zhen2010sigir}.
\section{Motivation}
\label{AH:motivation}

%\subsubsection{Limitations of Passive Hashing}
%\label{section:motivation}
In passive hashing methods such as \mbox{SSH}, there are two types of point pairs (similar and dissimilar) and both types are considered equally important. However, as illustrated in Fig.~\ref{fig:bias} below, treating both types of point pairs as equally important is not satisfactory when the data points belong to more than two classes.

%\vspace{-1cm}
\begin{figure}[htb]
\centering
\epsfig{figure=fig/ah/bias, width=0.7\textwidth}
\caption{Limitations of passive hashing}
\label{fig:bias}
\end{figure}

In Fig.~\ref{fig:bias}, we use circles, triangles, hexagons and smaller circles to denote labeled data of three different classes and unlabeled data respectively. There are two dissimilar pairs represented by a blue dashed line and a green dashed line. Obviously, the circle class is more similar (closer) to the hexagon class than the triangle class.  However, the two pairs induce the same weight ($-1$) in $\mathcal{D}$, making the learned hash functions biased and perform worse even when more labeled pairs are provided. As a result, it is very important to treat the pairs with different weights or informativeness, as we will discuss later, but passive hashing methods are unable to do so. 


To see some real examples of these limitations, we conduct a group of experiments on the \mbox{MNIST} data set.\footnote{Details of the data set and the evaluation methods will be introduced later.} In the experiments, we trained several semi-supervised hashing models~\cite{wang2010cvpr} by varying the number of labeled points\footnote{All the pairwise relations among these points will be used for training.}. The performance of these models, measured by average precision and recall over ten random repeats, is plotted in Fig.~\ref{fig:ah:demo-randompoint} below.

\begin{figure}[ht]
\vspace{-1.5cm}
\subfigure[Precision]{
    \begin{minipage}[b]{0.48\linewidth}
%        \centering
        \epsfig{figure=fig/activepoint-largemnist-precision500-24b-bias, width=1.0\textwidth}\vspace{-2cm}
    \end{minipage}}
\subfigure[Recall]{
    \begin{minipage}[b]{0.48\linewidth}
%        \centering
        \epsfig{figure=fig/activepoint-largemnist-recall500-24b-bias, width=1.0\textwidth}\vspace{-2cm}
    \end{minipage}}
\caption{Semi-supervised hashing with a varying number of labeled data points}
\label{fig:ah:demo-randompoint} %% label for entire figure
\end{figure}

We can see that the model performance in terms of both precision and recall has big variance, showing that the effectiveness of labeled points is quite different. Moreover, the model performance improves as the number of labeled data points increases until about one thousand. After that, however, the performance degrades as more labeled points are added, indicating that choosing labeled data is very critical to supervised and semi-supervised \mbox{HFL} algorithms. 

The motivation of active hashing is simple: if we select labeled data carefully, we can not only reduce the model variance but also keep improving the model quality without impairing it when more labeled data are added into the training set.

\section{Uncertainty-based Active Hashing}
\label{AH:UAH}




%In the first group of experiments, we randomly select $1,000$ data points and vary the number of observed pairwise relations among these points from $ 10^2 $ to $ 10^6 $. The selection of pairwise relations is repeated ten times and the average results are plotted in Fig.~\ref{fig:ah:demo-fixpoint}. Obviously, given that the data points are fixed, the larger the number of observed pairwise relations, the better performance the model achieves. But the variance is small, meaning that if we select a fixed number of pairwise relations among these points, different random selections make little difference.
%
%\begin{figure}[ht]
%\vspace{-2cm}
%\subfigure[Precision]{
%    \begin{minipage}[b]{0.48\linewidth}
%        \centering
%        \epsfig{figure=fig/demo-fixpoint-precision-8b, width=1.0\textwidth}\vspace{-2.5cm}
%    \end{minipage}}
%\subfigure[Recall]{
%    \begin{minipage}[b]{0.48\linewidth}
%        \centering
%        \epsfig{figure=fig/demo-fixpoint-recall-8b, width=1.0\textwidth}\vspace{-2.5cm}
%    \end{minipage}}
%\caption{Semi-supervised hashing with a varying number of labeled pairs}
%\label{fig:ah:demo-fixpoint} %% label for entire figure
%\end{figure}

%From above experiments, we see previous passive hashing methods suffer from aforementioned limitation, in the following, we present our solution which can overcome this limitation.

\subsection{Uncertainty-based Active Hashing}
The challenge of active hashing is how to identify the most informative points for hash function learning. Based on a previous semi-supervised \mbox{HFL} model, i.e., \mbox{SSH}~\cite{wang2010cvpr}, we propose an \textit{uncertainty-based active hashing} (\mbox{UAH}) method.

We have already known that, in~\cite{wang2010cvpr}, one bit code is obtained by thresholding a linear projection of a point $\x$, e.g., $h_m = \sgn(\w_m^T\x)$ for the $m$th bit. Intuitively speaking, the magnitude of $\w_m^T\x$ measures the certainty of current hash function $h_m$ on the $m$th bit code of $\x$. In other words, the larger $|\w_m^T\x|$ is, the more certain $h_m$ is on $\x$, and vice versa. Thus it is very straightforward to use $h_m$'s certainty $\bar{h}_m(\x) = |\w^T_m\x|$ to measure the informativeness of point $\x$; that is, the smaller $\bar{h_m}(\x)$ is, the less certain $h_m$ is on $\x$ and hence the more informative $\x$ is to $h_m$. Therefore, given a group of $M$ hash functions $\mathcal{H} =\{h_m\}_{m=1}^M$, we define the certainty of $\x$ \wrt $\mathcal{H}$ as follows,
\begin{mydef} Data Certainty
  \label{definition:certainty}
   \begin{align}
   f(\mathcal{H},\x) = \|\W^T\x\|_{2},\nonumber
    \end{align}
    where $\|\cdot\|_{2}$ denotes the vector $\ell_2$ norm.\footnote{Please note that other norms such as $\ell_1$ norm can also be used to define certainty.}
\end{mydef}

Based on this simple definition of informativeness, one simple active hashing algorithm is to select the point with the smallest $f$ value at each active iteration. However, as pointed out by~\cite{hoi2006www,Guo2007nips}, selecting data point in such a greedy manner, i.e., one at a time, could be very inefficient in practice. Moreover, greedy selection may be suboptimal because the selected points might be very similar to each other and hence provide redundant information to the learner. As such, we propose a \textit{batch mode active hashing} (\mbox{BUAH}) algorithm which can select a batch of points at a time.


Similar to the batch mode active learning algorithm for image classification studied in~\cite{hoi2006icml}, we define the objective of \mbox{BUAH} as follows,
\begin{eqnarray}
\label{equation:ah}
\min_{\muu}&\muu^T\tilde{\f}+\frac{\lambda}{M}\muu^T\K\muu\\
\subto& \muu\in \{0,1\}^{|\mathcal{C}|}, \muu^T\1=M,\nonumber
\end{eqnarray}
where $\tilde{\f}$ is the normalized certainty value of the points in a candidate set $\mathcal{C}$ and can be computed as $\tilde{f}_i = f(\mathcal{H},\x_i)/f_{\max}$ and $ f_{\max} = \max_{\x_i\in\mathcal{C}} f(\mathcal{H},\x_i)$, $\muu$ is an indicator vector whose entries control whether or not corresponding points are selected, i.e., $\mu_i = 1$ when $\x_i$ is selected and $\mu_i = 0$ when $\x_i$ is not selected. We call $\lambda$ the balancing parameter, since it controls the balance between the two terms. Moreover, $K$ is the number of points we want to select and $\K$ is a positive semi-definite similarity matrix defined on $\mathcal{C}$. We use cosine similarity in our experiments, however, using other similarities is also possible and we leave it as future work.

Intuitively, we can consider the first term of Eqn.~(\ref{equation:ah}), $\muu^T\tilde{\f}$, to be the sum of certainty of selected examples, and the second term, $\muu^T\K\muu$, to be the sum of similarity between these selected examples. Thus minimizing the objective of Problem~(\ref{equation:ah}) is to select a set of points that are the most uncertain and dissimilar with each other. However, Problem~(\ref{equation:ah}) is an integer programming problem and NP-hard. We relax the problem by replacing the integer constraint $\muu\in \{0,1\}^{|\mathcal{C}|}$ with continuous constraint $\0\le\muu\le\1$, and arrive at the following optimization problem,
 \begin{eqnarray}
 \label{equation:ah:relax}
\min_{\muu}&\muu^T\tilde{\f}+\frac{\lambda}{M}\muu^T\K\muu\\
 \subto& \0\le\muu\le\1,\muu^T\1=M.\nonumber
 \end{eqnarray}
The optimization problem~(\ref{equation:ah:relax}) is a standard \mbox{QP} problem that can be solved efficiently using existing solvers~\cite{boyd2004convex}. Finally, given the estimated $\muu$, we select $M$ unlabeled points with the largest $\mu_i$'s.

The algorithm of \mbox{BUAH} is summarized in Algorithm~\ref{algorithm:ah}, where the superscripts of $\mathcal{L}, \mathcal{U}, \mathcal{C}$ indicate the indices of active selection iterations and $T$ is the total number of iterations. Since solving Problem~(\ref{equation:ah:relax}) requires time complexity $O(|\mathcal{C}|^3)$, in practice, we can select a subset of $\mathcal{C}$ corresponding to the smallest $f_i$ values to reduce the computational cost without much loss of accuracy.

%\begin{algorithm}
%%\DontPrintSemicolon
%%\SetKwData{Left}{left}\SetKwData{This}{this}\SetKwData{Up}{up}
%%\SetKwFunction{Union}{Union}\SetKwFunction{FindCompress}{FindCompress}
%\SetKwInOut{Input}{Input}\SetKwInOut{Output}{Output}
%\Input{$\mathcal{L}^{0},\mathcal{U}^{0},\mathcal{C}^{0}, T$}
%\Output{$\mathcal{L}^{T}, \mathcal{C}^{T}$}
%\Begin{
%\For{$t=1$ to $T$}{
%Train hash functions $\mathcal{H}$ based on $\mathcal{L}^{t-1}$ and $\mathcal{U}^{t-1}$\;
%Compute certainty value $\f$ of $\mathcal{C}^{t-1}$ using Definition~\ref{definition:certainty}\;
%Solve Problem~(\ref{equation:ah:relax}) and obtain $\muu$\;
%Choose $M$ examples with the largest $\mu_i$ into $\mathcal{A}^{t}$\;
%Request the labels of points in $\mathcal{A}^{t}$\;
%Update $\mathcal{L}^{t}\leftarrow\mathcal{L}^{t-1}\cup\mathcal{A}^{t}$, $\mathcal{U}^{t}\leftarrow\mathcal{U}^{t-1}\setminus \mathcal{A}^{t}$ and $\mathcal{C}^{t}\leftarrow\mathcal{C}^{t-1}\setminus \mathcal{A}^{t}$\;  }
%}
%\caption{Algorithm of \mbox{BUAH}}
%\label{algorithm:ah}
%\end{algorithm}

\begin{algorithm}[tb]
   \caption{Algorithm of \mbox{BUAH}}
   \label{algorithm:ah}
\begin{algorithmic}
   \STATE {\bfseries Input:} \\
   $\mathcal{L}^{0}$ -- initial set of labeled data,\\
   $\mathcal{U}^{0}$ -- initial set of unlabeled data,\\
   $\mathcal{C}^{0}$ -- initial set of candidate data,\\
    $T$ -- number of iterations
	\STATE {\bfseries Procedure:}
%   \STATE Initialize $noChange = true$.
\FOR{$t=1$ {\bfseries to} $T$}
\STATE Train hash functions $\mathcal{H}$ based on $\mathcal{L}^{t-1}$ and $\mathcal{U}^{t-1}$\;
\STATE Compute certainty value $\f$ of $\mathcal{C}^{t-1}$ using Definition~\ref{definition:certainty}\;
\STATE Solve Problem~(\ref{equation:ah:relax}) and obtain $\muu$\;
\STATE Choose $M$ examples with the largest $\mu_i$ into $\mathcal{A}^{t}$\;
\STATE Request the labels of points in $\mathcal{A}^{t}$\;
\STATE Update $\mathcal{L}^{t}\leftarrow\mathcal{L}^{t-1}\cup\mathcal{A}^{t}$, $\mathcal{U}^{t}\leftarrow\mathcal{U}^{t-1}\setminus \mathcal{A}^{t}$ and $\mathcal{C}^{t}\leftarrow\mathcal{C}^{t-1}\setminus \mathcal{A}^{t}$\;
\ENDFOR
%
\end{algorithmic}
\end{algorithm}

 % % % % % % % % % % % % % % % % % % % % % % % % % % % % % %
%\section{Optimistic Active Hashing}
%\label{AH:LAH}
%
%We want to formulate the active sampling problem as a optimization problem, whose objective is the expected reduction of some loss.

 % % % % % % % % % % % % % % % % % % % % % % % % % % % % % %
%\section{Information-theoretic active hashing}
%\label{AH:ITAH}
%
%We require the sampler to select those points maximizing some information gain \wrt a hashing model. This approach should also be applied to other hashing models if the selected points are really informative to hash function learning. You can read the information theoretic active learning again.

 % % % % % % % % % % % % % % % % % % % % % % % % % % % % % %
\section{Experiments}
\label{AH:exps}


%%%%%%%%%%%%%%%%%%%%%%%%%%%%%%%%%%%%%%%
\subsection{Experimental Settings}

We conduct several experiments on two tasks, namely, image retrieval and text retrieval. For both tasks, we mean-center the data and normalize the feature vectors to have unit norm. Both data sets are partitioned into separate training and test sets.  We first use the training set to learn hash functions. Then we retrieve from the training set 500 points with smallest Hamming distances to the query point in the test set. Two evaluation measures, precision and recall, of the retrieved points are computed for each query and then averaged over all the queries. In all the experiments, we set the size of the candidate set $|\mathcal{C}| = 5000$ and the code length $K=24$.

Three algorithms are compared in the experiments:
\begin{itemize}
    \item \textbf{Random Sampling} (\mbox{Random}), which randomly selects examples to label. This is essentially a passive hashing method.\footnote{But unlike other passive hashing methods, the labeled set does grow in size.  This is similar to some active learning methods with random data selection.  So it is still ``active'' in some sense.}
    \item \textbf{Greedy Active Hashing} (\mbox{GAH}), which selects one point with the smallest $\tilde{f}_i$ in each iteration.
    \item \textbf{Batch Mode Active Hashing} (\mbox{BMAH}), which selects a batch of $M$ points by solving Problem~(\ref{equation:ah:relax}).
\end{itemize}


%%%%%%%%%%%%%%%%%%%%%%%%%%%%%%%%%%%%%%%
\subsection{Image Retrieval}

%%%%%%%%%%%%%%%%%%%%%%%%%%%%
\subsubsection{Data set}
The \mbox{MNIST} data set\footnote{http://yann.lecun.com/exdb/mnist/} consists of $70,000$ images of handwritten digits.  Each image has $28\times28$ pixels and is associated with a label from 0 to 9. We use the gray-scale intensity values of the images as features resulting in a 784-dimensional vector space.  The data set is split into a training set of $69,000$ images and a test set of $1,000$ images.

%%%%%%%%%%%%%%%%%%%
\subsubsection{Comparison of \mbox{BMAH} and \mbox{GAH}}
\label{section:comp-bmah-gah-mnist}

We first compare two \mbox{AH} algorithms, \mbox{BMAH} and \mbox{GAH}. Initially, we randomly choose 100 points and their labels from the training set to form $\mathcal{L}^0$ and use the remaining data as $\mathcal{U}^0$. We then use \mbox{BMAH} and \mbox{GAH} individually to select $M$ points\footnote{We note that \mbox{GAH} selects only one example in each iteration, so it needs $M$ iterations to select a total of $M$ points. However, \mbox{BMAH} can select all $M$ points in a single iteration.} and report the retrieval results of the hash functions learned from the updated training set.  We use $\lambda=0.4$ for \mbox{BMAH}.\footnote{The reason why we set $\lambda=0.4$ will be made clear later.} The procedure is repeated 10 times and the average results, as well as their standard deviations, and total time cost (in seconds) for different values of $M$ are reported in Table~\ref{table:mnist-greedy}.

\begin{table}[htb]
\centering
\caption{Comparison of \mbox{BMAH} and \mbox{GAH} for image retrieval}
\begin{tabular}{|c|c|c|c|c|}
\toprule[0.8pt]\addlinespace[0pt]
$M$ & Method &  Precision &  Recall& Time Cost (seconds)\\
\addlinespace[0pt]\midrule[0.8pt]\addlinespace[0pt]
\multirow{2}{1.5em}{\centering 50}&\mbox{BMAH}&${0.6096}{\pm0.0009}$&${0.0438}{\pm0.0001}$&195.03\\\cline{2-5}
&\mbox{GAH}&${0.6096}{\pm0.0007}$&${0.0438}{\pm0.0001}$&482.23\\
\addlinespace[0pt]\midrule[0.5pt]\addlinespace[0pt]
\multirow{2}{1.5em}{\centering 100}&\mbox{BMAH}&${0.6097}{\pm0.0007}$&${0.0438}{\pm0.0001}$&214.23\\\cline{2-5}
&\mbox{GAH}&${0.6100}{\pm0.0007}$&${0.0438}{\pm0.0001}$&887.60\\
\addlinespace[0pt]\midrule[0.5pt]\addlinespace[0pt]
\multirow{2}{1.5em}{\centering 150}&\mbox{BMAH}&${0.6107}{\pm0.0009}$&${0.0438}{\pm0.0001}$&352.49\\\cline{2-5}
&\mbox{GAH}&${0.6120}{\pm0.0009}$&${0.0439}{\pm0.0001}$&1563.03\\
\addlinespace[0pt]\midrule[0.5pt]\addlinespace[0pt]
\multirow{2}{1.5em}{\centering 200}&\mbox{BMAH}&${0.6113}{\pm0.0012}$&${0.0439}{\pm0.0001}$&433.81\\\cline{2-5}
&\mbox{GAH}&${0.6154}{\pm0.0013}$&${0.0442}{\pm0.0001}$&1744.36\\
 \addlinespace[0pt]\bottomrule[0.8pt]
\end{tabular}
\label{table:mnist-greedy}
\end{table}

From Table~\ref{table:mnist-greedy}, we observe that \mbox{BMAH} and \mbox{GAH} are comparable in performance in terms of both precision and recall, showing that there is not much redundancy among the most informative points. However, \mbox{GAH} has much higher computational time cost than \mbox{BMAH}. Therefore, we exclude \mbox{GAH} in the subsequent experiments.

%%%%%%%%%%%%%%%%%%%
\subsubsection{Comparison of \mbox{BMAH} and \mbox{Random}}

In this section, we compare \mbox{BMAH} with \mbox{Random}. We again choose $100$ random points from the training set to form $\mathcal{L}^{0}$ and use the remaining data as $\mathcal{U}^{0}$. For \mbox{BMAH}, we set the parameters to be $M = 100$ and $\lambda = 0.4$. The whole process is repeated 10 times and we plot the average results and their standard deviations in Fig.~\ref{fig:apt-largemnist-24b}.

\begin{figure}[htb]
\vspace{-1.5cm}
\subfigure[Learning curves \wrt precision]{
    \begin{minipage}[b]{0.48\linewidth}
%        \centering
        \epsfig{figure=fig/ah/activepoint-largemnist-precision500-24b5, width=1.0\textwidth}\vspace{-2cm}
        \label{fig:apt-largemnist-precision-24b}
    \end{minipage}}
\subfigure[Learning curves \wrt recall]{
    \begin{minipage}[b]{0.48\linewidth}
%        \centering
        \epsfig{figure=fig/ah/activepoint-largemnist-recall500-24b5, width=1.0\textwidth}\vspace{-2cm}
        \label{fig:apt-largemnist-recall-24b}
    \end{minipage}}
\caption{Learning curves of \mbox{BMAH} and \mbox{Random} for image retrieval}
\label{fig:apt-largemnist-24b}
\end{figure}

We can see from the figures that \mbox{BMAH} performs much better than \mbox{Random}, especially at the late stage. We observe that the performance of \mbox{Random} degrades as more labels are obtained. This can be attributed to the limitation of passive hashing as introduced in Section~\ref{AH:motivation}. On the other hand, \mbox{BMAH} can overcome this problem by selecting the most uncertain points because these points are likely to locate near the class boundary and hence the difference between points in a dissimilar pair may be small.


%%%%%%%%%%%%%%%%%%%%%%%%%%%%%%%%%%%%%%%%%%%%%%%%%%%%%
\subsubsection{Varying initial label size $|\mathcal{L}^{0}|$}
\label{section:comp-initsize-mnist}

To evaluate the effect of the initial label size on \mbox{BMAH} and \mbox{Random}, we conduct some experiments by varying the value of $|\mathcal{L}^0|$. The parameters of \mbox{BMAH} are $M=10$ and $\lambda = 0.4$. For each $|\mathcal{L}^0|$ value, the whole procedure is repeated 10 times and the average results, as well as standard deviations, after selecting $1,000$ points are reported in Table~\ref{table:mnist-initsize}, where boldface numbers indicate better results.

\begin{table}[htb]
\centering
\caption{Comparison of \mbox{BMAH} and \mbox{Random} with varying initial label size $|\mathcal{L}^{0}|$}
\vspace{0.2cm}
\label{table:mnist-initsize}
%\begin{center}
\begin{tabular}{|c|c|c|c|}
\toprule[0.8pt]\addlinespace[0pt]
$|\mathcal{L}^{0}|$ & Method &  Precision &  Recall\\
\addlinespace[0pt]\midrule[0.8pt]\addlinespace[0pt]
\multirow{2}{3.5em}{\centering 100}&Random&${0.6416}{\pm0.0103}$&${0.0460}{\pm0.0007}$\\\cline{2-4}
                        &\mbox{BMAH}&${\textbf{0.6793}}{\pm0.0033}$&${\textbf{0.0489}}{\pm0.0002}$\\
\addlinespace[0pt]\midrule[0.5pt]\addlinespace[0pt]
\multirow{2}{3.5em}{\centering 200}&Random&${0.6430}{\pm0.0087}$&${0.0461}{\pm0.0006}$\\\cline{2-4}
                    &\mbox{BMAH}&${\textbf{0.6941}}{\pm0.0025}$&${\textbf{0.0500}}{\pm0.0002}$\\
\addlinespace[0pt]\midrule[0.5pt]\addlinespace[0pt]
\multirow{2}{3.5em}{\centering 300}&Random&${0.6518}{\pm0.0119}$&${0.0467}{\pm0.0009}$\\\cline{2-4}
                &\mbox{BMAH}&${\textbf{0.6985}}{\pm0.0052}$&${\textbf{0.0502}}{\pm0.0004}$\\
\addlinespace[0pt]\midrule[0.5pt]\addlinespace[0pt]
\multirow{2}{3.5em}{\centering 400}&Random&${0.6474}{\pm0.0150}$&${0.0464}{\pm0.0011}$\\\cline{2-4}
                    &\mbox{BMAH}&${\textbf{0.7053}}{\pm0.0060}$&${\textbf{0.0507}}{\pm0.0004}$\\
\addlinespace[0pt]\midrule[0.5pt]\addlinespace[0pt]
\multirow{2}{3.5em}{\centering 500}&Random&${0.6417}{\pm0.0135}$&${0.0460}{\pm0.0010}$\\\cline{2-4}
                &\mbox{BMAH}&${\textbf{0.7073}}{\pm0.0090}$&${\textbf{0.0509}}{\pm0.0007}$\\
 \addlinespace[0pt]\bottomrule[0.8pt]
\end{tabular}
%\end{center}
\end{table}

From Table~\ref{table:mnist-initsize}, we can easily see that \mbox{BMAH} is consistently better than \mbox{Random}. It is interesting to observe that the precision and recall of \mbox{BMAH} keep increasing as $|\mathcal{L}^0|$ increases, but those of \mbox{Random} first increase and then decrease. This is reasonable since \mbox{BMAH} can find informative points more accurately with more initially labeled points whereas the active selection procedure of \mbox{Random} is independent of $\mathcal{L}^{0}$.


%%%%%%%%%%%%%%%%%%%
\subsubsection{Varying batch size $M$}
\label{section:comp-batch-mnist}

To study the effect of the batch size on \mbox{BMAH} and \mbox{Random}, we conduct several experiments by varying $M$. We set $|\mathcal{L}^{0}|=100$ for both methods and $\lambda = 0.4$ for \mbox{BMAH}. The average results over 10 repeats, and the corresponding standard deviations, after selecting $10M$ points are reported in Table~\ref{table:mnist-batchsize}.

%, where again boldface numbers indicate better results.
\begin{table}[htb] 
\centering
\caption{Comparison of \mbox{BMAH} and \mbox{Random} with varying batch size $M$}
%\vspace{0.2cm}
\label{table:mnist-batchsize}
%\begin{center}
\begin{tabular}{|c|c|c|c|}
\toprule[0.8pt]\addlinespace[0pt]
$M$ & Method &  Precision &  Recall\\
\addlinespace[0pt]\midrule[0.8pt]\addlinespace[0pt]
\multirow{2}{3.5em}{\centering 50}&Random&$\textbf{0.6292}{\pm0.0026}$&$\textbf{0.0452}{\pm0.0002}$\\\cline{2-4}
&\mbox{BMAH}&${{0.6262}}{\pm0.0017}$&${{0.0450}}{\pm0.0001}$\\
\addlinespace[0pt]\midrule[0.5pt]\addlinespace[0pt]
\multirow{2}{3.5em}{\centering 100}&Random&${0.6416}{\pm0.0103}$&${0.0460}{\pm0.0007}$\\\cline{2-4}
&\mbox{BMAH}&${\textbf{0.6793}}{\pm0.0033}$&${\textbf{0.0489}}{\pm0.0002}$\\
\addlinespace[0pt]\midrule[0.5pt]\addlinespace[0pt]
\multirow{2}{3.5em}{\centering 150}&Random&${0.6392}{\pm0.0082}$&${0.0458}{\pm0.0006}$\\\cline{2-4}
&\mbox{BMAH}&${\textbf{0.6988}}{\pm0.0084}$&$\textbf{{0.0504}}{\pm0.0006}$\\
\addlinespace[0pt]\midrule[0.5pt]\addlinespace[0pt]
\multirow{2}{3.5em}{\centering 200}&Random&${0.6302}{\pm0.0130}$&${0.0452}{\pm0.0009}$\\\cline{2-4}
&\mbox{BMAH}&${\textbf{0.7068}}{\pm0.0046}$&${\textbf{0.0509}}{\pm0.0003}$\\
 \addlinespace[0pt]\bottomrule[0.8pt]
\end{tabular}
%\end{center}
\end{table}

From Table~\ref{table:mnist-batchsize}, it is easy to see that a larger value of $M$ also induces a larger performance gap between \mbox{BMAH} and \mbox{Random}. This is quite reasonable because larger $M$ implies that 1) \mbox{BMAH} will select more informative points than \mbox{Random}; 2) the limitation of passive hashing will be reduced more by \mbox{BMAH}.

%%%%%%%%%%%%%%%%%%%%%%%%%%
\subsubsection{Varying balancing parameter $\lambda$}
\label{section:comp-lambda-mnist}
In Problem~(\ref{equation:ah:relax}), the balancing parameter $\lambda$ controls the balance between selecting the most uncertain points and reducing the redundancy among the selected data points. In this last group of experiments, we study the effect of $\lambda$ by varying its value from $0$ to $1$. For each value of $\lambda$, the result is averaged over 10 random repeats. We plot learning curves \wrt precision and recall in Fig.~\ref{fig:apt-largemnist-24b-qp}. Note that the value of $\lambda$ is indicated by the superscript of \mbox{BMAH}.

%In following experiments,. The whole procedure is repeated 10 times, and the average results and standard deviation after selecting $1,000$ points are reported.

\begin{figure}[htb]
%\vspace{-1.5cm}
%\centering
\subfigure[Learning curves \wrt precision]{
    \begin{minipage}[b]{0.48\linewidth}
%        \centering
        \epsfig{figure=fig/ah/activepoint-largemnist-precision500-24b-qp5, width=1.0\textwidth}%\vspace{-2cm}
        \label{fig:apt-largemnist-precision-24b-qp}
    \end{minipage}}
\subfigure[Learning curves \wrt recall]{
    \begin{minipage}[b]{0.48\linewidth}
%        \centering
        \epsfig{figure=fig/ah/activepoint-largemnist-recall500-24b-qp5, width=1.0\textwidth}%\vspace{-2cm}
        \label{fig:apt-largemnist-recall-24b-qp}
    \end{minipage}}
\caption{Learning curves of \mbox{BMAH} with varying balancing parameter $\lambda$ for image retrieval. The parameters of \mbox{BMAH} are set as $|\mathcal{L}^{0}|=100$ and $M = 100$.}
\label{fig:apt-largemnist-24b-qp}
\end{figure}

From Fig.~\ref{fig:apt-largemnist-24b-qp}, we observe that \mbox{BMAH} is not very sensitive to $\lambda$ and $\lambda=0.4$ achieves the best performance after 30 iterations. We believe that  there is little redundancy among the most uncertain data points, so varying $\lambda$ in such a small range does not have much effect on the model performance.
%%%%%%%%%%%%%%%%%%%%%%%%%%
\subsubsection{Varying code length $K$}

We have observed that \mbox{BMAH} is very effective with a fixed code length, i.e., $ K=24 $. But is it also effective when we set the code length $ K $ to other values? To answer this question, we conducted a group of experiments by running \mbox{BMAH} with different $ K $ values. The learning curves of \mbox{BMAH} averaged over 10 random repeats are plotted in Fig.~\ref{fig:apt-largemnist-fullb}. %Our expectation is \mbox{BMAH} outperforms \mbox{Random} in all code lengths.

\begin{figure}[htb]
%\vspace{-1.5cm}
%\centering
\subfigure[Learning curves \wrt precision]{
    \begin{minipage}[b]{0.48\linewidth}
%        \centering
        \epsfig{figure=fig/ah/activepoint-largemnist-precision500-fullb5, width=1.0\textwidth}%\vspace{-2cm}
        %\label{fig:apt-largemnist-precision-12b-qp}
    \end{minipage}}
\subfigure[Learning curves \wrt recall]{
    \begin{minipage}[b]{0.48\linewidth}
%        \centering
        \epsfig{figure=fig/ah/activepoint-largemnist-recall500-fullb5, width=1.0\textwidth}%\vspace{-2cm}
        %\label{fig:apt-largemnist-recall-12b-qp}
    \end{minipage}}
\caption{Learning curves of \mbox{BMAH} with varying code length $K$ for text retrieval. The parameters of \mbox{BMAH} are set as $\lambda=0.4,|\mathcal{L}^{0}|=100$ and $M = 100$.}
\label{fig:apt-largemnist-fullb}
\end{figure}

From Fig.~\ref{fig:apt-largemnist-fullb}, we observe that larger code length gives better performance and, for all three different code lengths, \mbox{BMAH} consistently improves as more labeled data are added. The results validate the effectiveness of \mbox{BMAH}.

%%%%%%%%%%%%%%%%%%%%%%%%%%%%%%%%%%%%%%%
\subsection{Text Retrieval}

\subsubsection{Data set}

The 20 Newsgroups (\mbox{NEWS}) data set\footnote{http://people.csail.mit.edu/jrennie/20Newsgroups/} was originally collected for document classification. We use the popular `bydate' version which contains $18{,}846$ documents evenly distributed across 20 categories. Each document is labeled by exactly one of the 20 labels. In our experiments, we randomly select $1{,}000$ documents as the test set and use the remaining documents as the training set. The original data set contains $26{,}214$ features generated by the \emph{tf$\cdot$idf} scheme~\cite{salton1988ipm}. In our experiments, we extract $1{,}000$ features by applying principal component analysis (\mbox{PCA}).

%%%%%%%%%%%%%%%%%%%
\subsubsection{Comparison of \mbox{BMAH} and \mbox{GAH}}
\label{section:comp-bmah-gah-news}

The first group of experiments is to compare \mbox{BMAH} and \mbox{GAH}. We randomly choose 50 points and their labels to form $\mathcal{L}^0$ and use the remaining data as $\mathcal{U}^0$. Moreover, we set $\lambda=0.4$ for \mbox{BMAH}. The procedure is repeated 10 times and the average results after selecting $M$ points are reported in Table~\ref{table:news-greedy}. The results in Table~\ref{table:news-greedy} are very similar to those in Table~\ref{table:mnist-greedy}, and for the same reason, we omit \mbox{GAH} in the following experiments.

%\vspace{-0.1cm}
 \begin{table}[htb]
 \centering
\caption{Comparison of \mbox{BMAH} and \mbox{GAH} for text retrieval}
%\vspace{-0.1cm}
\label{table:news-greedy}
%\begin{center}
\begin{tabular}{|c|c|c|c|c|}
\toprule[0.8pt]\addlinespace[0pt]
$M$ & Method &  Precision &  Recall& Time Cost (seconds)\\
\addlinespace[0pt]\midrule[0.8pt]\addlinespace[0pt]
\multirow{2}{1.5em}{\centering 50}&\mbox{BMAH}&${0.3373}{\pm0.0012}$&${0.1857}{\pm0.0006}$&67.44\\\cline{2-5}
&\mbox{GAH}&${0.3375}{\pm0.0011}$&${0.1858}{\pm0.0006}$&473.28\\
\addlinespace[0pt]\midrule[0.5pt]\addlinespace[0pt]
\multirow{2}{1.5em}{\centering 100}&\mbox{BMAH}&${0.3377}{\pm0.0022}$&${0.1859}{\pm0.0012}$&79.88\\\cline{2-5}
&\mbox{GAH}&${0.3374}{\pm0.0017}$&${0.1857}{\pm0.0009}$&958.04\\
\addlinespace[0pt]\midrule[0.5pt]\addlinespace[0pt]
\multirow{2}{1.5em}{\centering 150}&\mbox{BMAH}&${0.3441}{\pm0.0043}$&${0.1894}{\pm0.0023}$&219.34\\\cline{2-5}
&\mbox{GAH}&${0.3420}{\pm0.0033}$&${0.1882}{\pm0.0018}$&2058.31\\
\addlinespace[0pt]\midrule[0.5pt]\addlinespace[0pt]
\multirow{2}{1.5em}{\centering 200}&\mbox{BMAH}&${0.3520}{\pm0.0043}$&${0.1935}{\pm0.0023}$&469.08\\\cline{2-5}
&\mbox{GAH}&${0.3492}{\pm0.0033}$&${0.1921}{\pm0.0018}$&2612.79\\
 \addlinespace[0pt]\bottomrule[0.8pt]
\end{tabular}
%\end{center}
\end{table}



%%%%%%%%%%%%%%%%%%%
\subsubsection{Comparison of \mbox{BMAH} and \mbox{Random}}

We compare \mbox{BMAH} with \mbox{Random} in this section. Similarly, we set $|\mathcal{L}^{0}|=50$ for both methods and $M = 50$ and $\lambda = 0.4$ for \mbox{BMAH}. We plot the results averaged over 10 random repeats and their standard deviations in Fig.~\ref{fig:apt-news-24b}.

\begin{figure}[htb]
\vspace{-1.5cm}
\subfigure[Learning curves \wrt precision]{
    \begin{minipage}[b]{0.48\linewidth}
%        \centering
        \epsfig{figure=fig/ah/activepoint-news-precision500-24b5, width=1.0\textwidth}\vspace{-2cm}
        \label{fig:apt-news-precision-24b}
    \end{minipage}}
\subfigure[Learning curves \wrt recall]{
\begin{minipage}[b]{0.48\linewidth}
%    \centering
    \epsfig{figure=fig/ah/activepoint-news-recall500-24b5, width=1.0\textwidth}\vspace{-2cm}
    \label{fig:apt-news-recall-24b}
\end{minipage}}
\caption{Learning curves of \mbox{BMAH} and \mbox{Random} for text retrieval}
\label{fig:apt-news-24b}
\end{figure}

From the figures, we can easily see that the performance of \mbox{Random} is better than that of \mbox{BMAH} at the early stage but degrades very fast afterwards. This is reasonable because there are 20 categories in the data set and initially the selected points can only cover a small portion of them. In the beginning, \mbox{Random} uniformly selects points from all the categories but \mbox{BMAH} focuses on those points that are uncertain with respect to the current hash functions and hence might be biased. However, this kind of bias will be corrected when more labeled data points are added. Thus, the precision of \mbox{BMAH} continues to increase at a later stage but that of \mbox{Random} drops due to the limitation of passive hashing.


%%%%%%%%%%%%%%%%%%%%%%%%%%%
\subsubsection{Varying initial label size $|\mathcal{L}^{0}|$}
\label{section:comp-initsize-news}
In this group of experiments, we vary the initial label size $|\mathcal{L}^0|$ to study its effect on \mbox{BMAH} and \mbox{Random}. The parameters are set to $M=50$ and $\lambda = 0.4$. The whole procedure is repeated 10 times and the average results, together with standard deviations, after selecting $30M$ points are reported in Table~\ref{table:news-initsize}.

\begin{table}[htb]
\centering
\caption{Comparison of \mbox{BMAH} and \mbox{Random} with varying initial label size $|\mathcal{L}^{0}|$}
%\vspace{0.2cm}
\label{table:news-initsize}
%\begin{center}
\begin{tabular}{|c|c|c|c|}
\toprule[0.8pt]\addlinespace[0pt]
$|\mathcal{L}^{0}|$ & Method &  Precision &  Recall\\
\addlinespace[0pt]\midrule[0.8pt]\addlinespace[0pt]
\multirow{2}{3.5em}{\centering 50}&Random&${0.3765}{\pm0.0188}$&${0.2081}{\pm0.0103}$\\\cline{2-4}
&\mbox{BMAH}&${\textbf{0.4993}}{\pm0.0112}$&${\textbf{0.2752}}{\pm0.0062}$\\
\addlinespace[0pt]\midrule[0.5pt]\addlinespace[0pt]
\multirow{2}{3.5em}{\centering100}&Random&${0.3731}{\pm0.0250}$&${0.2063}{\pm0.0137}$\\\cline{2-4}
&\mbox{BMAH}&${\textbf{0.4961}}{\pm0.0140}$&${\textbf{0.2737}}{\pm0.0076}$\\
\addlinespace[0pt]\midrule[0.5pt]\addlinespace[0pt]
\multirow{2}{3.5em}{\centering150}&Random&${0.3726}{\pm0.0155}$&${0.2060}{\pm0.0083}$\\\cline{2-4}
&\mbox{BMAH}&${\textbf{0.4945}}{\pm0.0124}$&${\textbf{0.2730}}{\pm0.0065}$\\
\addlinespace[0pt]\midrule[0.5pt]\addlinespace[0pt]
\multirow{2}{3.5em}{\centering200}&Random&${0.3620}{\pm0.0147}$&${0.2002}{\pm0.0078}$\\\cline{2-4}
&\mbox{BMAH}&${\textbf{0.5068}}{\pm0.0140}$&${\textbf{0.2796}}{\pm0.0076}$\\
\addlinespace[0pt]\midrule[0.5pt]\addlinespace[0pt]
\multirow{2}{3.5em}{\centering250}&Random&${0.3666}{\pm0.0144}$&${0.2026}{\pm0.0079}$\\\cline{2-4}
&\mbox{BMAH}&${\textbf{0.5064}}{\pm0.0148}$&${\textbf{0.2792}}{\pm0.0080}$\\
 \addlinespace[0pt]\bottomrule[0.8pt]
\end{tabular}
%\end{center}
\end{table}

In Table~\ref{table:news-initsize}, \mbox{BMAH} consistently outperforms \mbox{Random}. It is interesting to observe that the precision and recall of \mbox{BMAH} keep increasing as $|\mathcal{L}^0|$ increases, but those of \mbox{Random} first increase and then drop. This can be explained by the same reason in Section~\ref{section:comp-initsize-mnist}.

%%%%%%%%%%%%%%%%%%%%%%%%%%%%%%
\subsubsection{Varying batch size $M$}
\label{section:comp-batch-news}

In this section, we evaluate the effect of the batch size $M$ on \mbox{BMAH} and \mbox{Random}. We set $|\mathcal{L}^{0}|=50$ for both methods and $\lambda = 0.4$ for \mbox{BMAH}. The whole process is repeated 10 times, and the average results and standard deviations, after selecting $20M$ points, are reported in Table~\ref{table:news-batchsize}.

\begin{table}[htb]
\centering
\caption{Comparison of \mbox{BMAH} and \mbox{Random} with varying batch size $M$}
%\vspace{0.2cm}
\label{table:news-batchsize}
%\begin{center}
\begin{tabular}{|c|c|c|c|}
\toprule[0.8pt]\addlinespace[0pt]
$M$ & Method &  Precision &  Recall\\
\addlinespace[0pt]\midrule[0.8pt]\addlinespace[0pt]
\multirow{2}{3.5em}{\centering 50}&Random&${0.4302}{\pm0.0201}$&${0.2374}{\pm0.0106}$\\\cline{2-4}
&\mbox{BMAH}&${\textbf{0.4399}}{\pm0.0121}$&${\textbf{0.2426}}{\pm0.0065}$\\
\addlinespace[0pt]\midrule[0.5pt]\addlinespace[0pt]
\multirow{2}{3.5em}{\centering 100}&Random&${0.3432}{\pm0.0173}$&${0.1900}{\pm0.0093}$\\\cline{2-4}
&\mbox{BMAH}&${\textbf{0.5001}}{\pm0.0220}$&${\textbf{0.2766}}{\pm0.0119}$\\
\addlinespace[0pt]\midrule[0.5pt]\addlinespace[0pt]
\multirow{2}{3.5em}{\centering 150}&Random&${0.2983}{\pm0.0170}$&${0.1655}{\pm0.0091}$\\\cline{2-4}
&\mbox{BMAH}&${\textbf{0.4076}}{\pm0.0239}$&${\textbf{0.2248}}{\pm0.0131}$\\
\addlinespace[0pt]\midrule[0.5pt]\addlinespace[0pt]
\multirow{2}{3.5em}{\centering 200}&Random&${0.2718}{\pm0.0132}$&${0.1511}{\pm0.0071}$\\\cline{2-4}
&\mbox{BMAH}&${\textbf{0.3266}}{\pm0.0267}$&${\textbf{0.1806}}{\pm0.0148}$\\
 \addlinespace[0pt]\bottomrule[0.8pt]
\end{tabular}
%\end{center}
\end{table}

From Table~\ref{table:news-batchsize}, we see that increasing $M$ will degrade the performance of both \mbox{BMAH} and \mbox{Random}. This observation is different from what we have observed in Section~\ref{section:comp-batch-mnist}. In this data set, the limitation of passive hashing is much more severe since there are 20 categories. Hence \mbox{BMAH} cannot completely eliminate the limitation as in the previous task. However, the performance drop of \mbox{BMAH} is much slower than that of \mbox{Random}. This still demonstrates that \mbox{BMAH} is capable of overcoming the limitation of passive hashing slightly.

%%%%%%%%%%%%%%%%%%%
\subsubsection{Varying balancing parameter $\lambda$}

To study the sensitivity of \mbox{BMAH} to the balancing parameter $\lambda$, we conduct some experiments by varying $\lambda$ from $0$ to $1$. The learning curves averaged over 10 random repeats are plotted in Fig.~\ref{fig:apt-news-24b-qp}. Similar to before, we use superscripts to indicate the values of $\lambda$.

\begin{figure}[htb]
%\vspace{-1.5cm}
\subfigure[Learning curves \wrt precision]{
    \begin{minipage}[b]{0.48\linewidth}
%        \centering
        \epsfig{figure=fig/ah/activepoint-news-precision500-24b-qp5, width=1.0\textwidth} %\vspace{-2cm}
        \label{apt-news-precision-24b-qp}
    \end{minipage}}
\subfigure[Learning curves \wrt recall]{
    \begin{minipage}[b]{0.48\linewidth}
%        \centering
        \epsfig{figure=fig/ah/activepoint-news-recall500-24b-qp5, width=1.0\textwidth} %\vspace{-2cm}
        \label{fig:apt-news-recall-24b-qp}
    \end{minipage}}
\caption{Learning curves of \mbox{BMAH} with varying balancing parameter $\lambda$ for text retrieval. The parameters of \mbox{BMAH} are set as $|\mathcal{L}^{0}|=50$ and $M = 50$.}
\label{fig:apt-news-24b-qp}
\end{figure}

From Fig.~\ref{fig:apt-news-24b-qp}, we see that \mbox{BMAH} is not very sensitive to $\lambda$, and $\lambda=0.4$ achieves the best performance after 30 iterations. This again can be explained by the same reason in Section~\ref{section:comp-lambda-mnist}.

%%%%%%%%%%%%%%%%%%%%%%%%%%
\subsubsection{Varying code length $K$}

In this last subsection, we study the performance of \mbox{BMAH} by varying the code length $ K $. The learning curves of \mbox{BMAH}, averaged over 10 random repeats, with different $ K $ are plotted in Fig.~\ref{fig:apt-news-fullb}. %Our expectation is \mbox{BMAH} outperforms \mbox{Random} in all code lengths.

\begin{figure}[htb]
%\vspace{-1.5cm}
%\centering
\subfigure[Learning curves \wrt precision]{
    \begin{minipage}[b]{0.48\linewidth}
%        \centering
        \epsfig{figure=fig/ah/activepoint-news-precision500-fullb5, width=1.0\textwidth}%\vspace{-2cm}
        \label{fig:apt-largemnist-precision-12b-qp}
    \end{minipage}}
\subfigure[Learning curves \wrt recall]{
    \begin{minipage}[b]{0.48\linewidth}
%        \centering
        \epsfig{figure=fig/ah/activepoint-news-recall500-fullb5, width=1.0\textwidth}%\vspace{-2cm}
        \label{fig:apt-largemnist-recall-12b-qp}
    \end{minipage}}
\caption{Learning curves of \mbox{BMAH} with varying code length $K$ for text retrieval. The parameters of \mbox{BMAH} are set as $\lambda=0.4,|\mathcal{L}^{0}|=50$ and $M = 50$.}
\label{fig:apt-news-fullb}
\end{figure}

From Fig.~\ref{fig:apt-news-fullb}, we observe that the performance of \mbox{BMAH} improves as more labeled data are incorporated no matter how long the hash codes are. This again validates the effectiveness of \mbox{BMAH}.

 % % % % % % % % % % % % % % % % % % % % % % % % % % % % % %
\section{Conclusion}
\label{AH:conclusion}
To summarize, we have proposed a new hashing framework to actively learn hash functions from both unlabeled and labeled data. We have utilized the uncertainty criterion, which is commonly used in many active learning methods, to give a simple and efficient active hashing algorithm. Experimental results show that our framework can effectively identify the most informative points for the expert to label and can overcome the limitations of existing supervised \mbox{HFL} methods. 

Though being effective and easy to implement, the uncertainty-based active hashing method does not directly optimize the quality of the learned hash functions. As such, to take this work further, we plan to explore other criteria of informativeness for active hashing. For example, we may select the points which maximize some information gain or minimize some expected loss functions which are directly related to the quality of hash functions. Moreover, although we have demonstrated that it is very effective to select points to label first and then construct point pairs, it might be better to select point pairs directly to make it computationally more efficient. Last but not least, another possible research direction is to explore other batch mode algorithms for active hashing.

%some other informativeness criteria which are more closely related to hash function quality are worthy of investigation.


\chapter{Multimodal Hashing for Aligned Data}
\label{chap:smh}

 % % % % % % % % % % % % % % % % % % % % % % % % % % % % % %
\section{Introduction}

As of now, almost all existing \mbox{HFL} methods assume that the data are unimodal, meaning that both the queries and the candidates are of the same modality. They cannot be adapted easily for multimodal search which is often encountered in many multimedia retrieval, image analysis and data mining applications. Take crossmodal multimedia retrieval for example, using an image about a historic event as query, one may want to retrieve relevant text articles that can provide more detailed information about the event. Obviously, existing unimodal methods cannot be applied directly to multimodal similarity search because they assume that all the data are of the same modality.  Designing \mbox{HFL} methods for multimodal data is thus a very worthwhile direction to pursue.

Recently, \mbox{Bronstein} \etal~\cite{bronstein2010cvpr} proposed a general framework which is referred to as \textit{multimodal hashing} (\mbox{MH}). As illustrated in Figure~\ref{smh:fig:framework} for the bimodal case, \mbox{MH} functions hash documents of different modalities into a common Hamming space so that fast similarity search can be performed. The key challenge of \mbox{MH} is to learn effective hash functions for multiple modalities efficiently from the provided information.

\begin{figure}[htb]
\centering
\epsfig{figure=fig/mh_illustration, width=0.7\textwidth}
\caption{Illustration of the multimodal hashing framework. Under this framework, similar documents (with bounding boxes of the same color) of different modalities are hashed to nearby points in the Hamming space whereas dissimilar documents (with bounding boxes of different colors) are hashed to points far apart.}
\label{smh:fig:framework}
\end{figure}

In this chapter, we study a simple case of multimodal hashing, in which the data from different modalities are aligned. For example, suppose there are two modalities, i.e., image and text, if each image has been aligned with one and only one text article and vice versa, we consider the data to be aligned. The alignment can be determined by applications at hand, e.g, an image and a text can be paired if they are referring to the same object. To learn \mbox{MH} functions for these data, we first give a basic model which learns hash functions through spectral analysis of the correlation between modalities. Besides the basic method which can only handle vectorial data and is linear, we provide a kernel extension to handle nonvectorial data and incorporate nonlinearity. We further incorporate \textit{Laplacian} regularization for situations in which side information is also available in the data.

%In the second method, the hash functions are learned by minimizing the multimodal reconstruction error. More specifically, given a collection of pairwise distance within each modality and pairwise relations across modalities, the model aims at minimizing the squared error between the pairwise reconstructive distance (normalized Hamming distance) computed based on the hash codes and the original distance as well as relations.

%The third method is based on latent factor models.

The rest of this chapter is organized as follows. In Section~\ref{smh:relatedwork}, we introduce some related work. We then present our model, \textit{spectral multimodal hashing}, in Section~\ref{smh:SMH}. Empirical studies conducted on real-world data sets are presented in Section~\ref{smh:exps}, before we conclude this chapter in Section~\ref{smh:conclusion}.

 % % % % % % % % % % % % % % % % % % % % % % % % % % % % % %
\section{Related Work}
\label{smh:relatedwork}



%\footnote{It is straightforward to extend the problem formulation to more than two modalities, but we focus on the bimodel case in this paper for notational simplicity.}
Under the framework of multimodal hashing, we first introduce a recent work called cross modal similarity sensitive hashing (\mbox{CMSSH})~\cite{bronstein2010cvpr}, which is, to the best of our knowledge, the first work on multimodal hashing. 

Suppose we have two sets of $N$ data points each from two modalities (\aka feature spaces), $\mathcal{X} = \{\x_i\in\mathbb{R}^{D_x}\}_{i=1}^{N}$ and $ \mathcal{Y} = \{\y_i\in\mathbb{R}^{D_y}\}_{i=1}^{N}$, and the corresponding points $(\x_i,\y_i)$ are paired.  For applications studied in this paper, a pair $(\x_i,\y_i)$ may represent a multimedia document where $\x_i$ is an image and $\y_i$ is the corresponding text article.  For notational convenience, we denote the data sets as matrices $\X\in\mathbb{R}^{D_x\times N}$ and $\Y\in\mathbb{R}^{D_y\times N}$ where each column corresponds to a data point. Without loss
of generality, we assume that $\X,\Y$ have been normalized to have zero mean.

CMSSH works as follows: Given a set of similar pairs $\{(\x_i,\y_i)\}$ and a set of dissimilar pairs $\{(\x_j,\y_j)\}$, where $\x\in\mathcal{X}$ and $\y\in\mathcal{Y}$ belong to two different modalities, \mbox{CMSSH} constructs two mapping functions $\xi:\mathcal{X}\rightarrow\mathbb{H}^{M}$ and $\eta:\mathcal{Y}\rightarrow\mathbb{H}^{M}$ such that, with high probability, the Hamming distance is small for similar points and large for dissimilar points. Specifically, the $m$th bit of Hamming representation $\mathbb{H}^{M}$ for $\mathcal{X}$ and $\mathcal{Y}$ can be defined by two functions, $\xi_ {m}$ and $\eta_{m}$, which are parameterized by projections $p_{m}$ and $q_{m}$, respectively. In their paper, $\xi_{m}$ and $\eta_{m}$ are assumed to have the form $\xi_{m}(\x) = \sgn(\p_{m}^{T}\x+a_{m})$ and $\eta_{m}(\y) =\sgn(\q_{m}^{T}\y+b_{m})$, where $\p_{m}$ and $\q_{m}$ are $D_{x}$- and $D_{y}$-dimensional unit vectors, $a_{m}$ and $b_{m}$ are scalars.

A method based on boosting is used to learn the mapping functions.  The algorithm is briefly described here.  First, it initializes the weight of each point pair to $w_{m}(k) = 1/K$ where $K$ is the total number of point pairs. Then, for the ${m}$th bit, it selects $\xi_{m}$ and $\eta_{m}$ that maximize the following objective function:
$$\sum\nolimits_{k=1}\nolimits^{K}\left(w_{m}(k)s_k\sgn(\p_{m}^{T}\x_k+a_{m})\sgn(\q_{m}^{T}\y_k+b_{m})\right),\nonumber$$
where $s_k=1$ if the $k$th pair is a similar pair and $s_k = -1$ otherwise. Since maximizing the objective function above is difficult, the $\sgn(\cdot)$ operator and bias terms $a_{m},b_{m}$ are dropped to give the following approximate objective function for maximization:
$$
\sum\nolimits_{k=1}\nolimits^{K}w_{m}(k)s_k(\p_{m}^{T}\x_k)(\q_{m}^{T}\y_k) = \p_{m}^{T}\left(\sum\nolimits_{k=1}\nolimits^{K}w_{m}(k)s_k\x_k\y_k^{T}\right)\q_{m}.
$$
It is easy to see that the $\p_{m}$ and $\q_{m}$ that maximize the above objective are the largest left and right singular vectors of $\C = \sum_{k=1}^{K}w_{m}(k)s_k\x_k\y_k^{T}$. After obtaining $\p_{m}$ and $\q_{m}$, the algorithm updates the weights with the update rule $w_{m+1}(k) = w_{m}(k)\exp(-s_{k}\xi_{m}(\x)\eta_{m}(\y))$ and then proceeds to learn $\p_{m+1},\q_{m+1}$ for the $(m+1)$st bit.

Roughly speaking, \mbox{CMSSH} tries to map similar points to similar codes and dissimilar points to different codes by exploiting pairwise relations across different modalities. However, it ignores relational information within the same modality which could be very useful for hash function learning~\cite{weiss2008nips,he2010kdd}. Furthermore, \mbox{CMSSH} can only handle vectorial data which might not be available in many applications.

Recently, Kumar \etal extended spectral hashing~\cite{weiss2008nips} to the multi-view case, leading to a method called cross-view hashing (\mbox{CVH})~\cite{kumar2011ijcai}. The objective of \mbox{CVH} is to minimize the inter-view and intra-view Hamming distances for similar points and maximize those for dissimilar points. The optimization problem is relaxed to several generalized eigenvalue problems which can be solved by off-the-shelf methods.

 % % % % % % % % % % % % % % % % % % % % % % % % % % % % % %
\section{Spectral Multimodal Hashing}
\label{smh:SMH}

In this section, we first formulate the multimodal hashing problem as a discrete embedding problem and show that it can be approximately solved by spectral decomposition followed by thresholding, which is similar to spectral hashing~\cite{weiss2008nips} for unimodal data. But unlike spectral hashing, our focus is multimodal data, which are often encountered in a vast range of multimedia applications. Therefore, we call the proposed method \textit{spectral multimodal hashing} (\mbox{SMH}). In the following, we first give a basic \mbox{SMH} model in Section~\ref{smh:Ssmh:FORMULATION} and then present the other two models as extensions in Section~\ref{smh:Ssmh:EXT}.

%present a basic multimodal hashing model in Section~\ref{sec:ssmh:smh}. In Section~\ref{sec:ssmh:ksmh}, we introduce its kernel extension (\mbox{KSMH}) which enables us to deal with nonvectorial data as well as nonlinearity.  At last, to exploit side information in the form of labels or pairwise similarity relations between points, we further extend \mbox{KSMH} by introducing two novel regularizers and propose a regularized \mbox{KSMH} (\mbox{RKSMH}) model in Section~\ref{sec:ssmh:rksmh}.


\subsection{Formulation}
\label{smh:Ssmh:FORMULATION}

Let there be two data matrices $\X^{D_x\times N}$ and $\Y^{D_y\times N}$ from different modalities and the corresponding points $(\x_i,\y_i)$ be paired. For applications studied in this paper, a pair $(\x_i,\y_i)$ may represent a multimedia document where $\x_i$ is an image and $\y_i$ is the corresponding text article. Without loss of generality, we assume that $\X,\Y$ have been normalized to have zero mean. We want to learn two sets of hash functions $\{h_{m}\}_{m=1}^{M}$ and $\{g_{m}\}_{m=1}^{M}$ to give $M$-bit binary codes of $\X$ and $\Y$, respectively. 

In this paper, we use thresholded linear projection to define the hash functions. More specifically, the $m$th hash functions for both modalities are defined as follows:
\begin{align}
h_{m}(\x)=\sgn(\x^T\w_{x}^{(m)}+t_x),\ \ %\mbox{or}
g_{m}(\y)=\sgn(\y^T\w_{y}^{(m)}+t_y)\nonumber,
\end{align}
where $\w_{x}^{(m)}\in\mathbb{R}^{D_{x}},\w_{y}^{(m)}\in\mathbb{R}^{D_{y}}$ correspond to two projection directions. The corresponding Hamming bits can be obtained as
\begin{align}
\label{eqn:bit}
b_{m}(\x) = \frac{1+h_{m}(\x)}{2}, \ \ 
%\mbox{or}
b_{m}(\y)= \frac{1+g_{m}(\y)}{2}.
\end{align}
%$(1+h_{m}(\x))/2$ and $(1+g_{m}(\y))/2$.

Let the binary vectors $\h(\x) = (h_{1}(\x),\dots,h_{M}(\x))^T$ and $\g(\y) = (g_{1}(\y)\dots,g_{M}(\y))^T$ denote the projections of points $\x$ and $\y$.  
The goal of our basic \mbox{SMH} model is to seek the projections that maximize the correlation between variables in the projected space (Hamming space). Intuitively, two hash codes are more correlated in the Hamming space if the corresponding points in the original space are similar and less correlated otherwise. Moreover, the hash codes should be balanced in the sense that each bit has equal chance of being 1 and $-1$ and the hash bits should be independent of each other~\cite{weiss2008nips}. As a result, \mbox{SMH} can be formulated as the following constrained optimization problem:
\begin{eqnarray}
\max_{\{\w_{x}^{(m)},\w_{y}^{(m)}\}_{m=1}^{M}}& \frac{\mathbb{E}(\h^{T}\g)}{\sqrt{\mathbb{E}(\h^{T}\h)\mathbb{E}(\g^{T}\g)}}\\
\subto&  \sum_{i=1}^N h_{m}(\x_i) =0, \ m=1,\dots,M\nonumber\\
&\sum_{i=1}^N g_{m}(\y_i) =0, \ m=1,\dots,M\nonumber\\
&\sum_{i=1}^{N}h_{m}(\x_i)h_{n}(\x_i) =0, \ \forall m\neq n\nonumber\\
&\sum_{i=1}^{N}g_{m}(\y_i)g_{n}(\y_i) =0, \ \forall m\neq n,\nonumber
\label{eqn:cmh1}
\end{eqnarray}
where the expectation is taken with respect to the data distribution in the corresponding feature space. This problem is difficult to solve even without the constraints since the objective function is non-differentiable. Moreover, the balancing constraints make the problem NP-hard~\cite{weiss2008nips}.

Similar to~\cite{wang2010cvpr}, we relax the problem by dropping the $\sgn(\cdot)$ operator, the thresholds $ t_x $ and $ t_y $ and the balancing constraints.  Instead, we implicitly enforce the constraints by preprocessing the data through mean-centering.  Hence we arrive at the following optimization problem for one bit:\footnote{For notational simplicity, we omit the indices of the hash functions.}
\begin{eqnarray}
\label{eqn:cca1}
\max_{\w_{x},\w_{y}}& \mathbb{E}(\w_{x}^{T}\x \, \w_{y}^{T}\y)\\
\subto&  \mathbb{E}((\w_{x}^{T}\x)^2)=1, \, \mathbb{E}((\w_{y}^{T}\y)^2)=1,\nonumber
\end{eqnarray}
which in fact is the standard form of \emph{canonical correlation analysis} (\mbox{CCA})~\cite{hotelling1936cca}. Approximating the expectation by empirical expectation, we rewrite Problem~(\ref{eqn:cca1}) as follows:
\begin{eqnarray}
\label{eqn:csmh:optprob}
\max_{\w_{x},\w_{y}}& \w_{x}\C_{xy}\w_{y}\\
\subto&  \w_{x}\C_{xx}\w_{x}=1, \, \w_{y}\C_{yy}\w_{y}=1, \nonumber
\end{eqnarray}
where $\C_{xy} = \frac{1}{N}\X\Y^{T}$, $\C_{xx} = \frac{1}{N}\X\X^{T}$, and $\C_{yy} = \frac{1}{N}\Y\Y^{T}$.

This problem is equivalent to the following generalized eigenvalue problem:
\begin{align}
\label{eqn:csmh:wx}
\C_{xy}\C_{yy}^{-1}\C_{xy}^{T}\w_{x} = \lambda^2\C_{xx}\w_{x}.
\end{align}
The solution $\w_{x}$ is the eigenvector that corresponds to the largest eigenvalue.  With the $\w_{x}$ thus computed, we can compute $\w_{y}$ as
\begin{align}
\label{eqn:csmh:wy}
\w_{y} = \frac{1}{\lambda}\C_{yy}^{-1}\C_{yx}\w_{x}.
\end{align}

With projection vectors $ \w_x $ and $ \w_y $ computed, one common approach of getting the binary codes is simply using the $ \sgn(\cdot) $ operator. However, this may separate the points located near the boundary, impairing the model especially when the data distribution is dense in that area. To overcome this shortcoming, we use two thresholds, a fixed threshold of zero and a learned threshold, to get the binary codes.

The learning-based threshold can be obtained as follows. For each projection, we first divide the range of projected values into $ N_b $ bins and then calculate the relative data density of each bin as $ P_{t} = N_t/N, t=1,\dots,N_b, $ with $ N_t $ stands for the number of points in the $ t $th bin. The cost of cutting the $ t $th bin is defined as follows,
\begin{align}
C_t = \left(\sum\nolimits_{\hat{t}=1}^{t-1}P_{\hat{t}}\right)^2 +   \left(\sum\nolimits_{\hat{t}=t+1}^{N_b}P_{\hat{t}}\right)^2 + P_t,\nonumber
\end{align}
which measures the relative density of the $ t $th bin and the relative density of its both sides. Intuitively, if $ C_{t} $ is small, a boundary cutting through the $ t $th bin will separate a sparse area and make the points located evenly at its both sides. Actually, $ C_t $ is an adapted surrogate of the average size of a proper hash bucket which should be as small as possible for nearest neighbor search~\cite{cayton2007nips}. We then use the center of the bin with the smallest $ C_t $ as the threshold and denote it as $ t_x $ or $ t_y $.

Now we are ready to generate binary bits with 0 and $ t_x $. For example, given $ \w_x $ and $ \x^{*} $, we have
\begin{align}
\label{eqn:csmh:hg1}
h_1(\x^{*}) = \sgn(\w_{x}^{T}\x^{*}), \ \ h_2(\x^{*}) = \sgn(\w_{x}^{T}\x^{*}-t_x),
\end{align}
and given $ \w_y $ and $ \y^{*} $, we have
\begin{align}
\label{eqn:csmh:hg2}
g_1(\y^{*}) = \sgn(\w_{y}^{T}\y^{*}), \ \ g_2(\y^{*}) = \sgn(\w_{y}^{T}\y^{*}-t_y).
\end{align}



%We adopt a alternating algorithm to find the best threshold. First initialize $t_y=0$, we change the value of $t_x$ gradually from the centers.
%
%We first sort the values $\w_x^T\x $, then we get a vector of threshold values $[\w_x^T\x_1-0.1,(\w_x^T\x_1+\w_x^T\x_2)/2, (\w_x^T\x_2+\w_x^T\x_3)/2, \dots, (\w_x^T\x_{n-1}+\w_x^T\x_n)/2, \w_x^T\x_n+0.001]$. For the first threshold, the initial precision is evaluated. For a new threshold $t_x^{*}$, only check those affected point pairs with one end point $\x_{*}$, the precision can be changed to $\frac{total \# of point pairs * previous precision +2(\# of correct pair-\# of incorrect pair)}{total \# of point pairs}$. Given each point is involved in only a small number of points, the algorithm can be very efficient with complexity $O(Nd+P)$, where $P$ is the total number of pairs and $d$ is the average number of pairs a point is involved in.

%Another extension is to generate multiple bits using one eigenvector. One bit uses threshold 0 and the other uses the threshold learned. This combination is also possible and might be useful.
%
%CCA may also depend on the first few eigenvectors. Let's try.
%
%We can compare these two approaches to see which is better.






The basic \mbox{SMH} algorithm is summarized in Algorithm~\ref{algorithm:cmh}.

\begin{algorithm}[ht]
\caption{Algorithm of \mbox{SMH}}
\label{algorithm:cmh}
\begin{algorithmic}
\STATE {\bfseries Input:} \\
$\X$, $\Y$ -- data matrices
\\ $M$ -- number of hash functions
\STATE {\bfseries Procedure:}
\STATE Compute $\C_{xx},\C_{xy},\C_{yy}$.
\STATE Obtain $M$ eigenvectors corresponding to the $M$ largest eigenvalues of the generalized eigenvalue problem~(\ref{eqn:csmh:wx}) as $\w_{x}$'s.
 \STATE Obtain the corresponding $\w_{y}$'s using Equation~(\ref{eqn:csmh:wy}).
 \STATE Learn thresholds $ t_x $ and $ t_y $.
 \STATE Obtain the hash codes of points $\x^{*}$ and $\y^{*}$ using Equations~(\ref{eqn:csmh:hg1}),~(\ref{eqn:csmh:hg2}) \& (\ref{eqn:bit}).


\end{algorithmic}
\end{algorithm}

%%%%%%%%%%%%%%%%%%%%%%%%%%%%%%%%%%%%%%%
\subsection{Extensions}
\label{smh:Ssmh:EXT}

%%%%%%%%%%%%%%%%%%%%%%%%%%%%%%%%%%%%%%%
\subsubsection{Kernel \mbox{SMH}}
\label{smh:Ssmh:EXT:KSMH}

The \mbox{SMH} model presented in the previous subsection has two limitations.  First, it can only handle vectorial data.  Second, the projection before thresholding is linear.  In this subsection, we propose a kernel extension of \mbox{SMH}, abbreviated as \mbox{KSMH} thereafter, to overcome these limitations. % by taking the kernel approach~\cite{shawe2004book}.

Let $\mathcal{K}(\cdot,\cdot)$ be a valid kernel function and $\phi(\cdot)$ be the corresponding function that maps data points in the original input space to the kernel-induced feature space. In the sequel, we use $\Ph(\X) = [\phi(\x_1),\dots,\phi(\x_{N})]$ and $\Ph(\Y) = [\phi(\y_1),\dots,\phi(\y_{N})]$ to denote the data matrices in the kernel-induced feature space.

%Suppose we have a set of $P$ landmark points\footnote{*** You should explain what landmark points are and why they are needed.} $\hat{\mathcal{X}} \subset\mathcal{X}$, which can also be represented by $\hat{\X}$ in the original input space and $\Ph(\hat{\X})$ in the kernel-induced feature space.  For the other modality, we define $\hat{\mathcal{Y}},\hat{\Y}$ and $\Ph(\hat{\Y})$ similarly.

Taking the kernel approach~\cite{scholkopf2001colt}\cite{kulis2009nips}, we represent $\w_{x}$ and $\w_{y}$ as linear combinations of two groups of landmark points in the kernel-induced feature space, i.e.,
\begin{align}
\w_{x} = \Ph(\hat{\X})^{T}\alpp, \ \
\w_{y} = \Ph(\hat{\Y})^{T}\bett,\nonumber
\end{align}
where $ \hat{\X}\in\mathbb{R}^{D_x\times P}$ and $\hat{\X}\in\mathbb{R}^{D_x\times P} $ are two landmark sets, in which the points are randomly chosen from $ \X $ and $ \Y  $, respectively. We note that although the landmark points should be sampled from the corresponding data distribution and be sufficiently representative, it is enough in practice to select the landmarks randomly from the training set. $\alpp\in\mathbb{R}^{P\times 1},\bett\in\mathbb{R}^{P\times 1}$ are combination coefficients. To reduce the computational cost, $P$ is usually a small number compared to $N$.

The objective function of Problem~(\ref{eqn:cmh1}) can now be rewritten as
\begin{align}
\frac{\alpp^{T}\K_{\hat{x}x}\K_{y\hat{y}}\bett}{\sqrt{\alpp^{T}\K_{\hat{x}x}\K_{x\hat{x}}\alpp\bett^{T}\K_{\hat{y}y}\K_{y\hat{y}}\bett}},
\label{eqn:kcsmh:obj1}
\end{align}
where $\K_{\hat{x}x} = \K_{x\hat{x}}^{T} = \Ph(\hat{\X})^{T}\Ph(\X)$ and $\K_{\hat{y}y} = \K_{y\hat{y}}^{T} = \Ph(\hat{\Y})^{T}\Ph(\Y)$.

Since the objective function above can lead to degenerate solutions as discussed in~\cite{hardoon2004nc}, we penalize the norms of $\w_{x}$ and $\w_{y}$ in the denominator of (\ref{eqn:kcsmh:obj1}) and arrive at the following alternative form:
\begin{align}
\frac{\alpp^{T}\K_{\hat{x}x}\K_{y\hat{y}}\bett}{\sqrt{\alpp^{T}(\K_{\hat{x}x}\K_{x\hat{x}}+\kappa\K_{\hat{x}\hat{x}})\alpp\bett^{T}(\K_{\hat{y}y}\K_{y\hat{y}}+\kappa\K_{\hat{y}\hat{y}})\bett}},
\label{eqn:kcsmh:obj2}
\end{align}
where $\K_{\hat{x}\hat{x}} = \Ph(\hat{\X})^{T}\Ph(\hat{\X}),\K_{\hat{y}y}= \Ph(\hat{\Y})^{T}\Ph(\hat{\Y})$ and $\kappa>0$ is a regularization parameter.
%
%We are now ready to formulate \mbox{KSMH} as the following constrained optimization problem:
%\begin{eqnarray}
%\max_{\alpp,\bett}& \alpp^{T}\K_{\hat{x}x}\K_{y\hat{y}}\bett\\
%\subto&  \alpp^{T}(\K_{\hat{x}x}\K_{x\hat{x}}+\kappa\K_{\hat{x}\hat{x}})\alpp=1\nonumber\\
%&  \bett^{T}(\K_{\hat{y}y}\K_{y\hat{y}}+\kappa\K_{\hat{y}\hat{y}})\bett=1.\nonumber
%\end{eqnarray}

After some simple relaxations and manipulations similar to \mbox{SMH}, $\alpp$ can be obtained by solving the following generalized eigenvalue problem:
\begin{align}
\label{eqn:kcsmh:alpha}
\K_{\hat{x}x}\K_{y\hat{y}}(\K_{\hat{y}y}\K_{y\hat{y}} + \kappa\K_{\hat{y}\hat{y}})^{-1}\K_{\hat{y}y}\K_{x\hat{x}}\alpp =\lambda^2(\K_{\hat{x}x}\K_{x\hat{x}} + \kappa\K_{\hat{x}\hat{x}})\alpp.
\end{align}
After obtaining $\alpp$, we compute
\begin{align}
\label{eqn:kcsmh:beta}
\bett =\frac{1}{\lambda} (\K_{\hat{y}y}\K_{y\hat{y}}+\kappa\K_{\hat{y}\hat{y}})^{-1}\K_{\hat{y}y}\K_{x\hat{x}}\alpp.
\end{align}

We can also learn the thresholds $ t_x $ and $ t_y $ using the same approach presented in the last section.  For any new point $\x^{*}$, two bits of binary code can be obtained as
\begin{align}
\label{eqn:kcsmh:hg1}
h_1(\x^{*}) = \sgn(\k_{x^{*}}^{T}\alpp), \ \ h_2(\x^{*}) = \sgn(\k_{x^{*}}^{T}\alpp-t_x)
\end{align}
and for $ \y^{*} $, we have
\begin{align}
\label{eqn:kcsmh:hg2}
g_1(\y^{*}) = \sgn(\k_{y^{*}}^{T}\bett), \ \ g_2(\y^{*}) = \sgn(\k_{y^{*}}^{T}\bett-t_y),
\end{align}
where $\k_{y^{*}} = \Ph(\hat{\X})^{T}\phi(\x^{*})$ and $\k_{y^{*}} = \Ph(\hat{\Y})^{T}\phi(\y^{*})$.

The algorithm of \mbox{KSMH} is summarized in Algorithm~\ref{algorithm:kcmh}.

\begin{algorithm}[ht]
\caption{Algorithm of \mbox{KSMH}}
\label{algorithm:kcmh}
\begin{algorithmic}
\STATE {\bfseries Input:} \\
$\X$, $\Y$ -- data matrices
\\$\mathcal{K}(\cdot,\cdot)$ -- kernel function
\\ $M$ -- number of hash functions
\\ $\kappa$ -- regularization parameter
\STATE {\bfseries Procedure:} \\

   \STATE Compute $\K_{\hat{x}x}, \K_{\hat{y}y}, \K_{\hat{x}\hat{x}}, \K_{\hat{y}\hat{y}}$.
   \STATE Obtain $M$ eigenvectors corresponding to the $M$ largest eigenvalues of the generalized \STATE eigenvalue problem~(\ref{eqn:kcsmh:alpha}) as $\alpp$'s.
   \STATE Obtain the corresponding $\bett$'s using Equation~(\ref{eqn:kcsmh:beta}).
   \STATE Learn thresholds $ t_x $ and $ t_y $.
   \STATE Obtain the hash codes of points $\x^{*}$ and $\y^{*}$ using Equations~(\ref{eqn:kcsmh:hg1}), (\ref{eqn:kcsmh:hg2}) \& (\ref{eqn:bit}).

\end{algorithmic}

\end{algorithm}


%%%%%%%%%%%%%%%%%%%%%%%%%%%%%%%%%%%%%%%
\subsubsection{Regularized kernel \mbox{SMH}}
\label{smh:Ssmh:EXT:RKSMH}

Both \mbox{SMH} and \mbox{KSMH} proposed above aim at maximizing the correlation between variables in different modalities while ignoring the relational information within each modularity.  Moreover, it is unclear how to make use of side information such as labels in the two models in case such information is available in the data.

Inspired by~\cite{blaschko2008ecml}, we further extend \mbox{KSMH} by adding two \textit{Laplacian} regularization terms to the objective~(\ref{eqn:kcsmh:obj2}). We name this new model \mbox{RKSMH}, whose objective is:
\begin{align}
\frac{\alpp^{T}\K_{\hat{x}x}\K_{y\hat{y}}\bett}{\sqrt{\alpp^{T}(\K_{\hat{x}x}\R_{x}\K_{x\hat{x}}+\kappa\K_{\hat{x}\hat{x}})\alpp\bett^{T}(\K_{\hat{y}y}\R_{y}\K_{y\hat{y}}+\kappa\K_{\hat{y}\hat{y}})\bett}},\nonumber
\end{align}
where $\R_{x} = (\I+\gamma\mathcal{L}_{x}),\R_{y} = (\I+\gamma\mathcal{L}_{y})$, $\gamma>0$ is a parameter controlling the impact of regularization, and $\mathcal{L}_{x},\mathcal{L}_{y}$ are graph \textit{Laplacians}~\cite{chung1997spectral} that incorporate some information about $\X$ and $\Y$, respectively. We note that the \textit{Laplacian} matrix is defined as $\mathcal{L} = \D-\W$, where $\D$ is a diagonal matrix with $D(i,i) = \sum_{j=1}^{N}W(i,j)$.\footnote{To avoid being cluttered, we omit the subscripts here.} We note that the \textit{Laplacian} matrix can be computed efficiently by using an anchor graph~\cite{liu2010icml}.

The regularizers $ \R_{x} $ and $ \R_{y} $ not only can exploit relational information of a single modality but can also incorporate into the model side information when it is available. For example, $\mathcal{L}_{x}$ can incorporate structural or geometric information in the input space $ \X $ by defining $\W_{x}$ as
\begin{align}
\label{eqn:wfeature}
W_{x}(i,j) = \left\{ \begin{array}{ll}
\exp\left(-\frac{d^2(\x_i,\x_j)}{\sigma^2}\right) & \textrm{if $\x_i,\x_j$ are neighbors}\\
0 & \textrm{otherwise}
\end{array} \right.
\end{align}
where $d(\cdot,\cdot)$ is the Euclidean distance between two points and $\sigma$ is a user-specified width parameter. In our experiments, we regard two points as neighbors if either one is among the $K$ nearest neighbors of the other one in the feature space. We call this type of \textit{Laplacian} the feature-based \textit{Laplacian}.


$\mathcal{L}_{x}$ can also be used to incorporate side information such as labels by defining $\W_{x}$ as
\begin{align}
\label{eqn:wlabel}
W_{x}(i,j) = \left\{ \begin{array}{ll}
1 & \textrm{if $\x_i$ and $\x_j$ have the same label}\\
0 & \textrm{otherwise}
\end{array} \right.
\end{align}
We call this \textit{Laplacian} the label-based \textit{Laplacian} thereafter. Note that $\mathcal{L}_{y}$ can be defined similarly.

In \mbox{RKSMH}, $\alpp$ can be obtained by solving the following generalized eigenvalue problem:
\begin{align}
\label{eqn:lrkcsmh:alpha}
\K_{\hat{x}x}\K_{y\hat{y}}(\K_{\hat{y}y}\R_{y}\K_{y\hat{y}} + \kappa\K_{\hat{y}\hat{y}})^{-1}\K_{\hat{y}y}\K_{x\hat{x}}\alpp= \lambda^2(\K_{\hat{x}x}\R_{x}\K_{x\hat{x}} + \kappa\K_{\hat{x}\hat{x}})\alpp,
\end{align}
and $\bett$ can be computed as
\begin{align}
\label{eqn:lrkcsmh:beta}
\bett =\frac{1}{\lambda} (\K_{\hat{y}y}\R_{y}\K_{y\hat{y}}+\kappa\K_{\hat{y}\hat{y}})^{-1}\K_{\hat{y}y}\K_{x\hat{x}}\alpp.
\end{align}


The thresholding procedure of \mbox{RKSMH} is the same as those of \mbox{KSMH} and \mbox{SMH}. The algorithm is summarized in Algorithm~\ref{algorithm:lrkcmh}.

\begin{algorithm}[htb]
\caption{Algorithm of \mbox{RKSMH}}
\label{algorithm:lrkcmh}
\begin{algorithmic}
%\SetKwInOut{Input}{Input}\SetKwInOut{Output}{Output}
\STATE {\bfseries Input:} \\
$\X$, $\Y$ -- data matrices
\\$\mathcal{K}(\cdot,\cdot)$ -- kernel function
\\ $M$ -- number of hash functions
\\ $\kappa,\gamma$ -- regularization parameters
\STATE {\bfseries Procedure:} \\

  \STATE Compute $\K_{\hat{x}x}, \K_{\hat{y}y}, \K_{\hat{x}\hat{x}}, \K_{\hat{y}\hat{y}}, \R_{x},\R_{y}$.
   \STATE Obtain $M$ eigenvectors corresponding to the $M$ largest eigenvalues of the generalized eigenvalue problem~(\ref{eqn:lrkcsmh:alpha}) as $\alpp$'s.
   \STATE Obtain the corresponding $\bett$'s using Equation~(\ref{eqn:lrkcsmh:beta}).
   \STATE Learn thresholds $ t_x $ and $ t_y $.
  \STATE  Obtain the hash codes of points $\x^{*}$ and $\y^{*}$ using Equations~(\ref{eqn:kcsmh:hg1}), (\ref{eqn:kcsmh:hg2}) \& (\ref{eqn:bit}).
\end{algorithmic}
\end{algorithm}

%%%%%%%%%%%%%%%%%%%%%%%%%%%%%%%%%%%%%%%
\subsubsection{Beyond two modalities}
\label{smh:Ssmh:EXT:BEYOND}

Our \mbox{SMH} models can easily accommodate more than two modalities, using the corresponding extensions of \mbox{CCA} and \mbox{KCCA}~\cite{hardoon2004nc}\cite{blaschko2008ecml}.

Taking \mbox{SMH} for example, suppose we have $ K $ modalities and want to learn $ K $ projection vectors $ \{\w_1,\cdots,\w_K\}$, we solve the following generalized eigenvalue problem:
\begin{align}
\begin{pmatrix}
\C_{11} & \cdots &\C_{1K} \\
\vdots & \ddots &\vdots \\
\C_{K1}&\cdots &\C_{KK} \end{pmatrix}\begin{pmatrix}\w_1\\\vdots\\\w_K\end{pmatrix} = \lambda  \begin{pmatrix}
\C_{11} & \cdots &\0 \\
\vdots & \ddots &\vdots \\
\0&\cdots &\C_{KK} \end{pmatrix}\begin{pmatrix}
\w_{1}\\
\vdots \\
\w_{K}\end{pmatrix},\nonumber
\end{align} 
where $ \C_{ij} $ is the covariance matrix between modalities $ i $ and $ j $, and $  \C_{ij} = \C_{ji}^T  $.

For \mbox{KSMH}, we select landmark points from each modality and index them with $ \{\hat{1},\cdots,\hat{K}\} $. The corresponding eigenvalue problem, for projection vectors $ \{\alpha_1,\cdots,\alpha_K\} $, is:
\begin{align}\footnotesize
\begin{pmatrix}
\K_{\hat{1}1}\K_{1\hat{1}} & \cdots &\K_{\hat{1}1}\K_{K\hat{K}} \\
\vdots & \ddots &\vdots \\
\K_{\hat{K}K}\K_{1\hat{1}}&\cdots &\K_{\hat{K}K}\K_{K\hat{K}} \end{pmatrix}\begin{pmatrix}\alpha_1\\\vdots\\\alpha_K\end{pmatrix} =
\lambda  \begin{pmatrix}
\K_{\hat{1}1}\K_{1\hat{1}}+\kappa\K_{\hat{1}\hat{1}} & \cdots &\0 \\
\vdots & \ddots &\vdots \\
\0&\cdots &\K_{\hat{K}K}\K_{K\hat{K}}+\kappa\K_{\hat{K}\hat{K}} \end{pmatrix}\begin{pmatrix}
\alpha_{1}\\
\vdots \\
\alpha_{K}\end{pmatrix},\nonumber
\end{align} 
where $ \K_{\hat{i}i} $ is the kernel matrix between the landmark points and the training data points and $ \K_{\hat{i}\hat{i}} $ is the kernel matrix for the landmark points, for the $ i $th modality. The generalized eigenvalue problems for \mbox{RKSMH} are similar, and we omit them here due to space limitations.

With the projection vectors learned, we can apply the same thresholding process to each modality and get the binary codes easily.

%For multiple views, we should talk about it here. At least two ways.

%%%%%%%%%%%%%%%%%%%%%%%%%%%%%%%%%%%%%%%%%%%%%%%%%%%%%%%%%%%%%%%%%%%%%%%%%%%%%%%%
%\section{Multimodal Binary Reconstructive Embedding}
%\label{smh:MBRE}
%The \mbox{SMH} model introduced in last subsection requires the data points in different modalities to be paired, which might not be the case in some applications. In this section, we extend a unimodal hashing method \mbox{BRE} to multimodal settings to given a novel method called \textit{multimodal binary reconstructive embedding} (\mbox{MBRE}), the inputs of which is pairwise distance or relations.
%
%%+++++++++++++++++++++++++++++++++++++++++++++++++++++++
%\subsection{Model}
%Let $M$ be the number of hash functions (\aka code length), $N$ be the number of data points, and $Q$ be the number of landmark points. Given two kinds of data points $\mathcal{X}$ and $\mathcal{Y}$,\footnote{Without loss of generality, here we assume there are two modalities and each modality has $N$ points.} similar to \mbox{BRE}, we define hash functions \wrt the $m$th bit for $\x\in\mathcal{X}$ and $\y\in\mathcal{Y}$, respectively, as follows,
%\begin{align}
%h_{m}(\x) = \frac{1+\sgn\left(\sum_{q=1}^{Q}W_{x}(m,q)\kappa(\x_q,\x)\right)}{2} \ \ \mbox{or} \ \ g_{m}(\x) = \frac{1+\sgn\left(\sum_{q=1}^{Q}W_{y}(m,q)\kappa(\y_q,\y)\right)}{2}\nonumber,
%\end{align}
%where $\W_{x}$ and $\W_{y}$ are two $M\times Q$ projection matrices, $\{\x_q\}_{q=1}^{Q}\subset\mathcal{X}$ and $\{\y_q\}_{q=1}^{Q}\subset\mathcal{Y}$ are landmark points for $\mathcal{X}$ and $\mathcal{Y}$ respectively, and $\kappa(\cdot,\cdot)$ is a kernel function. Note that defining hash functions this way is very common in kernel methods and brings us flexibility to work on a wide variety of data types. Therefore, given two points $\x\in\mathcal{X}$ and $\y\in\mathcal{Y}$, we denote their corresponding binary representations as $\tilde{\x}$ and $\tilde{\y}$ such that their $m$th bits can be evaluated by $\tilde{x}(m) = h_{m}(\x)$ and $\tilde{y}(m) = g_{m}(\y)$.
%
%Since in many real-world applications, it is much easier to obtain binary pairwise relationships rather than real-valued distance, here we simply define the \textit{original} distance between two points $\x_i,\x_j$ as follows,
%\begin{align}
%d(\x_i,\x_j) = \left\{ \begin{array}{ll}
%0 & \textrm{if $\x_i$ and $\x_j$ belong to the same class};\\
%1 & \textrm{otherwise},
%\end{array} \right. \nonumber%\\
%%d(\y_k,\y_l) = \left\{ \begin{array}{ll}
%%0 & \textrm{if $\y_k$ and $\y_l$ are similar};\\
%%1 & \textrm{if $\y_k$ and $\y_l$ are dissimilar},
%%\end{array} \right. \nonumber\\
%%%d(\x_i,\x_j) = \frac{1}{2}\|\x_i-\x_j\|^{2}_2, &\tilde{d}(\x_i,\x_j) = \frac{1}{M}\|\tilde{\x}_i-\tilde{\x}_j\|^{2}_2,\nonumber\\
%%%d(\y_k,\y_l) = \frac{1}{2}\|\y_k-\y_l\|^{2}_2, &\tilde{d}(\y_k,\y_l) = \frac{1}{M}\|\tilde{\y}_k-\tilde{\y}_l\|^{2}_2,\nonumber
%%d(\x_i,\y_k) = \left\{ \begin{array}{ll}
%%0 & \textrm{if $\x_i$ and $\y_k$ are similar};\\
%%1 & \textrm{if $\x_i$ and $\y_k$ are dissimilar},
%%\end{array} \right. \nonumber
%\end{align}
%$d(\y_k,\y_l)$ and $d(\x_i,\y_k)$ are defined similarly. Note that using binary values here to define distance is just a special case, and our model can accept other definitions of distance.
%
%We define the \textit{reconstructive} distance between two points as follows,
%\begin{align}
%\tilde{d}(\x_i,\x_j) = \frac{1}{M}\|\tilde{\x}_i-\tilde{\x}_j\|^{2}_2, \ \
%\tilde{d}(\y_k,\y_l) = \frac{1}{M}\|\tilde{\y}_k-\tilde{\y}_l\|^{2}_2, \ \
%\tilde{d}(\x_i,\y_k) = \frac{1}{M}\|\tilde{\x}_i-\tilde{\y}_k\|^{2}_2.\nonumber
%\end{align}
%
%Intuitively speaking, we try to find $\W_{x},\W_{y}$ such that the reconstructive distance are close to the original distance. More specifically, the goal of \mbox{MBRE} is to minimize the following objective,
%\begin{align}
%\mathcal{O}\left(\W_{x},\W_{y}\right)&=\sum_{(\x_i,\x_j)\in\mathcal{N}_{x}}\left(d(\x_i,\x_j)-\tilde{d}(\x_i,\x_j)\right)^{2}+\sum_{(\y_k,\y_l)\in\mathcal{N}_{y}}\left(d(\y_k,\y_l)-\tilde{d}(\y_k,\y_l)\right)^{2}\nonumber\\
%&+\sum_{(\x_i,\y_k)\in\mathcal{N}_{xy}}\left(d(\x_i,\y_k)-\tilde{d}(\x_i,\y_k)\right)^{2},
%\label{eqn:totalobj}
%\end{align}
%where $\mathcal{N}_{x}$ is a set of point pairs in $\mathcal{X}$, $\mathcal{N}_{y}$ is a set of point pairs in $\mathcal{Y}$ and $\mathcal{N}_{xy}$ is a set of pairs with one point in $\mathcal{X}$ and the other point in $\mathcal{Y}$. In our experiments, there are $k$ pairs for each point and so each set has size upper-bounded by $Nk$.\footnote{The total number of pairs might be smaller than $Nk$, since there might be some duplicate pairs. Besides, different sets may have different $k$ values.}  We note that the objective function of \mbox{BRE} is just the first term of that in Eqn.~(\ref{eqn:totalobj}).
%
%
%%######################################
%\subsection{Algorithm}
%To solve the above optimization problem, we adapt the coordinate descent algorithm used in~\cite{kulis2009nips} for our model. The major difference between the adapted algorithm and the original one is threefold: 1) we update all parameters sequentially but the original algorithm randomly updates only a small subset of them;\footnote{Note that original algorithm is slow to converge because of random update.} 2) we use a warm-start approach to improve the convergence rate and obtain better performance; 3) our algorithm involves more updating terms.
%
%We first introduce Lemma~\ref{lemma:updatex} as follows.
%\begin{mylem}
%Let $\bar{D}_{x}(i,j)=d(\x_i,\x_j)-\tilde{d}(\x_i,\x_j),\bar{D}_{xy}(i,k)=d(\x_i,\y_k)-\tilde{d}(\x_i,\y_k)$. Consider updating one hash function of $\mathcal{X}$ from $h_{o}$ to $h_{n}$, and let $\h_{o}$ and $\h_{n}$ be the $N\times 1$ vectors obtained by applying the old and new hash functions to each data point in $\mathcal{X}$. Furthermore, we denote the hash function of $\mathcal{Y}$ with the same bit index as $g$ and the corresponding binary vector as $\g$. Then the objective function of using $h_{n}$ instead of $h_{o}$ can be expressed as
%\begin{align}
%\mathcal{O} &= \sum_{(\x_i,\x_j)\in\mathcal{N}_{x}}\left(\bar{D}_{x}(i,j)+\frac{1}{M}(h_{o}(i)-h_{o}(j))^2-\frac{1}{M}(h_{n}(i)-h_{n}(j))^2\right)^2\nonumber\\
%&+ \sum_{(\x_i,\y_k)\in\mathcal{N}_{xy}}\left(\bar{D}_{xy}(i,k)+\frac{1-2g(k)}{M}(h_{o}(i)-h_{n}(i))\right)^2+C,
%\end{align}
%where $C$ is a constant independent of $h_{o}$ and $h_{n}$.
%\label{lemma:updatex}
%\end{mylem}
%\begin{myproof}
%Let $\tilde{\D}_{x}^{o}$ and $\tilde{\D}_{x}^{n}$ be the matrices of reconstructive distance using $h_{o}$ and $h_{n}$ respectively, $\H_{o}$ and $\H_{n}$ be the $N\times M$ matrices of old and new hash codes of $\mathcal{X}$ respectively, and $\G$ be the hash codes of $\mathcal{Y}$. Moreover, we use $\1_{t}$ to denote the $t$th standard basis vector and $\1$ to denote a vector of all ones, and their dimensionalities will be clear in the context.
%
%We can express $\tilde{\D}_{x}^{o}$ as follows,
%\begin{align}
%\tilde{\D}_{x}^{o} = \frac{1}{M}\left(\Ell_{xo}\1^{T}+\1\Ell^T_{o}-2\H_{o}\H_{o}^{T}\right)\nonumber,
%\end{align}
%where $\Ell_{xo}$ is the vector of squared norms of the rows of $\H_{o}$. Accordingly, we can express $\Ell_{xn}$ for $\H_{n}$ as $\Ell_{xn} = \Ell_{xo} - \h_{o}+\h_{n}$, since $\h_{o}$ and $\h_{n}$ are binary vectors.
%Moreover, we can easily obtain $\H_{n} = \H_{o} +(\h_{n}-\h_{o})\1^{T}_{m}$, where $m$ is the index of the hash function being updated. Therefore,
%\begin{align}
%\tilde{\D}_{x}^{n}
%&= \frac{1}{M}\left(\Ell_{xn}\1^{T}+\1\Ell_{xn}^T-2\H_{n}\H_{n}^{T}\right)\nonumber\\
%&= \frac{1}{M}\left((\Ell_{xo} - \h_{o}+\h_{n})\1^{T}+\1(\Ell_{xo} - \h_{o}+\h_{n})^{T}-2(\H_{o} +(\h_{n}-\h_{o})\1^{T}_{m})(\H_{o} +(\h_{n}-\h_{o})\1^{T}_{m})^{T}\right)\nonumber\\
%&= \tilde{\D}_{x}^{o}-\frac{1}{M}\left((\h_{o}\1^{T}+\1\h_{o}^{T}-2\h_{o}\h_{o}^{T})-(\h_{n}\1^{T}+\1\h_{n}^{T}-2\h_{n}\h_{n}^{T})\right).\nonumber
%\end{align}
%
%Similarly, we have the following crossmodel reconstructive distance matrix,
%\begin{align}
%\tilde{\D}_{xy}^{o} = \frac{1}{M}\left(\Ell_{xo}\1^{T}+\1\Ell_{y}^T-2\H_{o}\G^{T}\right)\nonumber,
%\end{align}
%where $\Ell_{y}$ is the vector of squared norms of the rows of $\G$. Therefore,
%\begin{align}
%\tilde{\D}_{xy}^{n}
%&= \frac{1}{M}\left(\Ell_{xn}\1^{T}+\1\Ell_{y}^T-2\H_{n}\G^{T}\right)\nonumber\\
%&= \frac{1}{M}\left((\Ell_{xo} - \h_{o}+\h_{n})\1^{T}+\1\Ell_{y}^{T}-2(\H_{o} +(\h_{n}-\h_{o})\1^{T}_{m})\G^{T}\right)\nonumber\\
%&= \tilde{\D}_{x}^{o}-\frac{1}{M}\left((\h_{o}\1^{T}-2\h_{o}\g^{T})-(\h_{n}\1^{T}-2\h_{n}\g^{T})\right).\nonumber
%\end{align}
%
%Thus we can write the objective function of using $h_{n}$ instead of $h_{o}$ as
%\begin{align}
%\mathcal{O} &= \sum_{(\x_i,\x_j)\in\mathcal{N}_{x}}\left(\bar{D}_{x}(i,j)+\tilde{D}_{x}^{o}(i,j)-\tilde{D}_{x}^{n}(i,j)\right)^2+\sum_{(\x_i,\y_k)\in\mathcal{N}_{xy}}\left(\bar{D}_{xy}(i,k)+\tilde{D}_{xy}^{o}(i,k)-\tilde{D}_{xy}^{n}(i,k)\right)^2\nonumber\\
%&=\sum_{(\x_i,\x_j)\in\mathcal{N}_{x}}\left(\bar{D}_{x}(i,j)+\frac{1}{M}(h_{o}(i)-h_{o}(j))^2-\frac{1}{M}(h_{n}(i)-h_{n}(j))^2\right)^2\nonumber\\
%&+\sum_{(\x_i,\y_k)\in\mathcal{N}_{xy}}\left(\bar{D}_{xy}(i,k)+\frac{1-2g(k)}{M}(h_{o}(i)-h_{n}(i))\right)^2+C,
%\end{align}
%where we have made use of $h_{o}(i)^2 = h_{o}(i)$ and $h_{n}(i)^2 = h_{n}(i)$ and grouped terms irrelevant to $h_{o},h_{n}$ into $C$. This completes the proof.
%\end{myproof}
%
%Now we move to the details of updating one element of $\W_{x}$, e.g., $W_{x}(m,q_0)$, with all the other elements in $\W_{x}$ fixed. Given a point $\x_i$, the $m$th hash code can be obtained by computing
%\begin{align}
%W_{x}(m,q_0)\kappa(\x_{q_0},\x_i)+\sum\nolimits_{q\neq q_0}W_{x}(m,q)\kappa(\x_{q},\x_i).
%\label{eqn:threshold-1bit}
%\end{align}
%Equating (\ref{eqn:threshold-1bit}) to zero, we can easily obtain the incremental value for $W_{x}(m,q_0)$ that can change the current bit of $\x_i$ as
%\begin{align}
%    \delta_{i} = \left(\sum\nolimits_{q\neq q_0}W_{x}(m,q)\kappa(\x_{q},\x_{i})\right)/\kappa(\x_{q_0},\x_{i}) - W_{x}(m,q_0).
%\end{align}
%
%If $h_m(\x_i)>0$, we should decrease $W_{x}(m,q_0)$ to flip the hash code, in another words, $\delta_i<0$. On the contrary, if $h_m(\x_i)<0$, we should increase $W_{x}(m,q_0)$ to flip the hash code, that is, $\delta_i>0$. As a result, we first find all the $\delta_{i}$'s for all $\x_{i}$'s. Then we sort $\{\delta_i\mid\delta_i>0\}$ in ascending order and $\{\delta_i\mid\delta_i<0\}$ in descending order, and thus obtain two sets of intervals. It is easy to observe that, in a fixed interval, changing $W_{x}(m,q_0)$ will not affect the hash code of any point. However, if we go across intervals, the hash code of exactly one point will be changed. As a result, starting from the current value of $W_{x}(m,q_0)$, we first increase it by adding $\delta_i+\epsilon>0$ from the smallest one to the largest one to obtain a set of possible values of objective function~(\ref{eqn:totalobj}). Note that $\epsilon$ is a very small positive number ensuring that only the $i$th bit is flipped. We then decrease $W_{x}(m,q_0)$ by adding $\delta_i-\epsilon<0$ to the starting value from the largest one to the smallest one to obtain another set of possible objective values. In total, we obtain a set of $N$ possible objective values. After getting all these values, we update $W_x(m,q_{0})$ by adding $\delta_i$ corresponding to the smallest objective $\mathcal{O}_i$ if it is smaller than original objective $\mathcal{O}$ before updating, or skip this iteration otherwise.
%
%The main idea of updating $W_{x}(m,q_0)$ is to find $\delta_i$ leading to the smallest objective function value.  We can compute the values sequentially in an efficient way based on Lemma~\ref{lemma:updateh}.
%\begin{mylem}
%Given two hash vectors $\h_{t}$ and $\h_{t-1}$ for $\mathcal{X}$ which are different in only one position, the objective w.r.t. $\h_{t}$ can be computed from that w.r.t. $\h_{t-1}$ in $O(k)$ time.
%\label{lemma:updateh}
%\end{mylem}
%\begin{myproof}
%Let the index of the point in which $\h_{t}$ and $\h_{t-1}$ are different be $a$. The only terms that change in the objective are $(\x_a,\x_j)\in\mathcal{N}_{x},(\x_i,\x_a)\in\mathcal{N}_{x}$, and $(\x_a,\y_k)\in\mathcal{N}_{xy}$. Let $f_a = 1$ if $h_{t-1}(a)=0,h_{t}(a)=1$, and $f_a=-1$ otherwise. Therefore the relevant terms in the objective function as given in Lemma~\ref{lemma:updatex} may be written as
%\begin{align}
%\mathcal{O}'&=\sum_{(\x_a,\x_j)\in\mathcal{N}_{x}}\left(\bar{D}_{x}(a,j)-\frac{f_a}{M}(1-2h_{t}(j))\right)^2+\sum_{(\x_i,\x_a)\in\mathcal{N}_{x}}\left(\bar{D}_{x}(i,a)-\frac{f_a}{M}(1-2h_{t}(i))\right)^2\nonumber\\
%&+\sum_{(\x_a,\y_k)\in\mathcal{N}_{xy}}\left(\bar{D}_{xy}(a,k)-\frac{f_a}{M}(1-2g(k))\right)^2.
%\label{eqn:updateO-1bit}\end{align}
%
%Since $\x_{a}$ has $k$ nearest neighbors and lives in the neighborhood of $k$ points on average, it costs $O(k)$ time to update the objective.
%\end{myproof}
%
%We can update each element of $\W_{y}$ similarly with the help of the following two lemmas. %Due to lack of space, we omit the proof here.
%
%\begin{mylem}
%Let $\bar{D}_{y}(k,l)=d(\y_k,\y_l)-\tilde{d}(\y_k,\y_l),\bar{D}_{xy}(i,k)=d(\x_i,\y_k)-\tilde{d}(\x_i,\y_k)$. Consider updating one hash function of $\mathcal{Y}$ from $g_{o}$ to $g_{n}$, and let $\g_{o}$ and $\g_{n}$ be the $N\times 1$ vectors obtained by applying the old and new hash functions to each data point in $\mathcal{Y}$. We further denote the hash function of $\mathcal{X}$ with the same index as $h$ and the corresponding binary vector of $\mathcal{X}$ as $\h$. Then the objective function of using $g_{n}$ instead of $g_{o}$ can be expressed as
%\begin{align}
%\mathcal{O} &= \sum_{(\y_k,\y_l)\in\mathcal{N}_{y}}\left(\bar{D}_{y}(k,l)+\frac{1}{M}(g_{o}(k)-g_{o}(l))^2-\frac{1}{M}(g_{n}(k)-g_{n}(l))^2\right)^2\nonumber\\
%&+ \sum_{(\x_i,\y_k)\in\mathcal{N}_{xy}}\left(\bar{D}_{xy}(i,k)+\frac{1-2h(i)}{M}(g_{o}(k)-g_{n}(k))\right)^2+C',
%\end{align}
%where $C'$ is a constant independent of $g_{o}$ and $g_{n}$.
%\label{lemma:updatey}
%\end{mylem}
%
%\begin{mylem}
%Given two hash vectors $\g_{t}$ and $\g_{t-1}$ for $\mathcal{Y}$ which are different in only one position, the objective w.r.t. $\g_{t}$ can be computed from that w.r.t. $\g_{t-1}$ in $O(k)$ time.
%\label{lemma:updateg}
%\end{mylem}
%
%As a result, the general procedure of our algorithm can be summarized as follows. We first initialize model parameters $\W_{x}, \W_{y}$. Then we update each element of $\W_{x}$ based on Lemma~\ref{lemma:updatex}\&\ref{lemma:updateh}, and each element of $\W_{y}$ based on Lemma~\ref{lemma:updatey}\&\ref{lemma:updateg}. This updating procedure iterates until $\W_{x}, \W_{y}$ converge. We then use current values of $\W_{x}, \W_{y}$ as initialization and retrain the model to get better $\W_{x}, \W_{y}$. In our experiments, this warm-start approach is very effective, $\W_{x}, \W_{y}$ will converge very fast to a better local optimum. To update one element of $\W_{x}$ or $\W_{y}$, sorting $N$ incremental values $\delta_i$'s needs $O(N\log N)$ time, obtaining all objective function values needs $O(Nk)$ time and finding the smallest $\mathcal{O}_i$'s needs $O(N)$ time. Putting everything together, the time complexity of updating one element is $O(N\log N+Nk)$. As a result, one full iteration of updating $\W_{x}$ or $\W_{y}$ requires $O(MQN(\log N+k))$ time.
%
%%\begin{algorithm}
%%%\DontPrintSemicolon
%%%\SetKwData{Left}{left}\SetKwData{This}{this}\SetKwData{Up}{up}
%%%\SetKwFunction{Union}{Union}\SetKwFunction{FindCompress}{FindCompress}
%%\SetKwInOut{Input}{Input}\SetKwInOut{Output}{Output}
%%
%%\Input{$\mathcal{N}_{x}, \mathcal{N}_{y}, \mathcal{N}_{xy}$.}
%%\Output{$\W_{x}, \W_{y}$.}
%%\Begin{
%%Initialize $\W_{x}, \W_{y}$.
%%\While{NOT Converge}{
%%\For{$m=1$ to $M$}{    \For{$q=1$ to $Q$}{ Update $W_{x}(m,q)$.}    }
%%\For{$m=1$ to $M$}{    \For{$q=1$ to $Q$}{ Update $W_{y}(m,q)$.}    }
%%}}
%%\caption{General procedure of coordinate descent}
%%\label{algo:cmh}
%%\end{algorithm}
%
%Note that local convergence in a finite number of updates is guaranteed since each update will never increase the objective function value which is lower-bounded by zero. Therefore, the algorithm is  efficient and can scale well even for large high-dimensional data sets.
%
% % % % % % % % % % % % % % % % % % % % % % % % % % % % % % %
%\section{Multimodal Latent Binary Embeddings}
%\label{smh:MLBE}
%
%Up to now, we have presented two models, namely, \mbox{SMH} and \mbox{MBRE}. To evaluate the data correlation, \mbox{SMH} requires paired or aligned input data which might not be easy to obtain. \mbox{MBRE} eliminates this constraint by directly finding a discrete embedding, so that the Hamming distance in the embedded space maximally approximates the original distance. In this section, we introduce an alternative multimodal hashing model to improve \mbox{SMH}, which is called \textit{multimodal latent binary embeddings} (\mbox{MLBE}), based on latent factor models. \mbox{MLBE} relates hash codes and observations of similarity, i.e., intramodel similarity and intermodel similarity, in a probabilistic model, and the hash codes can be learned easily by \mbox{MAP} estimation of the latent factors. %Among other things, \mbox{MLBE} can be easily extended to determine the proper length of hash codes.
%
%% the intramodel and intermodel similarities are generated based on latent binary factors and weighting matrices. 
%
%\subsection{Model}
%
%In the following, we focus on the bimodel case but it is easy to extend \mbox{MLBE} to support multiple modalities. Assume we have binary latent factors for each modality, for example, $ \U \in \{+1,-1\}^{N\times K} $ for $ \X  $ and $ \V \in \{+1,-1\}^{M\times K} $ for $ \Y  $. Correspondingly, we also have two weighting matrices, $\W^{x} \in \mathbb{R}^{K\times K}$ and $ \W^{y}  \in \mathbb{R}^{K\times K}$. The basic assumption of our model is that the observations of intramodel and intermodel similarities are determined by the latent factors and weighting matrices. The graphical representation of \mbox{MLBE} is depicted in Figure~\ref{fig:model}.
%
%\begin{figure}[tb]
%\centering
%\epsfig{figure=fig/mlbe/graphmodel, width=0.4\textwidth}
%\caption{Graphical representation of the model of multimodal latent binary embeddings. The shaded circles are observed variables and the empty ones are latent variables.}
%\label{fig:model}
%\end{figure}
%
%
%Given $ \U , \V , \W^{x} $ and $ \W^{y} $, the two symmetric intramodel similarity matrices $ \S^{x} \in \mathbb{R}^{N\times N}$ for  $ \X $ and $ \S^{y} \in \mathbb{R}^{M\times M}$ for $ \Y $ are generated from the following distributions, respectively:
%$$S^{x}_{ij} \mid \U, \W^{x}  \sim \mathcal{N}(\u_i^T\W^{x}\u_j,\theta_x^2 ), \ \ \forall i \ge j, \  i,j\in\{1,\cdots,N\}, $$
%$$S^{y}_{ij} \mid \V, \W^{y}  \sim \mathcal{N}(\v_i^T\W^{y}\v_j,\theta_y^2 ), \ \ \forall i \ge j, \  i,j\in\{1,\cdots,M\}, $$
%where $ \u_i $ and $ \u_j $ denote the $ i $th row and $ j $th row of $ \U  $. Similarly, $ \v_i $ and $ \v_j $ denote the $ i $th row and $ j $th row of $ \V  $.
%
%We also observe a intermodel similarity matrix $ \S^{xy} \in \{1,0\}^{N\times M}$, where 1 and 0 stand for similar and dissimilar, respectively. For example, if an image and a text document are both for a historic event, we label them with 1. If they are irrelevant, we label them with 0. Note that it is quite common and easy to define intermodel similarity using binary values $ \{1,0\} $ in practice, but our model can also accommodate other values by simply changing the distribution. We further assume only a subset of the intermodel similarity values are observed and use an indicator matrix $ \O\in \{0,1\}^{N\times M} $ to denote this, i.e., $ O_{ij}=1 $ if $ S_{ij}^{xy} $ is observed and $ O_{ij}=0 $ otherwise. Given $ \U  $ and $ \V  $, the observed elements in $ \S^{xy} $ are generated by
%$$S^{xy}_{ij} \mid \U, \V  \sim \mbox{Bernoulli}(\sigma(\u_i^{T}\v_j)),\ \ \forall i,j, \ O_{ij}=1,$$
%where $ \sigma(x) = 1/(1+\exp(-x))$ is the logistic sigmoid function.
%
%Assume each element in $ \U\in\{+1,-1\}^{N\times K}  $ is determined identically and independently the following way,\footnote{Conventional the Bernoulli distribution is for $ \{0,1\} $ valued variables. Here, without loss of generality, we can map them to $ \{-1,+1\} $ by linear transformation.}
%\begin{align}
%\pi \mid \alpha_u,\beta_u &\sim \mbox{Beta}(\alpha_u,\beta_u),\nonumber\\
%U_{ik} \mid \pi &\sim \mbox{Bernoulli}(\pi),\nonumber
%\end{align}
%where $ \alpha_u $ and $ \beta_u $ are hyperparameters, we can integrate out $ \pi $ to give the following prior on $ \U $:
%\begin{align}
%U_{ik} \mid \alpha_u,\beta_u  \sim \mbox{Bernoulli}(\frac{\alpha_u}{\alpha_u+\beta_u}), \ \ \forall i\in\{1,\cdots,N\}, \ k\in\{1,\cdots,K\}.\nonumber
%\end{align}
%
%Similarly, we define the prior on $ \V \in\{+1,-1\}^{M\times K} $ as
%\begin{align}
%V_{ik} \mid \alpha_v,\beta_v  \sim \mbox{Bernoulli}(\frac{\alpha_v}{\alpha_v+\beta_v}), \ \ 
%\forall i\in\{1,\cdots,M\}, \ k\in\{1,\cdots,K\}.\nonumber
%\end{align}
%
%%The prior terms for $ \U  \in \{+1,-1\}^{N\times K}$ and $ \V \in \{+1,-1\}^{M\times K} $ are from ~\cite{griffiths2006nips}:
%%$$\Pr(\U) = \prod_{k=1}^K\frac{\frac{\alpha}{K}\Gamma(N_k+\frac{\alpha}{K})\Gamma(N-N_k+1)}{\Gamma(N+1+\frac{\alpha}{K})}$$
%%and
%%$$\Pr(\V) = \prod_{k=1}^K\frac{\frac{\beta}{K}\Gamma(M_k+\frac{\beta}{K})\Gamma(M-M_k+1)}{\Gamma(M+1+\frac{\beta}{K})},$$
%%where $ N_k = \sum_{i=1}^{N}\delta(U_{ik}=1) $ and $ M_k = \sum_{i=1}^{M}\delta(V_{ik}=1) $ are the number of $ 1 $'s in the $ k $th column of $ \U  $ and $ \V  $, respectively.
%
%For $ \X  $, the entries of the symmetric weight matrix $ \W^{x}\in\mathbb{R}^{K\times K} $ are generated identically and independently by a standard Gaussian distribution:
%$$\W^{x}_{ij} \mid \phi_x^2  \sim \mathcal{N}(0,\phi_x^2 ), \ \  \forall i\ge j, \ i,j\in\{1,\cdots,K\}.$$
%We put a similar prior on  $ \W^{y}\in\mathbb{R}^{K\times K} $ for $ \Y  $:
%$$\W^{y}_{ij} \mid \phi_y^2  \sim \mathcal{N}(0,\phi_y^2 ), \ \ \forall i\ge j, \ i,j\in\{1,\cdots,K\}.$$
%%We put simple matrix Gaussian prior on $ \W_x $ and $ \W_y $, which can be written as:
%%$$\Pr(\w_{x}) = \mathcal{N}(\0,\phi_x\I ), \w_{x} = \W_x(:)$$
%%$$\Pr(\w_y) = \mathcal{N}(\0,\phi_y\I ), \w_y = \W_y(:)$$
%
%\subsection{Algorithm}
%
%Based on the observations, we can learn the parameters $ \U $ and $ \V $ to give the hash codes. But finding exact posterior distributions of $ \U  $ and $ \V  $ is intractable, as a result, we adopt an alternating algorithm to find an \mbox{MAP} estimation of $ \U , \V ,\W^x $ and $ \W^y  $.
%
%We first update $ U_{ik} $ while fixing the others. To decide the \mbox{MAP} estimation of $ U_{ik} $, we first define a loss function with respect to $ U_{ik}$ as in Definition~\ref{def:lossu}:
%
%%  $ Let $ \u_i $ be the $ i $th row of $ \U  $, $\w_{x} = \W_x(:) $ and $ \s^{x}_{i} = \S_x(:,i) $, we denote $ \A_i = \mbox{kron}(\u_i, \U) $ and have $ \Pr(\s^{x}_{i}\mid \A,\w_{x}) = \mathcal{N}(\A_i\w_{x},\theta_x\I) $. The loss function of updating one element $ \U_{ik}  $:
%
%\begin{mydef}
%\begin{align}
%\mathcal{L}_{U_{ik}} &=\log\frac{\alpha_u}{\beta_u}-\frac{1}{2\theta_{x}^2}\sum_{j\neq i}^{N}\left[-2 S^{x}_{ij} \u_j^T\W^x(\u_{i}^{+} - \u_{i}^{-}) - \u_j^T\W^x(\u_{i}^{+}{\u_{i}^{+}}^{T}-\u_{i}^{-}{\u_{i}^{-}}^{T})\W^x\u_j\right]\nonumber\\
%&+\sum_{j=1}^{M}O_{ij}\left[S_{ij}^{xy}\log \frac{\sigma_{ij}^{+}}{\sigma_{ij}^{-}} + (1-S_{ij}^{xy})\log \frac{1-\sigma_{ij}^{+}}{1-\sigma_{ij}^{-}}\right],
%\end{align}
%where $ U_{-ik} $ denotes all the elements in $ \U $ but $ U_{ik} $, $ \s^{x}_i $ denotes the $ i $th row of $ \S^{x} $, $ \u^{+}_i $ is the $ i $th row of $ \U  $ with $ U_{ik}=1 $ and $ \u^{-}_i $ is the $ i $th row of $ \U $ with $ U_{ik}=-1 $. We further define $ \sigma^{+}_{ij} = \sigma(\v_j^T\u_i^{+}) $ and $ \sigma^{-}_{ij} = \sigma(\v_j^T\u_i^{-}) $.
%\label{def:lossu}\end{mydef}
%
%Then we have the following lemma:
%\begin{mylem}
%The \mbox{MAP} solution of $ U_{ik} $ is $ U_{ik}=1 $ if $ \mathcal{L}_{U_{ik}}>0 $ and $ U_{ik}=-1 $ otherwise.
%\label{lemma:updateu}\end{mylem}
%
%\begin{myproof}
%To get the \mbox{MAP} estimation of $ U_{ik} $, we only need to compare the two posterior probabilities $ \Pr(U_{ik}=1) $ and $ \Pr(U_{ik}=-1) $ conditioned on the observations and all the other model parameters. Specifically, we compute the log ratio of the two probabilities which is larger than zero if $ \Pr(U_{ik}=1) > \Pr(U_{ik}=-1)  $ and smaller than zero otherwise. The log ratio can be evaluated as follows:
%\begin{align}
% & \log \frac{\Pr(U_{ik} = 1\mid U_{-ik},\V , \W_x, \S^{x}, \S^{xy})}{\Pr(U_{ik} = -1\mid U_{-ik},\V , \W_x, \S^{x}, \S^{xy})}\nonumber\\
%=& \log \frac{\Pr(U_{ik} = 1\mid \alpha,\beta)}{\Pr(U_{ik} = -1\mid \alpha,\beta)}
%+\log \frac{\Pr(\s^{x}_i\mid U_{ik}=1, U_{-ik}, \W^{x})}{\Pr(\s^{x}_i\mid U_{ik}=-1, U_{-ik}, \W^{x})}\nonumber\\
%+&\log \frac{\Pr(\S^{xy}\mid U_{ik}=1, U_{-ik}, \V)}{\Pr(\S^{xy}\mid U_{ik}=-1, U_{-ik}, \V)}\nonumber\\
%=&\log\frac{\alpha_u}{\beta_u}-\frac{1}{2\theta_{x}^2}\sum_{j\neq i}^{N}\left[-2 S^{x}_{ij} \u_j^T\W^x(\u_{i}^{+} - \u_{i}^{-})\right]\nonumber\\
%-&\frac{1}{2\theta_{x}^2}\sum_{j\neq i}^{N}\left[\u_j^T\W^x(\u_{i}^{+}{\u_{i}^{+}}^{T}-\u_{i}^{-}{\u_{i}^{-}}^{T})\W^x\u_j\right]\nonumber\\
%+&\sum_{j=1}^{M}O_{ij}\left[S_{ij}^{xy}\log \frac{\sigma_{ij}^{+}}{\sigma_{ij}^{-}} + (1-S_{ij}^{xy})\log \frac{1-\sigma_{ij}^{+}}{1-\sigma_{ij}^{-}}\right],
%%-\frac{1}{\theta_x}\left[{\s^{x}_i}^T (\A^{-}_{i} -  \A^{+}_{i} )\w_{x}\right]\nonumber\\
%%&-\frac{1}{2\theta_x}\left[\w_{x}^T({\A^{+}_{i}}^{T}\A^{+}_{i} - {\A^{-}_{i}}^{T}\A^{-}_{i})\w_{x}\right]\nonumber\\
%%&-\frac{1}{\mu}\sum_{i,j}I_{ij}\left[S^{xy}_{ij}(\sigma^{-}_{ij}-\sigma^{+}_{ij})+\frac{1}{2}({\sigma^{+}_{ij}}^2-{\sigma^{-}_{ij}}^2)\right],
%\label{eqn:lossu}\end{align}
%where $ U_{-ik} $ denotes all the elements in $ \U $ but $ U_{ik} $, $ \s^{x}_i $ denotes the $ i $th row of $ \S^{x} $, $ \u^{+}_i $ is the $ i $th row of $ \U  $ with $ U_{ik}=1 $ and $ \u^{-}_i $ is the $ i $th row of $ \U $ with $ U_{ik}=-1 $. We further define $ \sigma^{+}_{ij} = \sigma(\v_j^T\u_i^{+}) $ and $ \sigma^{-}_{ij} = \sigma(\v_j^T\u_i^{-}) $.
%
%The log ratio computed in Eqn.~(\ref{eqn:lossu}) gives exactly $ \mathcal{L}_{U_{ik}} $, hence the proof is completed.
%\end{myproof}
%
%%The details can be found in Appendix.
%
%%We group all the terms irrelevant to $ U_{ik} $ in $ C $.
%
%%$ N_{-ik} = \sum_{j\neq i}\delta(U_{jk}=1)$ is the number of $ +1 $ in $ k $th column and all rows but the $ i $th row and $ I_{ij}=1 $ if $ \S^{xy}_{ij} $ is observed and  $ I_{ij}= 0 $ otherwise. We define $ \A^{+}_{i} = \mbox{kron}(\u_i,\U ) $ and $ \sigma^{+}_{ij} = \sigma(\u_i^T\v_j) $ with $ U_{ik}=1 $. We define $ \A^{-}_{i} = \mbox{kron}(\u_i,\U ) $ and $ \sigma^{-}_{ij} = \sigma(\u_i^T\v_j) $ with $ U_{ik}=-1 $.
%
%%
%%$ (\hat{\u}_1-\hat{\u}_2)\w_x\S_x\nonumber\\&+(\hat{\u}_1(\W_x^T\W_x))(\hat{\u}_1-\hat{\u}_2)\nonumber\\& + (\hat{\u}_2(\W_x^T\W_x))(\hat{\u}_1-\hat{\u}_2)\nonumber\\& +\sum_{j}I_{ij}(S_{ij}-\sigma(\u_i^{T}\v_j)q)^2 $
%%We can easily evaluate loss function~(\ref{eqn:loss_u}) and set
%%\begin{align}
%%U_{ik} = \left\{ \begin{array}{ll}
%%+1 & \mathcal{L}_{U_{ik}}>0\\
%%-1 & \mbox{otherwise}
%%\end{array} \right.
%%\end{align}
%
%Similarly, we have Definition~\ref{def:lossv} and Lemma~\ref{lemma:updatev} for \mbox{MAP} estimation of $ \V $. %Due to space limitations, we omit the proof here.
%
%\begin{mydef}
%\begin{align}
%\mathcal{L}_{V_{ik}} &=\log\frac{\alpha_v}{\beta_v}-\frac{1}{2\theta_{y}^2}\sum_{j\neq i}^{N}\left[-2 S^{y}_{ij} \v_j^T\W^y(\v_{i}^{+} - \v_{i}^{-}) - \v_j^T\W^y(\v_{i}^{+}{\v_{i}^{+}}^{T}-\v_{i}^{-}{\v_{i}^{-}}^{T})\W^y\v_j\right]\nonumber\\
%&+\sum_{j=1}^{N}O_{ji}\left[S_{ji}^{xy}\log \frac{\sigma_{ji}^{+}}{\sigma_{ji}^{-}} + (1-S_{ji}^{xy})\log \frac{1-\sigma_{ji}^{+}}{1-\sigma_{ji}^{-}}\right],
%\end{align}
%where $ V_{-ik} $ denotes all the elements in $ \V $ but $ V_{ik} $, $ \s^{y}_i $ denotes the $ i $th row of $ \S^{y} $, $ \v^{+}_i $ is the $ i $th row of $ \V  $ with $ V_{ik}=1 $ and $ \v^{-}_i $ is the $ i $th row of $ \V $ with $ V_{ik}=-1 $. We further define $ \sigma^{+}_{ji} = \sigma(\u_j^T\v_i^{+}) $ and $ \sigma^{-}_{ji} = \sigma(\u_j^T\v_i^{-}) $.
%\label{def:lossv}\end{mydef}
%
%
%\begin{mylem}
%The \mbox{MAP} solution of $ V_{ik} $ is $ V_{ik}=1 $ if $ \mathcal{L}_{V_{ik}}>0 $ and $ V_{ik}=-1 $ otherwise.
%\label{lemma:updatev}\end{mylem}
%
%When fixing $ \U , \V  $ and $ \W^{y} $, we compute the \mbox{MAP} estimation of $ \W^{x} $ by maximizing the following loss function:
%\begin{align}
%\mathcal{L}_{\W^{x}}&= \log P(\W^{x}) + \log P(\S^{x}_{h}\mid \U ,\W^{x})\nonumber\\
%&=\sum_{ i\ge j}^{K}\sum_{ j=1}^{K}-\frac{{W^{x}_{ij}}^2}{2\phi_x^2} + \sum_{ i > j}^{N}\sum_{ j=1}^{N}-\frac{1}{2\theta_x^2}(S^{x}_{ij}-\u_i^T\W^x\u_j)^2\nonumber\\
%&=-\frac{1}{4\phi_x^2}\w_{x}^T(\I + \mbox{diag}(\m) )\w_{x}
%-\frac{1}{2\theta_x^2}\left[(\s^{x}_h-\A_h\w_x)^T(\s^{x}_h-\A_h\w_x)\right]\nonumber\\
%&= -\frac{1}{2}\w_{x}^T \left(\A_h^T\A_h +\frac{\theta^2_x}{4\phi^2_x}\left(\I + \mbox{diag}(\m) \right) \right)\w_{x}
% + \w_{x}^T\A^T_h \s_h^x+C'
%\label{eqn:loss_wx}\end{align}
%where $ \w_x  $ is a $ K^2 $-dimensional column vector taken column-wise from $ \W^x $, $ \m $ is a $ K^2 $-dimensional indicator vector in which the value should be 1 if the index corresponds to $ W^{x}_{ii},i=1,\cdots,K $ and 0 otherwise.
%Let $ \S^{x}_{h} $ denote the left-lower half of $ \S^{x} $ and its vector form be $ \s^{x}_h $. We define $ \A = \U\otimes \U $ and $ \A_h$ consists of the rows corresponding to $ S^x_{ij}, i>j $. We group all the terms irrelevant to $ \W^{x} $ in $ C' $.\footnote{Here we have used a property of Kronnecker multiplication: $ \u^T \W \v = \w^T(\u\otimes\v)  $ where $ \w $ is a column-wise vector of $ \W $ if $ \W $ is a symmetric matrix.}
%
%% $ \s^x = \S_x(:) $, we have $ \Pr(\s^x\mid \A,\w_{x}) = \mathcal{N}(\A\w_{x},\theta_x\I) $. The loss function of updating $ \w_{x} $ is:
%%\begin{align}
%%\mathcal{L}_x = -\frac{1}{2}\w_{x} (\A^T\A +\frac{\theta_x}{\phi_x}\I )\w_{x} + \s^x\A \w_{x},
%%\end{align}
%\begin{mylem}
%The \mbox{MAP} estimation of $\W^{x}$ can be evaluated by:
%\begin{align}
%\w_{x} =\left(\A_h^T\A_h +\frac{\theta^2_x}{4\phi^2_x}\left(\I + \mbox{diag}(\m) \right)\right)^{-1}\A_h^T \s^x. \nonumber
%\end{align}
%\label{lemma:updatewx}
%\end{mylem}
%
%Note that Lemma~\ref{lemma:updatewx} can be easily proved by setting the derivative of $ \mathcal{L}_{\W^{x}} $ with respect to $ \w_{x} $ to zero.\footnote{We can adopt gradient-based algorithms to find this global maximum, which may be much faster.} Similarly, we have Lemma~\ref{lemma:updatewy} for $ \W^{y} $.
%
%\begin{mylem}
%The \mbox{MAP} estimation of $\W^{y}$ can be evaluated by:
%\begin{align}
%\w_{y} =\left(\B_h^T\B_h +\frac{\theta^2_y}{4\phi^2_y}\left(\I + \mbox{diag}(\m) \right)\right)^{-1}\B_h^T \s^y,\nonumber
%\end{align}
%where $ \w_y  $ is a $ k^2 $-dimensional column vector taken column-wise from $ \W^y $, $ \m $ is a $ k^2 $-dimensional indicator vector in which the value should be 1 if the index corresponds to $ W^{y}_{ii},i=1,\cdots,K $ and 0 otherwise.
%Let $ \S^{y}_{h} $ denote the left-lower half of $ \S^{y} $ and its vector form be $ \s^{y}_h $. We define $ \B = \V\otimes \V $ and $ \B_h$ consists of the rows corresponding to $ S^y_{ij}, i>j $.
%\label{lemma:updatewy}
%\end{mylem}
%
%%We can update $ \W^{y} $ similarly.\footnote{Updating $ \W^{x} $ and $ \W^{y} $ needs playing with a very large matrix  $ \A_h$ which might not be handled in Matlab, so we use a small but sufficient reference set in $ \X $ and $ \Y  $ to learn $ \W^{x} $ and $ \W^{y} $ and fix them to learn $ \U  $ and $ \V $ for the whole database.}
%
%%Similarly, we have $ \w_y =(\B^T\B +\frac{\theta_y}{\phi_y}\I )^{-1}\B^T \s_y $, where $ \B = \mbox{kron}(\V, \V) , \w_y = \W_y(:) $ and $ \s^y = \S_y(:) $.
%
%
%%The algorithm should work as follows:
%%1 use training data to get Wx, Wy and U, V.
%%2 fix Wx, Wy and U, V for a reference set, we then paralelly update the U and V in the test set. Each update is conducted iteratively for the elements in U or V, should be converge very fast.
%%3 use the code to do retrieval.
%
%We summarize the algorithm of \mbox{MLBE} in Algorithm~\ref{algorithm:mlbe}. In our experiments, we use the log likelihood to determine the convergence.
%
%\begin{algorithm}[!t]
%%\DontPrintSemicolon
%%\SetKwData{Left}{left}\SetKwData{This}{this}\SetKwData{Up}{up}
%%\SetKwFunction{Union}{Union}\SetKwFunction{FindCompress}{FindCompress}
%\SetKwInOut{Input}{Input}\SetKwInOut{Output}{Output}
%\Input{$\S_{x}$, $\S_{y}$, $\S_{xy}$ -- similarity matrices
%\\ $\O_{xy}$ -- similarity matrices
%\\ $M$ -- number of hash functions
%\\ $\theta_x,\theta_y, \phi_x,\phi_y, \alpha_u,\alpha_v, \beta_u, \beta_v$ -- regularization parameters}
%\Begin{
%\textit{Training phase}:\\
%   Initialize $ \U  $ and $ \V  $ with $ \{-1,+1\}$ of equal probability.
%   \While{not converge}{
%   Update each element of $ \W_x $ sequentially using Lemma~\ref{lemma:updatewx}.
%   Update $ \U  $ using Lemma~\ref{lemma:updateu}.
%   Update each element of $ \W_y $ sequentially using Lemma~\ref{lemma:updatewy}.
%   Update $ \V $ using Lemma~\ref{lemma:updatev}.
%   }
%\textit{Testing phase}:\\
%   Obtain hash codes of points $\x^{*}$ and $\y^{*}$ using Lemma~\ref{lemma:updateu} and Lemma~\ref{lemma:updatev}, respectively.
%}
%\caption{Algorithm of \mbox{MLBE}}
%\label{algorithm:mlbe}
%\end{algorithm}
%
%%\subsection{Complexity Analysis}
%%-------------------------------------------------------------------------
%
%%\section{Max margin multimodal hashing}
%%\label{smh:MMMH}
%%
%%In this section, we introduce a new model that utilize the idea of margin, which is equivalent to hinge loss. The key challenge is how to define margin in multimodal setting. And how to optimize. It will be the best if we can find some convex formulation. Nevertheless, we can use CCCP to achieve some global optimality. This should also be inspired from other embedding algorithms.
%
\section{Experiments}
\label{smh:exps}

We conduct several experiments to compare \mbox{SMH} and its extensions with some other related methods. Through the experiments, we want to answer the following questions for each method:

\begin{enumerate}
\item How does \mbox{SMH} perform when compared with other state-of-the-art hashing models on crossmodal retrieval task?
\item How does \mbox{SMH} perform when compared with other state-of-the-art hashing models on unimodal retrieval task?
\end{enumerate}

%%%%%%%%%%%%%%%%%%%%%%%%%%%%%%%%%%%%%%%
\subsection{Data Sets}
\label{smh:exps:data}

In our experiments, we use two publicly available data sets that are, to the best of our knowledge, the only two up-to-date public data sets involving multiple modalities at large scale.

The first data set, named \textit{Wiki}, is based on a set of Wikipedia featured articles provided by~\cite{rasiwasia2010mm}.\footnote{\url{http://www.svcl.ucsd.edu/projects/crossmodal/}} It contains a total of 2,866 documents (image-text pairs), each of which consists of an image and a text article. Each document is annotated with a label chosen from ten semantic classes. The data set has been split into a training set of 2,173 documents and a test set of 693 documents. The image representation scheme is based on the popular \textit{scale invariant feature transformation} (\mbox{SIFT})~\cite{lowe2004ijcv} with a codebook of 128 words.  Representation of a text article is based on its probability distribution over topics derived from a \textit{latent Dirichlet allocation} (\mbox{LDA}) model~\cite{blei2003jmlr} with ten topics.

The second data set, named \textit{\mbox{Flickr}} thereafter, is a subset of the NUS-WIDE database\footnote{\url{http://lms.comp.nus.edu.sg/research/NUS-WIDE.htm}} which is based on images from \mbox{Flickr.com}~\cite{nus-wide-civr09}. We prune the original data set and keep only the points belonging to at least one of the ten largest classes. The data set contains a total of 186,577 image-text pairs, each of which belongs to at least one of ten possible labels (\aka concepts). The data set has been split into a training set of 185,577 pairs and a test set of 1,000 pairs. The images are represented by a $500$-dimensional \mbox{SIFT} representation. The text is simply represented by the number of occurrences of the 1,000 most frequently used tags according to the image.

Some characteristics of the two data sets are summarized in Table~\ref{table:data}.

\begin{table}[!t]
% increase table row spacing, adjust to taste
% \renewcommand{\arraystretch}{1.3}
% if using array.sty, it might be a good idea to tweak the value of \extrarowheight as needed to properly center the text within the cells
\caption{Characteristics of Data Sets}\vspace{0.5cm}
\label{table:data}
\centering
\begin{tabular}{|c|c|c|c|c|}
\hline
Data set & $D_{x}$ &  $D_{y}$ &  \# of points &\# of classes\\
\hline
Wiki& 128& 10& 2,866 & 10\\
\hline
\mbox{Flickr}& 500& 1000& 186,577 &10\\
\hline
\end{tabular}
\end{table}

%%%%%%%%%%%%%%%%%%%%%%%%%%%%%%%%%%%%%%%
\subsection{Experimental Settings}
\label{smh:exps:settings}

To mimic real multimedia retrieval systems, we consider two tasks in our experiments: crossmodel and uni-modal retrieval. In cross-modal retrieval, the query and the database belong to different modalities, for example, an image is used as a query and a set of text is used as a database. In uni-modal retrieval, the query and the database belong to the same modality. For each task, we first train the models on the training set, and then use documents in the test set as queries and the training set as database.

We use two evaluation measures in both cases, namely, \textit{mean average precision} (\mbox{MAP}) and precision at a fixed Hamming radius. \mbox{MAP} is a measure widely used by the information retrieval community~\cite{baeza1999book,rasiwasia2010mm}. Specifically, the \mbox{MAP} for a set of queries is the mean of the \textit{average precision} (\mbox{AP}) scores for each query, with
$$\textrm{AP} = \frac{1}{L}\sum\nolimits_{r=1}\nolimits^{N}P(r)\times\delta(r),$$
where $r$ is the rank position, $N$ is the number of retrieved documents, $\delta(r)$ is a binary function that returns 1 if the document at rank position $r$ is relevant\footnote{In the experiments, an image is relevant to a text if they share the same class label and vice versa.} to the query and 0 otherwise, $P(r)$ is the precision of relevance at position $r$, and $L$ is the total number of relevant documents in the retrieved set. \mbox{MAP} is sometimes referred to geometrically as the area under the precision-recall curve for a set of queries~\cite{turpin2006sigir}. Thus a larger value of \mbox{MAP} indicates a better performance. To get the precision at Hamming radius $d$, we first retrieve all the documents which have Hamming distance at most $d$ to the query and then compute the precision of the retrieved documents. Similar to \mbox{MAP}, larger values of precision indicate better performance. In all experiments, we set the rank position $r=100$ and the Hamming radius $d=2$.

%%%%%%%%%%%%%%%%%%%%%%%%%%%%%%%%%%%%%%%
\subsection{Results} % of \mbox{SMH}
\label{smh:exps:results_smh}

In the following experiments, we randomly select $P=500$  data points from the training set as landmarks for \mbox{KSMH} and \mbox{RKSMH} and repeat the process ten times. Hence for these two methods, we report the average results with the corresponding standard deviations. Moreover, linear kernel is used, Laplacians are defined based on labels and the parameters are set to $\kappa = 10^{-4}$, $\gamma = 0.1$ for the \mbox{Wiki} data set and $\gamma=100$ for the \mbox{Flickr} data set.  Besides, to reduce computational cost on the \mbox{Flickr} data set, we use a subset of 5,000 instances from the training set to train the models but the retrieval tasks are still conducted on the whole training set.

%%%%%%%%%%%%%%%%%%%
\subsubsection{Comparison for crossmodel retrieval}
\label{smh:exps:results:cross}

We first compare the four multimodal hashing methods, i.e., \mbox{CMSSH}, \mbox{SMH}, \mbox{KSMH} and \mbox{RKSMH}, for crossmodel retrieval. The results for different code lengths $M$ on the \mbox{Wiki} data set are reported in Table~\ref{table:comp-wiki-cross-it}~\&~\ref{table:comp-wiki-cross-ti}, and those on the \mbox{Flickr} data set are reported in Table~\ref{table:comp-flickr-cross-it}~\&~\ref{table:comp-flickr-cross-ti}.

\begin{table}[htb]\small
\caption{Performance comparison for Image-Text retrieval on \mbox{Wiki}}\label{table:comp-wiki-cross-it}\vspace{-0.5cm}
\begin{center}
\begin{tabular}{|c|c|c|c|c|}
\toprule[1pt]\addlinespace[0pt]
    \multirow{2}{*}{Method}&  \multirow{2}{*}{Measure}  &  \multicolumn{3}{|c|}{Image query -- Text database}  \\
\cline{3-5}%\addlinespace[0pt]\midrule[1pt]\addlinespace[0pt]
&&$M=4$&$M=8$&$M=16$\\
\hline
\multirow{2}{*}{CMSSH}&{MAP}    &    $0.1660     $        &  $    0.1640    $ &$ 0.1751$  \\
\cline{2-5}%\addlinespace[0pt]\cmidrule[0.5pt]{2-5}\addlinespace[0pt]
&{Precision}    &     	$0.1150   $         &     $   0.1501  $         &        $  {\bf0.3487} $       \\
\hline%\addlinespace[0pt]\midrule[0.5pt]\addlinespace[0pt]
\multirow{2}{*}{SMH}&MAP        &     $0.1937 $         &     $  0.2290    $      &  $ 0.2140$ \\
\cline{2-5}%\addlinespace[0pt]\cmidrule[0.5pt]{2-5}\addlinespace[0pt]
&{Precision}    &     	$0.1252   $         &   $ {\bf 0.1640}   $            &    $ 0.2200$         \\
\hline%\addlinespace[0pt]\midrule[0.5pt]\addlinespace[0pt]
\multirow{2}{*}{KSMH}&MAP        &   $0.1909\pm 0.0032$       &  ${\bf0.2194\pm0.0052}$         & $0.2177\pm 0.0058$  \\
\cline{2-5}%\addlinespace[0pt]\cmidrule[0.5pt]{2-5}\addlinespace[0pt]
&{Precision}    &  ${\bf0.1253\pm0.0008}$         & $0.1635\pm 0.0045$              &  $0.2186\pm0.0130$\\
\hline%\addlinespace[0pt]\midrule[0.5pt]\addlinespace[0pt]
\multirow{2}{*}{RKSMH}&MAP        &     ${\bf 0.1918\pm0.0021}$         & $0.2189\pm0.0036$          & ${\bf0.2200\pm0.0046}$  \\
\cline{2-5}%\addlinespace[0pt]\cmidrule[0.5pt]{2-5}\addlinespace[0pt]
&{Precision}    &    	$0.1219\pm 0.0008        $          &    $0.1633\pm0.0032$           &    $0.2172\pm0.0084$         \\
\addlinespace[0pt]\bottomrule[1pt]
\end{tabular}
\end{center}
\end{table}

\begin{table}[htb]\small
\caption{Performance comparison for Text-Image retrieval on \mbox{Wiki}}\label{table:comp-wiki-cross-ti}\vspace{-0.5cm}
\begin{center}
\begin{tabular}{|c|c|c|c|c|}
\toprule[1pt]\addlinespace[0pt]
    \multirow{2}{*}{Method}&  \multirow{2}{*}{Measure}  &  \multicolumn{3}{|c|}{Text query -- Image database}\\
\cline{3-5}%\addlinespace[0pt]\midrule[1pt]\addlinespace[0pt]
&&$M=4$&$M=8$&$M=16$\\
\hline
\multirow{2}{*}{CMSSH}&{MAP}     &$  0.1928  $&$   0.1746 $& $     0.1950   $\\
\cline{2-5}%\addlinespace[0pt]\cmidrule[0.5pt]{2-5}\addlinespace[0pt]
&{Precision}    & $0.1142$&$   0.1389    $&$    0.1320    $\\
\hline%\addlinespace[0pt]\midrule[0.5pt]\addlinespace[0pt]
\multirow{2}{*}{SMH}&MAP        &${\bf 0.2208}       $&$ {\bf0.2784}  $&$ {\bf0.3494} $\\
\cline{2-5}%\addlinespace[0pt]\cmidrule[0.5pt]{2-5}\addlinespace[0pt]
&{Precision}    &$0.1258      $&$ {\bf0.1800}   $&$ {\bf0.3047} $\\
\hline%\addlinespace[0pt]\midrule[0.5pt]\addlinespace[0pt]
\multirow{2}{*}{KSMH}&MAP        & $0.2207\pm0.0039$& $ 0.2765\pm 0.0052 $&$ 0.3108\pm 0.0059 $\\
\cline{2-5}%\addlinespace[0pt]\cmidrule[0.5pt]{2-5}\addlinespace[0pt]
&{Precision}    & ${\bf 0.1264\pm0.0013}$&$ 0.1782\pm 0.0071 $&$ 0.2780\pm 0.0118 $\\
\hline%\addlinespace[0pt]\midrule[0.5pt]\addlinespace[0pt]
\multirow{2}{*}{RKSMH}&MAP & $0.2088\pm0.0038$&$ 0.2559\pm 0.0065 $&$ 0.3171\pm 0.0075 $\\
\cline{2-5}%\addlinespace[0pt]\cmidrule[0.5pt]{2-5}\addlinespace[0pt]
&{Precision}   &$0.1228        \pm0.0008       $&$ 0.1740\pm  0.0045 $&$ 0.2774\pm 0.0106  $\\
\addlinespace[0pt]\bottomrule[1pt]
\end{tabular}
\end{center}
\end{table}

\begin{table}[htb]\small
\caption{Performance comparison for Image-Text retrieval on \mbox{Flickr}}\label{table:comp-flickr-cross-it}\vspace{-0.5cm}
\begin{center}
\begin{tabular}{|c|c|c|c|c|}
\toprule[1pt]\addlinespace[0pt]
    \multirow{2}{*}{Method}&  \multirow{2}{*}{Measure}  &  \multicolumn{3}{|c|}{Image query -- Text database}  \\
\cline{3-5}%\addlinespace[0pt]\midrule[1pt]\addlinespace[0pt]
&&$M=4$&$M=8$&$M=16$\\
\hline
\multirow{2}{*}{CMSSH}&{MAP}    &    $0.3723  $           &  $  0.3822  $ &$  0.4100$  \\
\cline{2-5}%\addlinespace[0pt]\cmidrule[0.5pt]{2-5}\addlinespace[0pt]
&{Precision}    &     	$0.3458 $         &     $   0.3503 $         &        $   0.4104$       \\
\hline%\addlinespace[0pt]\midrule[0.5pt]\addlinespace[0pt]
\multirow{2}{*}{SMH}&MAP        &     $0.3463  $         &     $  0.3872  $      &  $  0.4159$ \\
\cline{2-5}%\addlinespace[0pt]\cmidrule[0.5pt]{2-5}\addlinespace[0pt]
&{Precision}    &     	$0.3451  $         &   $   0.4130 $            &    $  0.4100$         \\
\hline%\addlinespace[0pt]\midrule[0.5pt]\addlinespace[0pt]
\multirow{2}{*}{KSMH}&MAP        &   ${\bf0.4604 \pm 0.0116}$       &  $   0.4747 \pm  0.0136  $         & $   0.4718\pm    0.0066$\\
\cline{2-5}%\addlinespace[0pt]\cmidrule[0.5pt]{2-5}\addlinespace[0pt]
&{Precision}    &     	${\bf 0.3778  \pm0.0052}  $         & $   0.4148  \pm   0.0066  $              &  $ 0.4390\pm  0.0111$ \\
\hline%\addlinespace[0pt]\midrule[0.5pt]\addlinespace[0pt]
\multirow{2}{*}{RKSMH}&MAP        &     $0.4482 \pm 0.0125  $         & $ {\bf  0.4782   \pm  0.0052}  $          & $ {\bf 0.4865\pm   0.0037}  $ \\
\cline{2-5}%\addlinespace[0pt]\cmidrule[0.5pt]{2-5}\addlinespace[0pt]
&{Precision}    &    	$0.3709  \pm0.0038 $          &    $ {\bf 0.4185   \pm   0.0065}$           &    $ {\bf 0.4690\pm  0.0094} $\\
\addlinespace[0pt]\bottomrule[1pt]
\end{tabular}
\end{center}
\end{table}

\begin{table}[htb]\small
\caption{Performance comparison for Text-Image retrieval on \mbox{Flickr}}\label{table:comp-flickr-cross-ti}\vspace{-0.5cm}
\begin{center}
\begin{tabular}{|c|c|c|c|c|}
\toprule[1pt]\addlinespace[0pt]
    \multirow{2}{*}{Method}&  \multirow{2}{*}{Measure}  &  \multicolumn{3}{|c|}{Text query -- Image database}\\
\cline{3-5}%\addlinespace[0pt]\midrule[1pt]\addlinespace[0pt]
&&$M=4$&$M=8$&$M=16$\\
\hline
\multirow{2}{*}{CMSSH}&{MAP}      &${\bf 0.4824}  $&$   {0.4829}  $& $    0.4712 $\\
\cline{2-5}%\addlinespace[0pt]\cmidrule[0.5pt]{2-5}\addlinespace[0pt]
&{Precision}    & $0.3467 $&$   0.3616   $&$  {\bf 0.5286 } $\\
\hline%\addlinespace[0pt]\midrule[0.5pt]\addlinespace[0pt]
\multirow{2}{*}{SMH}&MAP &$0.3590  $&$   0.4056  $&$    0.4520 $\\
\cline{2-5}%\addlinespace[0pt]\cmidrule[0.5pt]{2-5}\addlinespace[0pt]
&{Precision}     &$0.3447   $&$ {\bf  0.4346 }  $&$   0.4460 $\\
\hline%\addlinespace[0pt]\midrule[0.5pt]\addlinespace[0pt]
\multirow{2}{*}{KSMH}&MAP      & $0.4683   \pm0.0146$&$    0.4849  \pm     0.0119   $&$ 0.4860\pm   0.0102 $\\
\cline{2-5}%\addlinespace[0pt]\cmidrule[0.5pt]{2-5}\addlinespace[0pt]
&{Precision}   & $  {\bf 0.3839   \pm0.0054}  $&$   0.4260\pm    0.0073     $&$   0.4537\pm  0.0114 $\\
\hline%\addlinespace[0pt]\midrule[0.5pt]\addlinespace[0pt]
\multirow{2}{*}{RKSMH}&MAP & $0.4610 \pm0.0066 $&$   {\bf 0.5013  \pm   0.0081}    $&$   {\bf 0.5098 \pm   0.0050}$\\
\cline{2-5}%\addlinespace[0pt]\cmidrule[0.5pt]{2-5}\addlinespace[0pt]
&{Precision}    &$0.3750   \pm0.0057   $&$  0.4241   \pm  0.0082    $&$  0.4751\pm  0.0141  $\\
\addlinespace[0pt]\bottomrule[1pt]
\end{tabular}
\end{center}
\end{table}

From the tables, we can see that all three \mbox{SMH} models outperform \mbox{CMSSH} by a large margin on both data sets. Among our three models, \mbox{RKSMH} performs the best on both data sets, indicating the effectiveness of \textit{Laplacian} regularization. We also note that  \mbox{KSMH} achieves performance similar to that of \mbox{SMH} on the \mbox{Wiki} data set and better performance than \mbox{SMH} on the \mbox{Flickr} data set, showing that the kernel extension is quite useful.

%%%%%%%%%%%%%%%%%%%
\subsubsection{Comparison for unimodel retrieval}
\label{smh:exps:results:uni}

In this section, we compare the four multimodal and two well-known unimodel hashing-based methods for unimodel retrieval.  It should be noted that multimodel hashing algorithms learn hash functions from both modalities whereas the unimodel hashing algorithms learn hash functions from only one modality. The results are summarized in Table~\ref{table:comp-wiki-uni-ii}~\&~\ref{table:comp-wiki-uni-tt} for the \mbox{Wiki} data set, and Table~\ref{table:comp-flickr-uni-ii}~\&~\ref{table:comp-flickr-uni-tt} for the \mbox{Flickr} data set.

\begin{table}[htb]\small
\caption{Performance comparison for image retrieval on \mbox{Wiki}}\label{table:comp-wiki-uni-ii}\vspace{-0.5cm}
\begin{center}
\begin{tabular}{|c|c|c|c|c|}
\toprule[1pt]\addlinespace[0pt]
    \multirow{2}{*}{Method}&  \multirow{2}{*}{Measure}  &  \multicolumn{3}{|c|}{Image query -- Image database}  \\
\cline{3-5}%\addlinespace[0pt]\midrule[1pt]\addlinespace[0pt]
&&$M=4$&$M=8$&$M=16$\\
\hline
\multirow{2}{*}{SH}&{MAP}    & $0.1559$   &  $0.1545$&$ 0.1552 $ \\
\cline{2-5}%
&{Precision}    &      $0.1084$         &        $0.1084$      &  $ 0.1083 $\\
\hline %\addlinespace[0pt]\midrule[0.8pt]\addlinespace[0pt]
\multirow{2}{*}{CMSSH}&{MAP}    &    $0.1640  $           &  $    0.1683   $ &$ 0.1743$  \\
\cline{2-5}%\addlinespace[0pt]\cmidrule[0.5pt]{2-5}\addlinespace[0pt]
&{Precision}    &     	$0.1139  $         &     $    0.1171   $         &        $  0.1294 $       \\
\hline%\addlinespace[0pt]\midrule[0.5pt]\addlinespace[0pt]
\multirow{2}{*}{SMH}&MAP        &     ${\bf 0.1773}  $         &     $  {\bf 0.1930} $      &  $  {\bf 0.1903}$ \\
\cline{2-5}%\addlinespace[0pt]\cmidrule[0.5pt]{2-5}\addlinespace[0pt]
&{Precision}    &     	$ {\bf 0.1178}  $         &   $ {\bf 0.1352}  $            &    $  {\bf 0.1536}$         \\
\hline%\addlinespace[0pt]\midrule[0.5pt]\addlinespace[0pt]
\multirow{2}{*}{KSMH}&MAP        &   $0.1759        \pm 0.0014    $       &  $0.1912\pm 0.0031 $         & $0.1892\pm   0.0019$ \\
\cline{2-5}%\addlinespace[0pt]\cmidrule[0.5pt]{2-5}\addlinespace[0pt]
&{Precision}    &     	$0.1173  \pm0.0004  $         & $  0.1343 \pm  0.0011  $              &  $   0.1533\pm   0.0029$       \\
\hline%\addlinespace[0pt]\midrule[0.5pt]\addlinespace[0pt]
\multirow{2}{*}{RKSMH}&MAP        &     $0.1747        \pm0.0014       $         & $0.1848\pm 0.0020$          & $0.1884\pm0.0019$  \\
\cline{2-5}%\addlinespace[0pt]\cmidrule[0.5pt]{2-5}\addlinespace[0pt]
&{Precision}    &    	$0.1155        \pm0.0006       $          &    $0.1324\pm 0.0007$           &    $0.1512\pm0.0022$         \\
\addlinespace[0pt]\bottomrule[1pt]
\end{tabular}
\end{center}
\end{table}

\begin{table}[htb]\small
\caption{Performance comparison for text retrieval on \mbox{Wiki}}\label{table:comp-wiki-uni-tt}\vspace{-0.5cm}
\begin{center}
\begin{tabular}{|c|c|c|c|c|}
\toprule[1pt]\addlinespace[0pt]
    \multirow{2}{*}{Method}&  \multirow{2}{*}{Measure}  &  \multicolumn{3}{|c|}{Text query -- Text database}\\
\cline{3-5}%\addlinespace[0pt]\midrule[1pt]\addlinespace[0pt]
&&$M=4$&$M=8$&$M=16$\\
\hline
\multirow{2}{*}{SH}&{MAP}  & ${\bf 0.3068}$ & $0.3986$&$ 0.5590 $\\
\cline{2-5}%
&{Precision}    &  $0.1084$           &$0.1086$&$ 0.1721 $\\
\hline %\addlinespace[0pt]\midrule[0.8pt]\addlinespace[0pt]
\multirow{2}{*}{CMSSH}&{MAP}   &$0.3024   $&$    0.4737     $& $  0.5364 $\\
\cline{2-5}%\addlinespace[0pt]\cmidrule[0.5pt]{2-5}\addlinespace[0pt]
&{Precision}    & $0.1739   $&$   0.2752      $&$   0.4179  $\\
\hline%\addlinespace[0pt]\midrule[0.5pt]\addlinespace[0pt]
\multirow{2}{*}{SMH}&MAP      &$0.2850      $&$ 0.4461   $&$ 0.5563  $\\
\cline{2-5}%\addlinespace[0pt]\cmidrule[0.5pt]{2-5}\addlinespace[0pt]
&{Precision}     &$0.1358      $&$ 0.2648   $&$ 0.5704 $\\
\hline%\addlinespace[0pt]\midrule[0.5pt]\addlinespace[0pt]
\multirow{2}{*}{KSMH}&MAP        & $0.3066        \pm0.0220    $&$ 0.4627\pm  0.0173 $&$ 0.5590\pm    0.0068 $\\
\cline{2-5}%\addlinespace[0pt]\cmidrule[0.5pt]{2-5}\addlinespace[0pt]
&{Precision}   & $ {\bf 0.1397   \pm0.0041 } $&$  0.2742 \pm    0.0226    $&$    0.5741\pm 0.0217 $\\
\hline%\addlinespace[0pt]\midrule[0.5pt]\addlinespace[0pt]
\multirow{2}{*}{RKSMH}&MAP       & $0.2891        \pm0.0046       $&$ {\bf 0.5078\pm  0.0046} $&$ {\bf 0.5697\pm0.0041 } $\\
\cline{2-5}%\addlinespace[0pt]\cmidrule[0.5pt]{2-5}\addlinespace[0pt]
&{Precision}    &$0.1328       \pm0.0013      $&$ {\bf 0.2786 \pm  0.0144 } $&$ {\bf 0.5927 \pm 0.0143} $\\
\addlinespace[0pt]\bottomrule[1pt]
\end{tabular}
\end{center}
\end{table}


\begin{table}[htb]\small
\caption{Performance comparison for image retrieval on \mbox{Flickr}}\label{table:comp-flickr-uni-ii}\vspace{-0.5cm}
\begin{center}
\begin{tabular}{|c|c|c|c|c|}
\toprule[1pt]\addlinespace[0pt]
    \multirow{2}{*}{Method}&  \multirow{2}{*}{Measure}  &  \multicolumn{3}{|c|}{Image query -- Image database}  \\
\cline{3-5}%\addlinespace[0pt]\midrule[1pt]\addlinespace[0pt]
&&$M=4$&$M=8$&$M=16$\\
\hline
\multirow{2}{*}{SH}&{MAP}    &     $0.3743$          &  $0.3780$ &$0.3793  $\\
\cline{2-5}%\addlinespace[0pt]\cmidrule[0.5pt]{2-5}\addlinespace[0pt]
&{Precision}    &       $0.3449$        &      $0.3449$       &$0.3449$ \\
\hline %\addlinespace[0pt]\midrule[0.8pt]\addlinespace[0pt]
\multirow{2}{*}{CMSSH}&{MAP}    &    $0.4064  $           &  $   0.4262  $ &$ 0.4304$  \\
\cline{2-5}%\addlinespace[0pt]\cmidrule[0.5pt]{2-5}\addlinespace[0pt]
&{Precision}    &     	$0.3396   $         &     $ 0.3376  $         &        $  0.3424$       \\
\hline%\addlinespace[0pt]\midrule[0.5pt]\addlinespace[0pt]
\multirow{2}{*}{SMH}&MAP        &     $0.3753 $         &     $   0.4326  $      &  $  0.4388$ \\
\cline{2-5}%\addlinespace[0pt]\cmidrule[0.5pt]{2-5}\addlinespace[0pt]
&{Precision}    &     	$0.3450 $         &   $   0.3786  $            &    $  0.4075$         \\
\hline%\addlinespace[0pt]\midrule[0.5pt]\addlinespace[0pt]
\multirow{2}{*}{KSMH}&MAP        &   ${\bf 0.4498   \pm 0.0061 } $       &  $   0.4642 \pm   0.0051  $         & $ 0.4606\pm  0.0034$  \\
\cline{2-5}%\addlinespace[0pt]\cmidrule[0.5pt]{2-5}\addlinespace[0pt]
&{Precision}    &     	${\bf 0.3746 \pm0.0059 } $         & $ {\bf 0.4095  \pm   0.0103}   $              &  $  0.4342\pm  0.0126$       \\
\hline%\addlinespace[0pt]\midrule[0.5pt]\addlinespace[0pt]
\multirow{2}{*}{RKSMH}&MAP        &     $0.4390 \pm0.0062 $         & $ {\bf  0.4726  \pm 0.0055}   $          & $ {\bf  0.4783\pm    0.0029}$  \\
\cline{2-5}%\addlinespace[0pt]\cmidrule[0.5pt]{2-5}\addlinespace[0pt]
&{Precision}    &    	$0.3668 \pm0.0050 $          &    $  0.4020   \pm   0.0070 $           &    $ {\bf  0.4403\pm   0.0082}$         \\
\addlinespace[0pt]\bottomrule[1pt]
\end{tabular}
\end{center}
\end{table}

\begin{table}[htb]\small
\caption{Performance comparison for text retrieval on \mbox{Flickr}}\label{table:comp-flickr-uni-tt}\vspace{-0.5cm}
\begin{center}
\begin{tabular}{|c|c|c|c|c|}
\toprule[1pt]\addlinespace[0pt]
    \multirow{2}{*}{Method}&  \multirow{2}{*}{Measure} &  \multicolumn{3}{|c|}{Text query -- Text database}\\
\cline{3-5}%\addlinespace[0pt]\midrule[1pt]\addlinespace[0pt]
&&$M=4$&$M=8$&$M=16$\\
\hline
\multirow{2}{*}{SH}&{MAP}    & $0.3753$ & $0.3756$&$ 0.3752 $\\
\cline{2-5}%\addlinespace[0pt]\cmidrule[0.5pt]{2-5}\addlinespace[0pt]
&{Precision}    &      $0.3449$         &$0.3449$&$0.3449$\\
\hline %\addlinespace[0pt]\midrule[0.8pt]\addlinespace[0pt]
\multirow{2}{*}{CMSSH}&{MAP}  &$0.4762 $&$   0.5197   $&$  {\bf 0.5832 } $ \\
\cline{2-5}%\addlinespace[0pt]\cmidrule[0.5pt]{2-5}\addlinespace[0pt]
&{Precision}  & $0.3824  $&$   0.3962   $&$   0.4112 $\\
\hline%\addlinespace[0pt]\midrule[0.5pt]\addlinespace[0pt]
\multirow{2}{*}{SMH}&MAP &$0.3769 $&$    0.4650   $&$   0.5031 $\\
\cline{2-5}%\addlinespace[0pt]\cmidrule[0.5pt]{2-5}\addlinespace[0pt]
&{Precision}    &$ 0.3449 $&$    0.3838   $&$   0.4356  $\\
\hline%\addlinespace[0pt]\midrule[0.5pt]\addlinespace[0pt]
\multirow{2}{*}{KSMH}&MAP       & $ {\bf 0.4866      \pm0.0135}$&$ 0.5132  \pm    0.0098    $&$ 0.5177\pm   0.0095  $\\
\cline{2-5}%\addlinespace[0pt]\cmidrule[0.5pt]{2-5}\addlinespace[0pt]
&{Precision}     & ${\bf 0.3839   \pm0.0062 } $&$    0.4342  \pm   0.0121   $& $ 0.4760\pm  0.0112  $\\
\hline%\addlinespace[0pt]\midrule[0.5pt]\addlinespace[0pt]
\multirow{2}{*}{RKSMH}&MAP      & $0.4723  \pm0.0153  $&$ {\bf 0.5245  \pm  0.0103 }     $&$   0.5441\pm  0.0068  $\\
\cline{2-5}%\addlinespace[0pt]\cmidrule[0.5pt]{2-5}\addlinespace[0pt]
&{Precision}       &$0.3777  \pm 0.0036 $&$  {\bf 0.4394 \pm   0.0073 }    $&$ {\bf  0.5117\pm  0.0127 }  $\\
\addlinespace[0pt]\bottomrule[1pt]
\end{tabular}
\end{center}
\end{table}

Similar to the results of crossmodal retrieval, our models outperform \mbox{CMSSH} on both data sets and the performance gap is larger on the \mbox{Flickr} data set. As expected, \mbox{RKSMH} achieves the best performance among our methods and \mbox{KSMH} is better than \mbox{SMH}. Note that our methods perform better than one state-of-the-art unimodal hashing methods, namely, \textit{spectral hashing}, indicating that information from other modalities can help to learn good hash codes for unimodal retrieval. As a result, \mbox{SMH}, especially \mbox{RKSMH}, is also very useful for unimodal retrieval systems.

% % % % % % % % % % % % % % % % % % % % % % % % % % % % % % %
\section{Conclusion}
\label{smh:conclusion}

In this chapter, we have proposed spectral multimodal hashing (\mbox{SMH}) under the framework of multimodal hashing, the goal of which is to perform similarity search on data of multiple modalities. \mbox{SMH} learns the hash codes through spectral analysis of the modality correlation. Experimental results show that our \mbox{SMH} model outperforms the state-of-the-art methods. %However, \mbox{SMH} has an apparent limitation, that is, it is only for the aligned data which may not be available in some applications. 

In the future, we wish to relax the data alignment assumption of SMH and develop more general multimodal hashing methods. In addition, we would like to apply SMH to other applications such as multimodal medical image registration.%In the next chapter, we propose a new multimodal hashing model for graph data which is more general than aligned data.



\chapter{Multimodal Hashing for Graph Data}
\label{chap:mlbe}

 % % % % % % % % % % % % % % % % % % % % % % % % % % % % % %
\section{Introduction}

So far as we know, existing multimodal hashing methods including CMSSH and CVH have achieved successes in several applications but they also have some apparent limitations.  First, both models can only deal with vectorial data which may not be available in some applications.  Besides, they both involve eigendecomposition operations which may be very costly especially when the data dimensionality is high.  Furthermore, \mbox{CMSSH} has been developed for shape retrieval and medical image alignment applications and \mbox{CVH} for people search applications in the natural language processing area. These applications are very different from those studied in this thesis.

Although spectral multimodal hashing has solved some of the issues mentioned above, it is also based on eigendecomposition and requires the data to be aligned. In this chapter, we study hashing-based similarity search in the context of multimodal graph data, in which pairwise similarity of data points is provided and different modalities are no longer aligned.

We propose a probabilistic latent factor model, called \textit{multimodal latent binary embedding}~(\mbox{MLBE}),
to learn hash functions for multiple modalities.  As a generative model, the hash codes are binary latent factors in a common Hamming space which determine the generation of both intra-modality and inter-modality similarities as observed either directly or indirectly.  Although getting a full Bayesian solution is intractable, we devise an efficient algorithm for learning the binary latent factors based on \textit{maximum a posteriori} (\mbox{MAP}) estimation. Compared to its counterparts~\cite{bronstein2010cvpr,kumar2011ijcai}, MLBE can:
\begin{itemize}
\item[(a)] be interpreted easily in a principled manner;
\item[(b)] be extended easily;
\item[(c)] avoid overfitting via parameter learning;
%\item accommodate both vectorial and non-vectorial data;
\item[(d)] support efficient learning algorithms.
\end{itemize}

The remainder of this chapter is organized as follows. Section~\ref{mlbe:model} presents the model formulation, the learning algorithm, some model extensions as well as discussion. Experimental validation of \mbox{MLBE} conducted using both synthetic data and two realistic data sets is presented in Section~\ref{mlbe:exps}. Finally, Section~\ref{mlbe:conclusion} concludes the chapter.

%Section~\ref{mlbe:relatedwork} briefly introduces some recent related work. 
%\section{Related Work}
%\label{mlbe:relatedwork}

%Our work bears some resemblance to metric learning which aims at learning similarity or distance measures from data~\cite{xing2002nips}.  Such data-dependent similarity measures are generally more effective than their data-independent counterparts.  Although a lot of research has been conducted on metric learning, multimodal metric learning is still largely unexplored even though multimodal data are commonly found in many applications.  Some recent efforts have been made on non-hashing-based methods~\cite{lee2009cvpr,quadrianto2011icml}.  Compared with hashing-based methods, these methods do not have the merits of low storage requirement and high search speed.

%To the best of our knowledge, Bronstein \etal proposed the first hashing-based model, called cross-modal similarity sensitive hashing (\mbox{CMSSH}) thereafter, for multimodal similarity search~\cite{bronstein2010cvpr}. Specifically, given a set of similar and dissimilar point pairs, \mbox{CMSSH} constructs two groups of linear hash functions (for the bimodal case) such that, with high probability, the Hamming distance after mapping is small for similar points and large for dissimilar points. In their formulation, each hash function (for one bit) can be obtained by solving a singular value decomposition (\mbox{SVD}) problem and the hash functions are learned sequentially in a standard boosting manner. However, \mbox{CMSSH} ignores the intra-modality relational information which could be very useful~\cite{weiss2008nips}.







%For multiple views, we should talk about it here. At least two ways.

%%%%%%%%%%%%%%%%%%%%%%%%%%%%%%%%%%%%%%%%%%%%%%%%%%%%%%%%%%%%%%%%%%%%%%%%%%%%%%%%
%\section{Multimodal Binary Reconstructive Embedding}
%\label{MH:MBRE}
%The \mbox{SMH} model introduced in last subsection requires the data points in different modalities to be paired, which might not be the case in some applications. In this section, we extend a uni-modal hashing method \mbox{BRE} to multimodal settings to given a novel method called \textit{multimodal binary reconstructive embedding} (\mbox{MBRE}), the inputs of which is pairwise distance or relations.
%
%%+++++++++++++++++++++++++++++++++++++++++++++++++++++++
%\subsection{Model}
%Let $M$ be the number of hash functions (\aka code length), $N$ be the number of data points, and $Q$ be the number of landmark points. Given two kinds of data points $\mathcal{X}$ and $\mathcal{Y}$,\footnote{Without loss of generality, here we assume there are two modalities and each modality has $N$ points.} similar to \mbox{BRE}, we define hash functions \wrt the $m$th bit for $\x\in\mathcal{X}$ and $\y\in\mathcal{Y}$, respectively, as follows,
%\begin{align}
%h_{m}(\x) = \frac{1+\sgn\left(\sum_{q=1}^{Q}W_{x}(m,q)\kappa(\x_q,\x)\right)}{2} \ \ \mbox{or} \ \ g_{m}(\x) = \frac{1+\sgn\left(\sum_{q=1}^{Q}W_{y}(m,q)\kappa(\y_q,\y)\right)}{2}\nonumber,
%\end{align}
%where $\W_{x}$ and $\W_{y}$ are two $M\times Q$ projection matrices, $\{\x_q\}_{q=1}^{Q}\subset\mathcal{X}$ and $\{\y_q\}_{q=1}^{Q}\subset\mathcal{Y}$ are landmark points for $\mathcal{X}$ and $\mathcal{Y}$ respectively, and $\kappa(\cdot,\cdot)$ is a kernel function. Note that defining hash functions this way is very common in kernel methods and brings us flexibility to work on a wide variety of data types. Therefore, given two points $\x\in\mathcal{X}$ and $\y\in\mathcal{Y}$, we denote their corresponding binary representations as $\tilde{\x}$ and $\tilde{\y}$ such that their $m$th bits can be evaluated by $\tilde{x}(m) = h_{m}(\x)$ and $\tilde{y}(m) = g_{m}(\y)$.
%
%Since in many real-world applications, it is much easier to obtain binary pairwise relationships rather than real-valued distance, here we simply define the \textit{original} distance between two points $\x_i,\x_j$ as follows,
%\begin{align}
%d(\x_i,\x_j) = \left\{ \begin{array}{ll}
%0 & \textrm{if $\x_i$ and $\x_j$ belong to the same class};\\
%1 & \textrm{otherwise},
%\end{array} \right. \nonumber%\\
%%d(\y_k,\y_l) = \left\{ \begin{array}{ll}
%%0 & \textrm{if $\y_k$ and $\y_l$ are similar};\\
%%1 & \textrm{if $\y_k$ and $\y_l$ are dissimilar},
%%\end{array} \right. \nonumber\\
%%%d(\x_i,\x_j) = \frac{1}{2}\|\x_i-\x_j\|^{2}_2, &\tilde{d}(\x_i,\x_j) = \frac{1}{M}\|\tilde{\x}_i-\tilde{\x}_j\|^{2}_2,\nonumber\\
%%%d(\y_k,\y_l) = \frac{1}{2}\|\y_k-\y_l\|^{2}_2, &\tilde{d}(\y_k,\y_l) = \frac{1}{M}\|\tilde{\y}_k-\tilde{\y}_l\|^{2}_2,\nonumber
%%d(\x_i,\y_k) = \left\{ \begin{array}{ll}
%%0 & \textrm{if $\x_i$ and $\y_k$ are similar};\\
%%1 & \textrm{if $\x_i$ and $\y_k$ are dissimilar},
%%\end{array} \right. \nonumber
%\end{align}
%$d(\y_k,\y_l)$ and $d(\x_i,\y_k)$ are defined similarly. Note that using binary values here to define distance is just a special case, and our model can accept other definitions of distance.
%
%We define the \textit{reconstructive} distance between two points as follows,
%\begin{align}
%\tilde{d}(\x_i,\x_j) = \frac{1}{M}\|\tilde{\x}_i-\tilde{\x}_j\|^{2}_2, \ \
%\tilde{d}(\y_k,\y_l) = \frac{1}{M}\|\tilde{\y}_k-\tilde{\y}_l\|^{2}_2, \ \
%\tilde{d}(\x_i,\y_k) = \frac{1}{M}\|\tilde{\x}_i-\tilde{\y}_k\|^{2}_2.\nonumber
%\end{align}
%
%Intuitively speaking, we try to find $\W_{x},\W_{y}$ such that the reconstructive distance are close to the original distance. More specifically, the goal of \mbox{MBRE} is to minimize the following objective,
%\begin{align}
%\mathcal{O}\left(\W_{x},\W_{y}\right)&=\sum_{(\x_i,\x_j)\in\mathcal{N}_{x}}\left(d(\x_i,\x_j)-\tilde{d}(\x_i,\x_j)\right)^{2}+\sum_{(\y_k,\y_l)\in\mathcal{N}_{y}}\left(d(\y_k,\y_l)-\tilde{d}(\y_k,\y_l)\right)^{2}\nonumber\\
%&+\sum_{(\x_i,\y_k)\in\mathcal{N}_{xy}}\left(d(\x_i,\y_k)-\tilde{d}(\x_i,\y_k)\right)^{2},
%\label{eqn:totalobj}
%\end{align}
%where $\mathcal{N}_{x}$ is a set of point pairs in $\mathcal{X}$, $\mathcal{N}_{y}$ is a set of point pairs in $\mathcal{Y}$ and $\mathcal{N}_{xy}$ is a set of pairs with one point in $\mathcal{X}$ and the other point in $\mathcal{Y}$. In our experiments, there are $k$ pairs for each point and so each set has size upper-bounded by $Nk$.\footnote{The total number of pairs might be smaller than $Nk$, since there might be some duplicate pairs. Besides, different sets may have different $k$ values.}  We note that the objective function of \mbox{BRE} is just the first term of that in Eqn.~(\ref{eqn:totalobj}).
%
%
%%######################################
%\subsection{Algorithm}
%To solve the above optimization problem, we adapt the coordinate descent algorithm used in~\cite{kulis2009nips} for our model. The major difference between the adapted algorithm and the original one is threefold: 1) we update all parameters sequentially but the original algorithm randomly updates only a small subset of them;\footnote{Note that original algorithm is slow to converge because of random update.} 2) we use a warm-start approach to improve the convergence rate and obtain better performance; 3) our algorithm involves more updating terms.
%
%We first introduce Lemma~\ref{lemma:updatex} as follows.
%\begin{mylem}
%Let $\bar{D}_{x}(i,j)=d(\x_i,\x_j)-\tilde{d}(\x_i,\x_j),\bar{D}_{xy}(i,k)=d(\x_i,\y_k)-\tilde{d}(\x_i,\y_k)$. Consider updating one hash function of $\mathcal{X}$ from $h_{o}$ to $h_{n}$, and let $\h_{o}$ and $\h_{n}$ be the $N\times 1$ vectors obtained by applying the old and new hash functions to each data point in $\mathcal{X}$. Furthermore, we denote the hash function of $\mathcal{Y}$ with the same bit index as $g$ and the corresponding binary vector as $\g$. Then the objective function of using $h_{n}$ instead of $h_{o}$ can be expressed as
%\begin{align}
%\mathcal{O} &= \sum_{(\x_i,\x_j)\in\mathcal{N}_{x}}\left(\bar{D}_{x}(i,j)+\frac{1}{M}(h_{o}(i)-h_{o}(j))^2-\frac{1}{M}(h_{n}(i)-h_{n}(j))^2\right)^2\nonumber\\
%&+ \sum_{(\x_i,\y_k)\in\mathcal{N}_{xy}}\left(\bar{D}_{xy}(i,k)+\frac{1-2g(k)}{M}(h_{o}(i)-h_{n}(i))\right)^2+C,
%\end{align}
%where $C$ is a constant independent of $h_{o}$ and $h_{n}$.
%\label{lemma:updatex}
%\end{mylem}
%\begin{myproof}
%Let $\tilde{\D}_{x}^{o}$ and $\tilde{\D}_{x}^{n}$ be the matrices of reconstructive distance using $h_{o}$ and $h_{n}$ respectively, $\H_{o}$ and $\H_{n}$ be the $N\times M$ matrices of old and new hash codes of $\mathcal{X}$ respectively, and $\G$ be the hash codes of $\mathcal{Y}$. Moreover, we use $\1_{t}$ to denote the $t$th standard basis vector and $\1$ to denote a vector of all ones, and their dimensionalities will be clear in the context.
%
%We can express $\tilde{\D}_{x}^{o}$ as follows,
%\begin{align}
%\tilde{\D}_{x}^{o} = \frac{1}{M}\left(\Ell_{xo}\1^{T}+\1\Ell^T_{o}-2\H_{o}\H_{o}^{T}\right)\nonumber,
%\end{align}
%where $\Ell_{xo}$ is the vector of squared norms of the rows of $\H_{o}$. Accordingly, we can express $\Ell_{xn}$ for $\H_{n}$ as $\Ell_{xn} = \Ell_{xo} - \h_{o}+\h_{n}$, since $\h_{o}$ and $\h_{n}$ are binary vectors.
%Moreover, we can easily obtain $\H_{n} = \H_{o} +(\h_{n}-\h_{o})\1^{T}_{m}$, where $m$ is the index of the hash function being updated. Therefore,
%\begin{align}
%\tilde{\D}_{x}^{n}
%&= \frac{1}{M}\left(\Ell_{xn}\1^{T}+\1\Ell_{xn}^T-2\H_{n}\H_{n}^{T}\right)\nonumber\\
%&= \frac{1}{M}\left((\Ell_{xo} - \h_{o}+\h_{n})\1^{T}+\1(\Ell_{xo} - \h_{o}+\h_{n})^{T}-2(\H_{o} +(\h_{n}-\h_{o})\1^{T}_{m})(\H_{o} +(\h_{n}-\h_{o})\1^{T}_{m})^{T}\right)\nonumber\\
%&= \tilde{\D}_{x}^{o}-\frac{1}{M}\left((\h_{o}\1^{T}+\1\h_{o}^{T}-2\h_{o}\h_{o}^{T})-(\h_{n}\1^{T}+\1\h_{n}^{T}-2\h_{n}\h_{n}^{T})\right).\nonumber
%\end{align}
%
%Similarly, we have the following cross-modal reconstructive distance matrix,
%\begin{align}
%\tilde{\D}_{xy}^{o} = \frac{1}{M}\left(\Ell_{xo}\1^{T}+\1\Ell_{y}^T-2\H_{o}\G^{T}\right)\nonumber,
%\end{align}
%where $\Ell_{y}$ is the vector of squared norms of the rows of $\G$. Therefore,
%\begin{align}
%\tilde{\D}_{xy}^{n}
%&= \frac{1}{M}\left(\Ell_{xn}\1^{T}+\1\Ell_{y}^T-2\H_{n}\G^{T}\right)\nonumber\\
%&= \frac{1}{M}\left((\Ell_{xo} - \h_{o}+\h_{n})\1^{T}+\1\Ell_{y}^{T}-2(\H_{o} +(\h_{n}-\h_{o})\1^{T}_{m})\G^{T}\right)\nonumber\\
%&= \tilde{\D}_{x}^{o}-\frac{1}{M}\left((\h_{o}\1^{T}-2\h_{o}\g^{T})-(\h_{n}\1^{T}-2\h_{n}\g^{T})\right).\nonumber
%\end{align}
%
%Thus we can write the objective function of using $h_{n}$ instead of $h_{o}$ as
%\begin{align}
%\mathcal{O} &= \sum_{(\x_i,\x_j)\in\mathcal{N}_{x}}\left(\bar{D}_{x}(i,j)+\tilde{D}_{x}^{o}(i,j)-\tilde{D}_{x}^{n}(i,j)\right)^2+\sum_{(\x_i,\y_k)\in\mathcal{N}_{xy}}\left(\bar{D}_{xy}(i,k)+\tilde{D}_{xy}^{o}(i,k)-\tilde{D}_{xy}^{n}(i,k)\right)^2\nonumber\\
%&=\sum_{(\x_i,\x_j)\in\mathcal{N}_{x}}\left(\bar{D}_{x}(i,j)+\frac{1}{M}(h_{o}(i)-h_{o}(j))^2-\frac{1}{M}(h_{n}(i)-h_{n}(j))^2\right)^2\nonumber\\
%&+\sum_{(\x_i,\y_k)\in\mathcal{N}_{xy}}\left(\bar{D}_{xy}(i,k)+\frac{1-2g(k)}{M}(h_{o}(i)-h_{n}(i))\right)^2+C,
%\end{align}
%where we have made use of $h_{o}(i)^2 = h_{o}(i)$ and $h_{n}(i)^2 = h_{n}(i)$ and grouped terms irrelevant to $h_{o},h_{n}$ into $C$. This completes the proof.
%\end{myproof}
%
%Now we move to the details of updating one element of $\W_{x}$, e.g., $W_{x}(m,q_0)$, with all the other elements in $\W_{x}$ fixed. Given a point $\x_i$, the $m$th hash code can be obtained by computing
%\begin{align}
%W_{x}(m,q_0)\kappa(\x_{q_0},\x_i)+\sum\nolimits_{q\neq q_0}W_{x}(m,q)\kappa(\x_{q},\x_i).
%\label{eqn:threshold-1bit}
%\end{align}
%Equating (\ref{eqn:threshold-1bit}) to zero, we can easily obtain the incremental value for $W_{x}(m,q_0)$ that can change the current bit of $\x_i$ as
%\begin{align}
%    \delta_{i} = \left(\sum\nolimits_{q\neq q_0}W_{x}(m,q)\kappa(\x_{q},\x_{i})\right)/\kappa(\x_{q_0},\x_{i}) - W_{x}(m,q_0).
%\end{align}
%
%If $h_m(\x_i)>0$, we should decrease $W_{x}(m,q_0)$ to flip the hash code, in another words, $\delta_i<0$. On the contrary, if $h_m(\x_i)<0$, we should increase $W_{x}(m,q_0)$ to flip the hash code, that is, $\delta_i>0$. As a result, we first find all the $\delta_{i}$'s for all $\x_{i}$'s. Then we sort $\{\delta_i\mid\delta_i>0\}$ in ascending order and $\{\delta_i\mid\delta_i<0\}$ in descending order, and thus obtain two sets of intervals. It is easy to observe that, in a fixed interval, changing $W_{x}(m,q_0)$ will not affect the hash code of any point. However, if we go across intervals, the hash code of exactly one point will be changed. As a result, starting from the current value of $W_{x}(m,q_0)$, we first increase it by adding $\delta_i+\epsilon>0$ from the smallest one to the largest one to obtain a set of possible values of objective function~(\ref{eqn:totalobj}). Note that $\epsilon$ is a very small positive number ensuring that only the $i$th bit is flipped. We then decrease $W_{x}(m,q_0)$ by adding $\delta_i-\epsilon<0$ to the starting value from the largest one to the smallest one to obtain another set of possible objective values. In total, we obtain a set of $N$ possible objective values. After getting all these values, we update $W_x(m,q_{0})$ by adding $\delta_i$ corresponding to the smallest objective $\mathcal{O}_i$ if it is smaller than original objective $\mathcal{O}$ before updating, or skip this iteration otherwise.
%
%The main idea of updating $W_{x}(m,q_0)$ is to find $\delta_i$ leading to the smallest objective function value.  We can compute the values sequentially in an efficient way based on Lemma~\ref{lemma:updateh}.
%\begin{mylem}
%Given two hash vectors $\h_{t}$ and $\h_{t-1}$ for $\mathcal{X}$ which are different in only one position, the objective w.r.t. $\h_{t}$ can be computed from that w.r.t. $\h_{t-1}$ in $O(k)$ time.
%\label{lemma:updateh}
%\end{mylem}
%\begin{myproof}
%Let the index of the point in which $\h_{t}$ and $\h_{t-1}$ are different be $a$. The only terms that change in the objective are $(\x_a,\x_j)\in\mathcal{N}_{x},(\x_i,\x_a)\in\mathcal{N}_{x}$, and $(\x_a,\y_k)\in\mathcal{N}_{xy}$. Let $f_a = 1$ if $h_{t-1}(a)=0,h_{t}(a)=1$, and $f_a=-1$ otherwise. Therefore the relevant terms in the objective function as given in Lemma~\ref{lemma:updatex} may be written as
%\begin{align}
%\mathcal{O}'&=\sum_{(\x_a,\x_j)\in\mathcal{N}_{x}}\left(\bar{D}_{x}(a,j)-\frac{f_a}{M}(1-2h_{t}(j))\right)^2+\sum_{(\x_i,\x_a)\in\mathcal{N}_{x}}\left(\bar{D}_{x}(i,a)-\frac{f_a}{M}(1-2h_{t}(i))\right)^2\nonumber\\
%&+\sum_{(\x_a,\y_k)\in\mathcal{N}_{xy}}\left(\bar{D}_{xy}(a,k)-\frac{f_a}{M}(1-2g(k))\right)^2.
%\label{eqn:updateO-1bit}\end{align}
%
%Since $\x_{a}$ has $k$ nearest neighbors and lives in the neighborhood of $k$ points on average, it costs $O(k)$ time to update the objective.
%\end{myproof}
%
%We can update each element of $\W_{y}$ similarly with the help of the following two lemmas. %Due to lack of space, we omit the proof here.
%
%\begin{mylem}
%Let $\bar{D}_{y}(k,l)=d(\y_k,\y_l)-\tilde{d}(\y_k,\y_l),\bar{D}_{xy}(i,k)=d(\x_i,\y_k)-\tilde{d}(\x_i,\y_k)$. Consider updating one hash function of $\mathcal{Y}$ from $g_{o}$ to $g_{n}$, and let $\g_{o}$ and $\g_{n}$ be the $N\times 1$ vectors obtained by applying the old and new hash functions to each data point in $\mathcal{Y}$. We further denote the hash function of $\mathcal{X}$ with the same index as $h$ and the corresponding binary vector of $\mathcal{X}$ as $\h$. Then the objective function of using $g_{n}$ instead of $g_{o}$ can be expressed as
%\begin{align}
%\mathcal{O} &= \sum_{(\y_k,\y_l)\in\mathcal{N}_{y}}\left(\bar{D}_{y}(k,l)+\frac{1}{M}(g_{o}(k)-g_{o}(l))^2-\frac{1}{M}(g_{n}(k)-g_{n}(l))^2\right)^2\nonumber\\
%&+ \sum_{(\x_i,\y_k)\in\mathcal{N}_{xy}}\left(\bar{D}_{xy}(i,k)+\frac{1-2h(i)}{M}(g_{o}(k)-g_{n}(k))\right)^2+C',
%\end{align}
%where $C'$ is a constant independent of $g_{o}$ and $g_{n}$.
%\label{lemma:updatey}
%\end{mylem}
%
%\begin{mylem}
%Given two hash vectors $\g_{t}$ and $\g_{t-1}$ for $\mathcal{Y}$ which are different in only one position, the objective w.r.t. $\g_{t}$ can be computed from that w.r.t. $\g_{t-1}$ in $O(k)$ time.
%\label{lemma:updateg}
%\end{mylem}
%
%As a result, the general procedure of our algorithm can be summarized as follows. We first initialize model parameters $\W_{x}, \W_{y}$. Then we update each element of $\W_{x}$ based on Lemma~\ref{lemma:updatex}\&\ref{lemma:updateh}, and each element of $\W_{y}$ based on Lemma~\ref{lemma:updatey}\&\ref{lemma:updateg}. This updating procedure iterates until $\W_{x}, \W_{y}$ converge. We then use current values of $\W_{x}, \W_{y}$ as initialization and retrain the model to get better $\W_{x}, \W_{y}$. In our experiments, this warm-start approach is very effective, $\W_{x}, \W_{y}$ will converge very fast to a better local optimum. To update one element of $\W_{x}$ or $\W_{y}$, sorting $N$ incremental values $\delta_i$'s needs $O(N\log N)$ time, obtaining all objective function values needs $O(Nk)$ time and finding the smallest $\mathcal{O}_i$'s needs $O(N)$ time. Putting everything together, the time complexity of updating one element is $O(N\log N+Nk)$. As a result, one full iteration of updating $\W_{x}$ or $\W_{y}$ requires $O(MQN(\log N+k))$ time.
%
%%\begin{algorithm}
%%%\DontPrintSemicolon
%%%\SetKwData{Left}{left}\SetKwData{This}{this}\SetKwData{Up}{up}
%%%\SetKwFunction{Union}{Union}\SetKwFunction{FindCompress}{FindCompress}
%%\SetKwInOut{Input}{Input}\SetKwInOut{Output}{Output}
%%
%%\Input{$\mathcal{N}_{x}, \mathcal{N}_{y}, \mathcal{N}_{xy}$.}
%%\Output{$\W_{x}, \W_{y}$.}
%%\Begin{
%%Initialize $\W_{x}, \W_{y}$.
%%\While{NOT Converge}{
%%\For{$m=1$ to $M$}{    \For{$q=1$ to $Q$}{ Update $W_{x}(m,q)$.}    }
%%\For{$m=1$ to $M$}{    \For{$q=1$ to $Q$}{ Update $W_{y}(m,q)$.}    }
%%}}
%%\caption{General procedure of coordinate descent}
%%\label{algo:cmh}
%%\end{algorithm}
%
%Note that local convergence in a finite number of updates is guaranteed since each update will never increase the objective function value which is lower-bounded by zero. Therefore, the algorithm is  efficient and can scale well even for large high-dimensional data sets.
%
% % % % % % % % % % % % % % % % % % % % % % % % % % % % % % %
%\section{Multimodal Latent Binary Embeddings}
%\label{MH:MLBE}
%
%Up to now, we have presented two models, namely, \mbox{SMH} and \mbox{MBRE}. To evaluate the data correlation, \mbox{SMH} requires paired or aligned input data which might not be easy to obtain. \mbox{MBRE} eliminates this constraint by directly finding a discrete embedding, so that the Hamming distance in the embedded space maximally approximates the original distance. In this section, we introduce an alternative multimodal hashing model to improve \mbox{SMH}, which is called \textit{multimodal latent binary embeddings} (\mbox{MLBE}), based on latent factor models. \mbox{MLBE} relates hash codes and observations of similarity, i.e., intra-modal similarity and inter-modal similarity, in a probabilistic model, and the hash codes can be learned easily by \mbox{MAP} estimation of the latent factors. %Among other things, \mbox{MLBE} can be easily extended to determine the proper length of hash codes.
%
%% the intra-modal and inter-modal similarities are generated based on latent binary factors and weighting matrices. 
%
%\subsection{Model}
%
%In the following, we focus on the bi-modal case but it is easy to extend \mbox{MLBE} to support multiple modalities. Assume we have binary latent factors for each modality, for example, $ \U \in \{+1,-1\}^{N\times K} $ for $ \X  $ and $ \V \in \{+1,-1\}^{M\times K} $ for $ \Y  $. Correspondingly, we also have two weighting matrices, $\W^{x} \in \mathbb{R}^{K\times K}$ and $ \W^{y}  \in \mathbb{R}^{K\times K}$. The basic assumption of our model is that the observations of intra-modal and inter-modal similarities are determined by the latent factors and weighting matrices. The graphical representation of \mbox{MLBE} is depicted in Figure~\ref{mlbe:fig:model}.
%
%\begin{figure}[tb]
%\centering
%\epsfig{figure=fig/mlbe/graphmodel, width=0.4\textwidth}
%\caption{Graphical representation of the model of multimodal latent binary embeddings. The shaded circles are observed variables and the empty ones are latent variables.}
%\label{mlbe:fig:model}
%\end{figure}
%
%
%Given $ \U , \V , \W^{x} $ and $ \W^{y} $, the two symmetric intra-modal similarity matrices $ \S^{x} \in \mathbb{R}^{N\times N}$ for  $ \X $ and $ \S^{y} \in \mathbb{R}^{M\times M}$ for $ \Y $ are generated from the following distributions, respectively:
%$$S^{x}_{ij} \mid \U, \W^{x}  \sim \mathcal{N}(\u_i^T\W^{x}\u_j,\theta_x^2 ), \ \ \forall i \ge j, \  i,j\in\{1,\cdots,N\}, $$
%$$S^{y}_{ij} \mid \V, \W^{y}  \sim \mathcal{N}(\v_i^T\W^{y}\v_j,\theta_y^2 ), \ \ \forall i \ge j, \  i,j\in\{1,\cdots,M\}, $$
%where $ \u_i $ and $ \u_j $ denote the $ i $th row and $ j $th row of $ \U  $. Similarly, $ \v_i $ and $ \v_j $ denote the $ i $th row and $ j $th row of $ \V  $.
%
%We also observe a inter-modal similarity matrix $ \S^{xy} \in \{1,0\}^{N\times M}$, where 1 and 0 stand for similar and dissimilar, respectively. For example, if an image and a text document are both for a historic event, we label them with 1. If they are irrelevant, we label them with 0. Note that it is quite common and easy to define inter-modal similarity using binary values $ \{1,0\} $ in practice, but our model can also accommodate other values by simply changing the distribution. We further assume only a subset of the inter-modal similarity values are observed and use an indicator matrix $ \O\in \{0,1\}^{N\times M} $ to denote this, i.e., $ O_{ij}=1 $ if $ S_{ij}^{xy} $ is observed and $ O_{ij}=0 $ otherwise. Given $ \U  $ and $ \V  $, the observed elements in $ \S^{xy} $ are generated by
%$$S^{xy}_{ij} \mid \U, \V  \sim \mbox{Bernoulli}(\sigma(\u_i^{T}\v_j)),\ \ \forall i,j, \ O_{ij}=1,$$
%where $ \sigma(x) = 1/(1+\exp(-x))$ is the logistic sigmoid function.
%
%Assume each element in $ \U\in\{+1,-1\}^{N\times K}  $ is determined identically and independently the following way,\footnote{Conventional the Bernoulli distribution is for $ \{0,1\} $ valued variables. Here, without loss of generality, we can map them to $ \{-1,+1\} $ by linear transformation.}
%\begin{align}
%\pi \mid \alpha_u,\beta_u &\sim \mbox{Beta}(\alpha_u,\beta_u),\nonumber\\
%U_{ik} \mid \pi &\sim \mbox{Bernoulli}(\pi),\nonumber
%\end{align}
%where $ \alpha_u $ and $ \beta_u $ are hyperparameters, we can integrate out $ \pi $ to give the following prior on $ \U $:
%\begin{align}
%U_{ik} \mid \alpha_u,\beta_u  \sim \mbox{Bernoulli}(\frac{\alpha_u}{\alpha_u+\beta_u}), \ \ \forall i\in\{1,\cdots,N\}, \ k\in\{1,\cdots,K\}.\nonumber
%\end{align}
%
%Similarly, we define the prior on $ \V \in\{+1,-1\}^{M\times K} $ as
%\begin{align}
%V_{ik} \mid \alpha_v,\beta_v  \sim \mbox{Bernoulli}(\frac{\alpha_v}{\alpha_v+\beta_v}), \ \ 
%\forall i\in\{1,\cdots,M\}, \ k\in\{1,\cdots,K\}.\nonumber
%\end{align}
%
%%The prior terms for $ \U  \in \{+1,-1\}^{N\times K}$ and $ \V \in \{+1,-1\}^{M\times K} $ are from ~\cite{griffiths2006nips}:
%%$$\Pr(\U) = \prod_{k=1}^K\frac{\frac{\alpha}{K}\Gamma(N_k+\frac{\alpha}{K})\Gamma(N-N_k+1)}{\Gamma(N+1+\frac{\alpha}{K})}$$
%%and
%%$$\Pr(\V) = \prod_{k=1}^K\frac{\frac{\beta}{K}\Gamma(M_k+\frac{\beta}{K})\Gamma(M-M_k+1)}{\Gamma(M+1+\frac{\beta}{K})},$$
%%where $ N_k = \sum_{i=1}^{N}\delta(U_{ik}=1) $ and $ M_k = \sum_{i=1}^{M}\delta(V_{ik}=1) $ are the number of $ 1 $'s in the $ k $th column of $ \U  $ and $ \V  $, respectively.
%
%For $ \X  $, the entries of the symmetric weight matrix $ \W^{x}\in\mathbb{R}^{K\times K} $ are generated identically and independently by a standard Gaussian distribution:
%$$\W^{x}_{ij} \mid \phi_x^2  \sim \mathcal{N}(0,\phi_x^2 ), \ \  \forall i\ge j, \ i,j\in\{1,\cdots,K\}.$$
%We put a similar prior on  $ \W^{y}\in\mathbb{R}^{K\times K} $ for $ \Y  $:
%$$\W^{y}_{ij} \mid \phi_y^2  \sim \mathcal{N}(0,\phi_y^2 ), \ \ \forall i\ge j, \ i,j\in\{1,\cdots,K\}.$$
%%We put simple matrix Gaussian prior on $ \W_x $ and $ \W_y $, which can be written as:
%%$$\Pr(\w_{x}) = \mathcal{N}(\0,\phi_x\I ), \w_{x} = \W_x(:)$$
%%$$\Pr(\w_y) = \mathcal{N}(\0,\phi_y\I ), \w_y = \W_y(:)$$
%
%\subsection{Algorithm}
%
%Based on the observations, we can learn the parameters $ \U $ and $ \V $ to give the hash codes. But finding exact posterior distributions of $ \U  $ and $ \V  $ is intractable, as a result, we adopt an alternating algorithm to find an \mbox{MAP} estimation of $ \U , \V ,\W^x $ and $ \W^y  $.
%
%We first update $ U_{ik} $ while fixing the others. To decide the \mbox{MAP} estimation of $ U_{ik} $, we first define a loss function with respect to $ U_{ik}$ as in Definition~\ref{def:lossu}:
%
%%  $ Let $ \u_i $ be the $ i $th row of $ \U  $, $\w_{x} = \W_x(:) $ and $ \s^{x}_{i} = \S_x(:,i) $, we denote $ \A_i = \mbox{kron}(\u_i, \U) $ and have $ \Pr(\s^{x}_{i}\mid \A,\w_{x}) = \mathcal{N}(\A_i\w_{x},\theta_x\I) $. The loss function of updating one element $ \U_{ik}  $:
%
%\begin{mydef}
%\begin{align}
%\mathcal{L}_{U_{ik}} &=\log\frac{\alpha_u}{\beta_u}-\frac{1}{2\theta_{x}^2}\sum_{j\neq i}^{N}\left[-2 S^{x}_{ij} \u_j^T\W^x(\u_{i}^{+} - \u_{i}^{-}) - \u_j^T\W^x(\u_{i}^{+}{\u_{i}^{+}}^{T}-\u_{i}^{-}{\u_{i}^{-}}^{T})\W^x\u_j\right]\nonumber\\
%&+\sum_{j=1}^{M}O_{ij}\left[S_{ij}^{xy}\log \frac{\sigma_{ij}^{+}}{\sigma_{ij}^{-}} + (1-S_{ij}^{xy})\log \frac{1-\sigma_{ij}^{+}}{1-\sigma_{ij}^{-}}\right],
%\end{align}
%where $ U_{-ik} $ denotes all the elements in $ \U $ but $ U_{ik} $, $ \s^{x}_i $ denotes the $ i $th row of $ \S^{x} $, $ \u^{+}_i $ is the $ i $th row of $ \U  $ with $ U_{ik}=1 $ and $ \u^{-}_i $ is the $ i $th row of $ \U $ with $ U_{ik}=-1 $. We further define $ \sigma^{+}_{ij} = \sigma(\v_j^T\u_i^{+}) $ and $ \sigma^{-}_{ij} = \sigma(\v_j^T\u_i^{-}) $.
%\label{def:lossu}\end{mydef}
%
%Then we have the following lemma:
%\begin{mylem}
%The \mbox{MAP} solution of $ U_{ik} $ is $ U_{ik}=1 $ if $ \mathcal{L}_{U_{ik}}>0 $ and $ U_{ik}=-1 $ otherwise.
%\label{lemma:updateu}\end{mylem}
%
%\begin{myproof}
%To get the \mbox{MAP} estimation of $ U_{ik} $, we only need to compare the two posterior probabilities $ \Pr(U_{ik}=1) $ and $ \Pr(U_{ik}=-1) $ conditioned on the observations and all the other model parameters. Specifically, we compute the log ratio of the two probabilities which is larger than zero if $ \Pr(U_{ik}=1) > \Pr(U_{ik}=-1)  $ and smaller than zero otherwise. The log ratio can be evaluated as follows:
%\begin{align}
% & \log \frac{\Pr(U_{ik} = 1\mid U_{-ik},\V , \W_x, \S^{x}, \S^{xy})}{\Pr(U_{ik} = -1\mid U_{-ik},\V , \W_x, \S^{x}, \S^{xy})}\nonumber\\
%=& \log \frac{\Pr(U_{ik} = 1\mid \alpha,\beta)}{\Pr(U_{ik} = -1\mid \alpha,\beta)}
%+\log \frac{\Pr(\s^{x}_i\mid U_{ik}=1, U_{-ik}, \W^{x})}{\Pr(\s^{x}_i\mid U_{ik}=-1, U_{-ik}, \W^{x})}\nonumber\\
%+&\log \frac{\Pr(\S^{xy}\mid U_{ik}=1, U_{-ik}, \V)}{\Pr(\S^{xy}\mid U_{ik}=-1, U_{-ik}, \V)}\nonumber\\
%=&\log\frac{\alpha_u}{\beta_u}-\frac{1}{2\theta_{x}^2}\sum_{j\neq i}^{N}\left[-2 S^{x}_{ij} \u_j^T\W^x(\u_{i}^{+} - \u_{i}^{-})\right]\nonumber\\
%-&\frac{1}{2\theta_{x}^2}\sum_{j\neq i}^{N}\left[\u_j^T\W^x(\u_{i}^{+}{\u_{i}^{+}}^{T}-\u_{i}^{-}{\u_{i}^{-}}^{T})\W^x\u_j\right]\nonumber\\
%+&\sum_{j=1}^{M}O_{ij}\left[S_{ij}^{xy}\log \frac{\sigma_{ij}^{+}}{\sigma_{ij}^{-}} + (1-S_{ij}^{xy})\log \frac{1-\sigma_{ij}^{+}}{1-\sigma_{ij}^{-}}\right],
%%-\frac{1}{\theta_x}\left[{\s^{x}_i}^T (\A^{-}_{i} -  \A^{+}_{i} )\w_{x}\right]\nonumber\\
%%&-\frac{1}{2\theta_x}\left[\w_{x}^T({\A^{+}_{i}}^{T}\A^{+}_{i} - {\A^{-}_{i}}^{T}\A^{-}_{i})\w_{x}\right]\nonumber\\
%%&-\frac{1}{\mu}\sum_{i,j}I_{ij}\left[S^{xy}_{ij}(\sigma^{-}_{ij}-\sigma^{+}_{ij})+\frac{1}{2}({\sigma^{+}_{ij}}^2-{\sigma^{-}_{ij}}^2)\right],
%\label{eqn:lossu}\end{align}
%where $ U_{-ik} $ denotes all the elements in $ \U $ but $ U_{ik} $, $ \s^{x}_i $ denotes the $ i $th row of $ \S^{x} $, $ \u^{+}_i $ is the $ i $th row of $ \U  $ with $ U_{ik}=1 $ and $ \u^{-}_i $ is the $ i $th row of $ \U $ with $ U_{ik}=-1 $. We further define $ \sigma^{+}_{ij} = \sigma(\v_j^T\u_i^{+}) $ and $ \sigma^{-}_{ij} = \sigma(\v_j^T\u_i^{-}) $.
%
%The log ratio computed in Eqn.~(\ref{eqn:lossu}) gives exactly $ \mathcal{L}_{U_{ik}} $, hence the proof is completed.
%\end{myproof}
%
%%The details can be found in Appendix.
%
%%We group all the terms irrelevant to $ U_{ik} $ in $ C $.
%
%%$ N_{-ik} = \sum_{j\neq i}\delta(U_{jk}=1)$ is the number of $ +1 $ in $ k $th column and all rows but the $ i $th row and $ I_{ij}=1 $ if $ \S^{xy}_{ij} $ is observed and  $ I_{ij}= 0 $ otherwise. We define $ \A^{+}_{i} = \mbox{kron}(\u_i,\U ) $ and $ \sigma^{+}_{ij} = \sigma(\u_i^T\v_j) $ with $ U_{ik}=1 $. We define $ \A^{-}_{i} = \mbox{kron}(\u_i,\U ) $ and $ \sigma^{-}_{ij} = \sigma(\u_i^T\v_j) $ with $ U_{ik}=-1 $.
%
%%
%%$ (\hat{\u}_1-\hat{\u}_2)\w_x\S_x\nonumber\\&+(\hat{\u}_1(\W_x^T\W_x))(\hat{\u}_1-\hat{\u}_2)\nonumber\\& + (\hat{\u}_2(\W_x^T\W_x))(\hat{\u}_1-\hat{\u}_2)\nonumber\\& +\sum_{j}I_{ij}(S_{ij}-\sigma(\u_i^{T}\v_j)q)^2 $
%%We can easily evaluate loss function~(\ref{eqn:loss_u}) and set
%%\begin{align}
%%U_{ik} = \left\{ \begin{array}{ll}
%%+1 & \mathcal{L}_{U_{ik}}>0\\
%%-1 & \mbox{otherwise}
%%\end{array} \right.
%%\end{align}
%
%Similarly, we have Definition~\ref{def:lossv} and Lemma~\ref{lemma:updatev} for \mbox{MAP} estimation of $ \V $. %Due to space limitations, we omit the proof here.
%
%\begin{mydef}
%\begin{align}
%\mathcal{L}_{V_{ik}} &=\log\frac{\alpha_v}{\beta_v}-\frac{1}{2\theta_{y}^2}\sum_{j\neq i}^{N}\left[-2 S^{y}_{ij} \v_j^T\W^y(\v_{i}^{+} - \v_{i}^{-}) - \v_j^T\W^y(\v_{i}^{+}{\v_{i}^{+}}^{T}-\v_{i}^{-}{\v_{i}^{-}}^{T})\W^y\v_j\right]\nonumber\\
%&+\sum_{j=1}^{N}O_{ji}\left[S_{ji}^{xy}\log \frac{\sigma_{ji}^{+}}{\sigma_{ji}^{-}} + (1-S_{ji}^{xy})\log \frac{1-\sigma_{ji}^{+}}{1-\sigma_{ji}^{-}}\right],
%\end{align}
%where $ V_{-ik} $ denotes all the elements in $ \V $ but $ V_{ik} $, $ \s^{y}_i $ denotes the $ i $th row of $ \S^{y} $, $ \v^{+}_i $ is the $ i $th row of $ \V  $ with $ V_{ik}=1 $ and $ \v^{-}_i $ is the $ i $th row of $ \V $ with $ V_{ik}=-1 $. We further define $ \sigma^{+}_{ji} = \sigma(\u_j^T\v_i^{+}) $ and $ \sigma^{-}_{ji} = \sigma(\u_j^T\v_i^{-}) $.
%\label{def:lossv}\end{mydef}
%
%
%\begin{mylem}
%The \mbox{MAP} solution of $ V_{ik} $ is $ V_{ik}=1 $ if $ \mathcal{L}_{V_{ik}}>0 $ and $ V_{ik}=-1 $ otherwise.
%\label{lemma:updatev}\end{mylem}
%
%When fixing $ \U , \V  $ and $ \W^{y} $, we compute the \mbox{MAP} estimation of $ \W^{x} $ by maximizing the following loss function:
%\begin{align}
%\mathcal{L}_{\W^{x}}&= \log P(\W^{x}) + \log P(\S^{x}_{h}\mid \U ,\W^{x})\nonumber\\
%&=\sum_{ i\ge j}^{K}\sum_{ j=1}^{K}-\frac{{W^{x}_{ij}}^2}{2\phi_x^2} + \sum_{ i > j}^{N}\sum_{ j=1}^{N}-\frac{1}{2\theta_x^2}(S^{x}_{ij}-\u_i^T\W^x\u_j)^2\nonumber\\
%&=-\frac{1}{4\phi_x^2}\w_{x}^T(\I + \mbox{diag}(\m) )\w_{x}
%-\frac{1}{2\theta_x^2}\left[(\s^{x}_h-\A_h\w_x)^T(\s^{x}_h-\A_h\w_x)\right]\nonumber\\
%&= -\frac{1}{2}\w_{x}^T \left(\A_h^T\A_h +\frac{\theta^2_x}{4\phi^2_x}\left(\I + \mbox{diag}(\m) \right) \right)\w_{x}
% + \w_{x}^T\A^T_h \s_h^x+C'
%\label{eqn:loss_wx}\end{align}
%where $ \w_x  $ is a $ K^2 $-dimensional column vector taken column-wise from $ \W^x $, $ \m $ is a $ K^2 $-dimensional indicator vector in which the value should be 1 if the index corresponds to $ W^{x}_{ii},i=1,\cdots,K $ and 0 otherwise.
%Let $ \S^{x}_{h} $ denote the left-lower half of $ \S^{x} $ and its vector form be $ \s^{x}_h $. We define $ \A = \U\otimes \U $ and $ \A_h$ consists of the rows corresponding to $ S^x_{ij}, i>j $. We group all the terms irrelevant to $ \W^{x} $ in $ C' $.\footnote{Here we have used a property of Kronnecker multiplication: $ \u^T \W \v = \w^T(\u\otimes\v)  $ where $ \w $ is a column-wise vector of $ \W $ if $ \W $ is a symmetric matrix.}
%
%% $ \s^x = \S_x(:) $, we have $ \Pr(\s^x\mid \A,\w_{x}) = \mathcal{N}(\A\w_{x},\theta_x\I) $. The loss function of updating $ \w_{x} $ is:
%%\begin{align}
%%\mathcal{L}_x = -\frac{1}{2}\w_{x} (\A^T\A +\frac{\theta_x}{\phi_x}\I )\w_{x} + \s^x\A \w_{x},
%%\end{align}
%\begin{mylem}
%The \mbox{MAP} estimation of $\W^{x}$ can be evaluated by:
%\begin{align}
%\w_{x} =\left(\A_h^T\A_h +\frac{\theta^2_x}{4\phi^2_x}\left(\I + \mbox{diag}(\m) \right)\right)^{-1}\A_h^T \s^x. \nonumber
%\end{align}
%\label{lemma:updatewx}
%\end{mylem}
%
%Note that Lemma~\ref{lemma:updatewx} can be easily proved by setting the derivative of $ \mathcal{L}_{\W^{x}} $ with respect to $ \w_{x} $ to zero.\footnote{We can adopt gradient-based algorithms to find this global maximum, which may be much faster.} Similarly, we have Lemma~\ref{lemma:updatewy} for $ \W^{y} $.
%
%\begin{mylem}
%The \mbox{MAP} estimation of $\W^{y}$ can be evaluated by:
%\begin{align}
%\w_{y} =\left(\B_h^T\B_h +\frac{\theta^2_y}{4\phi^2_y}\left(\I + \mbox{diag}(\m) \right)\right)^{-1}\B_h^T \s^y,\nonumber
%\end{align}
%where $ \w_y  $ is a $ k^2 $-dimensional column vector taken column-wise from $ \W^y $, $ \m $ is a $ k^2 $-dimensional indicator vector in which the value should be 1 if the index corresponds to $ W^{y}_{ii},i=1,\cdots,K $ and 0 otherwise.
%Let $ \S^{y}_{h} $ denote the left-lower half of $ \S^{y} $ and its vector form be $ \s^{y}_h $. We define $ \B = \V\otimes \V $ and $ \B_h$ consists of the rows corresponding to $ S^y_{ij}, i>j $.
%\label{lemma:updatewy}
%\end{mylem}
%
%%We can update $ \W^{y} $ similarly.\footnote{Updating $ \W^{x} $ and $ \W^{y} $ needs playing with a very large matrix  $ \A_h$ which might not be handled in Matlab, so we use a small but sufficient reference set in $ \X $ and $ \Y  $ to learn $ \W^{x} $ and $ \W^{y} $ and fix them to learn $ \U  $ and $ \V $ for the whole database.}
%
%%Similarly, we have $ \w_y =(\B^T\B +\frac{\theta_y}{\phi_y}\I )^{-1}\B^T \s_y $, where $ \B = \mbox{kron}(\V, \V) , \w_y = \W_y(:) $ and $ \s^y = \S_y(:) $.
%
%
%%The algorithm should work as follows:
%%1 use training data to get Wx, Wy and U, V.
%%2 fix Wx, Wy and U, V for a reference set, we then paralelly update the U and V in the test set. Each update is conducted iteratively for the elements in U or V, should be converge very fast.
%%3 use the code to do retrieval.
%
%We summarize the algorithm of \mbox{MLBE} in Algorithm~\ref{algorithm:mlbe}. In our experiments, we use the log likelihood to determine the convergence.
%
%\begin{algorithm}[!t]
%%\DontPrintSemicolon
%%\SetKwData{Left}{left}\SetKwData{This}{this}\SetKwData{Up}{up}
%%\SetKwFunction{Union}{Union}\SetKwFunction{FindCompress}{FindCompress}
%\SetKwInOut{Input}{Input}\SetKwInOut{Output}{Output}
%\Input{$\S_{x}$, $\S_{y}$, $\S_{xy}$ -- similarity matrices
%\\ $\O_{xy}$ -- similarity matrices
%\\ $M$ -- number of hash functions
%\\ $\theta_x,\theta_y, \phi_x,\phi_y, \alpha_u,\alpha_v, \beta_u, \beta_v$ -- regularization parameters}
%\Begin{
%\textit{Training phase}:\\
%   Initialize $ \U  $ and $ \V  $ with $ \{-1,+1\}$ of equal probability.
%   \While{not converge}{
%   Update each element of $ \W_x $ sequentially using Lemma~\ref{lemma:updatewx}.
%   Update $ \U  $ using Lemma~\ref{lemma:updateu}.
%   Update each element of $ \W_y $ sequentially using Lemma~\ref{lemma:updatewy}.
%   Update $ \V $ using Lemma~\ref{lemma:updatev}.
%   }
%\textit{Testing phase}:\\
%   Obtain hash codes of points $\x^{*}$ and $\y^{*}$ using Lemma~\ref{lemma:updateu} and Lemma~\ref{lemma:updatev}, respectively.
%}
%\caption{Algorithm of \mbox{MLBE}}
%\label{algorithm:mlbe}
%\end{algorithm}
%
%%\subsection{Complexity Analysis}
%%-------------------------------------------------------------------------
%
%%\section{Max margin multimodal hashing}
%%\label{MH:MMMH}
%%
%%In this section, we introduce a new model that utilize the idea of margin, which is equivalent to hinge loss. The key challenge is how to define margin in multimodal setting. And how to optimize. It will be the best if we can find some convex formulation. Nevertheless, we can use CCCP to achieve some global optimality. This should also be inspired from other embedding algorithms.
%

\section{Multimodal Latent Binary Embedding}
\label{mlbe:model}

%We present \mbox{MLBE} in detail in this section.  We use boldface uppercase and lowercase letters to denote matrices and vectors, respectively.  For a matrix $\A$, its $(i,j)$th element is denoted by $A_{ij}$.

%*******************************************************************************
\subsection{Model Formulation}

For simplicity of our presentation, we focus exclusively on the bimodal case in this chapter, but it is very easy to extend MLBE for more than two modalities. As a running example, we assume that the data come from two modalities $\mathcal{X}$ and $\mathcal{Y}$ corresponding to the image modality and text modality, respectively.

The observations in MLBE are intra-modality and inter-modality similarities.  Specifically, there are two symmetric intra-modality similarity matrices $\S^{x}\in\mathbb{R}^{I\times I}$ and $\S^{y}\in\mathbb{R}^{J\times J} $, where $I$ and $J$ denote the number of data points in modality $\mathcal{X}$ and that in modality $\mathcal{Y}$, respectively.  In case the observed data are only available in the form of feature vectors, different ways can be used to convert them into similarity matrices.  For the image data $\mathcal{X}$, the similarities in $\S^{x}$ could be computed from the corresponding Euclidean distances between feature vectors.  For the text data $\mathcal{Y}$, the similarities in $\S^{y}$ could be the cosine similarities between bag-of-words representations.

In addition, there is an inter-modality similarity matrix $\S^{xy} \in \{1,0\}^{I\times J}$, where 1 and 0 denote similar and dissimilar relationships, respectively, between the corresponding entities.  Note that it is common to specify cross-modality similarities this way, because it is very difficult if not impossible to specify real-valued cross-modality similarities objectively.  The binary similarity values in $\S^{xy}$ can often be determined based on their semantics.  Take multimedia retrieval for example, if an image and a text article are both for the same historic event, their similarity will be set to 1.  Otherwise, their similarity will be 0.

%Take another example in medical image analysis, images obtained from different measures may have similarity 1 if they are taken from the same part of the same patient, and have similarity 0 otherwise.

Our probabilistic generative model has latent variables represented by several matrices.  First, there are two sets of binary latent factors, $\U \in \{+1,-1\}^{I\times K}$ for $\mathcal{X}$ and $\V \in \{+1,-1\}^{J \times K}$ for $\mathcal{Y}$, where each row in $\U$ or $\V$ corresponds to one data point and can be interpreted as the hash code of that point.  In addition, there are two intra-modality weighting matrices, $\W^{x} \in \mathbb{R}^{K\times K}$ for $\mathcal{X}$ and $\W^{y} \in \mathbb{R}^{K\times K}$ for $\mathcal{Y}$, and an inter-modality weighting variable $w>0$.  The basic assumption of \mbox{MLBE} is that the observed intra-modality and inter-modality similarities are generated from the binary latent factors, intra-modality weighting matrices and inter-modality weighting variable.  Note that the real-valued weighting matrices and weighting variable are needed for generating the similarities because the values in the latent factors $\U$ and $\V$ are discrete.

\begin{figure}[t]
\centering
\epsfig{figure=fig/mlbe/mlbe_model, width=0.7\textwidth}
%\vspace{-0.2cm}
\caption{Graphical model representation of \mbox{MLBE}}
\label{mlbe:fig:model}
%\vspace{-0.5cm}
\end{figure}

The graphical model representation of \mbox{MLBE} is depicted in Figure~\ref{mlbe:fig:model}, in which shaded nodes are used for observed variables and empty ones for latent variables as well as parameters which are also defined as random variables.  The others are hyperparameters, which will be denoted collectively by $\Omega$ in the sequel for notational convenience.

We first consider the likelihood functions of \mbox{MLBE}. Given $\U, \V, \W^{x}, \W^{y}, \theta_{x}$ and $\theta_{y}$, the conditional probability density functions of the intra-modality similarity matrices $\S^{x}$ and $\S^{y}$ are defined as
\begin{align}
p(\S^{x} \mid \U, \W^{x},\theta_{x}) &= \prod\limits_{i=1}^{I}\prod\limits_{i'=1}^{I} \mathcal{N}(S^{x}_{ii'}\mid \u_i^T\W^{x}\u_{i'},\frac{1}{\theta_x} ),\nonumber\\
p(\S^{y} \mid \V, \W^{y}, \theta_y) &=  \prod\limits_{j=1}^{J}\prod\limits_{j'=1}^{J} \mathcal{N}(S^{y}_{jj'}\mid \v_j^T\W^{y}\v_{j'},\frac{1}{\theta_y} ), \nonumber
\end{align}
where $ \u_i $ and $ \u_{i'} $ denote the $ i $th and $ i' $th rows of $ \U  $, $ \v_i $ and $ \v_{j'} $ denote the $ j $th and $ j' $th rows of $ \V  $, and $\mathcal{N}(x\mid \mu, \sigma^2)$ is the probability density function of the univariate Gaussian distribution with mean $\mu $ and variance $\sigma^2$.

Given $ \U, \V  $ and $w$, the conditional probability mass function of the inter-modality similarity matrix $\S^{xy}$ is given by
\begin{align}
p(\S^{xy}\mid \U, \V, w ) = \prod\limits_{i=1}^{I}\prod\limits_{j=1}^{J} \left[\mbox{Bern}(S^{xy}_{ij}\mid\gamma(w\u_i^{T}\v_j))\right]^{O_{ij}},\nonumber
\end{align}
where $ \mbox{Bern}(x\mid \mu) $ is the probability mass function of the \mbox{Bernoulli} distribution with parameter $ \mu $, $O_{ij}$ is an indicator variable which is equal to $ 1 $ if $ S_{ij}^{xy} $ is observed and $ 0 $ otherwise, and $ \gamma(x) = 1/(1+\exp(-x))$ is the logistic sigmoid function to ensure that the parameter $\mu$ of the Bernoulli distribution is in the range $(0,1)$.

To complete the model formulation, we also need to define prior distributions on the latent variables and hyperprior distributions on the parameters.  For the matrix $\U$, we impose a prior independently and identically on each element of $\U$ as follows:\footnote{Conventionally, the Bernoulli distribution is defined for discrete random variables which take the value 1 for success and 0 for failure.  Here, the discrete random variables take values from $\{-1,+1\}$ instead assuming an implicit linear mapping from $\{0,1\}$.}
\begin{align}
p(U_{ik} \mid \pi_{ik})&=\mbox{Bern}(U_{ik}\mid\pi_{ik}),\nonumber\\
p(\pi_{ik} \mid \alpha_u,\beta_u) &= \mbox{Beta}(\pi_{ik}\mid\alpha_u,\beta_u),\nonumber
\end{align}
where $\mathrm{Beta}(\mu\mid a,b)$ is the probability density function of the \mbox{beta} distribution with hyperparameters $a$ and $b$. This particular choice is mainly due to the computational advantage of using conjugate distributions so that we can integrate out $ \pi_{ik} $, as a form of Bayesian averaging, to give the following prior distribution on $ \U $:
\begin{align}
p(\U \mid \alpha_u,\beta_u) = \prod\limits_{i=1}^{I}\prod\limits_{k=1}^{K}\mbox{Bern}(U_{ik}\mid \frac{\alpha_u}{\alpha_u+\beta_u}).\nonumber
\end{align}
Similarly, we define the prior distribution on $\V$ as:
\begin{align}
p(\V \mid \alpha_v,\beta_v) = \prod\limits_{j=1}^{J}\prod\limits_{k=1}^{K} \mbox{Bern}(V_{jk} \mid\frac{\alpha_v}{\alpha_v+\beta_v}). \nonumber
\end{align}

For the weighting matrices $\W^{x}$ and $\W^{y}$, we impose Gaussian prior distributions on them:
\begin{align}
p(\W^{x} \mid \phi_x)  &= \prod\limits_{k=1}^{K}\prod\limits_{d=k}^{K} \mathcal{N}(W^{x}_{kd} \mid 0,\frac{1}{\phi_x}),\nonumber\\
p(\W^{y} \mid \phi_y)  &=\prod\limits_{k=1}^{K}\prod\limits_{d=k}^{K} \mathcal{N}(W^{y}_{kd} \mid 0,\frac{1}{\phi_y}).\nonumber
\end{align}

The weighting variable $w$ has to be strictly positive to enforce a positive relationship between the inner product of two hash codes and the inter-modality similarity.  So we impose the half-normal prior distribution on $w$:
\begin{align}
p(w \mid \phi)  = \mathcal{HN}(w \mid \phi) = e^{-\frac{\phi}{2}w^2}\sqrt{\frac{2\phi}{\pi}}.\nonumber
\end{align}

Because the parameters $\theta_x, \theta_y, \phi_x, \phi_y$ and $\phi$ are all random variables, we also impose hyperprior distributions on them.  The gamma distribution is used for all these distributions:
\begin{align}
p(\theta_x\mid a_{\theta},b_{\theta}) &= \mbox{Gam}(\theta_x\mid a_{\theta},b_{\theta}),\nonumber\\
p(\theta_y\mid c_{\theta},d_{\theta}) &= \mbox{Gam}(\theta_y\mid c_{\theta},d_{\theta}),\nonumber\\
p(\phi_x\mid a_{\phi},b_{\phi}) &= \mbox{Gam}(\phi_x\mid a_{\phi},b_{\phi}),\nonumber\\
p(\phi_y\mid c_{\phi},d_{\phi}) &= \mbox{Gam}(\phi_y\mid c_{\phi},d_{\phi}),\nonumber\\
p(\phi\mid e_{\phi}, f_{\phi}) &= \mbox{Gam}(\phi\mid e_{\phi},f_{\phi}),\nonumber
\end{align}
where $\mbox{Gam}(\tau\mid a,b) = \frac{1}{\Gamma(a)}b^a\tau^{a-1}e^{-b\tau}$ denotes the probability density function of the gamma distribution with hyperparameters $a$ and $b$, and $\Gamma(\cdot)$ is the gamma function.

%*******************************************************************************
\subsection{Learning}

With the probabilistic graphical model formulated in the previous subsection, we can now devise an algorithm to learn the binary latent factors $\U$ and $\V$ which give the hash codes we need.  A fully Bayesian approach would infer the posterior distributions of $\U$ and $\V$, possibly using some sampling techniques.  However, such methods are often computationally demanding.  For computational efficiency, we devise an efficient alternating learning algorithm in this paper based on MAP estimation.


We first update $ U_{ik} $ while fixing all other variables. To find the \mbox{MAP} estimate of $ U_{ik} $, we define a loss function with respect to $ U_{ik}$ in Definition~\ref{def:lossu}:
\begin{mydef}
\begin{align}
\mathcal{L}_{ik} =&-\frac{\theta_{x}}{2}\sum_{l\neq i}^{I}\left[\left(S^{x}_{il}-\u_l^T\W^x\u_{i}^{+}\right)^2-\left(S^{x}_{il}-\u_l^T\W^x\u_{i}^{-}\right)^2\right]\nonumber\\
&-\frac{\theta_{x}}{2}\left[\left(S^{x}_{ii}-{\u_{i}^{+}}^T\W^x\u_{i}^{+}\right)^2-\left(S^{x}_{ii}-{\u_{i}^{-}}^T\W^x\u_{i}^{-}\right)^2\right]\nonumber\\
&+\sum_{j=1}^{J}O_{ij}\left[S_{ij}^{xy}\log \frac{\rho_{ij}^{+}}{\rho_{ij}^{-}} + (1-S_{ij}^{xy})\log \frac{1-\rho_{ij}^{+}}{1-\rho_{ij}^{-}}\right]+\log\frac{\alpha_u}{\beta_u},\nonumber
\end{align}
where $ \u^{+}_i $ is the $ i $th row of $ \U  $ with $ U_{ik}=1 $, $ \u^{-}_i $ is the $ i $th row of $ \U $ with $ U_{ik}=-1 $, $ \rho^{+}_{ij} = \gamma(w\v_j^T\u_i^{+}) $, and $ \rho^{-}_{ij} = \gamma(w\v_j^T\u_i^{-}) $.
\label{def:lossu}\end{mydef}

With $ \mathcal{L}_{ik} $, we can state the following theorem: %\footnote{*** You have a few theorems but no theorem.  Note that theorems are intermediate steps for theorems.}
\begin{mythe}
The \mbox{MAP} solution of $ U_{ik} $ is equal to 1 if $ \mathcal{L}_{ik}\ge 0 $ and $-1$ otherwise.
\label{theorem:updateu}\end{mythe}
\begin{myproof}
To obtain the \mbox{MAP} solution of $ U_{ik} $, it suffices to compare the following two posterior probabilities:
\begin{align}
p_{+} &= \Pr(U_{ik}=1\mid U_{-ik},\V , \W^{x}, w,\S^{x}, \S^{xy},\theta_{x}),\nonumber\\
p_{-} &= \Pr(U_{ik}=-1 \mid U_{-ik},\V , \W^{x}, w,\S^{x}, \S^{xy},\theta_{x}).\nonumber
\end{align}
Specifically, we compute the log ratio of the two probabilities, which is larger than or equal to zero if $ p_{+} \ge p_{-} $ and smaller than zero otherwise. The log ratio can be evaluated as follows:
\begin{align}
\lefteqn{\log \frac{\Pr(U_{ik} = 1\mid U_{-ik},\V , \W^{x}, w,\S^{x}, \S^{xy},\theta_{x})}{\Pr(U_{ik} = -1\mid U_{-ik},\V , \W^{x}, w \S^{x}, \S^{xy},\theta_{x})}}\nonumber\\ & \ \ \ \ \ \ 
= \log \frac{\Pr(\S^{x}\mid U_{ik}=1, U_{-ik}, \W^{x},\theta_{x})}{\Pr(\S^{x}\mid U_{ik}=-1, U_{-ik}, \W^{x},\theta_{x})}\nonumber\\ & \ \ \ \ \ \ 
+\log \frac{\Pr(\S^{xy}\mid U_{ik}=1, U_{-ik}, w,\V)}{\Pr(\S^{xy}\mid U_{ik}=-1, U_{-ik}, w,\V)}\nonumber\\& \ \ \ \ \ \ 
+ \log \frac{\Pr(U_{ik} = 1\mid \alpha_{u},\beta_{u})}{\Pr(U_{ik} = -1\mid \alpha_{u},\beta_{u})}
\nonumber\\ & \ \ \ \ \ \ 
=-\frac{\theta_{x}}{2}\sum_{l\neq i}^{I}\left[\left(S_{il}-\u_l^T\W^x\u_{i}^{+}\right)^2-\left(S_{il}-\u_l^T\W^x\u_{i}^{-}\right)^2\right]\nonumber\\ & \ \ \ \ \ \ 
-\frac{\theta_{x}}{2}\left[\left(S_{ii}-{\u_{i}^{+}}^T\W^x\u_{i}^{+}\right)^2-\left(S_{ii}-{\u_{i}^{-}}^T\W^x\u_{i}^{-}\right)^2\right]\nonumber\\ & \ \ \ \ \ \ 
+\sum_{j=1}^{J}O_{ij}\left[S_{ij}^{xy}\log \frac{\rho_{ij}^{+}}{\rho_{ij}^{-}} + (1-S_{ij}^{xy})\log \frac{1-\rho_{ij}^{+}}{1-\rho_{ij}^{-}}\right]+\log\frac{\alpha_u}{\beta_u}\nonumber,
\end{align}
where $ U_{-ik} $ denotes all the elements in $ \U $ except $ U_{ik} $. The log ratio thus computed gives exactly $ \mathcal{L}_{ik} $.  This completes the proof.
\end{myproof}
%The proof of Theorem~\ref{theorem:updateu} can be found in Appendix~\ref{appd:proofu}.

Similarly, we have Definition~\ref{def:lossv} and Theorem~\ref{theorem:updatev} for the \mbox{MAP} estimation of $ \V $. The proof of Theorem~\ref{theorem:updatev} is similar and so it is omitted in the paper due to page limitations.

\begin{mydef}
\begin{align}
\mathcal{Q}_{jl} =&-\frac{\theta_{y}}{2}\sum_{l\neq j}^{J}\left[\left(S^{y}_{jl}-\v_l^T\W^y\v_{j}^{+}\right)^2-\left(S^{y}_{jl}-\v_l^T\W^y\v_{j}^{-}\right)^2\right]\nonumber\\
&-\frac{\theta_{y}}{2}\left[\left(S^{y}_{jj}-{\v_{j}^{+}}^T\W^y\v_{j}^{+}\right)^2-\left(S^{y}_{jj}-{\v_{j}^{-}}^T\W^y\v_{j}^{-}\right)^2\right]\nonumber\\
&+\sum_{i=1}^{I}O_{ij}\left[S_{ij}^{xy}\log \frac{\lambda_{ij}^{+}}{\lambda_{ij}^{-}} + (1-S_{ij}^{xy})\log \frac{1-\lambda_{ij}^{+}}{1-\lambda_{ij}^{-}}\right]+\log\frac{\alpha_v}{\beta_v},\nonumber
\end{align}
where $ \v^{+}_j $ is the $ j $th row of $ \V  $ with $ V_{jk}=1 $, $ \v^{-}_j $ is the $ j $th row of $ \V $ with $ V_{jk}=-1 $, $ \lambda^{+}_{ij} = \gamma(w\u_i^T\v_j^{+}) $, and $ \lambda^{-}_{ij} = \gamma(w\u_i^T\v_i^{-}) $.
\label{def:lossv}\end{mydef}


\begin{mythe}
The \mbox{MAP} solution of $ V_{ik} $ is equal to 1 if $ \mathcal{Q}_{jl}\ge 0$ and $-1$ otherwise.
\label{theorem:updatev}\end{mythe}

With $ \U , \phi_{x}  $ and $ \theta_{x} $ fixed, we can compute the \mbox{MAP} estimate of $ \W^{x} $ using Theorem~\ref{theorem:updatewx} below. %The proof is in Appendix~\ref{appd:proofw}.


\begin{mythe}
The \mbox{MAP} solution of $\W^{x}$ is
\begin{align}
\w^{x} =\left(\A^T\M_2\A +\frac{\phi_x}{\theta_x}\M_1\right)^{-1}\A^T\M_2\s^x, \nonumber
\end{align}
where $ \w^{x}  $ is a $ K^2 $-dimensional column vector taken in a columnwise manner from $ \W^x $, $ \s^{x} $ is an $ I^2 $-dimensional column vector taken in a columnwise manner from $ \S^{x} $,  $ \A = \U\otimes \U $, $\M_1$ is a diagonal matrix with each diagonal entry equal to 1 if it is the linear index of the upper-right portion of $\W^{x}$ and 0 otherwise, and $\M_2$ is similarly defined but with a different size which is determined by $\S^{x}$.
\label{theorem:updatewx}
\end{mythe}
\begin{proof}
The negative log of the posterior distribution of $ \W^{x} $ can be written as:
\begin{align}
\label{eqn:loss_wx}
\lefteqn{-\log p(\W^{x}\mid \S^{x}, \U,\theta_{x},\phi_{x} )} \\ & \ \ \ \ \ 
=  -\log P(\W^{x}\mid \phi_{x}) - \log P(\S^{x}\mid \U ,\W^{x},\theta_{x}) + \tilde{C}\nonumber\\ & \ \ \ \ \ 
=\frac{\phi_x}{2}\sum_{ k=1}^{K}\sum_{ d=k}^{K}(W^{x}_{kd})^2 + \frac{\theta_x}{2}\sum_{ i=1}^{I}\sum_{ i'=i}^{I}\left(S^{x}_{ii'}-\u_i^T\W^x\u_{i'}\right)^2+\tilde{C}\nonumber\\ & \ \ \ \ \ 
=\frac{\phi_x}{2}{\w^{x}}^T\M_1\w^{x}
+\frac{\theta_x}{2}\left(\s^{x}-\A\w^{x}\right)^T\M_2\left(\s^{x}-\A\w^{x}\right)+\tilde{C}\nonumber\\ & \ \ \ \ \ 
= \frac{1}{2}{\w^{x}}^T \left(\theta_x\A^T\M_2\A+\phi_x\M_1\right)\w^{x}
-\theta_x {\s^x}^T\M_2\A\w^{x} +\tilde{C},\nonumber
\end{align}
%where $ \w^{x}  $ is a $ K^2 $-dimensional column vector taken column-wisely from $ \W^x $, $ \s^{x} $ is a $ I^2 $-dimensional column vector taken column-wisely from $ \S^{x} $, $ \A = \U\otimes \U $, $\M_1$ is a diagonal matrix in which the entries are equal to 1 if they are the linear indices of the upper-right part of $\W^{x}$, and $\M_2$ is defined similar to $\M_1$ but with different size, which is determined by $\S^{x}$.\footnote{*** There is no need to define them again.}
where $ \tilde{C} $ is a constant term independent of $ \W^{x} $.


Setting the derivative of Equation~(\ref{eqn:loss_wx}) to zero, we get
\begin{align}
\w^{x} =\left(\A^T\M_2\A +\frac{\phi_x}{\theta_x}\M_1\right)^{-1}\A^T\M_2\s^x. \nonumber
\end{align}
This completes the proof.
\end{proof}


Similarly, we have Theorem~\ref{theorem:updatewy} for $ \W^{y} $.


\begin{mythe}
The \mbox{MAP} solution of $\W^{y}$ is
\begin{align}
\w^{y} =\left(\B^T\tilde{\M}_2\B +\frac{\phi_y}{\theta_y}\M_1\right)^{-1}\B^T\tilde{\M}_2\s^y,\nonumber
\end{align}
%where $ \w^{y}  $ is a $ K^2 $-dimensional column vector taken column-wisely from $ \W^y $, $ \s^{y} $ is a $ J^2 $-dimensional column vector taken column-wisely from $ \S^{y} $, $ \B = \V\otimes \V $ and $\tilde{\M}_2$ is defined similar to $\M_2$ but with different size, which is determined by $\S^{y}$.
where $\w^{y}, \s^{y}, \B$ and $\tilde{\M}_2$ are also defined similarly.
\label{theorem:updatewy}
\end{mythe}

To obtain the \mbox{MAP} estimate of $w$, we minimize the negative log posterior $p(w\mid \U, \V, \S^{xy},\phi)$, which is equivalent to the following objective function:
\begin{align}
\mathcal{L}_{w} &= \frac{\phi }{2}w^2-\sum\limits_{i=1}^{I}\sum\limits_{j=1}^{J}\left\{O_{ij}\left[S^{xy}_{ij}\log\lambda_{ij} + \left(1-S^{xy}_{ij}\right)\log\left(1-\lambda_{ij}\right)\right]\right\},\nonumber
\end{align}
where $\lambda_{ij} = \gamma\left(w\u_{i}^{T}\v_{j}\right)$.

Although the objective function is convex with respect to $w$, there is no closed-form solution.  Nevertheless, due to its convexity, we can obtain the global minimum easily using a gradient descent algorithm.  The gradient can be evaluated as follows: %\footnote{*** There is an additional (hidden) parameter which is the learning rate of the gradient descent algorithm.}
\begin{align}
\label{equation:updatew}
\nabla w &=\phi \cdot w -\sum\limits_{i=1}^{I}\sum\limits_{j=1}^{J}\left\{O_{ij}\left[S^{xy}_{ij}\left(1-\lambda_{ij}\right)\u_{i}^{T}\v_{j}-\left(1-S^{xy}_{ij}\right)\lambda_{ij}\u_{i}^{T}\v_{j}\right]\right\}.
\end{align}

As for the parameters, closed-form solutions exist for their \mbox{MAP} estimates which are summarized in the following theorem.
\begin{mythe}
\label{theorem:updatehyper}
The \mbox{MAP} estimates of $\theta_{x},\theta_{y},\phi_{x},\phi_{y}$ and $\phi$ are:
\begin{align}
\theta_{x} &=\frac{I(I+1)+4(a_{\theta}-1)}{4b_{\theta}+2\sum\nolimits_{i=1}^{I}\sum\nolimits_{i'=i}^{I}\left(S^{x}_{ii'}-\u_i^{T}\W^{x}\u_{i'}\right)^2},\label{theorem:updatethetax}\\
\theta_{y} &=\frac{J(J+1)+4(c_{\theta}-1)}{4d_{\theta}+2\sum\nolimits_{j=1}^{J}\sum\nolimits_{j'=j}^{J}\left(S^{y}_{jj'}-\v_j^{T}\W^{y}\v_{j'}\right)^2},\label{theorem:updatethetay}\\
\phi_{x} &=\frac{K(K+1)+4(a_{\phi}-1)}{4b_{\phi}+2\sum\nolimits_{k=1}^{K}\sum\nolimits_{d=k}^{K}\left(W^{x}_{kd}\right)^2},\label{theorem:updatephix}\\
\phi_{y} &=\frac{K(K+1)+4(c_{\phi}-1)}{4d_{\phi}+2\sum\nolimits_{k=1}^{K}\sum\nolimits_{d=k}^{K}\left(W^{y}_{kd}\right)^2},\label{theorem:updatephiy}\\
\phi &=\frac{2e_{\phi}-1}{2f_{\phi}+w^2}.\label{theorem:updatephi}
\end{align}

\end{mythe}

Theorem~\ref{theorem:updatehyper} can be proved easily. Briefly speaking, we first find the posterior distribution of each parameter and then compute the optimal value by setting its derivative to zero.  Details of the proof are omitted here.

To summarize, the learning algorithm of \mbox{MLBE} is presented in Algorithm~\ref{algorithm:mlbe}.

%, where we use $\Omega = \{\alpha_u,\alpha_v, \beta_u, \beta_v, a_{\phi}, b_{\phi},c_{\phi},d_{\phi},e_{\phi},f_{\phi},a_{\theta},b_{\theta},c_{\theta},d_{\theta}\}$ to denote all the hyperparameters.

\begin{algorithm}[ht]
\caption{Learning algorithm of \mbox{MLBE}}
\label{algorithm:mlbe}
\begin{algorithmic}
\STATE {\bfseries Input:} \\
$\S^{x}$, $\S^{y}$, $\S^{xy}$ -- similarity matrices
\\ $\O$ -- observation indicator variables for $\S^{xy}$
\\ $K$ -- number of hash functions
\\ $\Omega$ --  hyperparameters
\STATE {\bfseries Procedure:} \\
%\textit{Training stage}:\\
   \STATE  Initialize all the latent variables and parameters except $ \W^{x},\W^{y} $.
   \WHILE{not converged}
   \STATE Update $ \W^{x} $ using Theorem~\ref{theorem:updatewx}.
   \STATE Update $\phi_x$ using Equation~(\ref{theorem:updatephix}).
   \STATE Update each element of $ \U  $ using Theorem~\ref{theorem:updateu}.
   \STATE Update $\theta_x$ using Equation~(\ref{theorem:updatethetax}).
   \STATE Update $ \W^y $ using Theorem~\ref{theorem:updatewy}.
   \STATE Update $\phi_y$ using Equation~(\ref{theorem:updatephiy}).
   \STATE Update each element of $ \V $ using Theorem~\ref{theorem:updatev}.
   \STATE Update $\theta_y$ using Equation~(\ref{theorem:updatethetay}).
   \STATE Update $w$ by gradient descent using Equation~(\ref{equation:updatew}).
   \STATE Update $\phi$ using Equation~(\ref{theorem:updatephi}).
	\ENDWHILE

\end{algorithmic}
\end{algorithm}
\vspace{-0.2cm}
%*******************************************************************************
\subsection{Out-of-Sample Extension}

Algorithm~\ref{algorithm:mlbe} tells us how to learn the hash functions for the observed bimodal data based on their intra-modality and inter-modality similarities.  However, the hash codes can only be computed this way for the training data. In many applications, after learning the hash functions, it is necessary to obtain the hash codes for out-of-sample data points as well.  One naive approach would be to incorporate the out-of-sample points into the original training set and then learn the hash functions from scratch.  However, this approach is computationally unappealing due to its high computational cost especially when the training set is large.

In this subsection, we propose a simple yet very effective method for finding the hash codes of out-of-sample points.  The method is based on a simple and natural assumption that the latent variables and parameters for the training data can be fixed while computing the hash codes for the out-of-sample points.

Specifically, we first train the \mbox{MLBE} model using some 
training data selected from both modalities.\footnote{We do not make any assumption on how the training data are selected.  They may be selected randomly for simplicity or carefully based on how representative they are.  Random selection is used in our experiments.}
Using the latent variables and parameters learned from the training points, we can find the hash codes for the out-of-sample points by applying Theorem~\ref{theorem:updateu} or Theorem~\ref{theorem:updatev}.  For illustration, Algorithm~\ref{algorithm:mlbe-ext} shows how to compute the hash code for an out-of-sample point $\x^{*}$ from modality $\mathcal{X}$ using the latent variables and parameters learned from two training sets $\hat{\mathcal{X} }$ and $\hat{\mathcal{Y} }$.  It is worth noting that the hash code for each out-of-sample point can be computed independently.  The implication is that the algorithm is highly parallelizable, making it potentially applicable to very large data sets.  The same can also be done to out-of-sample points from the other modality with some straightforward modifications.

%\footnote{*** The terminology mess has not been sorted out: training data, landmark points, landmark sets, out-of-sample points, new point, etc.}

\begin{algorithm}[ht]
\caption{Algorithm for out-of-sample extension}
\label{algorithm:mlbe-ext} %
\begin{algorithmic}
\STATE {\bfseries Input:} \\
$\hat{\S}^{x}$ -- intra-modality similarities for $\x^{*}$ and $\hat{\mathcal{X} }$
\\$\hat{\S}^{xy}$ -- inter-modality similarities for $\x^{*}$ and $\hat{\mathcal{Y} }$
\\$\hat{\U},\hat{\V},\hat{\W}^{x},\hat{w},\hat{\theta}_x $ -- learned variables
\\$\alpha_u,\beta_u$ --  hyperparameters
\STATE {\bfseries Procedure:}
   \STATE Initialize $ \u^{*}  $.
   \WHILE{not converged}
   \STATE Update each element of $ \u^{*}  $ using Theorem~\ref{theorem:updateu}.
   \ENDWHILE

\end{algorithmic}
\end{algorithm}
\vspace{-0.3cm}

%*******************************************************************************
\subsection{Complexity Analysis}

The computational cost of the above learning algorithm is mainly spent on updating $\U ,\V ,\W^{x}, \W^{y},$ and $w$.

The complexity of updating an entry $U_{ik}$ is $O(IK^{2}+JK)$, which grows linearly with the number of points in each modality. Updating $\W^{x}$ requires inverting a $K^2\times K^2$ matrix. Since $K$ is usually very small, this step can be performed efficiently. The complexity of evaluating the gradient $\nabla w$ is linear in the number of observations of the inter-modality similarities. We note that the complexity can be greatly reduced if the similarity matrices are sparse, which is often the case in real applications.

%*******************************************************************************
\subsection{Discussion}
\label{MH:exps:disc}
Our work is closely related to binary latent factor models~\cite{meeds2006nips,heller2007aistats} but there exist some significant differences.  First, the binary latent factors in MLBE are used as hash codes for multimodal similarity search, while the latent factors in~\cite{meeds2006nips,heller2007aistats} are treated as cluster membership indicators which are used for clustering and link prediction applications. Moreover, the model formulations are very different. In MLBE, the prior distributions on the binary latent factors are simple Bernoulli distributions, but in~\cite{meeds2006nips,heller2007aistats}, the priors on the binary latent factors are Indian buffet processes~\cite{griffiths2005nips}. Furthermore, from a matrix factorization point of view, MLBE simultaneously factorizes multiple matrices but the method in~\cite{meeds2006nips} factorizes only one matrix.

%===============================================================================
\section{Experiments}
\label{mlbe:exps}

We first present an illustrative example on synthetic data in Section~\ref{sec:exps:synthetic}.  It is then followed by experiments on two publicly available real-world data sets.  Section~\ref{sec:exps:settings} presents the experimental settings and then Sections~\ref{sec:exps:wiki} and \ref{sec:exps:flickr} present the results.

%Section~\ref{sec:exps:settings} presents the experimental settings. After that, experimental results on two public real data sets are presented and discussed in Section~\ref{sec:exps:wiki} and Section~\ref{sec:exps:flickr}, respectively.

%*******************************************************************************
\subsection{Illustration on Synthetic Data}
\label{sec:exps:synthetic}

There are four groups of data points with each group consisting of 50 points.  We associate each group with one of four hash codes, namely, $ [1, 1, -1, -1] $, $ [-1, -1, 1, 1] $, $ [1, -1, 1, -1] $ and $ [-1, 1, -1, 1] $, and use
a $200\times 4 $ matrix $\H $ to denote the hash codes of all 200 points.  We generate $ \W^{x} $ and $ \W^{y} $ from $ \mathcal{N}(\cdot\mid 1, 0.01) $ and  $ \mathcal{N}(\cdot\mid 5, 0.01) $, respectively.  Based on the latent representation, we generate the similarity matrices $ \S^{x}$ and $ \S^{y} $ using $ \mathcal{N}(S^{x}_{il} \mid \h_{i}^{T}\W^{x}\h_{l}, 0.01) $ and $ \mathcal{N}(S^{y}_{jl} \mid \h_{i}^{T}\W^{y}\h_{l}, 0.01) $, respectively. Moreover, we set $ S^{xy}_{ij} = 1 $ if $  \h_{i}=\h_j $ and $ S^{xy}_{ij} = 0$ otherwise, assuming that all entries in $ \S^{xy}$ are observed, i.e., $ O_{ij} = 1, \forall i,j$.

Based on the similarities generated, we train \mbox{MLBE} to obtain the hash codes $\U$ and $\V$.  Because the bits of the hash codes are interchangeable, it is more appropriate to use inner products of the hash codes to illustrate the similarity structures, as shown in Figure~\ref{mlbe:fig:toy-compare}. Note that the whiter the area is, the more similar the points involved are. Figure~\ref{mlbe:fig:toy:truth} depicts the ground-truth similarity structure, Figure~\ref{mlbe:fig:toy:uu} and~\ref{mlbe:fig:toy:vv} show the learned intra-modality similarity structure for each modality, and Figure~\ref{mlbe:fig:toy:uv} shows the learned inter-modality similarity structure.  As we can see, the whiter areas in the last three subfigures are in the same locations as those in Figure~\ref{mlbe:fig:toy:truth}.  In other words, both the intra-modality and inter-modality similarity structures revealed by the learned hash codes are very close to the ground truth, showing the effectiveness of \mbox{MLBE}. %\footnote{*** I don't know how to interpret properly.  In what sense is it `close'?  If the goal is to learn the hash codes and ground-truth hash codes are available, I wonder if it makes more sense to compare the learned and ground-truth hash codes?}

\begin{figure}[t]
\subfigure[$\H\H^T$]{\label{mlbe:fig:toy:truth}
    \begin{minipage}[b]{0.5\linewidth}\vspace{-0.4cm}
        \centering %\vspace{-1cm}
        \epsfig{figure=fig/mlbe/synthetic-hh, width=0.8\textwidth} %\vspace{-1.5cm}
    \end{minipage}}
\subfigure[$\U\U^T$]{\label{mlbe:fig:toy:uu}
    \begin{minipage}[b]{0.5\linewidth}\vspace{-0.4cm}
        \centering %\vspace{-1cm}
        \epsfig{figure=fig/mlbe/synthetic-uu, width=0.8\textwidth} %\vspace{-1.5cm}
    \end{minipage}}
\\\subfigure[$\V\V^T$]{\label{mlbe:fig:toy:vv}
    \begin{minipage}[b]{0.5\linewidth}\vspace{-0.4cm}
        \centering %\vspace{-1.5cm}
        \epsfig{figure=fig/mlbe/synthetic-vv, width=0.8\textwidth} %\vspace{-1.5cm}
    \end{minipage}}
\subfigure[$\U\V^T$]{\label{mlbe:fig:toy:uv}
    \begin{minipage}[b]{0.5\linewidth}\vspace{-0.4cm}
        \centering %\vspace{-1.5cm}
        \epsfig{figure=fig/mlbe/synthetic-uv, width=0.8\textwidth} %\vspace{-1.5cm}
    \end{minipage}}
\vspace{-0.3cm}
\caption{Illustration of \mbox{MLBE}}
\label{mlbe:fig:toy-compare}
\vspace{-0.4cm}
\end{figure}


%*******************************************************************************
\subsection{Experimental Settings}
\label{sec:exps:settings}

We have conducted several comparative experiments on two real-world data sets, which are, to the best of our knowledge, the largest publicly available multimodal data sets that are fully paired and labeled.  Both data sets are bimodal with the image and text modalities but the feature representations are different.  Each data set is partitioned into a database set and a separate query set.

We compare \mbox{MLBE} with \mbox{CMSSH}\footnote{The implementation is kindly provided by the authors.} and \mbox{CVH}\footnote{Because the code is not publicly available, we implemented the method ourselves.} on two common crossmodal retrieval tasks. Specifically, we use a text query to retrieve similar images from the image database and use an image query to retrieve similar texts from the text database. Since the data sets are fully labeled, meaning that each document (image or text) has one or more semantic labels, it is convenient to use these labels to decide the ground-truth neighbors. 
%All models are trained on the same landmark set random selected from the database set.

We use \textit{mean Average Precision} (\mbox{mAP}) as the performance measure. Given a query and a set of $R$ retrieved documents, the \textit{Average Precision} (\mbox{AP}) is defined as
\begin{align}
\mbox{AP} = \frac{1}{L}\sum\nolimits_{r=1}\nolimits^{R} P(r) \, \delta(r),\nonumber
\end{align}
where $L$ is the number of true neighbors in the retrieved set, $P(r)$ denotes the precision of the top $r$ retrieved documents, and $\delta(r)=1$ if the $r$th retrieved document is a true neighbor and $\delta(r)=0$ otherwise. We then average the \mbox{AP} values over all the queries in the query set to obtain the \mbox{mAP} measure. The larger the \mbox{mAP}, the better the performance. In the experiments, we set $R=50$.

We also report two types of performance curves, namely, \textit{precision-recall} curve and \textit{recall} curve. Given a query set and a database, both curves can be obtained by varying the Hamming radius of the retrieved points and evaluating the precision, recall and number of retrieved points accordingly.

For \mbox{MLBE}, the intra-modality similarity matrices are computed based on the feature vectors. We first compute the Euclidean distance $d$ between two feature vectors and then transform it into a similarity measure $s = e^{-d^2/2\sigma^2} $, where the parameter $\sigma^2$ is fixed to 1 for both data sets. The inter-modality similarity matrices are simply determined by the class labels. Since \mbox{MLBE} is not sensitive to the hyperparameters, we simply set all of them to 1. Besides, we initialize $\U $ and $\V$ using the results of \mbox{CVH}, set the initial values of $\theta_{x},\theta_{y},\phi_{x},\phi_{y}$ to $0.01$, and use a fixed learning rate $10^{-4}$ for updating $w$.

In all the experiments, the size of the training set, which is randomly selected from the database set for each modality, is set to 300 and only 0.1\% of the random selected entries of $\S^{xy}$ are observed.\footnote{We have tried larger training sets, e.g., of sizes $500$ and $1{,}000$, in our experiments but there is no significant performance improvement. So we omit the results due to space limitations.} To be fair, all three models are trained on the same training set. %\footnote{*** One dimension is missing in the experiments, which is to vary the size of the training/landmark set.}

%*******************************************************************************
\subsection{Results on \mbox{Wiki} Data Set}
\label{sec:exps:wiki}

The \mbox{Wiki} data set is generated from a group of $2{,}866$ wikipedia documents provided by~\cite{rasiwasia2010mm}. Each document is an image-text pair and is labeled with exactly one of 10 semantic classes. The images are represented by 128-dimensional \mbox{SIFT}~\cite{lowe2004ijcv} feature vectors.  The text articles are represented by the probability distributions over 10 topics, which are derived from a latent Dirichlet allocation (\mbox{LDA}) model~\cite{blei2003jmlr}. We use 80\% of the data as the database set and the remaining 20\% to form the query set.

%The results of \mbox{CMSSH} are based on 5 repetitions.

The \mbox{mAP} values for \mbox{MLBE} and the two baselines are reported in Table~\ref{mlbe:table:wiki-compare-map}.  The \textit{precision-recall} curves and \textit{recall} curves for the three methods are plotted in Figure~\ref{mlbe:fig:wiki-compare-curve}.

\begin{table}[htb] %\vspace{-0.4cm}
\caption{\mbox{mAP} comparison on \mbox{Wiki}}\label{mlbe:table:wiki-compare-map}\vspace{-0.5cm}
\begin{center}
\begin{tabular}{|c|c|c|c|c|}
\toprule[1pt]\addlinespace[0pt]
\multirow{2}{7em}{\centering Task}&\multirow{2}{4em}{\centering Method}&\multicolumn{3}{|c|}{Code Length}\\
\cline{3-5}
& &  $K=8$&  $K=16$&  $K=24$\\
\addlinespace[0pt]\midrule[1pt]\addlinespace[0pt]
\multirow{3}{7em}{\centering Image Query \\ vs. \\Text Database}
&\mbox{MLBE}&${\bf 0.3810}$&${\bf0.2561}$&${\bf0.1915}$\\
\cline{2-5}
&\mbox{CVH}&${{0.2592}}$&${0.2190}$&${0.1767}$\\
\cline{2-5}
&\mbox{CMSSH}&${0.2438}$&${0.2014}$&${0.1757}$\\
\addlinespace[0pt]\midrule[0.7pt]\addlinespace[0pt]
\multirow{3}{7em}{\centering Text Query \\ vs. \\Image Database}
&\mbox{MLBE}&${\bf{0.4955}}$&${\bf0.3209}$&${0.2143}$\\
\cline{2-5}
&\mbox{CVH}&${0.3474}$&${0.3094}$&${\bf 0.2576}$\\
\cline{2-5}
&\mbox{CMSSH}&${0.2044}$&${0.2286}$&${0.2256}$\\
\addlinespace[0pt]\bottomrule[1pt]
\end{tabular}
\end{center}
\end{table}
%\vspace{-0.5cm}

\begin{figure}[htb]
\subfigure[$K=8$]{
    \begin{minipage}[b]{0.32\linewidth}\vspace{-0.3cm}
%        \centering\vspace{-1cm}
        \epsfig{figure=fig/mlbe/comp-pr-xy-8b, width=1.0\textwidth} %\vspace{-1.5cm}
    \end{minipage}}
\subfigure[$K=16$]{
    \begin{minipage}[b]{0.32\linewidth}\vspace{-0.3cm}
%        \centering\vspace{-1cm}
        \epsfig{figure=fig/mlbe/comp-pr-xy-16b, width=1.0\textwidth} %\vspace{-1.5cm}
    \end{minipage}}
\subfigure[$K=24$]{
    \begin{minipage}[b]{0.32\linewidth}\vspace{-0.3cm}
%        \centering\vspace{-1cm}
        \epsfig{figure=fig/mlbe/comp-pr-xy-24b, width=1.0\textwidth} %\vspace{-1.5cm}
    \end{minipage}}
\\        
\subfigure[$K=8$]{
    \begin{minipage}[b]{0.32\linewidth}\vspace{-0.3cm}
%        \centering\vspace{-1cm}
        \epsfig{figure=fig/mlbe/comp-pr-yx-8b, width=1.0\textwidth} %\vspace{-1.5cm}
    \end{minipage}}
\subfigure[$K=16$]{
    \begin{minipage}[b]{0.32\linewidth}\vspace{-0.3cm}
%        \centering\vspace{-1cm}
        \epsfig{figure=fig/mlbe/comp-pr-yx-16b, width=1.0\textwidth} %\vspace{-1.5cm}
    \end{minipage}}
\subfigure[$K=24$]{
    \begin{minipage}[b]{0.32\linewidth}\vspace{-0.3cm}
%        \centering\vspace{-1cm}
        \epsfig{figure=fig/mlbe/comp-pr-yx-24b, width=1.0\textwidth} %\vspace{-1.5cm}
    \end{minipage}}
\\
\subfigure[$K=8$]{
    \begin{minipage}[b]{0.32\linewidth}\vspace{-0.3cm}
%        \centering\vspace{-1cm}
        \epsfig{figure=fig/mlbe/comp-rec-xy-8b, width=1.0\textwidth} %\vspace{-1.5cm}
    \end{minipage}}
\subfigure[$K=16$]{
    \begin{minipage}[b]{0.32\linewidth}\vspace{-0.3cm}
%        \centering\vspace{-1cm}
        \epsfig{figure=fig/mlbe/comp-rec-xy-16b, width=1.0\textwidth} %\vspace{-1.5cm}
    \end{minipage}}
\subfigure[$K=24$]{
    \begin{minipage}[b]{0.32\linewidth}\vspace{-0.3cm}
%        \centering\vspace{-1cm}
        \epsfig{figure=fig/mlbe/comp-rec-xy-24b, width=1.0\textwidth} %\vspace{-1.5cm}
    \end{minipage}}
\\        
\subfigure[$K=8$]{
    \begin{minipage}[b]{0.32\linewidth}\vspace{-0.3cm}
%        \centering\vspace{-1cm}
        \epsfig{figure=fig/mlbe/comp-rec-yx-8b, width=1.0\textwidth} %\vspace{-1.5cm}
    \end{minipage}}
\subfigure[$K=16$]{
    \begin{minipage}[b]{0.32\linewidth}\vspace{-0.3cm}
%        \centering\vspace{-1cm}
        \epsfig{figure=fig/mlbe/comp-rec-yx-16b, width=1.0\textwidth} %\vspace{-1.5cm}
    \end{minipage}}
\subfigure[$K=24$]{
    \begin{minipage}[b]{0.32\linewidth}\vspace{-0.3cm}
%        \centering\vspace{-1cm}
        \epsfig{figure=fig/mlbe/comp-rec-yx-24b, width=1.0\textwidth} %\vspace{-1.5cm}
    \end{minipage}}\vspace{-0.3cm}
\caption{Precision-recall curves and recall curves on \mbox{Wiki}}
\label{mlbe:fig:wiki-compare-curve} \vspace{-0.4cm}
\end{figure}

We can see that \mbox{MLBE} significantly outperforms \mbox{CVH} and \mbox{CMSSH} when the code length is small.  As the code length increases, the performance gap gets smaller. We conjecture that \mbox{MLBE} may get trapped in local minima during the learning process when the code length is too large. %\footnote{*** If this is the case, it is a limitation of the learning algorithm.}

Besides, we observe that as the code length increases, the performance of all three methods degrades.  We note that this phenomenon has also been observed in~\cite{wang2010cvpr,liu2011icml}.  A possible reason is that the learned models will be farther from the optimal solutions when the code length gets larger. %\footnote{*** This sounds weird.  In any case the speculation has not been verified.  Moreover, one may question why this problem does not arise in other matrix factorization methods such as for CF.}

%characteristics of the data.\footnote{*** This sounds vague to me.}
%In most cases, small code length is good enough.
%*******************************************************************************
\subsection{Results on \mbox{Flickr} Data Set}
\label{sec:exps:flickr}




The \mbox{Flickr} data set consists of $186{,}577$ image-tag pairs, which are pruned from the \mbox{NUS} dataset~\cite{nus-wide-civr09} by keeping the pairs that belong to one of the 10 largest classes. Each pair is annotated by at least one of 10 labels. The image features are 500-dimensional \mbox{SIFT} features and the text features are 1000-dimensional vectors obtained by performing \mbox{PCA} on the original tag occurrence features. We use 99\% of the data as the database set and the remaining 1\% to form the query set.

\begin{table}[ht] %\vspace{-0.4cm}
\caption{\mbox{mAP} comparison on \mbox{Flickr}}\label{mlbe:table:flickr-compare-map}\vspace{-0.5cm}
\begin{center}
\begin{tabular}{|c|c|c|c|c|}
\toprule[1pt]\addlinespace[0pt]
\multirow{2}{7em}{\centering Task}&\multirow{2}{4em}{\centering Method}&\multicolumn{3}{|c|}{Code Length}\\
\cline{3-5}
& &  $K=8$&  $K=16$&  $K=24$\\
\addlinespace[0pt]\midrule[1pt]\addlinespace[0pt]
\multirow{3}{7em}{\centering Image Query \\ vs. \\Text Database}
&\mbox{MLBE}&${\bf 0.6322}$&${\bf 0.6608 }$&${0.5104}$\\
\cline{2-5}
&\mbox{CVH}&${{0.5361}}$&${0.4871}$&${0.4605}$\\
\cline{2-5}
&\mbox{CMSSH}&${0.5155}$&${0.5333}$&${\bf 0.5150}$\\
\addlinespace[0pt]\midrule[0.7pt]\addlinespace[0pt]
\multirow{3}{7em}{\centering Text Query \\ vs. \\Image Database}
&\mbox{MLBE}&${\bf{0.5626}}$&${\bf0.5970}$&${0.4296}$\\
\cline{2-5}
&\mbox{CVH}&${0.5260}$&${0.4856}$&${\bf 0.4553}$\\
\cline{2-5}
&\mbox{CMSSH}&${0.5093}$&${0.4594}$&${0.4053}$\\
\addlinespace[0pt]\bottomrule[1pt]
\end{tabular}
\end{center}
\end{table}
%\vspace{-0.4cm}


The \mbox{mAP} results are reported in Table~\ref{mlbe:table:flickr-compare-map}. Similar to the results on \mbox{Wiki}, we observe that \mbox{MLBE} outperforms its counterparts by a large margin when the code length is small.



The corresponding \textit{precision-recall} curves and \textit{recall} curves are plotted in Figure~\ref{mlbe:fig:flickr-compare-curve}.  We note that \mbox{MLBE} has the best overall performance.




\begin{figure}[htb]
\subfigure[$K=8$]{
    \begin{minipage}[b]{0.32\linewidth}\vspace{-0.3cm}
%        \centering\vspace{-1cm}
        \epsfig{figure=fig/mlbe/flickr-pr-xy-8b, width=1.0\textwidth} %\vspace{-1.5cm}
    \end{minipage}}
\subfigure[$K=16$]{
    \begin{minipage}[b]{0.32\linewidth}\vspace{-0.3cm}
%        \centering\vspace{-1cm}
        \epsfig{figure=fig/mlbe/flickr-pr-xy-16b, width=1.0\textwidth} %\vspace{-1.5cm}
    \end{minipage}}
\subfigure[$K=24$]{
    \begin{minipage}[b]{0.32\linewidth}\vspace{-0.3cm}
%        \centering\vspace{-1cm}
        \epsfig{figure=fig/mlbe/flickr-pr-xy-24b, width=1.0\textwidth} %\vspace{-1.5cm}
    \end{minipage}}
\\    
\subfigure[$K=8$]{
    \begin{minipage}[b]{0.32\linewidth}\vspace{-0.3cm}
%        \centering\vspace{-1cm}
        \epsfig{figure=fig/mlbe/flickr-pr-yx-8b, width=1.0\textwidth} %\vspace{-1.5cm}
    \end{minipage}}
\subfigure[$K=16$]{
    \begin{minipage}[b]{0.32\linewidth}\vspace{-0.3cm}
%        \centering\vspace{-1cm}
        \epsfig{figure=fig/mlbe/flickr-pr-yx-16b, width=1.0\textwidth} %\vspace{-1.5cm}
    \end{minipage}}
\subfigure[$K=24$]{
    \begin{minipage}[b]{0.32\linewidth}\vspace{-0.3cm}
%        \centering\vspace{-1cm}
        \epsfig{figure=fig/mlbe/flickr-pr-yx-24b, width=1.0\textwidth} %\vspace{-1.5cm}
    \end{minipage}}
\\
\subfigure[$K=8$]{
    \begin{minipage}[b]{0.32\linewidth}\vspace{-0.3cm}
%        \centering\vspace{-1cm}
        \epsfig{figure=fig/mlbe/flickr-rec-xy-8b, width=1.0\textwidth} %\vspace{-1.5cm}
    \end{minipage}}
\subfigure[$K=16$]{
    \begin{minipage}[b]{0.32\linewidth}\vspace{-0.3cm}
%        \centering\vspace{-1cm}
        \epsfig{figure=fig/mlbe/flickr-rec-xy-16b, width=1.0\textwidth} %\vspace{-1.5cm}
    \end{minipage}}
\subfigure[$K=24$]{
    \begin{minipage}[b]{0.32\linewidth}\vspace{-0.3cm}
%        \centering\vspace{-1cm}
        \epsfig{figure=fig/mlbe/flickr-rec-xy-24b, width=1.0\textwidth} %\vspace{-1.5cm}
    \end{minipage}}
\\
\subfigure[$K=8$]{
    \begin{minipage}[b]{0.32\linewidth}\vspace{-0.3cm}
%        \centering\vspace{-1cm}
        \epsfig{figure=fig/mlbe/flickr-rec-yx-8b, width=1.0\textwidth} %\vspace{-1.5cm}
    \end{minipage}}
\subfigure[$K=16$]{
    \begin{minipage}[b]{0.32\linewidth}\vspace{-0.3cm}
%        \centering\vspace{-1cm}
        \epsfig{figure=fig/mlbe/flickr-rec-yx-16b, width=1.0\textwidth} %\vspace{-1.5cm}
    \end{minipage}}
\subfigure[$K=24$]{
    \begin{minipage}[b]{0.32\linewidth}\vspace{-0.3cm}
%        \centering\vspace{-1cm}
        \epsfig{figure=fig/mlbe/flickr-rec-yx-24b, width=1.0\textwidth} %\vspace{-1.5cm}
    \end{minipage}}\vspace{-0.3cm}
\caption{Precision-recall curves and recall curves on \mbox{Flickr}}
\label{mlbe:fig:flickr-compare-curve} %\vspace{-0.4cm}
\end{figure}

%*******************************************************************************
%\subsection{Results on \mbox{Flickr1M} Data Set}
%\label{sec:exps:flickr1m}
%
%The data set is described here.
%
%
%
%
%The \mbox{mAP} results are reported in Table~\ref{mlbe:table:flickr1m-compare-map}. Similar to the results on \mbox{Wiki}, we observe that \mbox{MLBE} outperforms its counterparts by a large margin when the code length is small.
%
%\begin{table}[ht] %\vspace{-0.4cm}
%\caption{\mbox{mAP} comparison on \mbox{Flickr}}\label{mlbe:table:flickr1m-compare-map}
%\begin{center}
%\begin{tabular}{|c|c|c|c|c|}
%\toprule[1pt]\addlinespace[0pt]
%\multirow{2}{7em}{\centering Task}&\multirow{2}{4em}{\centering Method}&\multicolumn{3}{|c|}{Code Length}\\
%\cline{3-5}
%& &  $K=8$&  $K=16$&  $K=24$\\
%\addlinespace[0pt]\midrule[1pt]\addlinespace[0pt]
%\multirow{3}{7em}{\centering Image Query \\ vs. \\Text Database}
%&\mbox{MLBE}&${\bf 0.6322}$&${\bf 0.6608 }$&${0.5104}$\\
%\cline{2-5}
%&\mbox{CVH}&${{0.5361}}$&${0.4871}$&${0.4605}$\\
%\cline{2-5}
%&\mbox{CMSSH}&${0.5155}$&${0.5333}$&${\bf 0.5150}$\\
%\addlinespace[0pt]\midrule[0.7pt]\addlinespace[0pt]
%\multirow{3}{7em}{\centering Text Query \\ vs. \\Image Database}
%&\mbox{MLBE}&${\bf{0.5626}}$&${\bf0.5970}$&${0.4296}$\\
%\cline{2-5}
%&\mbox{CVH}&${0.5260}$&${0.4856}$&${\bf 0.4553}$\\
%\cline{2-5}
%&\mbox{CMSSH}&${0.5093}$&${0.4594}$&${0.4053}$\\
%\addlinespace[0pt]\bottomrule[1pt]
%\end{tabular}
%\end{center}
%\end{table}
%%\vspace{-0.4cm}
%
%The corresponding \textit{precision-recall} curves and \textit{recall} curves are plotted in Figure~\ref{mlbe:fig:flickr1m-compare-curve}.  We note that \mbox{MLBE} has the best overall performance.
%
%
%
%
%\begin{figure}[htb]
%\subfigure[$K=8$]{
%    \begin{minipage}[b]{0.32\linewidth}\vspace{-0.3cm}
%%        \centering\vspace{-1cm}
%        \epsfig{figure=fig/mlbe/flickr-pr-xy-8b, width=1.0\textwidth} %\vspace{-1.5cm}
%    \end{minipage}}
%\subfigure[$K=8$]{
%    \begin{minipage}[b]{0.32\linewidth}\vspace{-0.3cm}
%%        \centering\vspace{-1cm}
%        \epsfig{figure=fig/mlbe/flickr-pr-yx-8b, width=1.0\textwidth} %\vspace{-1.5cm}
%    \end{minipage}}
%\subfigure[$K=8$]{
%    \begin{minipage}[b]{0.32\linewidth}\vspace{-0.3cm}
%%        \centering\vspace{-1cm}
%        \epsfig{figure=fig/mlbe/flickr-rec-xy-8b, width=1.0\textwidth} %\vspace{-1.5cm}
%    \end{minipage}}
%\subfigure[$K=8$]{
%    \begin{minipage}[b]{0.32\linewidth}\vspace{-0.3cm}
%%        \centering\vspace{-1cm}
%        \epsfig{figure=fig/mlbe/flickr-rec-yx-8b, width=1.0\textwidth} %\vspace{-1.5cm}
%    \end{minipage}}
%    \\
%\subfigure[$K=16$]{
%    \begin{minipage}[b]{0.32\linewidth}\vspace{-0.3cm}
%%        \centering\vspace{-1cm}
%        \epsfig{figure=fig/mlbe/flickr-pr-xy-16b, width=1.0\textwidth} %\vspace{-1.5cm}
%    \end{minipage}}
%\subfigure[$K=16$]{
%    \begin{minipage}[b]{0.32\linewidth}\vspace{-0.3cm}
%%        \centering\vspace{-1cm}
%        \epsfig{figure=fig/mlbe/flickr-pr-yx-16b, width=1.0\textwidth} %\vspace{-1.5cm}
%    \end{minipage}}
%\subfigure[$K=16$]{
%    \begin{minipage}[b]{0.32\linewidth}\vspace{-0.3cm}
%%        \centering\vspace{-1cm}
%        \epsfig{figure=fig/mlbe/flickr-rec-xy-16b, width=1.0\textwidth} %\vspace{-1.5cm}
%    \end{minipage}}
%\subfigure[$K=16$]{
%    \begin{minipage}[b]{0.32\linewidth}\vspace{-0.3cm}
%%        \centering\vspace{-1cm}
%        \epsfig{figure=fig/mlbe/flickr-rec-yx-16b, width=1.0\textwidth} %\vspace{-1.5cm}
%    \end{minipage}}
%    \\
%\subfigure[$K=24$]{
%    \begin{minipage}[b]{0.32\linewidth}\vspace{-0.3cm}
%%        \centering\vspace{-1cm}
%        \epsfig{figure=fig/mlbe/flickr-pr-xy-24b, width=1.0\textwidth} %\vspace{-1.5cm}
%    \end{minipage}}
%\subfigure[$K=24$]{
%    \begin{minipage}[b]{0.32\linewidth}\vspace{-0.3cm}
%%        \centering\vspace{-1cm}
%        \epsfig{figure=fig/mlbe/flickr-pr-yx-24b, width=1.0\textwidth} %\vspace{-1.5cm}
%    \end{minipage}}
%\subfigure[$K=24$]{
%    \begin{minipage}[b]{0.32\linewidth}\vspace{-0.3cm}
%%        \centering\vspace{-1cm}
%        \epsfig{figure=fig/mlbe/flickr-rec-xy-24b, width=1.0\textwidth} %\vspace{-1.5cm}
%    \end{minipage}}
%\subfigure[$K=24$]{
%    \begin{minipage}[b]{0.32\linewidth}\vspace{-0.3cm}
%%        \centering\vspace{-1cm}
%        \epsfig{figure=fig/mlbe/flickr-rec-yx-24b, width=1.0\textwidth} %\vspace{-1.5cm}
%    \end{minipage}}\vspace{-0.3cm}
%\caption{Precision-recall curves and recall curves on \mbox{Flickr1M}}\label{mlbe:fig:flickr1m-compare-curve} %\vspace{-0.4cm}
%\end{figure}

% % % % % % % % % % % % % % % % % % % % % % % % % % % % % % %
\section{Conclusion}
\label{mlbe:conclusion}

In this chapter, we have proposed a novel probabilistic model, called multimodal latent binary embedding (MLBE), for multimodal hash function learning for graph data.  As a latent factor model, MLBE regards the binary latent factors as hash codes and hence maps data points from multiple modalities to a common Hamming space in a principled manner.  Although finding exact posterior distributions of the latent factors is intractable, we have devised an efficient alternating learning algorithm based on \mbox{MAP} estimation. Experimental results show that MLBE compares favorably with two state-of-the-art models. 

For our future research, we will go beyond the point estimation approach presented in this paper to explore a more Bayesian treatment based on variational inference for enhanced robustness and efficiency. We would also like to extend \mbox{MLBE} to determine the code length $K$ automatically from data.  This is an important yet largely unaddressed issue in existing methods. Moreover, we also plan to apply \mbox{MLBE} to other tasks such as multimodal medical image registration.

%In next chapter, we move forward from graph data to general data and present a new multimodal hashing method.


\chapter{Multimodal Hashing for General Data}
\label{chap:crh}

%-------------------------------------------------------------------------------
\section{Introduction}

As stated earlier, the \mbox{SMH} model and the \mbox{MLBE} model require the data points in different modalities to be aligned or organized in graphs. However, these assumptions might not be the case in some applications. 

In this chapter, we present a novel model for data in general form. Specifically, we develop Co-Regularized Hashing (\mbox{CRH}), which is based on a boosted co-regularization framework. For each bit of the hash codes, \mbox{CRH} learns a group of hash functions, one for each modality, by minimizing a novel loss function. Although the loss function is non-convex, it is in a special form which can be expressed as a difference of convex functions.  As a consequence, the Concave-Convex Procedure~(CCCP)~\cite{yuille2001nips}
can be applied to solve the optimization problem iteratively.  We use a stochastic gradient method in each \mbox{CCCP} iteration. After learning the hash functions for one bit, \mbox{CRH} proceeds to learn more bits via a boosting procedure such that the bias introduced by the hash functions can be sequentially minimized.

In the following, we present the \mbox{CRH} model in Section~\ref{crh:model}. Empirical study conducted on three real data sets is reported in Section~\ref{crh:exps} before the conclusion in Section~\ref{crh:conclusion}.

%-------------------------------------------------------------------------------
\section{Co-Regularized Hashing}
\label{crh:model}

%We use boldface lowercase letters and calligraphic letters to denote vectors and sets, respectively. For a vector $\x $, $\x^{T} $ denotes its transpose and $\|\x\| $ its $\ell_2$ norm. 


%*******************************************************************************
\subsection{Objective Function}

Suppose that there are two sets of data points from two modalities,\footnote{For simplicity of our presentation, we focus on the bimodal case here and leave the discussion on extension to more than two modalities to Section~\ref{sec:moel:ext}.}
e.g., $\{\x_{i}\in\mathcal{X}\}_{i=1}^{I}$ for a set of $I$ images from some feature space $\mathcal{X}$ and $\{\y_{j}\in\mathcal{Y}\}_{j=1}^{J}$ for a set of $J$ textual documents from another feature space $\mathcal{Y}$. We also have a set of $N$ inter-modality point pairs $\Theta = \{(\x_{a_1},\y_{b_1}), (\x_{a_2},\y_{b_2}),\dots, (\x_{a_N},\y_{b_N})\}$, where, for the $n$th pair, ${a_n}$ and $b_{n}$ are indices of the points in $\mathcal{X}$ and $\mathcal{Y}$, respectively. We further assume that each pair has a label $s_{n} = 1$ if $\x_{a_n}$ and $\y_{b_n}$ are similar and $s_{n}=0$ otherwise.  The notion of inter-modality similarity varies from application to application.  For example, if an image includes a tiger and a textual document is a research paper on tigers, they should be labeled as similar.  On the other hand, it is highly unlikely to label the image as similar to a textual document on basketball.
%\footnote{*** In general, you should write it as $\Theta = \{(\x_{a_1},\y_{b_1}), (\x_{a_2},\y_{b_2}),\dots, (\x_{a_N},\y_{b_N})\}$.}

For each bit of the hash codes, we define two linear hash functions as follows:
\begin{align}
f(\x ) = \sgn(\w_{x}^{T}\x) \ \ \mbox{and} \ \ g(\y ) &= \sgn(\w_{y}^{T}\y),\nonumber
\end{align}
where $\sgn(\cdot)$ denotes the sign function, and $\w_{x}$ and $\w_{y}$ are projection vectors which, ideally, should map similar points to the same hash bin and dissimilar points to different bins.  Our goal is to achieve \mbox{HFL} by learning $\w_{x}$ and $\w_{y}$ from the multimodal data.

To achieve this goal, we propose to minimize the following objective function \wrt~(with respect to) $\w_{x}$ and $\w_{y}$:
\begin{align}
\mathcal{O}= \frac{1}{I}\sum\limits_{i=1}^{I}\ell_{i}^{x}+\frac{1}{J}\sum\limits_{j=1}^{J}\ell_{j}^{y}+\gamma\sum_{n=1}^{N}\omega_{n}\ell_{n}^{*}+\frac{\lambda_{x}}{2}\|\w_{x}\|^{2}+\frac{\lambda_{y}}{2}\|\w_{y}\|^{2},
\label{eqn:loss}
\end{align}
where $\ell_{i}^{x}$ and $\ell_{j}^{y}$ are intra-modality loss terms for modalities $\mathcal{X}$ and $\mathcal{Y}$, respectively. In this work, we define them as:
\begin{align}
\ell_{i}^{x}&=\big[1-f(\x_i)(\w_{x}^{T}\x_i)\big]_{+}=\big[1-|\w_{x}^{T}\x_{i}|\big]_{+},\nonumber\\
\ell_{j}^{y}&=\big[1-g(\y_j)(\w_{y}^{T}\y_j)\big]_{+}=\big[1-|\w_{y}^{T}\y_{j}|\big]_{+},\nonumber
\end{align}
where $[a]_{+}$ is equal to $a$ if $a \ge 0$ and 0 otherwise.  We note that the intra-modality loss terms are similar to the hinge loss in the (linear) support vector machine but have quite different meaning. Conceptually, we want the projected values to be far away from 0 and hence expect the hash functions learned to have good generalization ability~\cite{mu2010cvpr}.
For the inter-modality loss term $\ell_{n}^{*}$, we associate with each point pair a weight $\omega_{n}$, with $\sum\nolimits_{n=1}^{N}\omega_n=1$, to normalize the loss as well as compute the bias of the hash functions. In this paper, we define $\ell_{n}^{*}$ as
\begin{align}
\ell_{n}^{*} = s_{n}d_{n}^{2}+(1-s_{n})\tau(d_{n}),\nonumber
\end{align}
where $d_{n} = \w_{x}^{T}\x_{a_n}-\w_{y}^{T}\y_{b_n}$ and $\tau(d)$ is called the smoothly clipped inverted squared deviation (\mbox{SCISD}) function. 


%\footnote{*** But the nature of the problem is different.  You are not dealing with a classification problem here.  Implicitly you want the projected values to be far away from 0.  It is only in this sense that it is similar to the maximum margin criterion.}

The \mbox{SCISD} function was first proposed in~\cite{quadrianto2011icml}.  It can be defined as follows:
\begin{align}
\tau(d) & =\left\{ \begin{array}{ll}
         -\frac{1}{2}d^{2}+\frac{a\lambda^2}{2} & \mbox{if ~} |d| \le \lambda \\
         \frac{d^{2}-2a\lambda|d|+a^2\lambda^2}{2(a-1)}& \mbox{if ~}\lambda < |d|\le a\lambda\\
         0 & \mbox{if ~} a\lambda<|d|,\nonumber
                          \end{array} \right.
\end{align}
where $a$ and $\lambda$ are two user-specified parameters.  The \mbox{SCISD} function penalizes projection vectors that result in small distance between dissimilar points after projection.  A more important property is that it can be expressed as a difference of two convex functions.  Specifically, we can express $\tau(d) = \tau_{1}(d) - \tau_{2}(d)$ where
\begin{align}
\tau_{1}(d) = \left\{ \begin{array}{ll}
         0 & \mbox{if ~}|d| \le \lambda \\
         \frac{ad^{2}-2a\lambda|d|+a\lambda^2}{2(a-1)}& \mbox{if ~}\lambda < |d|\le a\lambda\\
         \frac{1}{2}d^{2}-\frac{a\lambda^2}{2} & \mbox{if ~} a\lambda<|d|\nonumber
                          \end{array} \right.\ \mbox{and}  \ \ 
\tau_{2}(d) = \frac{1}{2}d^{2}-\frac{a\lambda^2}{2}.\nonumber
\end{align}
%Obviously, both $\tau_{1}(\cdot)$ and $\tau_{2}(\cdot)$ are convex functions. 

%*******************************************************************************
%\subsection{Relaxation via \mbox{CCCP}}
%
%It is easy to realize that all the terms in the objective function~(\ref{eqn:loss}) are convex except $\tau(\cdot)$, which, as explained above, can be expressed as a difference of two convex functions. As a consequence, we can use \mbox{CCCP} to solve the non-convex optimization problem iteratively with each iteration minimizing a convex upper bound of the original objective function.
%
%%
%%The only non-convex term in the objective function is $\tau(d_{n})$, but it can be easily decomposed to the difference of two convex functions~\cite{quadrianto2011icml}. 
%%
%%Because both $\tau_{1}(d_{n})$ and $\tau_{2}(d_{n})$ are convex functions \wrt (with respect to) $\w_{x}$ or $\w_{y}$, and we can use a general \mbox{CCCP} approach to get two convex upper bounds, respectively.
%
%Briefly speaking, given an objective function $f_{0}(x)-g_{0}(x)$ where both $f_{0}$ and $g_{0}$ are convex, \mbox{CCCP} works iteratively as follows.  The variable $x$ is first randomly initialized to $x_0$.  At the $t$th iteration, \mbox{CCCP} minimizes the following convex upper bound of $f_{0}(x)-g_{0}(x)$ at location $x_{t}$:
%$$f_{0}(x)-\big(g_{0}(x_{t})+\partial_{x}g_{0}(x_{t})(x-x_{t})\big),$$
%where $\partial_{x}g_{0}(x_{t})$ is the first derivative of $g(x)$ at $x_{t}$. This optimization problem can be solved using any convex optimization solver to obtain $x_{t+1}$.  Given an initial value $x_{0}$, the solution sequence $\{x_{t}\}$ found by \mbox{CCCP} is guaranteed to reach a local minimum or a saddle point. 
%
%For our problem, at the $t$th \mbox{CCCP} iteration, the convex upper bound of $\tau(d_{n})$ \wrt~$\w_{x}$ is:
%\begin{align}
%%\label{eqn:upperx}
%\hat{\tau}_{x}(d_{n}) &= \tau_{1}(d_{n})-\frac{({d}^{(t)}_{n})^{2}}{2}+\frac{a\lambda^2}{2}-{d}^{(t)}_{n}\x_{a_n}^{T}(\w_{x}-\w_{x}^{(t)})\nonumber,
%\end{align}
%where $\w_{x}^{(t)}$ is the value of $\w_{x}$ at the $t$th iteration and ${d}^{(t)}_{n} = (\w^{(t)}_{x})^{T}\x_{a_n}-\w_{y}^{T}\y_{b_n}$.
%
%
%Similarly, the convex upper bound of $\tau(d_{n})$ \wrt~$\w_{y}$ at the $t$th iteration is:
%\begin{align}
%%\label{eqn:uppery}
%\hat{\tau}_{y}(d_{n}) &= \tau_{1}(d_{n})-\frac{({d}^{(t)}_{n})^{2}}{2}+\frac{a\lambda^2}{2}+{d}^{(t)}_{n}\y_{b_n}^{T}(\w_{y}-\w_{y}^{(t)}),\nonumber
%\end{align}
%where $\w_{y}^{(t)}$ is the value of $\w_{y}$ at the $t$th iteration and ${d}^{(t)}_{n} = \w_{x}^{T}\x_{a_n}-(\w^{(t)}_{y})^{T}\y_{b_n}$.
%%The convex upper bound \wrt $\w_{y}$ is:
%%\begin{align}
%%\hat{\tau}(n) &= \tau_{1}(n)-\frac{1}{2}(\hat{d}^{(t)}_{n})^{2}+\frac{a\lambda^2}{2}+d^{(t)}_{n}\y_{b_n}^{T}(\w_{2}-\w_{2}^{t})\nonumber,
%%\end{align}
%%where $\hat{d}^{(t)}_{n} = \w_{1}^{T}\x_{a_n}-(\w^{(t)}_{2})^{T}\y_{b_n}$.


%*******************************************************************************
\subsection{Optimization}

Though the objective function (\ref{eqn:loss}) is nonconvex \wrt $\w_{x}$ and $\w_{y}$, we can optimize it \wrt~$\w_{x}$ and $\w_{y}$ in an alternating manner. Take $\w_{x}$ for example, we remove the irrelevant terms and get the following objective:
\begin{align}
\frac{1}{I}\sum\limits_{i=1}^{I}\ell_{i}^ {x}+\frac{\lambda_{x}}{2}\|\w_{x}\|^{2}+\gamma\sum\limits_{n=1}^{N}\omega_{n}\ell_{n}^{*},
\label{obj:x}
\end{align}
where
\begin{align}
\ell_{i}^{x}=\left\{ \begin{array}{ll}
         0 & \mbox{if ~}|\w_{x}^{T}\x_{i}| \ge 1 \\
         1- \w_{x}^{T}\x_{i} & \mbox{if ~}0\le\w_{x}^{T}\x_{i} < 1 \\
         1+\w_{x}^{T}\x_{i} & \mbox{if ~} -1 <\w_{x}^{T}\x_{i}<0\nonumber
                          \end{array} \right..
%
\end{align}

It is easy to realize that the objective function~(\ref{obj:x}) can be expressed as a difference of two convex functions in different cases. As a consequence, we can use \mbox{CCCP} to solve the non-convex optimization problem iteratively with each iteration minimizing a convex upper bound of the original objective function.

Briefly speaking, given an objective function $f_{0}(x)-g_{0}(x)$ where both $f_{0}$ and $g_{0}$ are convex, \mbox{CCCP} works iteratively as follows.  The variable $x$ is first randomly initialized to $x^{(0)}$.  At the $t$th iteration, \mbox{CCCP} minimizes the following convex upper bound of $f_{0}(x)-g_{0}(x)$ at location $x^{(t)}$:
$$f_{0}(x)-\big(g_{0}(x^{(t)})+\partial_{x}g_{0}(x^{(t)})(x-x^{(t)})\big),$$
where $\partial_{x}g_{0}(x^{(t)})$ is the first derivative of $g(x)$ at $x^{(t)}$. This optimization problem can be solved using any convex optimization solver to obtain $x^{(t+1)}$.  Given an initial value $x^{(0)}$, the solution sequence $\{x^{(t)}\}$ found by \mbox{CCCP} is guaranteed to reach a local minimum or a saddle point.

For our problem, the optimization problem at the $t$th iteration is minimizing the following upper bound of Equation~(\ref{obj:x}) \wrt $\w_x$:
\begin{align}
\mathcal{O}_{x}=\frac{\lambda_{x}\|\w_{x}\|^{2}}{2}+\gamma\sum\limits_{n=1}^{N}\omega_{n}\left(s_{n}d_{n}^{2}+(1-s_{n})\zeta^{x}_{n}\right) + \frac{1}{I}\sum\limits_{i=1}^{I}\ell_{i}^{x},
\label{eqn:objx}
\end{align}
where $\zeta^{x}_{n} = \tau_{1}(d_{n})-\tau_{2}(d_{n}^{(t)})-{d}^{(t)}_{n}\x_{a_n}^{T}(\w_{x}-\w_{x}^{(t)}),{d}^{(t)}_{n} = (\w^{(t)}_{x})^{T}\x_{a_n}-\w_{y}^{T}\y_{b_n},$
%\begin{align}
%\iota_{i}^{x}=& \left\{ \begin{array}{ll}
%      0 & \mbox{if ~}|\w_{x}^{T}\x_{i}| \ge 1 \\
%      1-\left(|(\w^{(t)}_{x})^{T}\x_{i}|+\sgn\left((\w^{(t)}_{x})^{T}\x_{i}\right)\x_{i}^{T}(\w_{x}-\w_{x}^{(t)})\right)& \mbox{if ~} |\w_{x}^{T}\x_{i}|<1\nonumber
%                       \end{array} \right.,
%\end{align}
and $\w_{x}^{(t)}$ is the value of $\w_{x}$ at the $t$th iteration.

%Note that we use subgradient of the absolute function since it is non-differentiable at some locations.
%where $\zeta^{x}_{n} = \tau_{1}(d_{n})-\tau_{2}(d_{n}^{(t)})-{d}^{(t)}_{n}\x_{a_n}^{T}(\w_{x}-\w_{x}^{(t)})$, $\iota_{i}^{x}=|(\w^{(t)}_{x})^{T}\x_{i}|+\sgn((\w^{(t)}_{x})^{T}\x_{i})\x_{i}^{T}(\w_{x}-\w_{x}^{(t)})$, $\w_{x}^{(t)}$ is the value of $\w_{x}$ at the $t$th iteration and ${d}^{(t)}_{n} = (\w^{(t)}_{x})^{T}\x_{a_n}-\w_{y}^{T}\y_{b_n}$.

%For each vector $\w_{x}$ or $\w_{y}$, the corresponding convex upper bound at the current location is minimized.
%\footnote{*** What do you mean by a new objective?  The objective function remains unchanged.  You are just taking a coordinate descent approach by optimizing the SAME objective function but with respect to DIFFERENT optimization variables. Answer: The convex upper bound \wrt different optimization variables are different, because the DC part is replaced by Taylor expansions at different locations.}
To find a local optimal solution of problem~(\ref{eqn:objx}), we can use any gradient based methods. In this work, we develop a stochastic gradient solver based on Pegasos~\cite{shalev2007icml}, which is known to be one of the fastest solvers for margin-based classifiers. Specifically, we randomly select $k$ point from each modality and $l$ point pairs to evaluate the gradient at each iteration. %Note that since the subproblem in each \mbox{CCCP} iteration is convex, in fact any off-the-shelf solver could be used.  We use the aforementioned solver due mainly to its efficiency. 
%\footnote{*** The logic of this paragraph is confusing.  First, the objective function is always the same.  Alternating between $\w_x$ and $\w_y$ is just the nature of the coordinate descent procedure.  Second, since the optimization problem in each CCCP iteration is convex, in principle any off-the-shelf solver can give the optimal solution.  However, the stochastic subgradient solver is used here for the efficiency concern.}

% each \mbox{CCCP} iteration, the objective involved is convex and any . In this paper, we  proposed by. 

%with respect to their convex upper bound, respectively.

%At the $t$th iteration, the optimization problem \wrt~$\w_{x}$ uses the following objective function which includes only those terms in $\mathcal{O}$ that depend on $\w_{x}$:
%\begin{align}
%\mathcal{O}_{x}
%=& \frac{1}{I}\sum\limits_{i=1}^{I}\big[1-|\w_{x}^{T}\x|\big]_{+}+\frac{\lambda_{x}}{2}\|\w_{x}\|^{2}\nonumber\\
%&+\gamma\sum_{n=1}^{N}\omega_{n}\left(s_{n}d_{n}^{2}+(1-s_{n})\hat{\tau}_{x}(d_{n})\right).
%\label{eqn:objx}
%\end{align}

The key step of our method is to evaluate the gradient of objective function~(\ref{eqn:objx}) \wrt~$\w_{x}$, which can be computed as
\begin{align}
\frac{\partial \mathcal{O}_{x}}{\partial \w_{x} }=2\gamma\sum\limits_{n=1}^{N}\omega_{n}s_{n}d_{n}\x_{a_n}+\gamma\sum\limits_{n=1}^{N}\omega_{n}\muu_{n}^{x}+\lambda_{x}\w_{x}- \frac{1}{I}\sum\limits_{i=1}^{I}\pii^{x}_{i},
\end{align}
%\begin{align}
%\frac{\partial \mathcal{O}_{x}}{\partial \w_{x} }=\left\{ \begin{array}{ll}
%      2\gamma\sum\limits_{n=1}^{N}\omega_{n}s_{n}d_{n}\x_{a_n}+\gamma\sum\limits_{n=1}^{N}\omega_{n}\muu_{n}^{x}+\lambda_{x}\w_{x} & \mbox{if ~}|\w_{x}^{T}\x| \ge 1 \\
%      2\gamma\sum\limits_{n=1}^{N}\omega_{n}s_{n}d_{n}\x_{a_n}+\gamma\sum\limits_{n=1}^{N}\omega_{n}\muu_{n}^{x}+\lambda_{x}\w_{x}- \frac{1}{I}\sum\limits_{i=1}^{I}\left(\sgn((\w^{(t)}_{x})^{T}\x_{i})\x_{i}\right)& \mbox{if ~} |\w_{x}^{T}\x|<1\nonumber
%                       \end{array} \right.
%\end{align}
where $\muu_{n}^{x}=(1-s_{n}) \left(\frac{\partial \tau_1}{\partial d_{n}}-{d}^{(t)}_{n}\right)\x_{a_n} $,
\begin{align}
\frac{\partial \tau_1}{\partial d_{n}} & =\left\{ \begin{array}{ll}
         0 & \mbox{if ~} |d_{n}| \le \lambda \\
         \frac{ad_{n}-2a\lambda\sgn(d_{n})}{(a-1)}& \mbox{if ~}\lambda < |d_{n}|\le a\lambda\\
         d_{n} & \mbox{if ~}a\lambda<|d_{n}|.\nonumber
                          \end{array} \right.\ \mbox{and} \ 
                          \pii^{x}_{i}=\left\{ \begin{array}{ll}
                                0 & \mbox{if ~}|\w_{x}^{T}\x_{i}| \ge 1 \\
                                \sgn\left(\w^{T}_{x}\x_{i}\right)\x_{i}& \mbox{if ~} |\w_{x}^{T}\x_{i}|<1\nonumber
                                                 \end{array} \right..
\end{align}

%To update $\w_{x}$, we use \mbox{CCCP} coupled with stochastic gradient descent (\mbox{SGD})~\cite{bottou2007nips}. Specifically, at one \mbox{CCCP} iteration, the objective function for $\w_{x}$ is:

%Then we use stochastic gradient descent to find a local minimum, which is also a global minimum, of $\mathcal{L}_{x}$. We note that at some locations, the gradient of $f(x) = |x|$ may not exist, and we use subgradient $\nabla f(x) = \sgn(x)$ instead.

Similarly, the objective function for the optimization problem \wrt~$\w_{y}$ at the $t$th \mbox{CCCP} iteration is:
\begin{align}
\mathcal{O}_{y}=\frac{\lambda_{y}\|\w_{y}\|^{2}}{2}+\gamma\sum\limits_{n=1}^{N}\omega_{n}\left(s_{n}d_{n}^{2}+(1-s_{n})\zeta^{y}_{n}\right) + \frac{1}{J}\sum\limits_{j=1}^{I}\ell_{j}^{y},
\label{eqn:objy}
\end{align}
where $\zeta^{y}_{n} = \tau_{1}(d_{n})-\tau_{2}(d_{n}^{(t)})+{d}^{(t)}_{n}\y_{b_n}^{T}(\w_{y}-\w_{y}^{(t)}), {d}^{(t)}_{n} = \w_{x}^{T}\x_{a_n}-(\w^{(t)}_{y})^{T}\y_{b_n}$, $\w_{y}^{(t)}$ is the value of $\w_{y}$ at the $t$th iteration and 
\begin{align}
\ell_{j}^{y}=\left\{ \begin{array}{ll}
         0 & \mbox{if ~}|\w_{y}^{T}\y_{j}| \ge 1 \\
         1- \w_{y}^{T}\y_{j} & \mbox{if ~}0\le\w_{y}^{T}\y_{j} < 1 \\
         1+\w_{y}^{T}\y_{j} & \mbox{if ~} -1 <\w_{y}^{T}\y_{j}<0\nonumber
                          \end{array} \right..
%
\end{align}
%\begin{align}
%\iota_{j}^{y}=& \left\{ \begin{array}{ll}
%      0 & \mbox{if ~}|\w_{y}^{T}\y_{j}| \ge 1 \\
%      1-\left(|(\w^{(t)}_{y})^{T}\y_{j}|+\sgn\left((\w^{(t)}_{y})^{T}\y_{j}\right)\y_{j}^{T}(\w_{y}-\w_{y}^{(t)})\right)& \mbox{if ~} |\w_{y}^{T}\y_{j}|<1\nonumber
%                       \end{array} \right.,
%%\label{eqn:objx}
%\end{align}


%\begin{align}
%\mathcal{O}_{y}=\left\{ \begin{array}{ll}
%      \frac{\lambda_{y}\|\w_{y}\|^{2}}{2}+\gamma\sum\limits_{n=1}^{N}\omega_{n}\left(s_{n}d_{n}^{2}+(1-s_{n})\zeta^{y}_{n}\right) & \mbox{if ~}|\w_{y}^{T}\y| \ge 1 \\
%      1+\frac{\lambda_{y}\|\w_{y}\|^{2}}{2}+\gamma\sum\limits_{n=1}^{N}\omega_{n}\left(s_{n}d_{n}^{2}+(1-s_{n})\zeta^{y}_{n}\right) - \frac{1}{J}\sum\limits_{j=1}^{J}\iota_{j}^{y}& \mbox{if ~} |\w_{y}^{T}\y|<1\nonumber
%                       \end{array} \right.
%%\label{eqn:objx}
%\end{align}
%where $\zeta^{y}_{n} = \tau_{1}(d_{n})-\tau_{2}(d_{n}^{(t)})+{d}^{(t)}_{n}\y_{b_n}^{T}(\w_{y}-\w_{y}^{(t)})$, $\iota_{j}^{y}=|(\w^{(t)}_{y})^{T}\y_{i}|+\sgn((\w^{(t)}_{y})^{T}\y_{i})\y_{i}^{T}(\w_{y}-\w_{y}^{(t)})$, $\w_{y}^{(t)}$ is the value of $\w_{y}$ at the $t$th iteration and ${d}^{(t)}_{n} = \w_{x}^{T}\x_{a_n}-(\w^{(t)}_{y})^{T}\y_{b_n}$.

The corresponding gradient is given by
\begin{align}
\frac{\partial \mathcal{O}_{y}}{\partial \w_{y} }=-2\gamma\sum\limits_{n=1}^{N}\omega_{n}s_{n}d_{n}\y_{b_n}-\gamma\sum\limits_{n=1}^{N}\omega_{n}\muu_{n}^{y}+\lambda_{y}\w_{y}- \frac{1}{J}\sum\limits_{j=1}^{I}
\pii^{y}_{j},
\end{align}
where $\muu_{n}^{y}=(1-s_{n}) \left(\frac{\partial \tau_1}{\partial d_{n}}-{d}^{(t)}_{n}\right)\y_{b_n}$ and
\begin{align}
\pii^{y}_{j}=\left\{ \begin{array}{ll}
    0 & \mbox{if ~}|\w_{y}^{T}\y_{j}| \ge 1 \\
    \sgn\left(\w_{y}^{T}\y_{j}\right)\y_{j}& \mbox{if ~} |\w_{y}^{T}\y_{j}|<1\nonumber
                                     \end{array} \right..
\end{align}

%\begin{align}
%\frac{\partial \mathcal{O}_{y}}{\partial \w_{y} }=\left\{ \begin{array}{ll}
%      -2\gamma\sum\limits_{n=1}^{N}\omega_{n}s_{n}d_{n}\y_{b_n}-\gamma\sum\limits_{n=1}^{N}\omega_{n}\muu_{n}^{y}+\lambda_{y}\w_{y} & \mbox{if ~}|\w_{y}^{T}\y| \ge 1 \\
%      -2\gamma\sum\limits_{n=1}^{N}\omega_{n}s_{n}d_{n}\y_{b_n}-\gamma\sum\limits_{n=1}^{N}\omega_{n}\muu_{n}^{y}+\lambda_{y}\w_{y}- \frac{1}{J}\sum\limits_{j=1}^{I}\left(\sgn((\w^{(t)}_{y})^{T}\y_{j})\y_{j}\right)& \mbox{if ~} |\w_{y}^{T}\y|<1\nonumber
%                       \end{array} \right.
%\end{align}
%where $\muu_{n}^{y}=(1-s_{n}) \left(\frac{\partial \tau_1}{\partial d_{n}}-{d}^{(t)}_{n}\right)\y_{b_n} $.

%The optimization for each bit is stochastic gradient descent, and we will use subgradient~\cite{boyd2004convex} for those hinge loss functions. We should list the gradient here.



%*******************************************************************************
\subsection{Algorithm}

So far we have only discussed how to learn the hash functions for one bit of the hash codes.  To learn the hash functions for multiple bits, one could repeat the same procedure and treat the learning for each bit independently.  However, as reported in previous studies~\cite{wang2010cvpr,liu2011icml}, it is very important to take into consideration the relationships between different bits in \mbox{HFL}. In other words, to learn compact hash codes, we should coordinate the learning of hash functions for different bits.

To this end, we take the standard \mbox{AdaBoost}~\cite{freund1997adaboost} approach to learn multiple bits sequentially.
%\footnote{*** Has this same approach been used for \mbox{HFL} by others? ~\cite{bronstein2010cvpr} also used boosting to update $\omega_n$ for each pair, but their rule is different from ours.}
Intuitively, this approach allows learning of the hash functions in later stages to be aware of the bias introduced by their antecedents.  The overall algorithm of \mbox{CRH} is summarized in Algorithm~\ref{alg:CRH}. %\footnote{*** You are abusing the symbols $w$ and $N$ (which have been used before), leading to confusion.}

\begin{algorithm}[htb]
   \caption{Co-Regularized Hashing}
   \label{alg:CRH}
\begin{algorithmic}
%\begin{multicols}{2}
   \STATE {\bfseries Input:} \\
   $\mathcal{X},\mathcal{Y}$ -- multimodal data \\
   $\Theta$ -- inter-modality point pairs\\
   $K$ -- code length\\
   $\lambda_{x},\lambda_{y},\gamma$ -- regularization parameters\\
   $a,\lambda$ -- parameters for \mbox{SCISD} function
   \STATE {\bfseries Output:} \\
   $\w_{x}^{(k)}, k=1,\dots, K$ -- projection vectors for $\mathcal{X}$ \\
      $\w_{y}^{(k)}, k=1,\dots, K$ -- projection vectors for $\mathcal{Y}$ 
   \STATE
	\STATE {\bfseries Procedure:}
%   \STATE Initialize $noChange = true$.
   \STATE Initialize $\omega_{n}^{(1)} = 1/N, \, \forall n \in \{1,2,\dots,N\}$.
   \FOR{$k=1$ {\bfseries to} $K$}
%   \IF{$x_i > x_{i+1}$}
   \REPEAT
   \STATE Optimize Equation~(\ref{eqn:objx}) to get $\w_{x}^{(k)}$;
   \STATE Optimize Equation~(\ref{eqn:objy}) to get $\w_{y}^{(k)}$;
   \UNTIL{convergence.}

   \STATE Compute error of current hash functions:
   \begin{align}
   \epsilon_{k} = \sum\nolimits_{n=1}^{N}\omega^{(k)}_{n}\I_{[s_{n}\ne h_{n} ]},\nonumber
   \end{align}
   where $\I_{[a]} = 1$ if $a$ is true and $\I_{[a]} = 0$ otherwise, and
   \begin{align}
   h_{n} & = \left\{ \begin{array}{ll}
            1 & f(\x_{a_n}) = g(\y_{b_n}) \\
            0 & f(\x_{a_n}) \ne g(\y_{b_n})\nonumber
                             \end{array} \right..
   \end{align}
   \STATE Set $\beta_{k} = \epsilon_{k}/(1-\epsilon_{k}).$
   \STATE Update the weight for each point pair:
   $$\omega^{(k+1)}_{n} =\omega^{(k)}_{n}\beta_{k}^{1-\I_{[s_{n}\ne h_{n}]}}.$$
%   \ENDIF
   \ENDFOR
%\end{multicols}
\end{algorithmic}
\end{algorithm}

The first computationally expensive part of the algorithm is to evaluate the gradients. The time complexity is $O((k+l)d)$, where $d$ is the data dimensionality, and $k$ and $l$ are the numbers of random points and random pairs, respectively, for the stochastic gradient solver. In our experiments, we set $k=1$ and $l=500$.  We notice that further increasing the two numbers brings no significant performance improvement. We leave the theoretical study of the impact of $k$ and $l$ to our future work. Another major computational cost comes from updating the weights of the inter-modality point pairs.  The time complexity is $O(dN)$, where $N$ is the number of inter-modality point pairs.

To summarize, our algorithm scales linearly with the number of inter-modality point pairs and the data dimensionality. In practice, the number of inter-modality point pairs is usually small, making our algorithm very efficient.
%\begin{figure}[htb] %\vspace{-0.3cm}
%\begin{wrapfigure}{r}{0.6\textwidth}
%\subfigure[Upating $\w_{x}$]{\label{fig:convergex}
%    \begin{minipage}[b]{0.45\linewidth} %\vspace{-0.4cm}
%%        \centering\vspace{-1cm}
%        \epsfig{figure=fig/convergence_x, width=0.8\textwidth} %\vspace{-1.5cm}
%    \end{minipage}}
%\subfigure[Upating $\w_{y}$]{\label{fig:convergey}
%    \begin{minipage}[b]{0.45\linewidth} %\vspace{-0.4cm}
%%        \centering\vspace{-1cm}
%        \epsfig{figure=fig/convergence_y, width=0.8\textwidth} %\vspace{-1.5cm}
%    \end{minipage}}
%%\vspace{-0.25cm}
%\caption{Illustration of convergence behavior}\label{fig:converge}\end{wrapfigure}
%%\vspace{-0.4cm}
%\end{figure}

%\begin{multicols}{2}
%\begin{minipage}[b]{0.4\linewidth}
%In Figure~\ref{fig:converge}, we empirically show the convergence behavior of our algorithm by plotting the objective function values and the corresponding upper bounds \wrt~the number of iterations. We can see that the bounds are very tight and our algorithm converges very fast and becomes stable after 20 iterations.
%\end{minipage}
%\begin{minipage}[b]{0.3\linewidth} %\vspace{-0.4cm}
%        \centering %\vspace{-1cm}
%        \epsfig{figure=fig/convergence_x, width=0.8\textwidth} %\vspace{-1.5cm}
%        \caption{Upating $\w_{x}$}
%    \end{minipage}
%\begin{minipage}[b]{0.3\linewidth} %\vspace{-0.4cm}
%        \centering %\vspace{-1cm}
%        \epsfig{figure=fig/convergence_y, width=0.8\textwidth} %\vspace{-1.5cm}
%        \caption{Upating $\w_{x}$}
%    \end{minipage}
%%\begin{minipage}{0.65\linewidth}



%\end{minipage}
%\end{multicols}
%*******************************************************************************
\subsection{Extensions}
\label{sec:moel:ext}
%\subsection{Kernelization}
We briefly discuss two possible extensions of \mbox{CRH} in this subsection.  First, we note that it is easy to extend \mbox{CRH} to learn nonlinear hash functions via the kernel trick~\cite{shawe2004book}. Specifically, according to the generalized representer theorem~\cite{scholkopf2001colt}, we can represent the projection vectors $\w_{x}$ and $\w_{y}$ as
\begin{align}
\w_{x} = \sum\nolimits_{i=1}^{I}\alpha_{i}\phi_{x}(\x_i) \ \ \mbox{and} \ \ \w_{y} = \sum\nolimits_{j=1}^{J}\beta_{j}\phi_{y}(\y_j),\nonumber
\end{align}
where $\phi_{x}(\cdot)$ and $\phi_{y}(\cdot)$ are kernel-induced feature maps for modalities $\mathcal{X}$ and $\mathcal{Y}$, respectively. Then the objective function~(\ref{eqn:loss}) can be expressed in kernel form and kernel-based hash functions can be learned by minimizing a new but very similar objective function.
%Because we do not use kernel in our experiments and discussion, and kernel is extremely important for machine learning and vision problems, we should clearly talk about the kernel extension and convince the reviewers.

%\subsection{Beyond Two Modalities}

Another possible extension is to make \mbox{CRH} support more than two modalities. Taking a new modality $\mathcal{Z}$ for example, we need to incorporate into Equation~(\ref{eqn:loss}) the following terms: loss and regularization terms for $\mathcal{Z}$, and all pairwise loss terms involving $\mathcal{Z} $ and other modalities, e.g., $\mathcal{X} $ and $\mathcal{Y}$.

For both extensions, it is straightforward to adapt the algorithm presented above to solve the new optimization problems.

%*******************************************************************************
\subsection{Discussion}
\mbox{CRH} is closely related to a recent multimodal metric learning method called MultiNPP~\cite{quadrianto2011icml}, because \mbox{CRH} uses a loss function for inter-modality point pairs which is similar to MultiNPP. However, \mbox{CRH} is a general framework and other loss functions for inter-modality point pairs can also be adopted. The two methods have at least three significant differences.  First, our focus is on \mbox{HFL} while MultiNPP is on metric learning through embedding. Second, in addition to the inter-modality loss term, the objective function in \mbox{CRH} includes two intra-modality loss terms for large margin HFL while MultiNPP only has a loss term for the inter-modality point pairs.
Third, CRH uses boosting to sequentially learn the hash functions but MultiNPP does not take this aspect into consideration. 

As discussed briefly in~\cite{quadrianto2011icml}, one may first use MultiNPP to map multimodal data into a common real space and then apply any unimodal \mbox{HFL} method for multimodal hashing. However, this naive two-stage approach has some limitations.  First, both stages can introduce information loss which impairs the quality of the hash functions learned.  Second, a two-stage approach generally needs more computational resources.  These two limitations can be overcome by using a one-stage method such as \mbox{CRH}.

% % % % % % % % % % % % % % % % % % % % % % % % % % % % % % %
\section{Experiments}
\label{crh:exps}

%-------------------------------------------------------------------------------
\subsection{Experimental Settings}

In our experiments, we compare \mbox{CRH} with two state-of-the-art multimodal hashing methods, namely, \mbox{CMSSH}~\cite{bronstein2010cvpr}\footnote{We used the implementation generously provided by the authors.} and \mbox{CVH}~\cite{kumar2011ijcai},\footnote{We implemented the method ourselves because the code is not publicly available.} for two crossmodal retrieval tasks: (1)~\textit{image query vs.\ text database}; (2)~\textit{text query vs.\ image database}. The goal of each retrieval task is to find from the text (image) database the nearest neighbors for the image (text) query.

%\footnote{We have tried several two-stage methods which combine  \mbox{Multi-NPP} and some representative unimodal \mbox{HFL} methods, e.g., \mbox{Multi-NPP+SH} and \mbox{Multi-NPP+LSH}, but did not obtain comparable results. We do not report them in the paper due to page limitations.}

We use two benchmark data sets which are, to the best of our knowledge, the largest fully paired and labeled multimodal data sets. We further divide each data set into a database set and a query set. To train the models, we randomly select a group of documents from the database set to form the training set. Moreover, we randomly select 0.1\% of the point pairs from the training set. For fair comparison, all models are trained on the same training set and the experiments are repeated 5 times.
%\footnote{*** Do you mean: we randomly select 0.1\% of the point pairs from the training set?}
%$$\mbox{AP} = \frac{1}{L}\sum_{r=1}^{R} P(r) \, \delta(r),$$

The mean average precision (\mbox{mAP}) is used as the performance measure. To compute the \mbox{mAP}, we first evaluate the average precision (\mbox{AP}) of a set of $R$ retrieved documents as $\mbox{AP} = \frac{1}{L}\sum_{r=1}^{R} P(r) \, \delta(r)$, where $L$ is the number of true neighbors in the retrieved set, $P(r)$ denotes the precision of the top $r$ retrieved documents, and $\delta(r)=1$ if the $r$th retrieved document is a true neighbor and $\delta(r)=0$ otherwise.  The \mbox{mAP} is then computed by averaging the \mbox{AP} values over all the queries in the query set. The larger the \mbox{mAP}, the better the performance. In the experiments, we set $R=50$. Besides, we also report the precision and recall within a fixed Hamming radius.

We use cross-validation to choose the parameters for \mbox{CRH} and find that the model performance is only mildly sensitive to the parameters. As a result, in all experiments, we set $\lambda_{x}=0.01, \lambda_{y}=0.01, \gamma = 1000, a=3.7$, and $\lambda=1/a$. Besides, unless specified otherwise, we fix the training set size to $2{,}000$ and the code length $K$ to 24.

%Our model is mildly sensitive to the parameters. 

%*******************************************************************************
\subsection{Results on \mbox{Wiki} Data Set}

The \mbox{Wiki} data set, generated from Wikipedia featured articles, consists of $2{,}866$ image-text pairs.\footnote{\url{http://www.svcl.ucsd.edu/projects/crossmodal/}} In each pair, the text is an article describing some events or people and the image is closely related to the content of the article. The images are represented by 128-dimensional \mbox{SIFT}~\cite{lowe2004ijcv} feature vectors, while the text articles are represented by the probability distributions over 10 topics learned by a latent Dirichlet allocation (\mbox{LDA}) model~\cite{blei2003jmlr}. Each pair is labeled with one of 10 semantic classes.  We simply use these class labels to identify the neighbors. Moreover, we use 80\% of the data as the database set and the remaining 20\% to form the query set.

The mAP values of the three methods and a method based on binarizing MultiNPP (Bin-MultiNPP) are reported in Table~\ref{crh:table:wiki-compare-map}.  We can see that \mbox{CRH} outperforms \mbox{CVH} and \mbox{CMSSH} under all settings and \mbox{CVH} performs slightly better than \mbox{CMSSH}.  We note that \mbox{CMSSH} ignores the intra-modality relational information and \mbox{CVH} simply treats each bit independently.  Hence the performance difference is expected. Also, we can see that binarizing MultiNPP directly always achieves the worst performance.

%\vspace{-0.4cm}
%\begin{table}[htb] 
%\caption{\mbox{mAP} comparison on \mbox{Wiki}} %\vspace{0.05in}
%\label{crh:table:wiki-compare-map}
%\begin{center}
%\begin{tabular}{|c|c|c|c|}
%\toprule[1pt]\addlinespace[0pt]
%\multirow{2}{7em}{\centering Task}&\multirow{2}{1.5cm}{\centering Method}&\multicolumn{2}{|c|}{Code Length}\\
%\cline{3-4}
%& &  $K=24$&  $K=48$\\
%\addlinespace[0pt]\midrule[1pt]\addlinespace[0pt]
%\multirow{3}{7em}{\centering Image Query \\ vs. \\Text Database}
%&\mbox{CRH}&${\bf 0.2607}$&${\bf 0.2393}$\\
%\cline{2-4}
%&\mbox{CVH}&${{0.1843}}$&${0.1894}$\\
%\cline{2-4}
%&\mbox{CMSSH}&${0.1785}$&${0.1666}$\\
%%\cline{2-4}
%%&\mbox{MultiNPP+SH}&${0.1577}$&${0.1577}$\\
%\addlinespace[0pt]\midrule[0.7pt]\addlinespace[0pt]
%\multirow{3}{7em}{\centering Text Query \\ vs. \\Image Database}
%&\mbox{CRH}&${\bf{0.3407}}$&${\bf 0.3167}$\\
%\cline{2-4}
%&\mbox{CVH}&${0.2839}$&${0.1812}$\\
%\cline{2-4}
%&\mbox{CMSSH}&${0.1977}$&${0.2030}$\\
%%\cline{2-4}
%%&\mbox{MultiNPP+SH}&${0.1577}$&${0.1577}$\\
%\addlinespace[0pt]\bottomrule[1pt]
%\end{tabular} %\vspace{-0.25cm}
%\end{center}
%\end{table}

\begin{table}[htb] 
\caption{\mbox{mAP} comparison on \mbox{Wiki}}\label{crh:table:wiki-compare-map}\vspace{-0.5cm}
\begin{center}
{\small
\begin{tabular}{|c|c|c|c|c|}
\toprule[1pt]\addlinespace[0pt]
\multirow{2}{7em}{\centering Task}&\multirow{2}{1.5cm}{\centering Method}&\multicolumn{3}{|c|}{Code Length}\\
\cline{3-5}
& &  $K=24$&  $K=48$&  $K=64$\\
\addlinespace[0pt]\midrule[1pt]\addlinespace[0pt]
\multirow{4}{7em}{\centering Image Query \\ vs. \\Text Database}
&\mbox{CRH}&${\bf 0.2537\pm0.0206}$&${\bf 0.2399\pm0.0185}$&${\bf 0.2392\pm0.0131}$\\
\cline{2-5}
&\mbox{CVH}&${{0.2043\pm0.0150}}$&${0.1788\pm 0.0149}$&${0.1732\pm0.0072}$\\
\cline{2-5}
&\mbox{CMSSH}&${0.1965\pm 0.0123}$&${0.1780\pm0.0080}$&${0.1624\pm0.0073}$\\
\cline{2-5}
&\mbox{Bin-MultiNPP}&${0.1790\pm 0.0202}$&${0.1672\pm0.0130}$&${0.1628\pm0.0175}$\\
\addlinespace[0pt]\midrule[0.7pt]\addlinespace[0pt]
\multirow{4}{7em}{\centering Text Query \\ vs. \\Image Database}
&\mbox{CRH}&${\bf{0.2896\pm0.0214}}$&${\bf 0.2882\pm0.0261}$&${\bf 0.2989\pm0.0293}$\\
\cline{2-5}
&\mbox{CVH}&${0.2714\pm0.0164}$&${0.2304\pm0.0104}$&${0.2156\pm0.0202}$\\
\cline{2-5}
&\mbox{CMSSH}&${0.2179\pm 0.0161}$&${0.2094\pm0.0072}$&${0.2040\pm0.0135}$\\
\cline{2-5}
&\mbox{Bin-MultiNPP}&${0.1925\pm0.0173}$&${0.1927\pm0.0284}$&${0.1847\pm0.0195}$\\
\addlinespace[0pt]\bottomrule[1pt]
\end{tabular} \vspace{-0.25cm}
}
\end{center}
\end{table}

\begin{figure}[ht]
\begin{center}
\subfigure[Varying Code Length]{\label{crh:fig:wiki-code-xy}
    \begin{minipage}[b]{0.45\linewidth} %\vspace{-0.3cm}
%        \centering\vspace{-1cm}
        \epsfig{figure=fig/crh/wiki-comp-code-xy, width=0.8\textwidth}%, height=3.4cm} %\vspace{-1.5cm}
    \end{minipage}}
\subfigure[Varying Code Length]{\label{crh:fig:wiki-code-yx}
    \begin{minipage}[b]{0.45\linewidth} %\vspace{-0.3cm}
%        \centering\vspace{-1cm}
        \epsfig{figure=fig/crh/wiki-comp-code-yx, width=0.8\textwidth}%, height=3.4cm} %\vspace{-1.5cm}
    \end{minipage}}
\\
\subfigure[Varying Training Set]{\label{crh:fig:wiki-train-xy}
    \begin{minipage}[b]{0.45\linewidth} %\vspace{-0.3cm}
%        \centering\vspace{-1cm}
        \epsfig{figure=fig/crh/wiki-comp-train-xy, width=0.8\textwidth}%, height=3.4cm} %\vspace{-1.5cm}
    \end{minipage}}
\subfigure[Varying Training Set]{\label{crh:fig:wiki-train-yx}
    \begin{minipage}[b]{0.45\linewidth} %\vspace{-0.3cm}
%        \centering\vspace{-1cm}
        \epsfig{figure=fig/crh/wiki-comp-train-yx, width=0.8\textwidth}%, height=3.4cm} %\vspace{-1.5cm}
    \end{minipage}}
\\
\subfigure[Pre-Rec Curve]{\label{crh:fig:wiki-pr-xy}
    \begin{minipage}[b]{0.45\linewidth} %\vspace{-0.3cm}
%        \centering\vspace{-1cm}
        \epsfig{figure=fig/crh/wiki-comp-pr-xy, width=0.8\textwidth}%, height=3.4cm} %\vspace{-1.5cm}
    \end{minipage}}
\subfigure[Pre-Rec Curve]{\label{crh:fig:wiki-pr-yx}
    \begin{minipage}[b]{0.45\linewidth} %\vspace{-0.3cm}
%        \centering\vspace{-1cm}
        \epsfig{figure=fig/crh/wiki-comp-pr-yx, width=0.8\textwidth}%, height=3.4cm} %\vspace{-1.5cm}
    \end{minipage}}
\\
\subfigure[Recall Curve]{\label{crh:fig:wiki-rec-xy}
    \begin{minipage}[b]{0.45\linewidth} %\vspace{-0.3cm}
%        \centering\vspace{-1cm}
        \epsfig{figure=fig/crh/wiki-comp-rec-xy, width=0.8\textwidth}%, height=3.4cm} %\vspace{-1.5cm}
    \end{minipage}}
\subfigure[Recall Curve]{\label{crh:fig:wiki-rec-yx}
    \begin{minipage}[b]{0.45\linewidth} %\vspace{-0.3cm}
%        \centering\vspace{-1cm}
        \epsfig{figure=fig/crh/wiki-comp-rec-yx, width=0.8\textwidth}%, height=3.4cm} %\vspace{-1.5cm}
    \end{minipage}} %\vspace{-0.2cm}
\end{center} \vspace{-0.5cm}
\caption{Results on \mbox{Wiki}}\label{crh:fig:wiki-compare-curve}
\end{figure}

We further compare the three methods on several aspects in Figure~\ref{crh:fig:wiki-compare-curve}. We first vary the code length $K$ and plot the precision within a Hamming radius of 2 in subfigures~\ref{crh:fig:wiki-code-xy} and~\ref{crh:fig:wiki-code-yx}. As $K$ increases, the performance of \mbox{CRH} also improves but the other two methods cannot benefit from increasing $K$. We then vary the size of the training set in subfigures~\ref{crh:fig:wiki-train-xy} and~\ref{crh:fig:wiki-train-yx}. Although \mbox{CVH} performs the best when the training set is small, its performance is gradually surpassed by \mbox{CRH} as the size increases.  In the remaining subfigures, we plot the precision-recall curves and recall curves for all three methods. It is obvious that \mbox{CRH} outperforms its two counterparts by a large margin.

%The precision-recall curves and recall curves show that \mbox{CRH} achieves the best performance. We set the code length $K=24$ and training set size $2{,}000$ to get the figures, if not specifically indicated.


%The first row is for the task of \textit{image query vs.\ text database} and the second row is for the task of \textit{text query vs.\ image database}.

%*******************************************************************************
\subsection{Results on \mbox{Flickr} Data Set}

The \mbox{Flickr} data set consists of $186{,}577$ image-tag pairs  pruned from the \mbox{NUS} data set\footnote{\url{http://lms.comp.nus.edu.sg/research/NUS-WIDE.htm}}~\cite{nus-wide-civr09} by keeping the pairs that belong to one of the 10 largest classes. The images are represented by 500-dimensional \mbox{SIFT} vectors. To obtain more compact representations of the tags, we perform \mbox{PCA} on the original tag occurrence features and obtain 1000-dimensional feature vectors. Each pair is annotated by at least one of 10 semantic labels, and two points are defined as neighbors if they share at least one label. We use 99\% of the data as the database set and the remaining 1\% to form the query set.


The mAP values of the three methods are reported in Table~\ref{crh:table:flickr-compare-map}. In the task of image query vs. text database, \mbox{CRH} performs comparably to \mbox{CMSSH}, which is better than \mbox{CVH}. However, in the other task, \mbox{CRH} achieves the best performance. At the same time, we observe that CRH beat Bin-MultiNPP by a large margin.


%\vspace{-0.5cm}
%\begin{table}[htb] %
%\caption{\mbox{mAP} comparison on \mbox{Flickr}} %\vspace{0.05in}
%\label{crh:table:flickr-compare-map}
%\begin{center}
%\begin{tabular}{|c|c|c|c|}
%\toprule[1pt]\addlinespace[0pt]
%\multirow{2}{7em}{\centering Task}&\multirow{2}{1.5cm}{\centering Method}&\multicolumn{2}{|c|}{Code Length}\\
%\cline{3-4}
%& &  $K=24$&  $K=48$\\
%\addlinespace[0pt]\midrule[1pt]\addlinespace[0pt]
%\multirow{3}{7em}{\centering Image Query \\ vs. \\Text Database}
%&\mbox{CRH}&${\bf 0.5393}$&${0.5336}$\\
%\cline{2-4}
%&\mbox{CVH}&${{0.4704}}$&${0.4512}$\\
%\cline{2-4}
%&\mbox{CMSSH}&${0.5356}$&${\bf 0.5361}$\\
%%\cline{2-4}
%%&\mbox{MultiNPP+SH}&${0.}$&${0.}$\\
%\addlinespace[0pt]\midrule[0.7pt]\addlinespace[0pt]
%\multirow{3}{7em}{\centering Text Query \\ vs. \\Image Database}
%&\mbox{CRH}&${\bf{0.5244}}$&${\bf 0.5170}$\\
%\cline{2-4}
%&\mbox{CVH}&${0.4560}$&${0.4511}$\\
%\cline{2-4}
%&\mbox{CMSSH}&${0.4970}$&${0.4790}$\\
%%\cline{2-4}
%%&\mbox{MultiNPP+SH}&${0.}$&${0.}$\\
%\addlinespace[0pt]\bottomrule[1pt]
%\end{tabular}
%\end{center} %
%\end{table}
%\vspace{-0.2cm}


\begin{table}[htb] %
\caption{\mbox{mAP} comparison on \mbox{Flickr}}\label{crh:table:flickr-compare-map}\vspace{-0.5cm}
\begin{center}
{\small
\begin{tabular}{|c|c|c|c|c|}
\toprule[1pt]\addlinespace[0pt]
\multirow{2}{7em}{\centering Task}&\multirow{2}{1.5cm}{\centering Method}&\multicolumn{3}{|c|}{Code Length}\\
\cline{3-5}
& &  $K=24$&  $K=48$&  $K=64$\\
\addlinespace[0pt]\midrule[1pt]\addlinespace[0pt]
\multirow{4}{7em}{\centering Image Query \\ vs. \\Text Database}
&\mbox{CRH}&${0.5259\pm 0.0094}$&${0.4990\pm 0.0075}$&${\bf 0.4929\pm 0.0064}$\\
\cline{2-5}
&\mbox{CVH}&${{0.4717\pm0.0035}}$&${0.4515\pm0.0041}$&$0.4471\pm 0.0023$\\
\cline{2-5}
&\mbox{CMSSH}&${\bf 0.5287\pm 0.0123}$&${\bf 0.5098\pm0.0141}$&$0.4911\pm 0.0220$\\
\cline{2-5}
&\mbox{Bin-MultiNPP}&${0.4775\pm0.0211}$&${0.4527\pm 0.0143}$&${0.4446\pm0.0259}$\\
\addlinespace[0pt]\midrule[0.7pt]\addlinespace[0pt]
\multirow{4}{7em}{\centering Text Query \\ vs. \\Image Database}
&\mbox{CRH}&${\bf{0.5364\pm 0.0021}}$&${\bf 0.5185\pm 0.0050}$&${\bf 0.5064\pm 0.0055}$\\
\cline{2-5}
&\mbox{CVH}&${0.4598\pm0.0020}$&${0.4519\pm0.0029}$&$0.4477\pm0.0058$\\
\cline{2-5}
&\mbox{CMSSH}&${0.5029\pm 0.0321}$&${0.4815\pm 0.0101}$&$0.4660\pm0.0298$\\
\cline{2-5}
&\mbox{Bin-MultiNPP}&${0.4767\pm 0.0147}$&${0.4524\pm 0.0115}$&${0.4450\pm0.0125}$\\
\addlinespace[0pt]\bottomrule[1pt]
\end{tabular}
}
\end{center} %
\end{table}

Similar to the previous subsection, we have conducted a group of experiments to compare the three methods on several aspects and report the results in Figure~\ref{crh:fig:flickr-compare-curve}. We first compare the precision under different code lengths in subfigures~\ref{crh:fig:flickr-code-xy} and~\ref{crh:fig:flickr-code-xy}. In almost all code lengths, \mbox{CRH} outperforms the other two methods. The results for varying the size of the training set are plotted in subfigures~\ref{crh:fig:flickr-train-xy} and~\ref{crh:fig:flickr-train-xy}. As more training data are used, \mbox{CRH} always performs better but the performance of \mbox{CVH} and \mbox{CMSSH} has high variance. Finally, the precision-recall curves and recall curves are shown in the remaining subfigures. Similar to the results on \mbox{Wiki}, \mbox{CRH} performs the best. However, the performance gap is smaller here.

%that \mbox{CRH} achieves the best performance. We set the code length $K=24$ and training set size $2{,}000$ to get the figures, if not specifically indicated.

\begin{figure}[ht]
\begin{center}
\subfigure[Varying Code Length]{\label{crh:fig:flickr-code-xy}
    \begin{minipage}[b]{0.45\linewidth} %\vspace{-0.3cm}
%        \centering\vspace{-1cm}
        \epsfig{figure=fig/crh/flickr-comp-code-xy, width=0.8\textwidth}%, height=3.4cm} %\vspace{-1.5cm}
    \end{minipage}}
\subfigure[Varying Code Length]{\label{crh:fig:flickr-code-yx}
    \begin{minipage}[b]{0.45\linewidth} %\vspace{-0.3cm}
%        \centering\vspace{-1cm}
        \epsfig{figure=fig/crh/flickr-comp-code-yx, width=0.8\textwidth}%, height=3.4cm} %\vspace{-1.5cm}
    \end{minipage}}
\\
\subfigure[Varying Training Set]{\label{crh:fig:flickr-train-xy}
    \begin{minipage}[b]{0.45\linewidth} %\vspace{-0.3cm}
%        \centering\vspace{-1cm}
        \epsfig{figure=fig/crh/flickr-comp-train-xy, width=0.8\textwidth}%, height=3.4cm} %\vspace{-1.5cm}
    \end{minipage}}
\subfigure[Varying Training Set]{\label{crh:fig:flickr-train-yx}
    \begin{minipage}[b]{0.45\linewidth} %\vspace{-0.3cm}
%        \centering\vspace{-1cm}
        \epsfig{figure=fig/crh/flickr-comp-train-yx, width=0.8\textwidth}%, height=3.4cm} %\vspace{-1.5cm}
    \end{minipage}}
\\
\subfigure[Pre-Rec Curve]{\label{crh:fig:flickr-pr-xy}
    \begin{minipage}[b]{0.45\linewidth} %\vspace{-0.3cm}
%        \centering\vspace{-1cm}
        \epsfig{figure=fig/crh/flickr-comp-pr-xy, width=0.8\textwidth}%, height=3.4cm} %\vspace{-1.5cm}
    \end{minipage}}
\subfigure[Pre-Rec Curve]{\label{crh:fig:flickr-pr-yx}
    \begin{minipage}[b]{0.45\linewidth} %\vspace{-0.3cm}
%        \centering\vspace{-1cm}
        \epsfig{figure=fig/crh/flickr-comp-pr-yx, width=0.8\textwidth}%, height=3.4cm} %\vspace{-1.5cm}
    \end{minipage}}
\\
\subfigure[Recall Curve]{\label{crh:fig:flickr-rec-xy}
    \begin{minipage}[b]{0.45\linewidth} %\vspace{-0.3cm}
%        \centering\vspace{-1cm}
        \epsfig{figure=fig/crh/flickr-comp-rec-xy, width=0.8\textwidth}%, height=3.4cm} %\vspace{-1.5cm}
    \end{minipage}}
\subfigure[Recall Curve]{\label{crh:fig:flickr-rec-yx}
    \begin{minipage}[b]{0.45\linewidth} %\vspace{-0.3cm}
%        \centering\vspace{-1cm}
        \epsfig{figure=fig/crh/flickr-comp-rec-yx, width=0.8\textwidth}%, height=3.4cm} %\vspace{-1.5cm}
    \end{minipage}}
\end{center}\vspace{-0.5cm}
\caption{Results on \mbox{Flickr}}\label{crh:fig:flickr-compare-curve}
 %\vspace{-0.4cm}
\end{figure}
%%\subsubsection{Discussion}
%%\label{MH:exps:disc}


%\subsection{Results on MIRFlickr Data Set}
%
%The MIRFlickr data set is ffered by the LIACS Medialab at Leiden University, The Netherlands.\footnote{\url{http://press.liacs.nl/mirflickr/}} There are 1 million images downloaded from the flickr.com, and each image is represented by two different descriptors, that is, the 150 dimensional EH descriptor and the 43 dimensional HT descriptor. Each image is also labeled with several tags. There are in total $1{,}386$ tags based on which we define similarity, meaning that we label two images as similar if the share one or more tags and dissimilar otherwise. We randomly choose $5{,}000$ points to form the query set and use the remaining data as the database set.
%
%The averaged precision values within the Hamming radius 2 are reported in Table~\ref{table:flickr1m-compare-precision}.
%
%\begin{table}[htb] %
%\caption{Precision comparison on \mbox{MIRFlickr}}\label{table:flickr1m-compare-precision} \vspace{-0.5cm}
%\begin{center}
%{\small
%\begin{tabular}{|c|c|c|c|c|}
%\toprule[1pt]\addlinespace[0pt]
%\multirow{2}{7em}{\centering Task}&\multirow{2}{1.5cm}{\centering Method}&\multicolumn{3}{|c|}{Code Length}\\
%\cline{3-5}
%& &  $K=24$&  $K=48$&  $K=64$\\
%\addlinespace[0pt]\midrule[1pt]\addlinespace[0pt]
%\multirow{4}{7em}{\centering Image Query \\ vs. \\Text Database}
%&\mbox{CRH}&${\bf 0.\pm 0.}$&${0.\pm 0.}$&${0.\pm 0.}$\\
%\cline{2-5}
%&\mbox{CVH}&${{0.}}$&${0.}$&\\
%\cline{2-5}
%&\mbox{CMSSH}&${0.}$&${\bf 0.}$&\\
%\cline{2-5}
%&\mbox{Bin-MultiNPP}&${0.}$&${0.}$&\\
%\addlinespace[0pt]\midrule[0.7pt]\addlinespace[0pt]
%\multirow{4}{7em}{\centering Text Query \\ vs. \\Image Database}
%&\mbox{CRH}&${\bf{0.\pm 0.}}$&${\bf 0.\pm 0.}$&${0.\pm 0.}$\\
%\cline{2-5}
%&\mbox{CVH}&${0.}$&${0.}$&\\
%\cline{2-5}
%&\mbox{CMSSH}&${0.}$&${0.}$&\\
%\cline{2-5}
%&\mbox{Bin-MultiNPP}&${0.}$&${0.}$&\\
%\addlinespace[0pt]\bottomrule[1pt]
%\end{tabular}
%}
%\end{center} %
%\end{table}

%The four different aspects are studied in Figure~\ref{crh:fig:flickr1m-compare-curve}.
%
%\begin{figure}[ht]
%\begin{center}
%\subfigure[Varying Code Length]{\label{crh:fig:flickr1m-code-xy}
%    \begin{minipage}[b]{0.45\linewidth} %\vspace{-0.3cm}
%%        \centering\vspace{-1cm}
%        \epsfig{figure=fig/crh/flickr-comp-code-xy, width=0.8\textwidth}%, height=3.4cm} %\vspace{-1.5cm}
%    \end{minipage}}
%\subfigure[Varying Training Set]{\label{crh:fig:flickr1m-train-xy}
%    \begin{minipage}[b]{0.45\linewidth} %\vspace{-0.3cm}
%%        \centering\vspace{-1cm}
%        \epsfig{figure=fig/crh/flickr-comp-train-xy, width=0.8\textwidth}%, height=3.4cm} %\vspace{-1.5cm}
%    \end{minipage}}
%\subfigure[Pre-Rec Curve]{\label{crh:fig:flickr1m-pr-xy}
%    \begin{minipage}[b]{0.45\linewidth} %\vspace{-0.3cm}
%%        \centering\vspace{-1cm}
%        \epsfig{figure=fig/crh/flickr-comp-pr-xy, width=0.8\textwidth}%, height=3.4cm} %\vspace{-1.5cm}
%    \end{minipage}}
%\subfigure[Recall Curve]{\label{crh:fig:flickr1m-rec-xy}
%    \begin{minipage}[b]{0.45\linewidth} %\vspace{-0.3cm}
%%        \centering\vspace{-1cm}
%        \epsfig{figure=fig/crh/flickr-comp-rec-xy, width=0.8\textwidth}%, height=3.4cm} %\vspace{-1.5cm}
%    \end{minipage}}
%    \\
%\subfigure[Varying Code Length]{\label{crh:fig:flickr1m-code-yx}
%    \begin{minipage}[b]{0.45\linewidth} %\vspace{-0.3cm}
%%        \centering\vspace{-1cm}
%        \epsfig{figure=fig/crh/flickr-comp-code-yx, width=0.8\textwidth}%, height=3.4cm} %\vspace{-1.5cm}
%    \end{minipage}}
%\subfigure[Varying Training Set]{\label{crh:fig:flickr1m-train-yx}
%    \begin{minipage}[b]{0.45\linewidth} %\vspace{-0.3cm}
%%        \centering\vspace{-1cm}
%        \epsfig{figure=fig/crh/flickr-comp-train-yx, width=0.8\textwidth}%, height=3.4cm} %\vspace{-1.5cm}
%    \end{minipage}}
%\subfigure[Pre-Rec Curve]{\label{crh:fig:flickr1m-pr-yx}
%    \begin{minipage}[b]{0.45\linewidth} %\vspace{-0.3cm}
%%        \centering\vspace{-1cm}
%        \epsfig{figure=fig/crh/flickr-comp-pr-yx, width=0.8\textwidth}%, height=3.4cm} %\vspace{-1.5cm}
%    \end{minipage}}
%\subfigure[Recall Curve]{\label{crh:fig:flickr1m-rec-yx}
%    \begin{minipage}[b]{0.45\linewidth} %\vspace{-0.3cm}
%%        \centering\vspace{-1cm}
%        \epsfig{figure=fig/crh/flickr-comp-rec-yx, width=0.8\textwidth}%, height=3.4cm} %\vspace{-1.5cm}
%    \end{minipage}}\vspace{-0.2cm}
%\end{center}
%\caption{Results on \mbox{Flickr}}\label{crh:fig:flickr1m-compare-curve}
% %\vspace{-0.4cm}
%\end{figure}

%-------------------------------------------------------------------------------
\section{Conclusion}
\label{crh:conclusion}

In this chapter, we have presented a novel method for multimodal hash function learning based on a boosted co-regularization framework which is named co-regularized hashing (CRH). In \mbox{CRH}, there is no data assumption such as those the data are aligned or organized in graphs. Because the objective function of the optimization problem is in the form of a difference of convex functions, we develop an efficient learning algorithm based on \mbox{CCCP} and a stochastic gradient method.  Experimental study based on two benchmark data sets shows that \mbox{CRH} outperforms two state-of-the-art multimodal hashing methods.

To take this work further, we would like to conduct theoretical analysis of \mbox{CRH} and apply it to some other tasks such as multimodal medical image alignment. Another possible research issue is to develop sublinear optimization algorithms to further improve the scalability of \mbox{CRH}.



\chapter{Conclusion and Future Work}
\label{chap:conclusion}

%In this chapter, we conclude the whole thesis and propose several possible directions for future pursuit.

%-------------------------------------------------------------------------------
\section{Conclusion}
%\label{Conclusion:con}

As a new research topic in the machine learning area, hash function learning has attracted plenty of interest in recent years. Different from locality sensitive hashing methods which design data-independent hash functions for specific similarity measures, hash function learning methods learn data-dependent hash functions directly from the data.

In the present thesis, we explore two research issues in hash function learning, namely, active hashing and multimodal hashing. Active hashing aims to actively select the data from which to learn hash functions, whereas multimodal hashing focuses on learning the hash functions for data involving multiple modalities. The contributions of this thesis are summarized as follows:
\begin{description}
\item[Active Hashing] To eliminate the data selection bias and reduce the labeling cost of existing supervised or semi-supervised hash function learning methods, we propose the active hashing framework. Under this framework, the hash functions are learned from the data which are carefully selected before learning. To evaluate the data informativeness, a simple yet effective criterion which is based on uncertainty is proposed. We also present a batch mode algorithm for the data selection procedure. Through extensive experiments, we demonstrate that active hashing outperforms passive hashing by a large margin.

\item[Spectral Multimodal Hashing] To learn hash functions for multimodal data, we propose the spectral multimodal hashing method (\mbox{SMH}). Given that the data in different modalities are aligned, that is, every object has a representation in each modality, \mbox{SMH} learns the hash functions through spectral analysis of modality correlation. Due to its resemblance to canonical correlation analysis in statistics, SMH is easy to extend to support kernel similarity and more than two modalities. We also propose a novel method to learn the thresholds for the hash functions. Experimental results show that our \mbox{SMH} model outperforms the state-of-the-art methods. %However, \mbox{SMH} has an apparent limitation, that is, it is only for the aligned data which may not be available in some applications. In the next chapter, we propose a new multimodal hashing model for graph data which is more general than aligned data.

\item[Multimodal Latent Binary Embedding] For the multimodal data which are organized in similarity graphs, we propose a novel probabilistic model called multimodal latent binary embedding (MLBE) to learn the hash functions. As a latent factor model, MLBE regards the binary latent factors as hash codes and hence maps data points from multiple modalities to a common Hamming space in a principled probabilistic manner. Although finding exact posterior distributions of the latent factors (hash codes) is intractable, we devise an efficient alternating learning algorithm based on \mbox{MAP} estimation. Experimental results show that MLBE compares favorably with state-of-the-art multimodal hashing models. %For our future research, we will go beyond the point estimation approach presented in this paper to explore a more Bayesian treatment based on variational inference for enhanced robustness and efficiency. We would also like to extend \mbox{MLBE} to determine the code length $K$ automatically from data.  This is an important yet largely unaddressed issue in existing methods. Besides, we also plan to apply \mbox{MLBE} to other tasks such as multimodal medical image registration.

\item[Co-regularized Hashing] For the general multimodal data, we propose a new method based on two popular supervised learning techniques: boosting and co-regularization, and name it co-regularized hashing (CRH). We no longer assume that the data are aligned or organized in graphs in CRH. To learn each bit of the hash codes, the objective function is in the form of a difference of convex functions, and we develop an efficient learning algorithm based on the convex-concave procedure (\mbox{CCCP}) and a stochastic gradient method. Hash functions for different bits are learned using a standard boosting method. Comparative studies based on benchmark data sets show that \mbox{CRH} outperforms two state-of-the-art multimodal hashing methods.
\end{description}
%\begin{itemize}
%%\item[Active Hashing] We first present the active hashing framework, which provides a general approach to select labeled data from which existing semi-supervised hash function learning methods can learn. 
%
%\item We first study a simple case in which multimodal data are aligned, and propose one method based on spectral analysis.
%
%\item Then for graph data, which is more general than aligned data, we give a probabilistic model based on latent feature models.
%
%\item Finally, we move to the general case and present co-regularized hashing, which takes advantage of boosting and optimization in learning multimodal hash functions. Complete comparative experiments validate the effectiveness of our models.
%\end{itemize}

%Multimodal hashing is more general and challenging than most existing hashing methods which are for uni-modal data.

%-------------------------------------------------------------------------------
\section{Future Work}

%Though we have tried to solved some basic problems of active hashing and multimodal hashing, there are still several open questions needing to be answered. 

To take our work further, we wish to study the following topics in the near future:

\begin{description}
\item[Challenging issues in Multimodal Hashing] In multimodal hash function learning, there are several challenging issues worth further studying. Among other things, it is the most important to learn an appropriate length of hash codes automatically from data. It is well known that the hash codes generated by HFL methods are always much shorter than those generated by LSH methods, however, there is no benchmark result on learning the code length. %Moreover, we would like to study some other applications of multimodal hash function learning, including medical image alignment and image or video de-duplication.

\item[Hierarchical Hash Function Learning] Most existing \mbox{HFL} methods assume that different hash bits are equally important, in other words, there is no structure in the hash codes. This assumption can be easily violated in applications where similarity search should be performed in different resolutions or levels. Although several researchers have made some preliminary attempts on generating multiple bits based on one hash function~\cite{liu2011icml}, to the best of our knowledge, no result has been reported on hash functions with structures.

We propose to develop hierarchical \mbox{HFL} methods. In addition to supporting different levels of search, the structural information of hash codes can be used to reduce the search space and hence further improve the scalability. For example, one can use coarse-level hash codes to find a candidate space and then fine-level hash codes for nearest neighbors. Moreover, hierarchical \mbox{HFL} can be used to combine different hashing models. For example, given that some hash functions are sensitive to recall while others are sensitive to precision, we can organize different kinds of hash codes in such a hierarchy that the new model can take advantages of different models.

\item[Sublinear Hash Function Learning]
Many \mbox{HFL} algorithms rely on expensive matrix operations such as eigen-decompostion and inversion. Obviously, they are not suitable for very high-dimensional data, such as text and image, unless the data are properly preprocessed. Although recent progress shows that linear time complexity can be expected~\cite{lin2010cvpr}, complicated approximation techniques are required in these methods. Furthermore, even linear time algorithms are sometimes not feasible in real large-scale applications. 

Recently, a family of sublinear algorithms has emerged in the machine learning area~\cite{hazan2011nips}. These algorithms have been applied to learning support vector machines (\mbox{SVM}) for classification problems. We believe that it is also possible to develop sublinear \mbox{HFL} algorithm, because some \mbox{HFL} algorithms such as CRH are similar to \mbox{SVM} training algorithms in some sense.

\item[Collaborative Filtering and Hash Function Learning]
Almost all \mbox{HFL} methods focus on similarity search applications. Nevertheless, other large-scale data mining tasks can also benefit from \mbox{HFL}. One outstanding example is collaborative filtering (\mbox{CF}), which is a very important technique for data mining and, more specifically, the recommender systems. So far as we know, several non-learning based hashing methods have been used to improve collaborative filtering~\cite{shi2009aistats,weinberger2009icml}, and we believe that \mbox{HFL} can make further improvement.

One possible approach may be to formulate \mbox{CF} as a multimodal hashing problem. To recommend movies to users, we just find nearest neighbor movies of the users in a common Hamming space. The learning problem needs to take into consideration the collaborative rating matrix as well as other side information, and the whole learning procedure should be conducted in sublinear time.

%\item[Applications of Hash Function Learning] 
%To further explore the topic of hash function learning, in the near future, we would like to explore the idea of learning hash functions for problems other than similarity search, such as classification, regression, ranking and collaborative filtering. 

\end{description}


%\subsection{Active Hashing}
%For active hashing, we plan to explore other criteria of data informativeness. For example, we may select the points which maximize information gain or minimize the expected error on test data. Moreover, although we have demonstrated that it is more effective to select and label points rather than pairwise constraints, it would be interesting to explore the informativeness of pairwise constraints directly for some supervised \mbox{HFL} methods. Last but not least, another possible research direction is to explore other batch mode algorithms for active hashing.






%\subsubsection{Unified Framework for Hash Function Learning }
%
%%\paragraph{What makes good codes?}
%%Yes, it depends on applications. But we should be clear when and why and how.
%There are different categorizations of \mbox{HFL} methods, such as supervised/unsupervised algorithms. However, little research effort has been put on summarize the assumptions in the unifying framework.
%
%We would like to propose a general framework, such as most current methods can be put into proper positions.







%%%%%%%%%%%%%%%%%%%%%%%%%%%%%%%%%%%%%%%%%%%%%%%%%%%%%%%%%%%%%%%%%%%%%%%%%
%                                                                       %
%      9) BIBLIOGRAPHY                                                  %
%                                                                       %
% This example uses bibtex to generate the required Bibliography. Refer %
% to the % the file ustthesis_test.bib for the entries of the           %
% Bibliography. Note that only the cited entries are printed.           %
%                                                                       %
% If BibTeX is not used to typeset the bibliography, replace the        %
% following line with the \begin{thebibliography} and \end{bibliography}%
% commands (the "thebibliography" environment) to process the           %
% Bibliography.                                                         %
%                                                                       %
%%%%%%%%%%%%%%%%%%%%%%%%%%%%%%%%%%%%%%%%%%%%%%%%%%%%%%%%%%%%%%%%%%%%%%%%%

%%%%%%%%%%%%%%%%%%%%%%%%%%%%%%%%%%%%%%%%%%%%%%%%%%%%%%%%%%%%%%%%%%%%%%%%%
%                                                                       %
% The recommended bibliography style is the IEEE bibliography style.    %
% "ustbib" defines the IEEE bibliography standard with the added        %
% ability of sorting the items by name of author.                       %
%                                                                       %
% If you are not using BibTeX to process your Bibliography, comment out %
% the following line.                                                   %
%                                                                       %
%%%%%%%%%%%%%%%%%%%%%%%%%%%%%%%%%%%%%%%%%%%%%%%%%%%%%%%%%%%%%%%%%%%%%%%%%

%\nocite{Wang+Zhang_PGM06,Wang+al_AAAI08,Wang+al_JAIR08,Zhang+al_JACM08,Zhang+al_JAIM08}
%\nocite{Zhang+al_JSSC08,Chen+al_PGM08,Zhang+al_AIME07}


%\bibliographystyle{apalike}
\bibliographystyle{theapa}
%\bibliographystyle{unsrt}
\bibliography{library_full}


% Please run "bibtex ustthesis_test" before the bibliography can be
% included.

%%%%%%%%%%%%%%%%%%%%%%%%%%%%%%%%%%%%%%%%%%%%%%%%%%%%%%%%%%%%%%%%%%%%%%%%%
%                                                                       %
%     10) APPENDIX (If Any)                                              %
%                                                                       %
% \appendix command marks the beginning of the APPENDIX part of the     %
% Thesis. The usual \chapter command is used for the different chapters %
% of the Appendix.                                                      %
%                                                                       %
%%%%%%%%%%%%%%%%%%%%%%%%%%%%%%%%%%%%%%%%%%%%%%%%%%%%%%%%%%%%%%%%%%%%%%%%%

\appendix
%%\input{TexFile/Appendix}

%%%%%%%%%%%%%%%%%%%%%%%%%%%%%%%%%%%%%%%%%%%%%%%%%%%%%%%%%%%%%%%%%%%%%%%%%
%                                                                       %
%     11) BIOGRAPHY (Optional)                                          %
%                                                                       %
% \biography and \endbiography are used to define the optional          %
% Biography of the author of the Thesis.                                %
%                                                                       %
%%%%%%%%%%%%%%%%%%%%%%%%%%%%%%%%%%%%%%%%%%%%%%%%%%%%%%%%%%%%%%%%%%%%%%%%%

% \biography
% The biography of the student is ALSO optional.
% \endbiography

\end{document}
