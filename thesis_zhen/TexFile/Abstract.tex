


To simplify modeling procedures, traditional statistical machine learning methods always assume that the instances are independent and identically distributed (i.i.d.). However, it is not uncommon for some real-world data, such as web pages and research papers, to contain relationships (links) between the instances. Different instances in such data are \emph{correlated} (linked) with each other, which implies that the common i.i.d. assumption is unreasonable for such \emph{relational data}. Hence, naively applying traditional statistical learning methods to relational data may lead to misleading conclusions about the data.

Statistical relational learning (SRL), which attempts to perform learning and inference in domains with complex relational structures, has become an emerging research area because relational data widely exist in a large variety of application areas, such as web mining, social network analysis, bioinformatics, economics and marketing. The existing mainstream SRL models extend traditional graphical models, such as Bayesian networks and Markov networks, by eliminating their underlying i.i.d. assumption. Some typical examples of such SRL models include \emph{relational Bayesian networks}, \emph{relational Markov networks}, and \emph{Markov logic networks}. Because the dependency structure in relational data is typically very complex, structure learning for these relational graphical models is often very time-consuming. Hence, it might be impractical to apply these models to large-scale relational data sets.

In this thesis, we propose a series of novel SRL models, called \emph{relational factor models} (RFMs), by extending traditional \emph{latent factor models} from i.i.d. domains to relational domains. These proposed RFMs provide a toolbox for different learning settings: some of them are well suited for transductive inference while others can be used for inductive inference; some of them are parametric while others are nonparametric; some of them can be used to model data with undirected relationships while others can be used for data with directed relationships. One promising advantage of our RFMs is that there is no need for time-consuming structure learning and the time complexity of most of them is linear to the number of observed links in the data. This implies that our RFMs can be used to model large-scale data sets. Experimental results show that our models can achieve state-of-the-art performance in many real-world applications such as linked-document classification and social network analysis.

